%%%%%%%%%%%%%%%%%%%%%%%%%%%%%%%%%%%%%%%%%%%%%%%%%%%%%%%%%%%%%%%%%%%%%%%%%%%%%%
%
% Chapter file included in main project file using \input{}
%
% Assumes that LaTeX2e macros and packages defined in rgb_laser_physics.sty
%   are available
%
%%%%%%%%%%%%%%%%%%%%%%%%%%%%%%%%%%%%%%%%%%%%%%%%%%%%%%%%%%%%%%%%%%%%%%%%%%%%%%

 \chapter{One-dimensional Multi-Mode Laser Dynamics\label{chp:laser_dynamics_1d_mml}}

In this chapter, we describe the dynamics of one-dimensional laser amplifiers and oscillators by applying the quasi-normal mode expansions derived in \sct{laser_resonators_1d_qnm} to the wave equation given by \eqn{wave_eqn_1d} and the density matrix evolution equations defined by \eqn{fls_mbe_rwa_pol} and \eqn{fls_mbe_rwa_pop_diff}. Under the right experimental conditions, these multimode representations (approximate as they are) can provide remarkably illuminating descriptions of laser behavior, including optimum output coupling, frequency pulling, wave mixing, and mode-locking.

When multiple modes oscillate in a laser, they give rise to coherent modulations of the populations in the nonlinear gain medium that create interactions between those modes. The frequencies of these modulations are integer multiples of the free spectral range $\Delta \omega_\text{FSR}$ --- defined by \eqn{delta_w_fsr_def} --- between adjacent intracavity field modes. In \fig{multimode_gain_spectrum_1d}, we show a plot of a gain medium with a peak at frequency $\omega_0 = \omega_{a b}$ as a function of the frequency detuning. We have superimposed the frequency modal structure --- over several free-spectral ranges --- of a cavity containing that medium. In the following sections, we will find that for a particular frequency $\Delta \omega_q \equiv 2 q \pi$, fluctuations in the gain medium at frequency $2 (q - p) \pi$ couple the electric field amplitude with frequency $2 p \pi$ to the macroscopic polarization component at frequency $2 q \pi$.

 \begin{figure}
  \centering
  \includegraphics[width=4.5in]{figures/multimode_gain_spectrum_1d}
  \caption{\label{fig:multimode_gain_spectrum_1d} Plot of a gain medium with a peak at frequency $\omega_0 = \omega_{a b}$ as a function of the frequency detuning. We have superimposed the frequency modal structure --- over several free-spectral ranges --- of a cavity containing the medium. We will find that for a particular frequency $\Delta \omega_q \equiv 2 q \pi$, fluctuations in the gain medium at frequency $2 (q - p) \pi$ couple the electric field amplitude with frequency $2 p \pi$ to the macroscopic polarization component at frequency $2 q \pi$.}
 \end{figure}

\section{One-Dimensional Multi-Mode Laser Evolution Equations\label{sct:laser_dynamics_1d_mml_evol_eqns}}

We begin by developing evolution equations for the complex longitudinal modal amplitudes of unidirectional and standing-wave intracavity laser fields based on the four-level Maxwell-Bloch equations given by \eqn{laser_statics_1d_sml_scaled} and \eqn{cw_sml_ftz_scaled}. We have
\begin{subequations}\label{eqn:laser_dynamics_1d_mml_scaled}
  \begin{align}
    \label{eqn:mml_etz_scaled}
    \ppt E^\pm\zt \pm \ppz E^\pm\zt &= \left[ i\, \widehat{\mathcal{D}}_0 - \half\, \an \right] E^\pm\zt + F^\pm\zt\, , \\
    \label{eqn:mml_ftz_scaled} \ppt \widetilde{F}\zt &= -\frac{1}{\tau_\perp} \left[ \mathcal{B}\, \widetilde{F}\zt - \frac{\mathcal{A}}{2}\, \widetilde{G}\zt \widetilde{E}\zt \right]\, , \nd \\
    \label{eqn:mml_gtz_scaled} \ppt \widetilde{G}\zt &= -\frac{1}{\tau_\parallel} \left\{ \widetilde{G}\zt - \Gn\zt + 2 \Re \left[ \widetilde{E}^\ast\zt \widetilde{F}\zt \right] \right\}\, ,
  \end{align}
\end{subequations}
where $\widehat{\mathcal{D}}_0$ is the differential operator defined by \eqn{cw_sml_disp_op}. Here we will defer the effects of frequency dispersion to \sct{laser_dynamics_1d_mml_frq_dis} by setting $\widehat{\mathcal{D}}_0 = 0$ in \eqn{mml_etz_scaled}.

Our goal will be to develop a set of nonlinear ordinary differential equations representing the time evolution of modal amplitudes of the electromagnetic field. Let's follow an approach similar to that used in \sct{laser_statics_1d_approx} and use the results of \sct{laser_resonators_1d_qnm} to expand $E^\pm\zt$ in terms of the quasi-normal modes of the laser resonator. For example, in the case of the one-dimensional unidirectional ring laser shown in \fig{resonator_1d_ring_gain}, $E^{-}\zt = 0$, and we can write the slowly-varying forward-propagating electric field amplitude as
\begin{equation}
  \label{eqn:mml_e_1d_zt_url}
  E^{+}\zt \equiv \sum_{q = -\infty}^\infty u_q\z\, e^{-i\, \Delta \omega_q\, t}\, E_q(t)\, ,
\end{equation}
where $u_q\z$ and the corresponding biorthogonal eigenfunction $v_q\z$ in the range $0 < z < 1$ are given by \eqn{laser_resonator_1d_u_unnorm} and \eqn{laser_resonator_1d_v_unnorm} as
\begin{subequations}
  \begin{align}
    \label{eqn:sml_1d_uq_url} u_q\z &=\mathcal{C}_\mathrm{URL}\, \exp\left[ +\left( i\, 2 q \pi + \ln\frac{1}{\sqrt{R}} \right) z \right]\, , \\
    \label{eqn:sml_1d_vq_url} v_q\z &=\mathcal{C}^{-1}_\mathrm{URL}\, \exp\left[ -\left( i\, 2 q \pi + \ln\frac{1}{\sqrt{R}} \right) z \right]\, ,
  \end{align}
\end{subequations}
$\mathcal{C}_\mathrm{URL}$ is given by \eqn{laser_resonator_1d_u_norm_url}, and
\begin{equation}\label{eqn:mml_1d_delta_w_q_def}
  \Delta \omega_q = 2 q \pi + \delta \omega_q\, ,
\end{equation}
consistent with both \eqn{cw_sml_etz_scaled} and \eqn{delta_w_q_def}. We apply the biorthogonality relation given by \eqn{laser_resonator_1d_uv_biortho} to \eqn{mml_etz_scaled} by substituting \eqn{mml_e_1d_zt_url} (with $q \longrightarrow p$) and a similar expression for $F^{+}\zt$; multiplying both sides through by $e^{+i\ \Delta \omega_q\, t}\, v_q\z$; and then integrating the result from $z = 0$ to $z = 1$. We find
\begin{equation} \label{eqn:mml_edot_temp}
  \dot{E}_q(t) = \left(-\frac{1}{2 \tau_\lambda} + i\, \delta \omega_q\right) E_q(t) + F_q(t)\, ,
\end{equation}
where $\tau_\lambda \equiv 1/\ln[1 / R \exp(-\anb)]$ is the photon lifetime\index{Photon lifetime} given by \eqn{f_fwhm} and $\anb \equiv \int_0^1 dz\, \alpha_0(z)$.

We shouldn't apply the rate-equation approximation (REA) to \eqn{mml_ftz_scaled} just yet, because a laser operating with $q_\text{max}$ longitudinal modes such that $q_\text{max}\, \Delta \omega_\text{FSR} \gtrsim 1 / \tau_\perp$ will exhibit a significant dependence of the unsaturated gain on the value of $q$. Instead, we will first substitute

because we want to keep the macroscopic polarization term $F_q(t)$ general for now. In the unidirectional ring laser case, we define $F_q(t)$ as

For the one-dimensional standing-wave laser shown in \fig{resonator_1d_sw_gain}, the slowly-varying biorthogonal eigenfunctions are $\mathbf{u}_q\z$ and $\mathbf{v}_q\z$, given by \eqn{laser_resonator_1d_u_sw_vec}, \eqn{laser_resonator_1d_u_sw}, \eqn{laser_resonator_1d_v_sw_vec}, and \eqn{laser_resonator_1d_v_sw}, and the corresponding normalization constant $\mathcal{C}_\mathrm{SWL}$ is given by \eqn{laser_resonator_1d_u_norm_swl}. Then
\begin{subequations} \label{eqn:sml_1d_uvq_swl}
  \begin{align}
    \label{eqn:sml_1d_uq_swl}
    \mathbf{u}_q\z &\equiv \begin{bmatrix} u^{+}_q\z \\ u^{-}_q\z \end{bmatrix} = \mathcal{C}_\mathrm{SWL} \begin{bmatrix} e^{+\left[ i\, 2 q \pi + \ln\left(1/\sqrt{R_1 R_2}\right) \right] z} \\ -\frac{1}{\sqrt{R_1}}\, e^{-\left[ i\, 2 q \pi + \ln\left(1/\sqrt{R_1 R_2}\right) \right] z} \end{bmatrix}\, , \nd \\
    \label{eqn:sml_1d_vq_swl}
    \mathbf{v}_q\z &\equiv \begin{bmatrix} v^{+}_q\z \\ v^{-}_q\z \end{bmatrix} = \mathcal{C}^{-1}_\mathrm{SWL} \begin{bmatrix} e^{-\left[ i\, 2 q \pi + \ln\left(1/\sqrt{R_1 R_2}\right) \right] z} \\ -\sqrt{R_1}\, e^{+\left[ i\, 2 q \pi + \ln\left(1/\sqrt{R_1 R_2}\right) \right] z} \end{bmatrix}\, ,
  \end{align}
\end{subequations}
where $0 < z < 1/2$ In this case, we apply the biorthogonality relation given by \eqn{laser_resonator_1d_uv_biortho_sw} to \eqn{cw_sml_ez_scaled} by substituting $E^{\pm}\z = \sum_p u^{\pm}_p\z\, \, e^{-i\, \Delta \omega_p\, t}\, E_p(t)$ and $F^{\pm}\z = \sum_p u^{\pm}_p\z\, \, e^{-i\, \Delta \omega_p\, t}\, F_p(t)$; forming the inner product of both sides with $e^{i\, \Delta \omega_q\, t}\, \mathbf{v}_q\z$; and then integrating the result from $z = 0$ to $z = 1/2$. Therefore, \eqn{e0_temp} remains valid for the standing-wave case with $\tau_\lambda \equiv 1 / \ln[1 / R_1 R_2 \exp(-\anb)]$ and $\anb \equiv 2 \int_0^{1/2} dz\, \alpha_0(z)$.

As a general representation of the spatially rapidly-varying fields in both unidirectional ring and standing-wave resonator configurations, we follow \sct{laser_resonators_1d_swl} and represent the electric field amplitude function as
\begin{equation}
  \label{eqn:mml_e_1d_zt_rv}
  \widetilde{E}\zt \equiv \sum_{q = -\infty}^\infty \widetilde{u}_q\z\, e^{-i\, \Delta \omega_q\, t}\, E_q(t)\, .
\end{equation}
In the unidirectional ring laser case,
\begin{subequations}
  \label{eqn:mml_1d_uvq_url}
  \begin{align}
    \label{eqn:mml_1d_uq_url} \widetilde{u}_q\z &= u^{+}_q\z\, e^{+i k_0 z}\, , \nd \\
    \label{eqn:mml_1d_vq_url} \widetilde{v}_q\z &= v^{+}_q\z\, e^{-i k_0 z}\, ,
  \end{align}
\end{subequations}
where $k_0$ is the propagation constant associated with the carrier frequency $\omega_0$. For a standing-wave resonator,
\begin{subequations}
  \label{eqn:mml_1d_uvq_swl}
  \begin{align}
    \label{eqn:mml_1d_uq_swl} \widetilde{u}_q\z &= u^{+}_q\z\, e^{+i k_0 z} + u^{-}_q\z\, e^{-i k_0 z}\, , \nd \\
    \label{eqn:mml_1d_vq_swl} \widetilde{v}_q\z &= v^{+}_q\z\, e^{-i k_0 z} + v^{-}_q\z\, e^{+i k_0 z}\, .
  \end{align}
\end{subequations}
We use a similar approach to the expansion of the amplitude of the macroscopic polarization.

\begin{multline} \label{eqn:mml_ftz_expansion}
  \sum_p \left[ \dot{F}_p(t) - i\, \Delta \omega_p\, F_p(t) \right] \widetilde{u}_p\z\, e^{-i\, \Delta \omega_p\, t} \\
  = -\frac{1}{\tau_\perp} \sum_p \left[ \mathcal{B}\, F_p(t) - \half\, \mathcal{A}\, \widetilde{G}\zt\, E_p(t) \right] \widetilde{u}_p\z\, e^{-i\, \Delta \omega_p\, t}\, .
\end{multline}
Now, rather than the rate-equation approximation (REA), we will apply the \emph{slowly-varying envelope approximation}\index{Slowly-varying envelope approximation} (SVEA) to \eqn{mml_ftz_expansion} by assuming that $|\dot{F}_p(t)| \ll |\Delta \omega_p F_p(t)|$ and neglecting the terms $\dot{F}_p(t)$ on the \lhs. This is valid when the time scale for changes in the modal polarization amplitudes $F_p(t)$ is long compared to the polarization relaxation time $\tau_\perp$.

\begin{equation} \label{eqn:mml_zop_def}
  \int d z \equiv \begin{cases}
    \int_0^{1} d z & \mbox{(URL)}\, , \\
    \int_0^{1/2} d z\, \frac{k_0}{2 \pi} \int_{z - \pi/k_0}^{z + \pi/k_0} d z' & \mbox{(SWL or SHB)}\, ,
  \end{cases}
\end{equation}

\begin{equation} %\label{eqn:mml_fq_sol_temp}
  \sum_p \mathcal{N}_{q p}\, F_p(t) = \half\, \Lq\, \sum_p e^{i \left(\Delta \omega_q - \Delta \omega_p\right) t}\, G_{q p}(t)\, E_p(t)\, , 
\end{equation}
\begin{equation} \label{eqn:mml_fq_sol_temp}
  \sum_p \mathcal{N}_{q p}\, \left(\mathcal{B} - i\, \Omega_p\right) F_p(t) = \half\, \mathcal{A}\, \sum_p e^{i \left(\Delta \omega_q - \Delta \omega_p\right) t}\, G_{q p}(t)\, E_p(t)\, , 
\end{equation}
where $\Omega_p \equiv \Delta \omega_q\, \tau_\perp$,
\begin{equation} \label{eqn:mml_nqp_def}
  \mathcal{N}_{q p} \equiv \int d z\, \widetilde{v}_q\z\, \widetilde{u}_p\z\, \nd
\end{equation}
\begin{equation} \label{eqn:mml_gqp_def}  
  G_{q p}(t) \equiv \int d z\, \widetilde{v}_q\z\, \widetilde{u}_p\z\, \widetilde{G}\zt\, .
\end{equation}
Applying \eqn{mml_zop_def} to \eqn{mml_nqp_def}, we find that for both unidirectional ring and standing-wave resonators, $\mathcal{N}_{q p} = \delta_{q p}$, and \eqn{mml_fq_sol_temp} simplifies to
\begin{equation} \label{eqn:mml_fq_sol}
  F_q(t) = \half\, \Lq\, \sum_p e^{i \left(\Delta \omega_q - \Delta \omega_p\right) t}\, G_{q p}(t)\, E_p(t)\, ,
\end{equation}
where
\begin{equation} \label{eqn:mml_lq_def}
  \Lq \equiv \frac{\mathcal{A}}{\mathcal{B} - i\, \Omega_q}\, .
\end{equation}

Let's now determine the evolution equation for $G_{q p}(t)$ by applying \eqn{mml_gqp_def} and \eqn{mml_zop_def} to \eqn{mml_gtz_scaled}. Using \eqn{mml_e_1d_zt_rv} and the corresponding representation for the macroscopic polarization, the nonlinear term on the \rhs can be written as
\begin{equation*}
  \begin{split}
    2 \Re \left[ \widetilde{E}^\ast\zt\, \widetilde{F}\zt \right] &= \widetilde{E}^\ast\zt\, \widetilde{F}\zt + c.c. \\
    &= \sum_{m n}  e^{-i\, \left(\Delta \omega_m - \Delta \omega_n\right) t}\, \widetilde{u}_m\z\, \widetilde{u}_n^\ast\z \left[ E_m(t)\, F_n^\ast(t) + F_m(t)\, E_n^\ast(t) \right]\, .
  \end{split}
\end{equation*}
We obtain
\begin{equation} \label{eqn:mml_gtz_gqp_temp}
  \dot{G}_{q p}(t) = -\frac{1}{\tau_\parallel} \left\{ G_{q p}(t) - \overline{G}_{q p}(t)  + \sum_{m n} e^{-i\, \left(\Delta \omega_m - \Delta \omega_n\right) t} \kappa_{q p m n} \left[ E_m(t)\, F_n^\ast(t) + F_m(t)\, E_n^\ast(t) \right] \right\}\, ,
\end{equation}
where
\begin{equation} \label{eqn:mml_gqp_bar_def}
  \overline{G}_{q p}(t) \equiv \int d z\, \widetilde{v}_q\z\, \widetilde{u}_p\z\, \Gn\zt\, ,
\end{equation}
and
\begin{equation} \label{eqn:mml_kqp_mn_def}
  \kappa_{q p m n} \equiv \int d z\, \widetilde{v}_q\z\, \widetilde{u}_p\z\, \widetilde{u}_m\z\, \widetilde{u}_n^\ast\z\, .
\end{equation}

Let's evaluate $\overline{G}_{q p}(t)$ and $\kappa_{q p m n}$ for the unidirectional ring resonator. Using \eqn{mml_1d_uvq_url} and \eqn{mml_zop_def}, we find
\begin{equation} %\label{eqn:mll_url_zqp_spec}
    \overline{G}_{q p}(t) = \int_{0}^{1} d z\, e^{i\, 2 (p - q)\, \pi\, z}\, \Gn\zt\, ,
\end{equation}
and we see that $\overline{G}_{q p}(t)$ is the complex exponential Fourier series coefficient of order $p - q$ for $\Gn\zt$ in the resonator. Suppose that $\Gn\zt = \Gnb(t)/(z_2 - z_1)$ for $0 < z_1 \le z \le z_2 < 1$, and is zero otherwise. In this (common) special case, when $q \ne p$ we have
\begin{equation} %\label{eqn:mll_url_zqp_spec}
    \overline{G}_{q p}(t) = \frac{\Gnb(t)}{z_2 - z_1}\, \int_{z_1}^{z_2} d z\, e^{i\, 2 (p - q)\, \pi\, z} = \frac{\exp[i\, 2 \left(p - q\right) \pi\, z_2] - \exp[i\, 2 \left(p - q\right) \pi\, z_1]}{i\, 2 \left(p - q\right) \pi \left(z_2 - z_1\right)} \, \Gnb(t)\, ,
\end{equation}
and $\overline{G}_{q q}(t) = \Gnb(t)$. Note that when $\{z_1, z_2\} \longrightarrow \{0, 1\}$, $\overline{G}_{q p}(t) \longrightarrow \delta_{q p}\, \Gnb(t)$.
For the URL, the spatial mode coupling coefficient defined by \eqn{mml_kqp_mn_def} becomes
\begin{equation}
  \label{eqn:mml_1d_kqpmn_url}
  \kappa_{q p m n} = \mathcal{C}^2_\mathrm{URL} \int_0^1 d z\, e^{i\, 2\, (-q + p + m - n)\, \pi\, z} = \Delta_{-q + p + m - n}(R)\, .
\end{equation}

In the case of the standing-wave resonator, we use \eqn{mml_1d_uvq_swl} and \eqn{mml_zop_def} to find
\begin{equation} %\label{eqn:mll_swl_zqp_spec}
  \overline{G}_{q p}(t) = \int_0^{1/2} d z\, \mathbf{v}_q\z \dotp \mathbf{u}_p\z\, \Gn\zt = 2 \int_0^{1/2} d z\, \cos\left[ 2\, (q - p)\, \pi\, z \right]\, \Gn\zt\, , 
\end{equation}
showing that $\overline{G}_{q p}(t)$ is the cosine Fourier series coefficient of order $q - p$ for $\Gn\zt$ in the SWL resonator. As we did above for the URL case, let's suppose that $\Gn\zt = \Gnb(t)/2 (z_2 - z_1)$ for $0 < z_1 \le z \le z_2 < 1/2$, and is zero otherwise. Then for $p \ne q$
 \begin{equation} %\label{eqn:mml_1d_gqp_swl}
\overline{G}_{q p}(t) = \frac{\sin[2\, (q - p)\, \pi\, z_2] - \sin[2\, (q - p)\, \pi\, z_1]}{2\, (q - p)\, \pi\, (z_2 - z_1)}\, \Gnb(t)\, .
  \end{equation}
When $\{z_1, z_2\} \longrightarrow \{0, 1/2\}$, $\overline{G}_{q p}(t) \longrightarrow \delta_{q p}\, \Gnb(t)$.
For the SWL, the spatial mode coupling coefficient defined by \eqn{mml_kqp_mn_def} becomes
\begin{equation*}
  \begin{split}
    \kappa_{q p m n} &= \int_0^{1/2} d z\, \frac{k_0}{2 \pi} \int_{z - \pi/k_0}^{z + \pi/k_0} d z^\prime \widetilde{v}_q\zp\, \widetilde{u}_p\zp\, \widetilde{u}_m\zp\, \widetilde{u}_n^\ast\zp \\
    &= \int_0^{1/2} d z\,
      \left\{ \left[ v_q^+\z\, u_p^+\z\, u_m^+\z\, u_n^{+\ast}\z + v_q^-\z\, u_p^-\z\, u_m^-\z\, u_n^{-\ast}\z \right]\right. \\
      &\quad\quad\quad\quad\;\;\: + \left[ v_q^+\z\, u_p^+\z\, u_m^-\z\, u_n^{-\ast}\z + v_q^-\z\, u_p^-\z\, u_m^+\z\, u_n^{+\ast}\z \right] \\
      &\quad\quad\quad\quad\;\;\, + \left. \left[ v_q^+\z\, u_p^-\z\, u_m^+\z\, u_n^{-\ast}\z + v_q^-\z\, u_p^+\z\, u_m^-\z\, u_n^{+\ast}\z \right] \right\}\\
    &= \mathcal{C}^2_\mathrm{SWL} \int_0^{1/2} d z\,
      \left\{ \left[ e^{\left[i\, 2 (-q + p + m - n) \pi\, z + \ln(1/R_1 R_2)\right] z} +  \frac{1}{R_1}\, e^{-\left[i\, 2 (-q + p + m - n) \pi\, z + \ln(1/R_1 R_2)\right] z} \right]\right. \\
      &\quad\quad\quad\quad\quad\quad\quad\, + \left[ e^{\left[i\, 2 (q - p + m - n) \pi\, z + \ln(1/R_1 R_2)\right] z} +  \frac{1}{R_1}\, e^{-\left[i\, 2 (q - p + m - n) \pi\, z + \ln(1/R_1 R_2)\right] z} \right] \\
      &\quad\quad\quad\quad\quad\quad\quad\, + \left. \left[ e^{\left[i\, 2 (q + p - m - n) \pi\, z + \ln(1/R_1 R_2)\right] z} +  \frac{1}{R_1}\, e^{-\left[i\, 2 (q + p - m - n) \pi\, z + \ln(1/R_1 R_2)\right] z} \right] \right\}\, ,
  \end{split}
\end{equation*}
or
\begin{equation}
  \label{eqn:mml_1d_kqpmn_swl}
  \kappa_{q p m n} = \Delta^\prime_{-q + p + m - n}\left(R_1\, R_2\right) + \Delta^\prime_{q - p + m - n}\left(R_1\, R_2\right) + \Delta^\prime_{q + p - m - n}\left(R_1\, R_2\right)\, ,
\end{equation}
where $\Delta^\prime_{q}\left(R_1\, R_2\right)$ is defined by \eqn{laser_resonator_1d_Deltap_qR}. The first term on the right of this equation couples co-propagating spatial modes (and is identical to the URL coupling term given by \eqn{mml_1d_kqpmn_url} when $R_2 = 1$), the second term couples counter-propagating spatial modes neglecting interference, and the third term couples modes incorporating spatial interference.

% In the case of the one-dimensional unidirectional ring laser shown in \fig{resonator_1d_ring_gain}, we can write the spatially rapidly-varying electric field amplitude and macroscopic polarization --- assumed to be propagating in the $+\hatb{z}$ direction --- in terms of the corresponding slowly-varying fields as $\widetilde{E}\zt = E\zt e^{i k_0 z}$ and $\widetilde{F}\zt = F\zt e^{i k_0 z}$, respectively. This common factor of $\exp(i\, k_0\, z)$ has already been canceled from both sides of \eqn{cw_sml_etz_scaled}, which holds for the slowly-varying field amplitudes. Let's use \eqn{laser_resonator_1d_ezt_expansion} to write the slowly-varying electric field envelope function $\Ezt$ as
%  \begin{equation} \label{eqn:mml_e_field_1d_t}
% \Ezt \equiv \sum_{p = -\infty}^\infty u_{p}\z\, e^{-i\, \Delta \omega_p\, t}\, E_{p}(t)\, ,
%  \end{equation}
% consistent with both \eqn{cw_sml_etz_scaled} and \eqn{delta_w_q_def}. In \sct{laser_dynamics_1d_mml_frq}, we'll use $\delta \omega_p$ to represent the majority of the frequency shifts due to frequency pulling and dispersion, thereby reducing the magnitude and increasing the time scale of the phase fluctuations of the field envelope amplitude $E_p(t)$. We build our wave equation by substituting \eqn{mml_e_field_1d_t} into \eqn{cw_sml_etz_scaled}, and then applying \eqn{laser_resonator_1d_u_hlde} to obtain
%  \begin{equation}%\label{}
%    \sum_{p = -\infty}^\infty u_{p}\z\, e^{-i\, \Delta \omega_p\, t} \left[\dot{E}_{p}(t) + \left(\frac{1}{2 \tau_p} - i\, \delta \omega_p\right) E_{p}(t)\right] = F\zt\, ,
%  \end{equation}
% where $\dot{E}_{p}(t) \equiv d E_{p}(t)/d t$, and $\tau_p$ is the photon lifetime of the cavity defined by \eqn{f_fwhm} with $|\Gamma|^2 = R e^{-\alpha\wn L}$. Now we multiply both sides of this equation by $v_q\z$, and then integrate over $z$ from $0$ to $1$ to obtain
% where we have used \eqn{laser_resonator_1d_uv_biortho} and defined
%  \begin{equation} \label{eqn:mml_1d_fq_def_url}
% F_q(t) \equiv e^{+i\, \Delta \omega_q\, t} \int_0^1 d z\, v_q\z\, F\zt\, .
%  \end{equation}

% We use a similar approach to formulate the corresponding multimode field amplitude evolution equation for the one-dimensional standing-wave laser shown in \fig{resonator_1d_sw_gain}. In this case, we must write the spatially rapidly-varying electric field amplitude and macroscopic polarization in terms of the corresponding slowly-varying fields as $\widetilde{E}\zt = E^+\zt e^{+i k_0 z} + E^-\zt e^{-i k_0 z}$ and $\widetilde{F}\zt = F^+\zt e^{+i k_0 z} + F^-\zt e^{-i k_0 z}$, respectively. We will follow \sct{laser_resonators_1d_swl}, and write both $\mathbf{E}\zt$ and $\mathbf{F}\zt$ as column vectors, as we did in \eqn{laser_resonators_1d_e_sw_def}, with the electric field amplitude defined in \sct{laser_resonators_1d_swl} by \eqn{laser_resonator_1d_ezt_expansion_sw}:
%  \begin{equation*}
% \mathbf{E}\zt \equiv \sum_{p = -\infty}^\infty \mathbf{u}_{p}\z\, e^{-i\, \Delta \omega_p\, t}\, E_{p}(t)\, .
%  \end{equation*}
% We use \eqn{cw_sml_etz_scaled} to write the wave equation for the slowly-varying amplitudes as a vector operator equation, given by
%  \begin{equation}
%  \hat{\mathcal{L}}\, \mathbf{E}\zt = \mathbf{F}\zt ,
%  \end{equation}
% where
%  \begin{equation}
% \hat{\mathcal{L}} =  \begin{bmatrix} \ppt + \ppz + \half \alpha\wn L & 0  \\ 0 & \ppt - \ppz + \half \alpha\wn L \end{bmatrix} .
%  \end{equation}
% Applying this operator to \eqn{laser_resonator_1d_ezt_expansion_sw}, we find
%  \begin{equation}%\label{}
%    \sum_{p = -\infty}^\infty \mathbf{u}_{p}\z\, e^{-i\, \Delta \omega_p\, t} \left[\dot{E}_{p}(t) + \left(\frac{1}{2 \tau_p} - i\, \delta \omega_p\right) E_{p}(t)\right] = \mathbf{F}\zt\, .
%  \end{equation}
% Now we take the dot product of both sides of this equation with $\mathbf{v}_q\z$, and then integrate over $z$ from $0$ to $1/2$ to reproduce \eqn{mml_edot_temp} in the standing-wave case, but with $F_q(t)$ defined through \eqn{laser_resonator_1d_uv_biortho_sw} as
%  \begin{equation} \label{eqn:mml_1d_fq_def_swl}
% F_q(t) \equiv e^{+i\, \Delta \omega_p\, t} \int_0^{1/2} d z\, \mathbf{v}_q\z \dotp \mathbf{F}\zt .
%  \end{equation}

In the next two sections, we use these multimode evolution equations to study the dynamics of \emph{injection-seeded gain-switched} and \emph{passively mode-locked} lasers. In \sct{laser_dynamics_1d_mml_qsl}, we will apply the \emph{strong} rate-equation approximation (REA) to construct a non-perturbative theory of a high-intensity pulsed laser that is driven by a short-duration pump and ``primed'' by a slowly-varying input field. In \sct{laser_dynamics_1d_mml_mll}, we will relax the REA to allow rapid intermodal interactions and build a weak-field perturbative model of \emph{coherent population pulsations}\index{Coherent population pulsations}\cite{ref:sargent1974lp} that lead to passive \emph{mode-locking}\index{Mode-locking} in either the time or the frequency domain.


%%%%%%%%%%%%%%%%%%%%%%%%%%%%%%%%%%%%%%%%%%%%%%%%%%%%%%%%%%%%%%%%%%%%%%%%%%%%%%
%
% Section file included in main project file using \input{}
%
% Assumes that LaTeX2e macros and packages defined in notes_qdcl.sty are
%   available
%
%%%%%%%%%%%%%%%%%%%%%%%%%%%%%%%%%%%%%%%%%%%%%%%%%%%%%%%%%%%%%%%%%%%%%%%%%%%%%%

 \section{Frequency Shifts in One-Dimensional Multimode Lasers\label{sct:laser_dynamics_1d_mml_frq}}

 \subsection{Frequency Pulling\label{sct:laser_dynamics_1d_mml_frq_frp}}

As an example of a particular choice of $\delta \omega_q$, we will attempt to capture the bulk of the final frequency-pulling effects we expect in a multimode laser by applying the approximate single-longitudinal-mode laser theory developed in \sct{laser_statics_1d_approx}. We define the complex amplitude $E_q(t)$ in terms of a real amplitude $A_q(t)$ and a real phase $\phi_q(t)$ as
 \begin{equation}\label{eqn:mml_1d_aq_phiq_def}
E_q(t) \equiv A_q(t)\, e^{-i\, \phi_q(t)} .
 \end{equation}
Following \sct{laser_statics_1d_approx}, we assume that $\dot{A}_q(t) = 0$ and $\dot{\phi}_q(t) = 0$, so that application of \eqn{mml_edot_temp} gives
 \begin{equation} \label{eqn:mml_1d_dwq}
\delta \omega_q = -\frac{1}{2\, \tau_p}\, \frac{\Im[f_q]}{\Re[f_q]}\, .
 \end{equation}
where $f_q(t) \equiv e^{i\, \phi_q(t)} F_q(t)$.
%Substituting this expression into \eqn{mml_edot}, and then separating the result into real and imaginary parts, we find
% \begin{subequations}\label{eqn:mml_1d_cq_phiq_sep}
% \begin{align}
% \label{eqn:mml_1d_cq_dot} \dot{c}_q(t) &= -\frac{1}{2\, \tau_p}\, c_q(t) + \Re\left[f_q(t)\right] , \nd \\
% \label{eqn:mml_1d_phiq_dot} \dot{\phi}_q(t) &= \delta \omega_q +  \frac{\Im\left[f_q(t)\right]}{c_q(t)},
% \end{align}
% \end{subequations}
%Our goal is to solve \eqn{mml_edot} in the (approximately) steady state for each amplitude $E_q(t)$. In this case, we should obtain both $\dot{c}_q(t) \approx 0$ and a small --- and ideally constant --- value of $\dot{\phi}_q(t)$. In \sct{laser_statics_1d_approx}, we found that $\Im[f_q(t)] = \Omega_q\, \Re[f_q(t)]$ in the single-mode case, and
In \sct{laser_statics_1d_approx}, we found that $\Im[f_0]/\Re[f_0] = \Im[\mathcal{L}_0]/\Re[\mathcal{L}_0]$, where $\mathcal{L}_0$ is the lineshape function, and we assume that this condition is approximately valid in the multimode case. That is,
 \begin{equation}
\frac{\Im[f_q]}{\Re[f_q]} \approx \frac{\Im[\mathcal{L}_q]}{\Re[\mathcal{L}_q]} = \Omega_q\, ,
 \end{equation}
where in the Lorentzian case
 \begin{equation} \label{eqn:mml_lmc_q_def}
\mathcal{L}_q \equiv \frac{1}{1 - i\, \Omega_q}\, .
 \end{equation}
%We note that in single-frequency laser theory, using \eqn{mml_1d_fq} to obtain $f_q(t)$ for the single mode $q$, we find
% \begin{equation}
%f_q(t) = \frac{\left(1 + \Omega_q^2\right) g_q}{1 + \Omega_q^2 + 2 c_q^2(t)}\, c_q(t) ,
% \end{equation}
%so that $\Im\left[f_q(t)\right] = 0$.
%In steady-state, then, $c_q(t) \approx 2\, \tau_p\, \Re[f_q(t)]$, and \eqn{mml_1d_phiq_dot} becomes
% \begin{equation} \label{eqn:mml_1d_phiq_dot_approx}
%\dot{\phi}_q(t) \approx \delta \omega_q + \frac{\Omega_q}{2\, \tau_p}\, .
% \end{equation}
Therefore, choosing $\delta \omega_q \equiv -\Omega_q/2\, \tau_p$ should give $\dot{\phi}_q(t) \approx 0$ when $\dot{A}_q(t) = 0$. If we assume that $\omega_0 = \omega_{a b}$, then
 \begin{equation} \label{eqn:mml_1d_omega_q_def}
\Omega_q = \Delta \omega_q\, \tau_\perp = \left(2 q \pi + \delta \omega_q\right) \tau_\perp\, ,
 \end{equation}
and we can solve for $\delta \omega_q$ to obtain
 \begin{subequations}\label{eqn:mml_1d_freq_pull}
 \begin{align}
 \delta \omega_q &= -\frac{\tau_\perp}{2\, \tau_p}\, \frac{2 q \pi}{1 + \tau_\perp/2\, \tau_p}\, , \nd \\
 \Delta \omega_q &= \frac{2 q \pi}{1 + \tau_\perp/2\, \tau_p}\, .
 \end{align}
 \end{subequations}
Finally, substituting $\delta \omega_q = -\Omega_q/2\, \tau_p$ into \eqn{mml_edot_temp} and simplifying, we find
 \begin{equation} \label{eqn:mml_1d_deq_dt_fp}
\dot{E}_q(t) = -\frac{1}{2 \tau_p} \left( 1 + i\, \Omega_q \right) E_q(t) + F_q(t) .
 \end{equation}

 \subsection{Dispersion\label{sct:laser_dynamics_1d_mml_frq_dis}}

In \sct{laser_amp_1d_pdes}, we derived the normalized wave equation in the time domain given by \eqn{cw_sml_etz_scaled} for a particular transverse mode of the electromagnetic field. Our goal here is to update \eqn{mml_edot_temp} to include the effects of dispersion. The relevant term in \eqn{cw_sml_etz_scaled} is
\begin{equation*}
    i\, \sum_{l = 2}^\infty \frac{D_l\wn}{l!} \left(i\, \frac{\partial}{\partial t}\right)^l E^\pm\zt\, ,
\end{equation*}
where $D_l\wn$ is given by \eqn{cw_sml_disp_coeff}. Let's consider the unidirectional case --- the result for the standing-wave laser will be the same --- and use the one-dimensional single-mode expansion of the slowly-varying complex field envelope function given by \eqn{laser_resonator_1d_ezt_expansion}. We find
\begin{equation}
    \begin{split}
        i\, \sum_{l = 2}^\infty \frac{D_l\wn}{l!} \left(i\, \frac{\partial}{\partial t}\right)^l E\zt &= i\, \sum_{l = 2}^\infty \frac{D_l\wn}{l!} \left(i\, \frac{\partial}{\partial t}\right)^l \sum_p u_p\z\, e^{-i\, \Delta \omega_p\, t}\, E_p(t) \\
        &= i\, \sum_{l p} i^l\, \frac{D_l\wn}{l!}\, u_p\z\, e^{-i\, \Delta \omega_p\, t}\, \sum_{j = 0}^{l} \binom{l}{j}\, (-i\, \Delta \omega_p)^j\, \frac{\partial^{l - j}}{\partial t^{l - j}}\, E_p(t)\, . 
    \end{split}
\end{equation}

If we keep only those terms proportional to $E_p(t)$ and $\dot{E}_p(t)$, then
\begin{equation*}
    \sum_{j = 0}^{l} \binom{l}{j}\, (-i\, \Delta \omega_p)^j\, \frac{\partial^{l - j}}{\partial t^{l - j}}\, E_p(t) \approx (-i\, \Delta \omega_p)^l\, E_p(t) + l\, (-i\, \Delta \omega_p)^{l - 1}\, \dot{E}_p(t)\, ,
\end{equation*}
and then
\begin{equation} \label{eqn:mml_1d_disp_cont}
    i\, \sum_{l = 2}^\infty \frac{D_l\wn}{l!} \left(i\, \frac{\partial}{\partial t}\right)^l E\zt = i\, \sum_p u_p\z\, e^{-i\, \Delta \omega_p\, t} \left[ \delta D_p\wn\, E_p(t) + i\, \delta \tau_p\wn\, \dot{E}_p(t) \right]\, ,
\end{equation}
where
\begin{subequations} \label{eqn:mml_1d_delta_dt_q_def}
    \begin{align}
        \label{eqn:mml_1d_delta_d_q_def}
        \delta D_p\wn &\equiv \sum_{l = 2}^\infty \frac{(2 p \pi)^l}{l!}\, D_l\wn\, , \\
        \label{eqn:mml_1d_delta_tau_q_def}
        \delta \tau_p\wn &\equiv \sum_{l = 2}^\infty \frac{(2 p \pi)^{l - 1}}{(l - 1)!}\, D_l\wn\, ,
    \end{align}
\end{subequations}
and we've used \eqn{mml_1d_delta_w_q_def} to make the approximation $\Delta \omega_p \approx 2 p \pi$.

% In the one-dimensional case, after applying the normalization procedure outlined in \sct{laser_amp_1d_pdes}, this wave equation becomes
% \begin{equation} \label{eqn:mml_1d_weq_w_norm}
%     \pm \frac{\partial}{\partial z} E^\pm\zw - i\, \omega\, E^\pm\zw - i\, \mathcal{D}(\omega_0, \omega)\, E^\pm\zw + \half \alpha\wn\, E^\pm\zw = F^\pm\zw\, ,
% \end{equation}
% where, using \eqn{idm_dispersion_def}, the dispersion is now given by
%  \begin{equation} \label{eqn:mml_1d_disp_def}
% \mathcal{D}(\omega_0, \omega) = \sum_{m = 2}^\infty \frac{\omega^m}{m!}\, D_m\wn
%  \end{equation}
% and
%  \begin{equation} \label{eqn:mml_1d_disp_coeff_redeff}
% D_m\wn = \frac{L}{\tau_g^m} \frac{d^m}{d \omega_0^m} \Re\left[\beta\wn\right]\, .
%  \end{equation}
% Recall that $z$ is expressed in units of $L$ (the round-trip physical length of the laser resonator), and $\alpha\wn$ in terms of $L^{-1}$. Similarly, $t$ has units of $\tau_g$ (the group round-trip propagation time), and $\omega$ has units of $\tau_g^{-1}$.

% We'll apply the same approach we used in the beginning of \sct{laser_dynamics_1d_mml} to build an expression for the time derivative of the electric field coefficient $E_q(t)$. We'll focus on the case of the unidirectional ring laser here, but a similar analysis of a standing-wave laser will yield the same result. Taking the Fourier transform of \eqn{mml_e_field_1d_t}, and applying the Fourier Shift Theorem discussed in \sct{math_prelim_fourier_transforms}, yields
%  \begin{equation} \label{eqn:mml_1d_e_field_w}
% E\zw = \sum_{q = -\infty}^\infty u_{q}\z\, E_{q}\left(\omega - \Delta \omega_q\right)\, ,
%  \end{equation}
% where, as usual, Fourier transform pairs are distinguished by their arguments. Let's define the small angular frequency $\nu \equiv \omega - \Delta \omega_q$. To first order in $\nu$ (the slowly-varying envelope approximation in the frequency domain),
%  \begin{equation} \label{eqn:mml_1d_eq_svea_w}
%  \begin{split}
% \omega^m\, E_q\left(\omega - \Delta \omega_q\right) &= (\Delta \omega_q + \nu)^m\, E_q(\nu) \\
% &\approx (2 q \pi)^m\, E_q(\nu) + m (2 q \pi)^{m - 1}\, \nu\, E_q(\nu)\, ,
%  \end{split}
%  \end{equation}
% where we've used \eqn{mml_1d_delta_w_q_def} to make the approximation $\Delta \omega_q \approx 2 q \pi$ in the exponentiated coefficients. Substituting \eqn{mml_1d_e_field_w} and \eqn{mml_1d_eq_svea_w} into \eqn{mml_1d_weq_w_norm}, and then applying \eqn{laser_resonator_1d_u_hlde} in the form
%  \begin{equation} \label{eqn:mml_1d_u_hlde}
% \ddz u_q\z = i\, \left(\Delta \omega_q - \delta \omega_q\right) u_q\z + \left[\frac{1}{2\, \tau_p} - \half\, \alpha\wn\right] u_q\z\, ,
%  \end{equation}
% we obtain
% % \begin{multline}
% %\sum_l u_l\z \left[ 1 + \delta \tau_l\wn \right] (-i \nu)\, E_l(\nu) = \\ \sum_l u_l\z \left\{-\frac{1}{2\, \tau_p} + i \left[ \delta \omega_l + \sum_{m = 2}^\infty \frac{(2 l \pi)^m}{m!}\, D_m\wn \right] \right\} E_l(\nu) + F\zw\, ,
% % \end{multline}
%  \begin{multline} \label{eqn:mml_1d_nu_e}
% \sum_l u_l\z \left[ 1 + \delta \tau_l\wn \right] (-i \nu)\, E_l(\nu) = \\ \sum_l u_l\z \left\{-\frac{1}{2 \tau_p} \left( 1 + i\, \Omega_l \right) + i\, \delta D_l\wn \right\} E_l(\nu) + F\zw\, ,
%  \end{multline}
% where we have chosen $\delta \omega_l = -\Omega_l/2\, \tau_p$ to incorporate frequency pulling,

We follow the approach we used to derive \eqn{mml_edot_temp}, and multiply both sides of \eqn{mml_1d_disp_cont} by $v_q\z$ and then integrate the result over the cavity length. Collecting the resulting dispersion terms allow us to obtain the updated field coefficient equation of motion
\begin{equation} \label{eqn:mml_1d_deq_dt_final}
    \dot{E}_q(t) = \frac{1}{1 + \delta \tau_q\wn} \left\{ \left[-\frac{1}{2\, \tau_p} \left( 1 + i\, \Omega_q \right) + i\, \delta D_q\wn\right] E_q(t) + F_q(t) \right\}\, ,
\end{equation}
where $F_q(t)$ is again given by \eqn{mml_fq_sol}. We see two primary effects of dispersion. First, there is an additional frequency shift for each mode that increases (in magnitude) nonlinearly with mode number $q$. Second, the group round-trip time is slightly different for each mode, changing with $q$ by a factor of $1 + \delta \tau_q\wn$.
   
% Using the Fourier Shift Theorem, we note that
%  \begin{equation}
% \int_{-\infty}^{\infty} \frac{d \omega}{2 \pi}\, e^{-i \omega t}\, E_{q}\left(\omega - \Delta \omega_q\right) = e^{i \Delta \omega_q t}\, \int_{-\infty}^{\infty} \frac{d \nu}{2 \pi}\, e^{-i \nu t}\, E_{q}(\nu)\, ,
%  \end{equation}
% and we apply this transform to \eqn{mml_1d_nu_e} to obtain the updated field coefficient equation of motion
% \begin{equation}
%\left[ 1 + \delta \tau_q\wn \right] \dot{E}_q(t) = \\ \left\{-\frac{1}{2\, \tau_p} + i \left[ \delta \omega_q + \sum_{m = 2}^\infty \frac{(2 q \pi)^m}{m!}\, D_m\wn \right] \right\} E_q(t) + F_q(t)\, ,
% \end{equation}
% \begin{equation} \label{eqn:mml_1d_deqdt}
%\left[ 1 + \delta \tau_q\wn \right] \dot{E}_q(t) = \\ \left\{-\frac{1}{2 \tau_p} \left( 1 + i\, \Omega_q \right) + i\, \delta D_q\wn \right\} E_q(t) + F_q(t)\, ,
% \end{equation}

%Following the same procedure leading to \eqn{mml_1d_phiq_dot_approx}, we find
% \begin{equation}%\label{eqn}
%\delta \omega_q + \sum_{m = 2}^\infty \frac{D_m\wn}{m!}\, (2 q \pi)^m = -\frac{\Omega_q}{2\, \tau_p}\, ,
% \end{equation}
%which we can solve using \eqn{mml_1d_omega_q_def} for $\delta \omega_q$. Therefore, the total frequency shift and frequency displacement for mode $q$ due to mode pulling and dispersion is given respectively by
% \begin{align}
%\label{eqn:mml_1d_freq_shift} \delta \omega_q &= -\frac{1}{1 + \tau_\perp/2\, \tau_p} \left[\frac{\tau_\perp}{2\, \tau_p}\, (2 q \pi) + \sum_{m = 2}^\infty \frac{(2 q \pi)^m}{m!}\, D_m\wn \right]\, , \nd \\
%\label{eqn:mml_1d_freq_disp} \Delta \omega_q &= \frac{1}{1 + \tau_\perp/2\, \tau_p} \left[2 q \pi - \sum_{m = 2}^\infty \frac{(2 q \pi)^m}{m!}\, D_m\wn \right]\, .
% \end{align}
% Collecting results, we have updated our field equation of motion to read
%Although it is not obvious from the form of \eqn{mml_1d_deq_dt_final}, as the magnitudes of the dispersion coefficients increase, the primary effect will be to change the phase of the nonlinear coupling driving the evolution of each mode.

%. Applying this result to the field in \eqn{mml_e_field_1d_t}, we have to third order
% \begin{equation}
%\mathcal{L}(t)\, \mathbf{E}\zt = \left[ i\, \beta_1\wn \frac{d}{d t} - \frac{\beta_2\wn}{2} \frac{d^2}{d t^2} - i\, \frac{\beta_3\wn}{6} \frac{d^3}{d t^3} \right] \mathbf{E}\zt
% \end{equation}
%In \eqn{mml_1d_deq_dt}, we have scaled the time by $\tau_s \equiv 2\, \tau_p$ and divided out a common factor of $\beta_1\wn$. For the moment, let's suppress the factor of $\tau_s$, and for convenience scale the time by $\tau_g^\parallel$. Then, multiplying $\mathcal{L}(t)$ through by a factor of $-i \tau_g^\parallel/\beta_1 = -i \tau_g^\parallel v_g = -i L$ (where $L$ is the round-trip physical path length, or twice the physical length of the cavity), we have a scaled dispersion operator and operand given by
%
%
%%keeping only the new second and third order terms, and approximating $\Delta \omega_q t \approx (2 q \pi/\tau_g^\parallel) (\tau_g^\parallel t) = (2 q \pi) t$ in \eqn{mml_e_field_1d_t}
% \begin{equation}%\label{}
%\mathcal{L}^\prime(t)\, E_q(t)\, e^{-i\, 2 q \pi\, t} = \left[ \frac{i}{2}\, D_2\wn\, \frac{d^2}{d t^2} - \frac{1}{6}\, D_3\wn\, \frac{d^3}{d t^3} \right] E_q(t)\, e^{-i\, 2 q \pi\, t} ,
% \end{equation}
%The second and third time derivatives are given by
% \begin{subequations}
% \begin{align}
%\frac{d^2}{d t^2} E_q(t)\, e^{-i\, 2 q \pi\, t} &= \left[ -i\, 2\, (2 q \pi)\, \dot{E}_q(t) - (2 q \pi)^2\, E_q(t) \right] e^{-i\, 2 q \pi\, t} , \nd \\
%\frac{d^3}{d t^3} E_q(t)\, e^{-i\, 2 q \pi\, t} &= \left[ -3\, (2 q \pi)^2\, \dot{E}_q(t) + i\, (2 q \pi)^3\, E_q(t) \right] e^{-i\, 2 q \pi\, t} .
% \end{align}
% \end{subequations}
%Therefore, collecting results, and cancelling the common factor of $\exp\left(-i\, 2 q \pi\, t\right)$, we find
% \begin{equation}
% \begin{split}
%e^{+i\, 2 q \pi\, t}\, \mathcal{L}^\prime(t)\, E_q(t)\, e^{-i\, 2 q \pi\, t} &= \left[ D_2\wn\, (2 q \pi) + \frac{D_3\wn}{2}\, (2 q \pi)^2 \right] \dot{E}_q(t)\\ &-i \left[ \frac{D_2\wn}{2}\, (2 q \pi)^2 + \frac{D_3\wn}{6}\, (2 q \pi)^3 \right] E_q(t),
% \end{split}
% \end{equation}
%

%%%%%%%%%%%%%%%%%%%%%%%%%%%%%%%%%%%%%%%%%%%%%%%%%%%%%%%%%%%%%%%%%%%%%%%%%%%%%%
%
% Subsection file included in section file using \input{}
%
% Assumes that LaTeX2e macros and packages defined in rgb_laser_physics.sty
%   are available
%
%%%%%%%%%%%%%%%%%%%%%%%%%%%%%%%%%%%%%%%%%%%%%%%%%%%%%%%%%%%%%%%%%%%%%%%%%%%%%%
 \section{Injection-Seeded Gain-Switched Lasers\label{sct:laser_dynamics_1d_mml_qsl}}

We consider the ideal four-level laser dynamical equations developed in \sct{laser_amp_1d_pdes}, and we assume that $\gamma_\perp \longrightarrow \infty$, so that $\Omega = 0$. The formal integral of \eqn{cw_sml_ftz_scaled} becomes
 \begin{equation} \label{eqn:qsl_ftzt_formal}
\widetilde{F}\zt = \frac{\gamma_\perp}{2}\, e^{-\gamma_\perp t} \int_{-\infty}^{t} d t^\prime\, e^{\gamma_\perp t^\prime}\, \widetilde{G}\left(z, t^\prime\right) \widetilde{E}\left(z, t^\prime\right) \, .
 \end{equation}
We now apply the strong REA in the same limit, and assume that $|\partial \widetilde{E}\zt / \partial t| \ll \gamma_\perp |\widetilde{E}\zt|$, and $|\partial \widetilde{G}\zt / \partial t| \ll \gamma_\perp |\widetilde{G}\zt|$. In this case, both $\widetilde{E}\left(z, t^\prime\right)$ and $\widetilde{G}\left(z, t^\prime\right)$ can be moved outside of the integral, yielding
 \begin{equation} \label{eqn:qsl_ftzt_rea}
\widetilde{F}\zt = \frac{1}{2}\, \widetilde{G}\zt\, \widetilde{E}\zt \, .
 \end{equation}
Substituting this result into \eqn{cw_sml_gtz_scaled} gives
 \begin{equation} \label{eqn:qsl_dgdt_rea}
\ppt \widetilde{G}\zt = \frac{1}{\tau_\parallel} \left[ \overline{G}\zt - \widetilde{G}\zt - \widetilde{G}\zt \left| \widetilde{E}\zt\right|^2 \right]\, ,
 \end{equation}

Let's allow the intracavity field to be supplemented by a quantity $\widetilde{J}\zt$ arising from a very weak input $F_1(t)$ injected through the output coupler mirror $\mathcal{M}_1$, as shown in \fig{resonator_1d_smat}. This additional field will contribute to the total macroscopic polarization, so that
 \begin{equation} \label{eqn:qsl_ftzt_inj}
\widetilde{F}\zt = \frac{1}{2}\, \widetilde{G}\zt\, \left[\widetilde{E}\zt + \widetilde{J}\zt\right]\, .
 \end{equation}
In \sct{laser_resonators_1d_tcm}, we learned how to expand $\widetilde{J}\zt$ as a series of quasi-normal spatial modes in both the unidirectional ring and standing-wave resonator cases. Because the injected field is so weak, we do not need to include it in the saturation term in \eqn{qsl_dgdt_rea}.

 \subsection{Unidirectional Ring Lasers\label{sct:laser_dynamics_1d_mml_qsl_url}}
As discussed in the introduction to \chp{laser_dynamics_1d_mml}, in the case of the URL the rapidly-varying spatial function $\exp(+i k_0 z)$ is common to both $\widetilde{E}\zt$ and $\widetilde{F}\zt$, and can therefore be ignored in \eqn{qsl_ftzt_rea} and \eqn{qsl_dgdt_rea}. Therefore, we find $F_q(t)$ in \eqn{mml_edot_temp} by substituting \eqn{mml_e_field_1d_t} --- and the corresponding expression for $J\zt$ --- into \eqn{qsl_ftzt_inj}, and then the result into \eqn{mml_1d_fq_def_url}. We obtain
 \begin{equation} \label{eqn:qsl_url_fqt}
F_q(t) = \half\, \sum_p e^{i 2 ( q - p ) \pi t}\, G_{q - p}(t) \left[ E_p(t) + J_p(t) \right]\, ,
 \end{equation}
where
 \begin{equation} \label{eqn:qsl_gqp_def}
G_{q - p}(t) \equiv \int_0^1 d z\, v_q\z\, u_p\z\, G\zt = \int_0^1 d z\, e^{-i 2 (q - p) \pi z}\, G\zt\, ,
 \end{equation}
and $J_p(t)$ is given by \eqn{eqt_inj}. \Eqn{qsl_url_fqt} and the slowly-spatially-varying partial differential equation
 \begin{equation} \label{eqn:qsl_dgdt_url}
\ppt G\zt = \frac{1}{\tau_\parallel} \left[ \overline{G}\zt - G\zt - G\zt \left| E\zt\right|^2 \right]\,
 \end{equation}
are the only tools we'll need to solve numerically a wide variety of gain-switched URL problems.

 \subsection{Standing-Wave Lasers\label{sct:laser_dynamics_1d_mml_qsl_swl}}
The calculation of the macroscopic polarization for a multimode standing-wave laser requires that we pay attention to the interference between the counterpropagating fields. We'll follow a strategy similar to that of the continuous-wave case described in \sct{laser_statics_1d_shb}. We begin with an explicit expression for the spatially rapidly-varying polarization of \eqn{qsl_ftzt_inj}, written as
 \begin{multline} \label{eqn:qsl_1d_fzt_swl}
F^+\zt\, e^{+i k_0 z} + F^-\zt\, e^{-i k_0 z} = \\ \half\, \widetilde{G}\zt \left\{\left[E^+\zt + J^+\zt\right] e^{+i k_0 z} + \left[E^-\zt + J^-\zt\right] e^{-i k_0 z}\right\}\, .
 \end{multline}
The envelope functions $F^\pm\zt$, $E^\pm\zt$, and $J^\pm\zt$ are spatially slowly varying, but we will need to average $\widetilde{G}\zt$ over a physical wavelength. Following the procedure outlined in \eqn{ld1d_sw_shb_pzp_full}, we find
 \begin{subequations} \label{eqn:qsl_1d_fpmzt_swl}
 \begin{align}
F^+\zt &= \half\, \mathcal{G}^{[0]}\zt \left[E^+\zt + J^+\zt\right] + \half\, \mathcal{G}^{[-2]}\zt \left[E^-\zt + J^-\zt\right]\, , \nd \\
F^-\zt &= \half\, \mathcal{G}^{[+2]}\zt \left[E^+\zt + J^+\zt\right] + \half\, \mathcal{G}^{[0]}\zt \left[E^-\zt + J^-\zt\right]\, ,
 \end{align}
 \end{subequations}
where
 \begin{equation} \label{eqn:qsl_1d_gnzt}
\mathcal{G}^{[n]}\zt \equiv \frac{k_0}{2 \pi} \int_{z - \pi/k_0}^{z + \pi/k_0} d z^\prime\, e^{+i n k_0 z^\prime}\, \widetilde{G}(z^\prime, t)\, .
 \end{equation}
Substituting \eqn{qsl_1d_fpmzt_swl} into \eqn{mml_1d_fq_def_swl} yields
 \begin{multline}
F_q(t) = \half \sum_p e^{i 2 (q - p) \pi t} \left[E_p(t) + J_p(t)\right]
\int_0^{1/2} d z\, \left[ \mathbf{v}_q\z \dotp \mathbf{u}_p\z\, \mathcal{G}^{[0]}\zt \right. \\
\left. + v_q^+\z\, u_p^-\z\, \mathcal{G}^{[-2]}\zt + v_q^-\z\, u_p^+\z\, \mathcal{G}^{[+2]}\zt \right]\, .
 \end{multline}

We'll need to construct partial differential equations for $\mathcal{G}^{[0]}\zt$ and $\mathcal{G}^{[\pm 2]}\zt$. We expand $|\widetilde{E}\zt|^2$ as
 \begin{equation}
\left|\widetilde{E}\zt\right|^2 = \left|E^{+}\zt\right|^2 +  \left|E^{-}\zt\right|^2 + E^{+}\zt\, E^{- \ast}\zt\, e^{+i 2 k_0 z} + E^{-}\zt\, E^{+ \ast}\zt\, e^{-i 2 k_0 z}\, ,
 \end{equation}
substitute this expression into \eqn{qsl_dgdt_rea}, and then apply the average specified by \eqn{qsl_1d_gnzt} to obtain
 \begin{equation} \label{eqn:qsl_dgndt_swl}
 \begin{split}
\ppt \mathcal{G}^{[n]}\zt &= \frac{1}{\tau_\parallel} \left\{ \delta_{n, 0}\, \overline{G}\zt - \mathcal{G}^{[n]}\zt - \mathcal{G}^{[n]}\zt \left[ \left|E^{+}\zt\right|^2 +  \left|E^{-}\zt\right|^2 \right] \right. \\
&\qquad \left. -~\mathcal{G}^{[n + 2]}\zt\, E^{+}\zt\, E^{- \ast}\zt - \mathcal{G}^{[n - 2]}\zt\, E^{-}\zt\, E^{+ \ast}\zt \right\}\, .
 \end{split}
 \end{equation}
If the pump has a short duration compared to $\tau_\parallel$, then rapid temporal oscillations in $E^{\pm}\zt\, E^{\mp \ast}\zt$ will diminish the contributions of higher-order spatial averages and allow us to neglect $\mathcal{G}^{[\pm 4]}\zt$. 
%%%%%%%%%%%%%%%%%%%%%%%%%%%%%%%%%%%%%%%%%%%%%%%%%%%%%%%%%%%%%%%%%%%%%%%%%%%%%%
%
% Subsection file included in section file using \input{}
%
% Assumes that LaTeX2e macros and packages defined in rgb_laser_physics.sty
%   are available
%
%%%%%%%%%%%%%%%%%%%%%%%%%%%%%%%%%%%%%%%%%%%%%%%%%%%%%%%%%%%%%%%%%%%%%%%%%%%%%%
 \section{Passively Mode-Locked Lasers\label{sct:laser_dynamics_1d_mml_mll}}

% \subsection{New and Busted: Mode-Locked Lasers}
% \begin{subequations}
%     \begin{align}
%         \widetilde{E}\zt &= \sum_{m n} E_{m n}\, \widetilde{u}_m(z)\, e^{-i\, \Delta \omega_n\, t}\, , \nd \\
%         \widetilde{F}\zt &= \sum_{l q} F_{l q}\, \widetilde{u}_l\z\, e^{-i\, \Delta \omega_q\, t}\, .
%     \end{align}
% \end{subequations}
% where $\Delta \omega_n \equiv 2\, n\, \pi$, and
% \begin{subequations}
%     \begin{align}
%         E_{m n} &\equiv \int_{z_\mathrm{min}}^{z_\mathrm{max}} d z\, \int_{-\half}^{+\half}\, d t\, \widetilde{v}_m(z)\, e^{i\, \omega_n\, t}\, E\zt\, , \nd \\
%         F_{l q} &\equiv \int_{z_\mathrm{min}}^{z_\mathrm{max}} d z\, \int_{-\half}^{+\half}\, \widetilde{v}_l\z\, e^{i\, \Delta \omega_q\, t}\, \widetilde{F}\zt\, .
%     \end{align}
% \end{subequations}

% \begin{equation}
%     2\, \Re\left[\widetilde{E}^\ast\zt\, \widetilde{F}\zt\right] = \sum_{i j m n} \widetilde{u}_i\z\, \widetilde{u}^\ast_{j}\z\, e^{-i\, \Delta \omega_{m - n}\, t} \left( E_{i m}\, F^\ast_{j n} + F_{i m}\, E^\ast_{j n} \right)\, .
% \end{equation}

% \begin{equation}
%     \widetilde{G}\zt = \Gnz - \sum_{i j m n} \widetilde{u}_i\z\, \widetilde{u}^\ast_{j}\z\, e^{-i\, \Delta \omega_{m - n}\, t}\, \mathcal{C}_{m - n} \left( E_{i m}\, F^\ast_{j n} + F_{i m}\, E^\ast_{j n} \right)\, ,
% \end{equation}
% where
% \begin{equation}
%     \mathcal{C}_q \equiv \left(1 - i\, \Delta \omega_q\, \tau_\parallel\right)^{-1}\, .
% \end{equation}

% \begin{multline}
%     \widetilde{E}\zt\, \widetilde{G}\zt = \Gnz \sum_{k p} \widetilde{u}_k\z\, e^{-i\, \Delta \omega_p\, t} E_{l p} \\
%     - \sum_{i j k m n p} \widetilde{u}_i\z\, \widetilde{u}^\ast_j\z\, \widetilde{u}_k\z\, e^{-i\, \Delta \omega_{m - n + p}\, t}\, \mathcal{C}_{m - n} \left( E_{i m}\, F^\ast_{j n} + F_{i m}\, E^\ast_{j n} \right) E_{k p}\, .
% \end{multline}

% \begin{multline}
%     \widetilde{F}\zt = \half\, \Gnz \sum_{k p} \widetilde{u}_k\z\, e^{-i\, \Delta \omega_p\, t}\, \mathcal{L}\left(\Omega_p\right)\, E_{k p} \\
%     - \half \sum_{i j k m n p} \widetilde{u}_i\z\, \widetilde{u}^\ast_j\z\, \widetilde{u}_k\z\, e^{-i\, (\Delta \omega_{m - n + p})\, t}\, \mathcal{L}\left(\Omega_{m - n + p}\right)\,  \mathcal{C}_{m - n} \left( E_{i m}\, F^\ast_{j n} + F_{i m}\, E^\ast_{j n} \right) E_{k p}\, .
% \end{multline}

% Perform the time integral to obtain
% \begin{equation}
%     \begin{split}
%         F_{l q} &= \half\, \mathcal{L}\left(\Omega_q\right) \sum_k\, E_{k q} \int d z\, \widetilde{v}_l\z\, \widetilde{u}_k\z\, \Gnz \\
%         &- \half\, \mathcal{L}\left(\Omega_q\right) \sum_{i j k m n} \mathcal{C}_{m - n} \left( E_{i m}\, F^\ast_{j n} + F_{i m}\, E^\ast_{j n} \right) E_{k, m - n + q}\, \int d z\, \widetilde{v}_l\z\, \widetilde{u}_i\z\, \widetilde{u}^\ast_j\z\, \widetilde{u}_k\z \\
%         &\equiv \half\, \mathcal{L}\left(\Omega_q\right) \left[ \sum_k\, \overline{G}_{k l}\, E_{k q} - \sum_{i j k m n} \kappa_{i j k l}\, \mathcal{C}_{m - n} \left( E_{i m}\, F^\ast_{j n} + F_{i m}\, E^\ast_{j n} \right) E_{k, m - n + q}\right]\, ,
%     \end{split}
% \end{equation}
% where
% \begin{align}
%     \overline{G}_{k l} &\equiv \int d z\, \widetilde{v}_l\z\, \widetilde{u}_k\z\, \Gnz\, , \nd \\
%     \kappa_{i j k l} &\equiv \int d z\, \widetilde{v}_l\z\, \widetilde{u}_i\z\, \widetilde{u}^\ast_j\z\, \widetilde{u}_k\z\, .
% \end{align}

% \begin{equation}
%     \begin{split}
%         F_{q p} &= \half\, \mathcal{L}\left(\Omega_q\right) \sum_k\, E_{q k} \int d z\, \widetilde{u}_k\z\, \widetilde{v}_p\z\, \Gnz \\
%         &- \half\, \mathcal{L}\left(\Omega_q\right) \sum_{i j k m n} \mathcal{C}_{m - n} \left( E_{m j}\, F^\ast_{n k} + F_{m j}\, E^\ast_{n k} \right) E_{m - n + q,\, l}\, \int d z\, \widetilde{u}_j\z\, \widetilde{u}^\ast_k\z\, \widetilde{u}_l\z\, \widetilde{v}_p\z \\
%         &\equiv \half\, \mathcal{L}\left(\Omega_q\right) \left[ \sum_k\, E_{q k}\, \overline{G}_{k p} - \sum_{j k l m n} \mathcal{C}_{m - n}\, \kappa_{j k l p} \left( E_{m j}\, F^\ast_{n k} + F_{m j}\, E^\ast_{n k} \right) E_{m - n + q,\, l}\right]\, ,
%     \end{split}
% \end{equation}
% where
% \begin{align}
%     \overline{G}_{k p} &\equiv \int d z\, \widetilde{u}_k\z\, \widetilde{v}_p\z\, \Gnz\, , \nd \\
%     \kappa_{j k l p} &\equiv \int d z\, \widetilde{u}_j\z\, \widetilde{u}^\ast_k\z\, \widetilde{u}_l\z\, \widetilde{v}_p\z\, .
% \end{align}

% \subsubsection{Unidirectional Ring Lasers}
% In the case of a unidirectional ring laser, the rapidly-varying quasi-normal spatial modes are given by
% \begin{subequations} %\label{eqn:laser_resonator_1d_uv}
%     \begin{align}
%        \widetilde{u}_q\z &\equiv \mathcal{C}\, e^{+\left[ i 2 q \pi + \ln(1/\sqrt{R}) \right] z}\, e^{i\, k_0\, z} , \nd \\ %\label{eqn:laser_resonator_1d_u} \\
%        \widetilde{v}_q\z &\equiv \frac{1}{\mathcal{C}}\, e^{-\left[ i 2 q \pi + \ln(1/\sqrt{R}) \right] z}\, e^{-i\, k_0\, z}\, , %\label{eqn:laser_resonator_1d_v}
%     \end{align}
% \end{subequations}
% where $\mathcal{C}$ is given by \eqn{laser_resonator_1d_u_norm_url}. Therefore,
% \begin{equation} %\label{eqn:mll_url_zqp_spec}
%     \overline{G}_{k l} = \int_{z_1}^{z_2} d z\, e^{i\, 2 (k - l)\, \pi\, z} \Gnz\, ,
% \end{equation}
% and we see that $\overline{G}_{k l}$ is the Fourier series coefficient of order $k - l$ for $\Gnz$ in the resonator. \red{In practice, we can use this representation as a guide to the range of values of $l$ that we need to include to provide a numerically accurate computation of the intracavity gain.} Suppose that $\Gnz = \Gnb/(z_2 - z_1)$ for $0 < z_1 \le z \le z_2 < 1$, and is zero otherwise. In this (common) special case, when $k \ne l$ we have
% \begin{equation} %\label{eqn:mll_url_zqp_spec}
%     \overline{G}_{k l} = \frac{\Gnb}{z_2 - z_1}\, \int_{z_1}^{z_2} d z\, e^{i\, 2 (k - l)\, \pi\, z} = -i\, \Gnb\, \frac{\exp[i\, 2 \left(k - l\right) \pi\, z_2] - \exp[i\, 2 \left(k - l\right) \pi\, z_1]}{2 \left(k - l\right) \pi \left(z_2 - z_1\right)}\, ,
% \end{equation}
% and $\overline{G}_{l l} = \Gnb$. Note that when $\{z_1, z_2\} \longrightarrow \{0, 1\}$, $\overline{G}_{k l} \longrightarrow \delta_{k l}\, \Gnb$.

% \begin{equation}
%     \kappa_{i j k l} = \mathcal{C}^2 \int_0^1 d z\, e^{\left[i\, 2 \left(i - j + k - l\right) \pi + \ln(1/R)\right] z} = \Delta_{i - j + k - l}(R)\, .
% \end{equation}

% \subsection{Old Hotness: Mode-Locked Lasers}
As in the case of the $Q$-switched laser discussed in \sct{laser_dynamics_1d_mml_qsl}, intermodal coupling through nonlinearities in the macroscopic polarization $\widetilde{F}\zt$ add dynamics to the gain and saturation of each mode that can lead to novel dynamical behavior. In a mode-locked laser, the amplitude and phases of the longitudinal modes are fixed in such a way that the output of the laser has particularly desirable properties, such as very short pulses or very stable (quasi-continuous-wave) behavior. We can understand this behavior through formal expansions of $\widetilde{F}\zt$ and $\widetilde{G}\zt$ in the Fourier frequency domain, but we must relax the rate-equation approximation that led to \eqn{qsl_ftzt_rea} to allow fluctuations in the polarization and gain that occur at integer multiples of the cavity free-spectral range $2 \pi/\tau_g$.

In mode-locked lasers, the gain is generally constant in time, and the dynamic fields arise from the nonlinear coupling of the longitudinal modes through the macroscopic polarization. However, the coefficients of both the electric field and macroscopic polarization vary slowly over the round-trip propagation time $\tau_g$, and the time derivatives of both $E_q(t)$ and $F_q(t)$ tend to zero as the intracavity laser amplifier reaches equilibrium. Therefore, we begin with the formal solution of \eqn{cw_sml_gtz_scaled}, finding
\begin{equation}
  G_{q p}(t) = \overline{G}_{q p} - \sum_{m n} e^{-i\, \Delta \omega_{m - n}\, t} \kappa_{q p m n}\, \mathcal{C}_{m n} \left( E_m\, F_n^\ast + F_m\, E_n^\ast \right)\, ,
\end{equation}
where
\begin{equation}
  \mathcal{C}_{m n} \equiv \left(1 - i\, \Delta \omega_{m - n}\, \tau_\parallel\right)^{-1}\, .
\end{equation}
Note that we have assumed that the pump $\Gn\zt$ is constant in time, so that its Fourier coefficients $\overline{G}_{q p}$ are also constant in time. Substituting this expression into \eqn{mml_fq_sol}, we find
\begin{equation}
    \label{eqn:mml_fqt_mll}
    \begin{split}
        F_q &= \half\, \Lq \sum_p e^{i\, \Delta \omega_{q - p}\, t}\, \overline{G}_{q p}\, E_p \\
        &- \half\, \Lq \sum_{m n p} e^{i\, \Delta \omega_{q - p - m + n}\, t}\, \kappa_{q p m n}\, \mathcal{C}_{m n} \left( E_m\, F_n^\ast + F_m\, E_n^\ast \right) E_p\, ,
    \end{split}
\end{equation}
Both $E_q(t)$ and $F_q(t)$ vary slowly in time compared to the rapid oscillations of the exponential functions in \eqn{mml_fqt_mll}, so terms with nonzero frequencies will average out. Therefore, only terms with $p = q$ in the first sum and $p = q - m + n$ in the second sum will contribute significantly to the value of $F_q$. Thus, we obtain the simplified expression
\begin{equation}
    \label{eqn:mml_fq_mml}
    F_q = \half\, \Lq \Gnb\, E_q
    - \half\, \Lq \sum_{m n} \kappa_{q m n}\, \mathcal{C}_{m n} \left( E_m\, F_n^\ast + F_m\, E_n^\ast \right) E_{q - m + n}\, ,
\end{equation}
where the contracted three-index spatial coupling coefficient is given by
\begin{equation}
    \kappa_{q m n} = \begin{cases}
      \Delta_0(R) & \text{(URL)}\\
      \Delta_0(R_1\, R_2) + \Delta_{2(m - n)}(R_1\, R_2) & \text{(SWL)} \\
      \Delta_0(R_1\, R_2) + \Delta_{2(m - n)}(R_1\, R_2) + \Delta_{2(q - m)}(R_1\, R_2) & \text{(SHB)}
    \end{cases}
\end{equation}

Referring to \eqn{mml_1d_deq_dt_final}, as the mode-locked laser oscillator reaches equilibrium, $F_q(t)$ becomes a constant and $\dot{E}_q(t) \longrightarrow 0$. In this case, we find that $F_q$ must also satisfy the expression
\begin{equation}
    F_q = R_q\, E_q\, ,
\end{equation}
where
\begin{equation}
    R_q \equiv \frac{1}{2\, \tau_\lambda} \left(1 + i\, \Omega_q\right) - i\, \delta D_q\, .
\end{equation}
Therefore,
% \begin{equation}
%     \sum_p e^{i\, \Delta \omega_{q - p}\, t}\, \overline{G}_{q p}\, E_p - 2\, \mathcal{L}^{-1}\bigl(\Omega_q\bigr)\, B_q\, E_q
%     = \sum_{m n p} e^{i\, \Delta \omega_{q - p - m + n}\, t}\, \kappa_{q p m n}\, C_{m - n} \left( B_m + B_n^\ast \right) E_m\, E_n^\ast\, E_p
% \end{equation}
\begin{equation}
    \left[ \Gnb - 2\, R_q  / \Lq\right] E_q
    = \sum_{m n} \kappa_{q m n}\, B_{m n}\, C_{m n}\, E_m\, E_n^\ast\, E_{q - m + n}\, ,
\end{equation}
where
\begin{equation}
    B_{m n} \equiv R_m + R_n^\ast = \frac{1}{\tau_\lambda} + i\, \frac{\Omega_m - \Omega_n}{2\, \tau_\lambda} - i \left(\delta D_m - \delta D_n\right)\, .
\end{equation}
\begin{equation}
    B_{m n} \equiv R_m + R_n^\ast = \frac{1}{\tau_\lambda} + i\, \frac{\tau_\perp}{2\, \tau_\lambda}\, \left(\omega_m - \omega_n\right) - i \left(\delta D_m - \delta D_n\right)\, .
\end{equation}

% This expression must be true at all times, so for convenience, let's use it to calculate $E_q$ at $t = 0$. We obtain
% \begin{equation}
%     \sum_p \overline{G}_{q p}\, E_p - 2\, \mathcal{L}^{-1}\bigl(\Omega_q\bigr)\, B_q\, E_q = \sum_{m n p} \kappa_{q p m n}\, C_{m n} \left( B_m + B_n^\ast \right) E_m\, E_n^\ast\, E_p
% \end{equation}


% As a general representation of the spatially rapidly-varying fields in both unidirectional ring and standing-wave resonator configurations, we follow \sct{laser_resonators_1d_swl} and represent the electric field amplitude function as
%  \begin{equation} \label{eqn:mml_e_1d_t_rv}
% \widetilde{E}\zt \equiv \sum_{p = -\infty}^\infty \widetilde{u}_p\z\, e^{-i\, \Delta \omega_p\, t}\, E_p(t)\, .
%  \end{equation}
% In the unidirectional ring case,
%  \begin{equation}
% \widetilde{u}_q\z = u^{+}_q\z\, e^{+i k_0 z}\, ,
%  \end{equation}
% where $k_0$ is the propagation constant associated with the carrier frequency $\omega_0$. For a standing-wave resonator,
%  \begin{equation}
% \widetilde{u}_q\z = u^{+}_q\z\, e^{+i k_0 z} + u^{-}_q\z\, e^{-i k_0 z}\, .
%  \end{equation}
% We use a similar approach to the expansion of the amplitude of the macroscopic polarization, with a subtle difference:
%  \begin{equation} \label{eqn:mml_f_1d_t_rv}
% \widetilde{F}\zt \equiv \sum_{q = -\infty}^\infty \widetilde{w}_q\z\, e^{-i\, \Delta \omega_q\, t}\, F_q(t)\, .
%  \end{equation}
% Here $\widetilde{w}_q(z)$ represents the spatial dependence of each frequency component of the macroscopic polarization. If we look carefully at \eqn{cw_sml_gtz_scaled}, we see that to first order in $\widetilde{E}\zt$, the gain is given by the function $\overline{G}\zt$ that describes the pump. Suppose that the pump is constant in time, so that $\overline{G}\zt \equiv \overline{G}\z$. Then we can write the pump function as
% \begin{equation} \label{eqn:mml_1d_pump_sep}
%   \overline{G}\z \equiv \Gn\, \mathcal{Z}\z\, ,
% \end{equation}
% \begin{equation}
%   G_0\z \equiv \Gn\, \mathcal{Z}\z\, ,
% \end{equation}
% where $\mathcal{Z}\z$ is a real function of $z$ normalized such that $\int_0^1 d z\, \mathcal{Z}\z = 1$ in the URL case or $2 \int_0^{1/2} d z\, \mathcal{Z}\z = 1$ in the SWL case, and $\Gn$ represents the \emph{small-signal (unsaturated) round-trip intensity gain}. Then a comparison of both sides of \eqn{cw_sml_ftz_scaled} suggests that
%  \begin{equation}
% \widetilde{w}_q\z \approx \mathcal{Z}\z\, \widetilde{u}_q\z\, .
%  \end{equation}
% In the following analysis, we'll also make a simplifying assumption: as discussed in \sct{laser_dynamics_1d_mml_frq}, $\delta \omega_q$ and, therefore, $\Delta \omega_q$ are linear in $q$.

% In this section, our primary tools will be the formal solutions of \eqn{cw_sml_ftz_scaled} and \eqn{cw_sml_gtz_scaled}, obtained through Fourier transform expansions. For example, consider the ordinary differential equation
%  \begin{equation}
% \ddt y(t) = -\frac{1}{\tau} \left[y(t) + s(t)\right]\, ,
%  \end{equation}
% for some function $s(t)$. Applying the Fourier Transform and using \eqn{fourier_freq} and \eqn{fourier_shift_thm}, we find
%  \begin{equation}
% y(\omega) = \frac{s(\omega)}{1 - i\, \omega\, \tau}\, .
%  \end{equation}
% %We obtain
% % \begin{equation}
% %A(t) = \left(1 + \tau\, \ddt\right)^{-1} B(t)\, ,
% % \end{equation}
% % \begin{equation}
% %\left(1 + \tau\, \ddt\right)^{-1} \equiv \sum_{l = 0}^{\infty} \left( -\tau\, \ddt \right)^l \, .
% % \end{equation}
% Suppose that $s(t)$ can be written as
%  \begin{equation}
% s(t) = \sum_q e^{-i\, \Delta \omega_q\, t}\, s_q(t)\, ,
%  \end{equation}
% giving the transform
%  \begin{equation}
% s(\omega) = \sum_q s_q(\omega - \Delta \omega_q)\, .
%  \end{equation}
% %If we define $\nu_q \equiv \omega - \Delta \omega_q$, and
% % \begin{equation}
% %c_q \equiv \frac{1}{1 - i\, \Delta \omega_q\, \tau}\, .
% % \end{equation}
% If we define $\nu_q \equiv \omega - \Delta \omega_q$, then we can rewrite $y(\omega)$ as
%  \begin{equation}
% y(\omega) = \sum_q \frac{1}{1 - i\, \Delta \omega_q\, \tau - i\, \nu_q\, \tau}\, s_q(\nu_q)\, .
%  \end{equation}
% With our usual casual indifference to mathematical rigor, we expand the denominator of this equation as a power series, and then apply the inverse Fourier transform over the frequency $\omega$ to each term separately. We obtain
%  \begin{equation}
% y(t) = \sum_q e^{-i\, \Delta \omega_q\, t}\, \left(1 - i\, \Delta \omega_q\, \tau + \tau\, \ddt\right)^{-1} s_q(t)\, ,
%  \end{equation}
% where for convenience we have defined the differential operator
% % \begin{equation}% \label{eqn:mll_diff_oper}
% %   \begin{split}
% %     \left(1 - i\, \Delta \omega_q\, \tau + \tau\, \ddt\right)^{-1} &= \sum_{l = 0}^{\infty} \left( i\, \Delta \omega_q\, \tau - \tau\, \ddt \right)^l \\
% %     &= \sum_{l = 0}^{\infty} \sum_{j = 0}^{l} \binom{l}{j} \left( i\, \Delta \omega_q\, \tau\right)^{l - j} (-\tau)^j\, \frac{d^j}{d t^j} \\
% %     &= \sum_{j = 0}^{\infty} \frac{(-\tau)^j}{\left(1 - i\, \Delta \omega_q\, \tau\right)^{j + 1}}\, \frac{d^j}{d t^j}\, .
% %   \end{split}
% % \end{equation}
% \begin{equation} \label{eqn:mll_diff_oper}
%   \begin{split}
%     \left(1 - i\, \Delta \omega_q\, \tau + \tau\, \ddt\right)^{-1} &= \left(1 - i\, \Delta \omega_q\, \tau\right)^{-1} \left(1 + \frac{\tau}{1 - i\, \Delta \omega_q\, \tau}\, \ddt\right)^{-1} \\
%     &= \frac{1}{1 - i\, \Delta \omega_q\, \tau}\, \sum_{l = 0}^{\infty} \left(-\frac{\tau}{1 - i\, \Delta \omega_q\, \tau}\right)^l\, \frac{d^l}{d t^l}\, .
%   \end{split}
% \end{equation}

% Let's apply this technique to solve the evolution equation for $\widetilde{G}\zt$ given by \eqn{cw_sml_gtz_scaled}. Using \eqn{mml_e_1d_t_rv} and \eqn{mml_f_1d_t_rv}, the nonlinear term on the \rhs can be written as
%  \begin{equation*}
%  \begin{split}
%     2 \Re \left[ \widetilde{E}^\ast\zt\, \widetilde{F}\zt \right] &= \widetilde{E}^\ast\zt\, \widetilde{F}\zt + c.c. \\
%     &= \sum_{m n}  e^{-i\, \Delta \omega_{m - n}\, t}\, \widetilde{u}_m\z\, \widetilde{u}_n^\ast\z\, \mathcal{Z}\z \left[ E_m(t)\, F_n^\ast(t) + F_m(t)\, E_n^\ast(t) \right]\, ,
%  \end{split}
%  \end{equation*}
% Therefore, using the Fourier transform expansion described above, we quickly find the formal solution
%  \begin{equation}  \label{eqn:mll_gzt_formal}
%  \begin{split}
% \widetilde{G}\zt &= \Gn\, \mathcal{Z}\z - \sum_{m n}  e^{-i\, \Delta \omega_{m - n}\, t}\, \widetilde{u}_m\z\, \widetilde{u}_n^\ast\z\, \mathcal{Z}\z \\
% &\qquad \times \left(1 - i\, \Delta \omega_{m - n} \, \tau_\parallel + \tau_\parallel\, \ddt\right)^{-1} \left[ E_m(t)\, F_n^\ast(t) + F_m(t)\, E_n^\ast(t) \right]\, .
%  \end{split}
%  \end{equation}
% %where
% % \begin{equation} \label{eqn:mll_1d_c_def}
% %C_q \equiv \frac{1}{1 - i\, \Delta \omega_q\, \tau_\parallel}\, .
% % \end{equation}
% We see that $\widetilde{G}\zt$ is rapidly-varying in space, and oscillates in time at a collection of frequencies that are approximately integer multiples of the free spectral range of the resonator. The Fourier transform approach to \eqn{cw_sml_ftz_scaled} is equally straightforward, giving the formal solution
%  \begin{equation}  \label{eqn:mll_fzt_formal}
%  \begin{split}
% \widetilde{F}\zt &= \frac{\Gn}{2}\, \sum_p e^{-i\, \Delta \omega_p\, t}\, \widetilde{u}_p\z\, \mathcal{Z}\z \left(1 - i\, \Omega_p + \tau_\perp\, \ddt\right)^{-1} E_p(t) \\
% &\quad - \half\, \sum_{m n p}  e^{-i\, \Delta \omega_{m - n + p}\, t}\, \widetilde{u}_m\z\, \widetilde{u}_n^\ast\z\, \widetilde{u}_p\z\, \mathcal{Z}\z \left(1 - i\, \Omega_{m - n + p} + \tau_\perp\, \ddt\right)^{-1} E_p(t) \\
% &\qquad \times \left(1 - i\, \Delta \omega_{m - n} \, \tau_\parallel + \tau_\parallel\, \ddt\right)^{-1} \left[ E_m(t)\, F_n^\ast(t) + F_m(t)\, E_n^\ast(t) \right]\, ,
%  \end{split}
%  \end{equation}
% where $\Omega_q \equiv \Omega_0 + \Delta \omega_q\, \tau_\perp$, and $\Omega_0$ is given by \eqn{tls_omega_0_def}.

% \begin{subequations} \label{eqn:mll_fgtzt_formal}
% \begin{align}
% \label{eqn:mll_ftzt_formal} \widetilde{F}\zt &= \frac{\gamma_\perp}{2}\, e^{-\gamma_\perp ( 1 - i\, \Omega_0 ) t} \int_{-\infty}^{t} d t^\prime\, e^{\gamma_\perp ( 1 - i\, \Omega_0 ) t^\prime}\, \widetilde{G}\left(z, t^\prime\right) \widetilde{E}\left(z, t^\prime\right) \, , \nd \\
% \label{eqn:mll_gtzt_formal} \widetilde{G}\zt &= \gamma_\parallel\, e^{-\gamma_\parallel t} \int_{-\infty}^{t} d t^\prime\, e^{\gamma_\parallel t^\prime}\, \left\{ \overline{G}\left(z, t^\prime\right) - 2 \Re \left[ \widetilde{E}^\ast\left(z, t^\prime\right) \widetilde{F}\left(z, t^\prime\right) \right] \right\} \, ,
% \end{align}
% \end{subequations}
%and our goal will be an expression for $\widetilde{F}\zt$ that is accurate to third order in $\widetilde{E}\zt$ \cite{ref:sargent1974lp}. Following the assumptions leading to \eqn{mml_1d_omega_q_def} and \eqn{mml_1d_freq_shift}, we'll take $\Omega_0 = 0$ in the remainder of this discussion.
%
%Suppose that the pump $\overline{G}\zt$ applied to the laser amplifier changes very slowly compared to the upper laser level lifetime $\tau_\parallel = \gamma_\parallel^{-1}$. (In most cases of practical interest, this constraint is equivalent to assuming that the pump is constant in time.) Then, to zeroth-order in the electric field amplitude, \eqn{mll_gtzt_formal} predicts that
% \begin{equation} \label{eqn:mll_gzt0}
%\widetilde{G}^{(0)}\zt = \gamma_\parallel\, e^{-\gamma_\parallel t} \int_{-\infty}^{t} d t^\prime\, e^{\gamma_\parallel t^\prime} \, \overline{G}\left(z, t^\prime\right) \cong \overline{G}\zt\, \gamma_\parallel\, e^{-\gamma_\parallel t} \int_{-\infty}^{t} d t^\prime\, e^{\gamma_\parallel t^\prime} = \overline{G}\zt\, .
% \end{equation}
%Next, we substitute this expression and \eqn{mml_e_field_1d_t} into \eqn{mll_ftzt_formal} to obtain an expression for $\widetilde{F}\zt$ that is accurate to first-order in $E_p(t)$. We find
% \begin{equation}
%\widetilde{F}^{(1)}\zt = \half\, \sum_p  \widetilde{u}_p\z\, \gamma_\perp\, e^{-\gamma_\perp\, t} \int_{-\infty}^{t} d t^\prime\, e^{\gamma_\perp ( 1 - i\, \Omega_p ) t^\prime}\, \overline{G}\left(z, t^\prime\right)\, E_p\left(t^\prime\right)\, ,
% \end{equation}
%where $\Omega_p = \Delta \omega_p\, \tau_\perp = (2 p \pi + \delta \omega_p) / \gamma_\perp$. Let's refine the rate equation approximation in this multimode case to assume that neither $\overline{G}\zt$ nor $E_p(t)$ change significantly during a time duration $\tau_\perp = \gamma_\perp^{-1}$. Moving both $\overline{G}\left(z, t^\prime\right)$ and $E_p\left(t^\prime\right)$ outside the time integral yields
% \begin{equation} \label{eqn:mll_fzt1}
%\widetilde{F}^{(1)}\zt = \frac{\overline{G}\zt}{2}\, \sum_p \frac{e^{-i\, \Delta \omega_p\, t}}{1 - i\, \Omega_p}\, E_p(t)\, \widetilde{u}_p\z\, .
% \end{equation}
%
%Our next assignment is to use \eqn{mml_e_field_1d_t}, \eqn{mll_gtzt_formal} and \eqn{mll_fzt1} to determine $\widetilde{G}^{(2)}\zt$. To second order in $E_q(t)$, the nonlinear term in \eqn{mll_gtzt_formal} becomes
% \begin{equation*}
% \begin{split}
%    2 \Re \left[ \widetilde{E}^\ast\zt\, \widetilde{F}^{(1)}\zt \right] &= \widetilde{E}^\ast\zt\, \widetilde{F}^{(1)}\zt + c.c. \\
%    &\equiv \overline{G}\zt\, \sum_{m n}  e^{-i (\Delta \omega_m - \Delta \omega_n) t}\, B_{m n}\, \widetilde{u}_m\z\, \widetilde{u}_n^\ast\z\, E_m(t)\, E_n^\ast(t)\, ,
% \end{split}
% \end{equation*}
%where
% \begin{equation} \label{eqn:mll_1d_b_def}
%B_{m n} \equiv \half\, \left( \frac{1}{1 - i\, \Omega_m} + \frac{1}{1 + i\, \Omega_n} \right)\, .
% \end{equation}
%Substituting this result into \eqn{mll_gtzt_formal} yields
% \begin{equation}
%\widetilde{G}^{(2)}\zt = -\overline{G}\zt\, \sum_{m n} B_{m n}\, \widetilde{u}_m\z\, \widetilde{u}_n^\ast\z\, \gamma_\parallel\, e^{-\gamma_\parallel t} \int_{-\infty}^{t} d t^\prime\, e^{[\gamma_\parallel - i (\Delta \omega_m - \Delta \omega_n)] t^\prime}\, E_m\left(t^\prime\right)\, E_n^\ast\left(t^\prime\right)\, .
% \end{equation}
%In practice, it may be difficult to claim that $E_q(t)$ will vary slowly relative to the timescale $\tau_\parallel = \gamma_\parallel^{-1}$. However, we can make the much more reasonable assumption that the dynamical variables will not change significantly during the group round-trip time $\tau_g$, so that $e^{- i (\Delta \omega_m - \Delta \omega_n) t}$ varies rapidly compared to $E_q(t)$. In this case, we have
% \begin{equation} \label{eqn:mll_gzt2}
%\widetilde{G}^{(2)}\zt \equiv -\overline{G}\zt\, \sum_{m n} e^{-i (\Delta \omega_m - \Delta \omega_n) t}\, B_{m n}\, C_{m n}\, \widetilde{u}_m\z\, \widetilde{u}_n^\ast\z\, E_m(t)\, E_n^\ast(t)\, .
% \end{equation}
%Finally, substitution of \eqn{mml_e_field_1d_t} and \eqn{mll_gzt2} into \eqn{mll_ftzt_formal}, followed by application of the rate-equation approximation,  yields for the third-order macroscopic polarization
% \begin{equation} \label{eqn:mll_fzt3}
% \begin{split}
%\widetilde{F}^{(3)}\zt &= -\frac{\overline{G}\zt}{2}\, \sum_{p m n}  \frac{e^{-i (\Delta \omega_p + \Delta \omega_m - \Delta \omega_n) t}}{1 - i (\Omega_p + \Omega_m - \Omega_n)}\, B_{m n}\, C_{m n} \\
%&\qquad \times \widetilde{u}_p\z\, \widetilde{u}_m\z\, \widetilde{u}_n^\ast\z\, E_p(t)\, E_m(t)\, E_n^\ast(t)\, .
% \end{split}
% \end{equation}
%The total macroscopic polarization, valid to third order in $E_q(t)$, is given by the sum of \eqn{mll_fzt1} and \eqn{mll_fzt3}.

%  \subsection{Unidirectional Ring Lasers\label{sct:laser_dynamics_1d_mml_mll_url}}
% As discussed above, in the case of the URL, the rapidly-varying spatial function $\exp(+i k_0 z)$ is common to both $\widetilde{E}\zt$ and $\widetilde{F}\zt$, and can therefore be ignored in \eqn{mll_fzt_formal}. If we substitute \eqn{mll_fzt_formal} into \eqn{mml_fq_sol}, we obtain
% % \begin{equation}
% % \begin{split}
% %F_q(t) &\cong \half\, \sum_p \frac{e^{i (\Delta \omega_q - \Delta \omega_p) t}}{1 - i\, \Omega_p}\, \overline{G}_{q - p}(t)\, E_p(t) \\
% %&\qquad - \half\, \sum_{p m n} \frac{e^{i (\Delta \omega_q - \Delta \omega_p - \Delta \omega_m + \Delta \omega_n) t}}{1 - i (\Omega_p + \Omega_m - \Omega_n)}\, B_{m n}\, C_{m n}\, \overline{G}_{q - p, m - n}(t)\, E_p(t)\, E_m(t)\, E_n^\ast(t)\,  ,
% % \end{split}
% % \end{equation}
%  \begin{equation} \label{eqn:mll_url_fqt_g}
%  \begin{split}
% F_q(t) &= \frac{\Gn}{2}\, \sum_p e^{i\, \Delta \omega_{q - p}\, t}\, \mathcal{Z}_{q - p}\, \left(1 - i\, \Omega_p + \tau_\perp\, \ddt\right)^{-1} E_p(t) \\
% &\quad - \half\, \sum_{m n p}  e^{i\, \Delta \omega_{q - m + n - p}\, t}\, \kappa_{q m n p}\, \left(1 - i\, \Omega_{m - n + p} + \tau_\perp\, \ddt\right)^{-1} E_p(t) \\
% &\qquad \times \left(1 - i\, \Delta \omega_{m - n} \, \tau_\parallel + \tau_\parallel\, \ddt\right)^{-1} \left[ E_m(t)\, F_n^\ast(t) + F_m(t)\, E_n^\ast(t) \right]\, ,
%  \end{split}
%  \end{equation}
% where
%  \begin{equation} \label{eqn:mll_url_zqp_def}
% \mathcal{Z}_{q - p} \equiv \int_0^1 d z\, v_q\z\, u_p\z\, \mathcal{Z}\z = \int_0^1 d z\, e^{-i 2 (q - p) \pi z}\, \mathcal{Z}\z
%  \end{equation}
% is essentially a one-dimensional discrete spatial Fourier transform of $\mathcal{Z}\z$, and
%  \begin{equation}
% \kappa_{q m n p} \equiv \int_0^1 d z\, v_q\z\, u_p\z\, u_m\z\, u_n^\ast\z\, \mathcal{Z}\z\, .
%  \end{equation}
%
% Then
% \begin{equation} \label{eqn:mll_url_gbqp_def}
%\overline{G}_{q - p} = \Gn \int_0^1 d z\, e^{-i 2 (q - p) \pi z}\, \mathcal{Z}\z \equiv \Gn\, \mathcal{Z}_{q - p}\, .
% \end{equation}
% \begin{equation} \label{eqn:mll_url_fqt1}
%F_q^{(1)}(t) = \frac{\Gnt}{2}\, \sum_p \frac{e^{i 2 ( q - p ) \pi t}}{1 - i\, \Omega_p}\, \mathcal{Z}_{q - p}\, E_p(t)\, .
% \end{equation}
% Suppose that $\mathcal{Z}\z = 1/(z_2 - z_1)$ for $0 < z_1 \le z \le z_2 < 1$, and is zero otherwise. In this (common) special case, when $q \ne p$ we have
%  \begin{equation} \label{eqn:mll_url_zqp_spec}
% \mathcal{Z}_{q - p} = i\, \frac{\exp[-i 2 (q - p) \pi z_2] - \exp[-i 2 (q - p) \pi z_1]}{2 (q - p) \pi (z_2 - z_1)}\, ,
%  \end{equation}
% and $\mathcal{Z}_0 = 1$ due to the normalization of $\mathcal{Z}\z$. Note that when $\{z_1, z_2\} \longrightarrow \{0, 1\}$, $\mathcal{Z}_{q - p} \longrightarrow \delta_{q p}$. However, if the intracavity laser amplifier does not fill the resonator, then \emph{in general} the quasi-normal spatial modes of the cavity will couple at first order through the Fourier transform included in \eqn{mll_url_zqp_def}.

% But we now make a crucial observation that will simplify our numerical analysis of mode-locked URLs. Since we have assumed that $|\dot{E}_q(t)| \ll |E_q(t)|/\tau_g$, terms in the first sum on the \rhs of \eqn{mll_url_fqt_g} with $p \ne q$, as well as terms in the second sum with $p \ne q - m + n$, are rapidly varying and will average out after the laser has reached stable operation. If we neglect these terms, then the polarization becomes
% \begin{equation}
%F_q(t) \cong \frac{\Gnt}{2 \left(1 - i\, \Omega_q\right)} \left[ E_q(t) - \kappa\, \sum_{m n} e^{-i\, \delta \phi_{q m n}(t)}\, B_{m n}\, C_{m n}\, E_{q - m + n}(t)\, E_m(t)\, E_n^\ast(t) \right]\,  ,
% \end{equation}
%  \begin{equation} \label{eqn:mll_url_fqt}
%  \begin{split}
% F_q(t) &= \half\, \Gn\, \left(1 - i\, \Omega_q + \tau_\perp\, \ddt\right)^{-1} E_q(t) - \frac{\kappa}{2}\, \left(1 - i\, \Omega_q + \tau_\perp\, \ddt\right)^{-1} \\
% &\qquad \times \sum_{m n} E_{q - m + n}(t)\, \left(1 - i\, \Delta \omega_{m - n} \, \tau_\parallel + \tau_\parallel\, \ddt\right)^{-1} \left[ E_m(t)\, F_n^\ast(t) + F_m(t)\, E_n^\ast(t) \right]\, ,
%  \end{split}
%  \end{equation}
% where
% \begin{equation}
%\delta \phi_{q m n}(t) \equiv (\delta \omega_{q - m + n} - \delta \omega_q + \delta \omega_m - \delta \omega_n)\, t\, , \nd
% \end{equation}
%  \begin{equation}
% \kappa \equiv \mathcal{C}^2 \int_0^1 d z\, e^{\ln(1/R)\, z}\, \mathcal{Z}\z\, .
%  \end{equation}
% In the case where $\mathcal{Z}\z = 1/(z_2 - z_1)$ for $0 < z_1 \le z \le z_2 < 1$, and is zero otherwise, we find
%  \begin{equation}
% \kappa = \frac{R}{1 - R}\, \frac{e^{\ln(1/R)\, z_2} - e^{\ln(1/R)\, z_1}}{z_2 - z_1} .
%  \end{equation}
% If $\{z_1, z_2\} \longrightarrow \{0, 1\}$, then $\kappa \longrightarrow 1$.

%In the third-order term of \eqn{mll_url_fqt}, we have made the approximation
% \begin{equation}
%\Omega_{q - m + n} + \Omega_m - \Omega_n = \Omega_q + \delta \phi_{q m n}(\tau_\perp) \approx \Omega_q
% \end{equation}
%because we have assumed throughout this analysis that $\delta \omega_q \ll 2 q \pi$. We can make a similar approximation for the coefficients $B_{m n}$ and $C_{m n}$ under the same assumption, but not for $\delta \phi_{q m n}(t)$. Because the shift caused by frequency pulling is linear in $q$ by \eqn{mml_1d_freq_shift}, if we can neglect dispersion then $\delta \phi_{q m n}(t) = 0$. However, if we include the effects of dispersion, then (to third order)
% \begin{equation}
%\delta \phi_{q m n}(t) = \frac{(q - m)(m - n)(2 \pi)^2}{1 + \tau_\perp/2\, \tau_p} \left[ D_2\wn + (q + n)\, \pi\, D_3\wn \right] t\, .
% \end{equation}
%Including this phase shift allows us to reduce the temporal fluctuations in the amplitudes $E_q(t)$, and in practice allows numerical solutions of \eqn{mml_1d_deq_dt_final} to converge more rapidly.

%  \subsubsection{Standing-Wave Lasers\label{sct:laser_dynamics_1d_mml_mll_swl}}

% The calculation of the macroscopic polarization for a standing-wave laser proceeds in essentially the same fashion as the unidirectional ring laser, but interference between the counterpropagating fields will complicate our calculations of the spatial coupling between electric field modes contributing to the nonlinear terms in the macroscopic polarization. Our strategy is straightforward, if a bit tedious. Following our approach in both \sct{laser_statics_1d_shb} and \sct{laser_dynamics_1d_mml_qsl}, we begin with the spatially rapidly-varying polarization given by \eqn{mll_fzt_formal}, now written explicitly as
% \begin{multline}
%F^{+}\zt\, e^{+i k_0 z} + F^{-}\zt\, e^{-i k_0 z} = \\ \frac{\overline{G}\zt}{2}\, \sum_p \frac{e^{-i\, \Delta \omega_p\, t}}{1 - i\, \Omega_p}\, \left[u^+_p\z\, e^{+i k_0 z} + u^-_p\z\, e^{-i k_0 z}\right] E_p(t)\, .
% \end{multline}
%  \begin{equation*}%  \label{eqn:mll_fzt_formal}
%  \begin{split}
% F^{+}\zt\, e^{+i k_0 z} + F^{-}\zt\, e^{-i k_0 z} &= \frac{\Gn}{2}\, \sum_p e^{-i\, \Delta \omega_p\, t}\, \left[u^+_p\z\, e^{+i k_0 z} + u^-_p\z\, e^{-i k_0 z}\right] \mathcal{Z}\z \\
% &\quad \times \left(1 - i\, \Omega_p + \tau_\perp\, \ddt\right)^{-1} E_p(t) \\
% &\quad - \half\, \sum_{m n p}  e^{-i\, \Delta \omega_{m - n + p}\, t}\, \widetilde{u}_m\z\, \widetilde{u}_n^\ast\z\, \widetilde{u}_p\z\, \mathcal{Z}\z \\
% &\quad \times \left(1 - i\, \Omega_{m - n + p} + \tau_\perp\, \ddt\right)^{-1} E_p(t) \\
% &\quad \times \left(1 - i\, \Delta \omega_{m - n} \, \tau_\parallel + \tau_\parallel\, \ddt\right)^{-1} \left[ E_m(t)\, F_n^\ast(t) + F_m(t)\, E_n^\ast(t) \right]\, .
%  \end{split}
%  \end{equation*}
% Let's start with the linear (first) term on the \rhs of this expression. If we make the reasonable assumption that $\mathcal{Z}\z$ is spatially slowly-varying on the scale of a wavelength, then the counterpropagating components of the polarization cleanly separate, and
%  \begin{equation}
% \mathbf{F}^{(1)}\zt = \frac{\Gn}{2}\, \sum_p e^{-i\, \Delta \omega_p\, t}\, \mathbf{u}_p\z\, \mathcal{Z}\z \left(1 - i\, \Omega_p + \tau_\perp\, \ddt\right)^{-1} E_p(t)\, ,
%  \end{equation}
% where we have used \eqn{laser_resonator_1d_u_sw_vec}. We substitute this result into \eqn{mml_fq_sol} to reproduce the first term on the \rhs of \eqn{mll_url_fqt_g}, where now
%  \begin{equation} \label{eqn:mll_swl_zqp_def}
% \mathcal{Z}_{q - p} \equiv \int_0^{1/2} d z\, \mathbf{v}_q\z \dotp \mathbf{u}_p\z\, \mathcal{Z}\z = 2 \int_0^{1/2} d z\, \cos\left[ 2\, (q - p)\, \pi\, z \right]\, \mathcal{Z}\z\, .
%  \end{equation}
% Let's suppose that $\mathcal{Z}\z = 1/2 (z_2 - z_1)$ for $0 <  z_1 \le z \le z_2 < 1/2$, and is zero otherwise. In this (common) special case, for $p \ne q$
%  \begin{equation} \label{eqn:mml_1d_zeta_12_swl}
% \mathcal{Z}_{q - p} = \frac{\sin[2\, (q - p)\, \pi\, z_2] - \sin[2\, (q - p)\, \pi\, z_1]}{2\, (q - p)\, \pi\, (z_2 - z_1)}\, .
%  \end{equation}
% When $\{z_1, z_2\} \longrightarrow \{0, 1/2\}$, $\mathcal{Z}_{q - p} \longrightarrow \delta_{q p}$.

% We must go through the same exercise with the nonlinear contribution to the macroscopic polarization given by \eqn{mll_fzt_formal}. First we expand the rapidly-varying spatial functions to explicitly show their net dependence on $e^{\pm i k_0 z}$. We find
%  \begin{equation}
%  \begin{split}
% \widetilde{U}_{m n p}\z & \equiv \widetilde{u}_p\z\, \widetilde{u}_m\z\, \widetilde{u}_n^\ast\z \\
% &= \left[u^+_p\z\, e^{+i k_0 z} + u^-_p\z\, e^{-i k_0 z}\right] \Big[u_m^{+}\z\, u_n^{+ \ast}\z + u_m^{-}\z\, u_n^{- \ast}\z \\
%     & \qquad \left. + u_m^{+}\z\, u_n^{- \ast}\z\, e^{+i 2 k_0 z} + u_m^{-}\z\, u_n^{+ \ast}\z\, e^{-i 2 k_0 z}\right]\, ,
%  \end{split}
%  \end{equation}
% or, neglecting terms proportional to $e^{\pm i 3 k_0 z}$, we follow \eqn{laser_resonator_1d_u_sw_vec} and write
%  \begin{equation}
% \mathbf{U}_{m n p}\z = \begin{bmatrix}
% u_p^+\z\, \left[u_m^{+}\z\, u_n^{+ \ast}\z + u_m^{-}\z\, u_n^{- \ast}\z\right] + u_p^-\z\, u_m^{+}\z\, u_n^{- \ast}\z \\
% u_p^-\z\, \left[u_m^{+}\z\, u_n^{+ \ast}\z + u_m^{-}\z\, u_n^{- \ast}\z\right] + u_p^+\z\, u_m^{-}\z\, u_n^{+ \ast}\z
%                    \end{bmatrix}
%  \end{equation}
% Let us again assume that we can neglect all terms in the macroscopic polarization that vary on timescales greater than or equal to $\tau_g$, leading to $p = q$ in the first-order contribution, and $p = q - m + n$ for the nonlinear term. Then in order to derive $F_q(t)$ for the standing-wave laser, we need to calculate the integral
%  \begin{equation}
% \kappa_{q m n} \equiv \int_0^{1/2} d z\, \mathbf{v}_q\z \dotp \mathbf{U}_{q - m + n, m, n}\z\, \mathcal{Z}\z\, .
%  \end{equation}
% Once this result is in hand, coefficient $q$ of the modal macroscopic polarization expansion becomes
%  \begin{equation} \label{eqn:mll_swl_fqt}
%  \begin{split}
% F_q(t) &= \half\, \Gn\, \left(1 - i\, \Omega_q + \tau_\perp\, \ddt\right)^{-1} E_q(t) - \half\, \left(1 - i\, \Omega_q + \tau_\perp\, \ddt\right)^{-1} \\
% &\qquad \times \sum_{m n} \kappa_{q m n}\, E_{q - m + n}(t)\, \left(1 - i\, \Delta \omega_{m - n} \, \tau_\parallel + \tau_\parallel\, \ddt\right)^{-1} \left[ E_m(t)\, F_n^\ast(t) + F_m(t)\, E_n^\ast(t) \right]\, ,
%  \end{split}
%  \end{equation}

% If $\mathcal{Z}\z = 1/2 (z_2 - z_1)$ for $0 < z_1 \le z \le z_2 < 1/2$, and is zero otherwise, then
%  \begin{multline} \label{eqn:mll_1d_kappa_def_swl}
% \kappa_{q m n} \equiv \frac{1}{2 \left(z_2 - z_1\right)}\, \int_{z_1}^{z_2} d z\, \left\{ \mathbf{v}_q\z \dotp \mathbf{u}_{q - m + n}\z \left[u_m^{+}\z\, u_n^{+ \ast}\z + u_m^{-}\z\, u_n^{- \ast}\z\right] \right. \\ \left. + v_q^{+}\z\, u_{q - m + n}^{-}\z\, u_m^+\z\, u_n^{- \ast}\z + v_q^{-}\z\, u_{q - m + n}^{+}\z\, u_m^-\z\, u_n^{+ \ast}\z \right\}\, .
%  \end{multline}
% Using \eqn{laser_resonator_1d_u_sw} and \eqn{laser_resonator_1d_v_sw}, this spatial coupling constant becomes
%  \begin{equation} \label{eqn:mll_1d_kappa_swl}
%  \begin{split}
% \kappa_{q m n} &= \frac{\mathcal{C}^{-1} \mathcal{C}^3}{2 \left(z_2 - z_1\right)}\, \int_{z_1}^{z_2} d z\, \left\{ \left[e^{i 2 (m - n) \pi z} + e^{-i 2 (m - n) \pi z}\right] \right. \\
% &\qquad\qquad \times \left[e^{i 2 (m - n) \pi z} e^{\ln(1/R_1 R_2) z} + \frac{1}{R_1}\, e^{-i 2 (m - n) \pi z} e^{-\ln(1/R_1 R_2) z}\right] \\
% &\qquad \left. + e^{i 4 (q -m) \pi z} e^{\ln(1/R_1 R_2) z} + \frac{1}{R_1}\, e^{-i 4 (q - m) \pi z} e^{-\ln(1/R_1 R_2) z} \right\} \\
% & \equiv \Delta^\prime_{0}\left(R_1, R_2\right) + \Delta^\prime_{2 (m - n)}\left(R_1, R_2\right) + \Delta^\prime_{2 (q - m)}\left(R_1, R_2\right)\, ,
%  \end{split}
%  \end{equation}
% where
%  \begin{equation}
%  \begin{split}
% \Delta^\prime_{2 q}\left(R_1, R_2\right) &\equiv \frac{\mathcal{C}^2}{2 \left(z_2 - z_1\right)}\, \int_{z_1}^{z_2} d z\, \left\{ e^{\left[ i 4 q \pi + \ln(1/R_1 R_2)\right] z} + \frac{1}{R_1}\, e^{-\left[ i 4 q \pi + \ln(1/R_1 R_2)\right] z} \right\} \\
% &= \Delta_{2 q}(R_1 R_2)\, \frac{\mathcal{C}^2}{2 \left(z_2 - z_1\right) \ln(1/R_1 R_2)} \left\{ \left[e^{\left[i 4 q \pi + \ln(1/R_1 R_2)\right] z_2} - e^{\left[i 4 q \pi + \ln(1/R_1 R_2)\right] z_1}\right] \right. \\
% &\qquad \left.- R_1^{-1} \left[e^{-\left[i 4 q \pi + \ln(1/R_1 R_2)\right] z_2} - e^{-\left[i 4 q \pi + \ln(1/R_1 R_2)\right] z_1}\right] \right\}\,
% \, ,
%  \end{split}
%  \end{equation}
% and $\Delta_q(R)$ is defined by \eqn{laser_resonator_1d_Delta_qR}. If $\{z_1, z_2\} \longrightarrow \{0, 1/2\}$, then $\Delta^\prime_{2 q}(R_1, R_2) \longrightarrow \Delta_{2 q}(R_1 R_2)$. In this case, $\kappa_{q m n}$ becomes
%  \begin{equation} \label{eqn:mll_1d_kappa_swl_smpl}
% \kappa_{q m n} = 1 + \Delta_{2 (m - n)}(R_1 R_2) + \Delta_{2 (q - m)}(R_1 R_2)\, ,
%  \end{equation}
% The first term on the \rhs of \eqn{mll_1d_kappa_swl} and \eqn{mll_1d_kappa_swl_smpl} is simply the nonlinear coupling for the unidirectional ring laser. The second term arises from cross-saturation \emph{neglecting} interference between the counterpropagating fields, while the third term adds these interference effects. Note that \eqn{mll_1d_kappa_swl_smpl} predicts that $\kappa_{qqq} = 3$, which is the saturation constant appropriate for weak fields as described toward the end of \sct{laser_statics_1d_shb}.

 \subsection{Numerics}
Let's assume that in all cases of practical interest the transverse coherence time $\tau_\perp$ (which has been scaled by the group round-trip time $\tau_g$) is small enough that we can ignore the corresponding differential operators on the \rhs of a former equation, giving
 \begin{equation} \label{eqn:mll_swl_fqt_prac}
 \begin{split}
F_q(t) &= \half\, \mathcal{L}_q\, \Gn\, E_q(t) - \half\, \mathcal{L}_q\, \sum_{m n} \kappa_{q m n}\, E_{q - m + n}(t) \\
&\qquad \times \left(1 - i\, \Delta \omega_{m - n} \, \tau_\parallel + \tau_\parallel\, \ddt\right)^{-1} \left[ E_m(t)\, F_n^\ast(t) + F_m(t)\, E_n^\ast(t) \right]\, ,
 \end{split}
 \end{equation}
where $\mathcal{L}_{q}$ is defined by \eqn{mml_lmc_q_def}.
% \begin{equation} \label{eqn:mll_1d_l_def}
%\mathcal{L}_{q} \equiv \frac{1}{1 - i\, \Omega_q}\, .
% \end{equation}
In general, we can't make assumptions about the scaled value of $\tau_\parallel$; it could be smaller or larger than unity. In the case of a single-mode laser with no dispersion, our incorporation of frequency pulling into \eqn{mml_1d_deq_dt_fp} means that a constant pump will eventually result in $\dot{E}_q(t) = 0$. One approach to estimating the impact of the differential operator on the \rhs of \eqn{mll_swl_fqt_prac} to a multimode laser is to expand the nonlinear contribution to $F_q(t)$ to third order in the electric field coefficients. Using
 \begin{equation}
F^{(1)}_q(t) = \half\, \mathcal{L}_q\, \Gn\, E_q(t)\, ,
 \end{equation}
we obtain
 \begin{equation} \label{eqn:mll_swl_fqt_fwm_prac}
 \begin{split}
F_q(t) &\cong \half\, \mathcal{L}_q\, \Gn E_q(t) \\
&\quad - \half\, \mathcal{L}_q\, \Gn \sum_{m n} \kappa_{q m n}\, B_{m n}\, E_{q - m + n}(t) \left(1 - i\, \Delta \omega_{m - n} \, \tau_\parallel + \tau_\parallel\, \ddt\right)^{-1} E_m(t)\, E_n^\ast(t)\, ,
 \end{split}
 \end{equation}
where
 \begin{equation} \label{eqn:mll_1d_b_def}
B_{m n} \equiv \half\, \left( \mathcal{L}_m + \mathcal{L}^\ast_n \right)\, .
 \end{equation}
We note that
 \begin{equation} %\label{eqn:mll_diff_oper}
\left(1 - i\, \Delta \omega_{m - n} \, \tau_\parallel + \tau_\parallel\, \ddt\right)^{-1} \left[E_m(t)\, E_n^\ast(t)\right] = \sum_{l = 0}^{\infty} \left( i\, \Delta \omega_{m - n} \, \tau_\parallel - \tau_\parallel\, \ddt \right)^l \left[ E_m(t)\, E_n^\ast(t)\right]\, .
 \end{equation}
The $l = 1$ term of the sum on the \rhs has the form
 \begin{equation}
 \begin{split}
\left( i\, \Delta \omega_{m - n} \, \tau_\parallel - \tau_\parallel\, \ddt \right) \left[ E_m(t)\, E_n^\ast(t)\right] &= i\, \Delta \omega_{m - n} \, \tau_\parallel \left[ E_m(t)\, E_n^\ast(t)\right] \\
&\quad - \tau_\parallel \left[E_n^\ast(t)\, \dot{E}_m(t) + E_m(t)\, \dot{E}_n^\ast(t)\right]\, .
 \end{split}
 \end{equation}
Consistent with our third-order expansion of $F_q(t)$, we use \eqn{mml_1d_deq_dt_final} to estimate $\dot{E}_q(t)$ to first order in $E_q(t)$. We obtain
 \begin{equation} %\label{eqn:mml_1d_deq_dt_final}
 \dot{E}_q(t) \approx \gamma_q\, E_q(t)\, ,
 \end{equation}
where
 \begin{equation} \label{eqn:mml_1d_gamma_q_def}
 \gamma_q \equiv \frac{1}{1 + \delta \tau_q\wn} \left[ \half \left( 1 + i\, \Omega_q \right) \left( \frac{\Gn}{1 + \Omega_q^2} - \frac{1}{\tau_p} \right) + i\, \delta D_q\wn \right] .
 \end{equation}
Therefore
 \begin{equation}
\left( i\, \Delta \omega_{m - n} \, \tau_\parallel - \tau_\parallel\, \ddt \right) \left[ E_m(t)\, E_n^\ast(t)\right] \approx \left[ i\, \Delta \omega_{m - n} - \left(\gamma_m + \gamma_n^\ast\right) \right] \tau_\parallel\, \, E_m(t)\, E_n^\ast(t)\, ,
 \end{equation}
and
 \begin{equation}
 \begin{split}
\left(1 - i\, \Delta \omega_{m - n} \, \tau_\parallel + \tau_\parallel\, \ddt\right)^{-1} E_m(t)\, E_n^\ast(t) &= \sum_{l = 0}^{\infty} \left\{\left[ i\, \Delta \omega_{m - n} - \left(\gamma_m + \gamma_n^\ast\right) \right] \tau_\parallel\right\}^l\, \left[ E_m(t)\, E_n^\ast(t)\right] \\
&\equiv C_{m n}\, E_m(t)\, E_n^\ast(t)\, ,
 \end{split}
 \end{equation}
where
 \begin{equation} \label{eqn:mll_1d_cp_def}
C_{m n} \equiv \frac{1}{1 + \left(\gamma_m + \gamma_n^\ast - i\, \Delta \omega_{m - n}\right)\, \tau_\parallel}\, .
 \end{equation}
Suppose that $\tau_\parallel \lesssim 1$, and that $\Gn$ is only moderately above threshold, so that $\gamma_0 < 1$. Then $\gamma_m + \gamma_n^\ast$ can be neglected in favor of $\Delta \omega_{m - n}$. If $\tau_\parallel \gg 1$, then even at moderate gains $C_{m n}$ will be strongly suppressed; we have
 \begin{equation}
C_{m n} \approx \frac{\delta_{m n}}{1 + 2 \Re (\gamma_n)\, \tau_\parallel}\, .
 \end{equation}
%where
% \begin{equation}
%2 \Re (\gamma_n)\, \tau_\parallel = \frac{\tau_\parallel}{1 + \delta \tau_n\wn} \left( \frac{\Gn}{1 + \Omega_n^2} - \frac{1}{\tau_p} \right)\, .
% \end{equation}
This effect is even more pronounced for multimode systems well above threshold. Note that modes \emph{below} threshold should have $C_{n n} = 1$.

Let's now investigate numerical solutions of \eqn{mml_1d_deq_dt_final} after replacing the differential operator in \eqn{mll_swl_fqt_prac} with $C_{m n}$ in our computations of $F_q(t)$. First, in the ``all-wave-mixing'' (AWM) case, we choose
 \begin{equation} \label{eqn:mll_swl_fqt_awm}
 F_q(t) \cong \half\, \mathcal{L}_q\, \left\{ \Gn\, E_q(t) - \sum_{m n} \kappa_{q m n}\, C_{m n}\, E_{q - m + n}(t) \left[ E_m(t)\, F_n^\ast(t) + F_m(t)\, E_n^\ast(t) \right] \right\}\, .
 \end{equation}
This equation can be rewritten as a matrix equation for $F_q(t)$ in the form
 \begin{equation}\label{eqn:mll_1d_fqt_swl_awm}
\sum_m \left[ A_{q m}(t)\, F_m(t) + B_{q m}(t)\, F^\ast_m(t) \right] = H_q(t)\, ,
 \end{equation}
where
 \begin{align*}
A_{q m}(t) &\equiv \delta_{q, m} + \sum_n \mathcal{K}_{qmn}\, E_{q - m + n}(t)\, E_n^\ast(t)\, , \\
B_{q m}(t) &\equiv \sum_n \mathcal{K}_{qnm}\, E_{q - n + m}(t)\, E_n(t)\, , \\
\mathcal{K}_{qmn} &\equiv \half\, \mathcal{L}_q\, \kappa_{q m n}\, C_{m n}\, , \nd \\
H_q(t) &\equiv \half\, \Gn\, \mathcal{L}_q\, E_q(t)\, .
 \end{align*}
Suppose that the total number of modes in our simulation is $\mathcal{N} \equiv 2 q_\text{max} + 1$. Then we can think of $A_{q m}(t)$ and $B_{q m}(t)$ as $\mathcal{N} \times \mathcal{N}$ complex square matrices, and $F_q(t)$ and $H_q(t)$ as $\mathcal{N} \times 1$ complex column vectors. Separating all of these variables into their real and imaginary parts, we can rewrite \eqn{mll_1d_fqt_swl_awm} as the $(2 \mathcal{N} \times 2 \mathcal{N}) \cdot (2 \mathcal{N} \times 1)$ real matrix equation
 \begin{equation}\label{eqn:mml_1d_fqt_sw_mat}
\begin{bmatrix}
  \Re[A(t) + B(t)] & -\Im[A(t) - B(t)] \\
  \Im[A(t) + B(t)] & \Re[A(t) - B(t)]
\end{bmatrix} \begin{bmatrix}
                  \Re[\mathbf{F}(t)] \\
                  \Im[\mathbf{F}(t)]
                \end{bmatrix}
                 = \begin{bmatrix}
                  \Re[\mathbf{H}(t)] \\
                  \Im[\mathbf{H}(t)]
                \end{bmatrix}\, ,
 \end{equation}
which can be solved using standard numerical linear algebra techniques.% However, as written, this approach --- with $\mathcal{K}_{qmn}(t)$ incorporating $e^{-i\, \delta \phi_{q m n}(t)}$ --- can be numerically inefficient. In practice, a better algorithm would be:
% \begin{enumerate}
% \item replace $E_q(t)$ in $H(t)$ with $E_q(t) e^{-i \delta \omega_q t}$;
% \item drop $e^{-i\, \delta \phi_{q m n}(t)}$ from $\mathcal{K}_{qmn}(t)$;
% \item solve \eqn{mml_1d_fqt_sw_mat};
% \item multiply $F_q(t)$ by $e^{+i \delta \omega_q t}$; and
% \item substitute the result into \eqn{mml_1d_deq_dt_final}.
% \end{enumerate}
%It is easy to see that precisely the same approach can be applied to $F_q(t)$ in \eqn{mll_swl_fqt}.

In the low-gain, weak-field case, we can use the expansion of $F_q(t)$ to third-order in the electric field amplitude --- the ``four-wave mixing'' (FWM) case:
 \begin{equation} \label{eqn:mll_swl_fqt_fwm}
F_q(t) \cong \half\, \mathcal{L}_q\, \Gn \left[ E_q(t) - \sum_{m n} \kappa_{q m n}\, B_{m n}\, C_{m n}\, E_{q - m + n}(t)\, E_m(t)\, E_n^\ast(t) \right]\, .
 \end{equation}
In principle, \eqn{mml_1d_deq_dt_final} can be solved much more efficiently with $F_q(t)$ obtained from \eqn{mll_swl_fqt_fwm} than with \eqn{mml_1d_fqt_sw_mat}.
%
%As the unsaturated gain increases, numerical solutions of \eqn{mml_edot} relying on the third-order expansion of the macroscopic polarization given by \eqn{mll_swl_fqt} can become unstable. We can improve this stability --- at the expense of some loss of accuracy at high gains --- by treating \eqn{mll_swl_fqt} as a geometric series, and (indirectly) ``re-summing'' the terms. In this case, we find
% \begin{multline}\label{eqn:mll_swl_fqt_num}
%F_q(t) \approx \frac{1}{2 \left( 1 - i\, \Omega_q \right)} \Bigg\{ \Gnt\, E_q(t) - \\ \sum_{m n}  e^{-i\, \delta \phi_{q m n}(t)}\, \kappa_{q m n}\, C_{m n}\, E_{q - m + n}(t)  \left[ F_m(t)\, E_n^\ast(t) + E_m(t)\, F_n^\ast(t) \right] \Bigg\}\, .
% \end{multline}
%
%
%It is straightforward to show that a perturbative expansion of this expression reproduces \eqn{mll_swl_fqt}.

\subsubsection{Preliminary Solver}
\begin{equation}
  \left|E_q(t)\right|^2 -2 \Re\left[E_q^\ast(t)\, F_q(t)\right] = 0\, .
\end{equation}

In our code, we scale the time variable by the photon lifetime $\tau_p$, and compute the derivative using
\begin{equation}
  \dot{E}_q(t) = \left[-\half + i\, \left(\delta \omega_q\, \tau_p + \delta D_q\right)\right] E_q(t) + F_q(t)\, ,
\end{equation}
where $\delta \omega_q$ and $\delta D_q$ are given by \eqn{mml_1d_freq_pull} and \eqn{mml_1d_delta_d_q_def}, respectively.
Therefore,
\begin{equation}
  E^\ast_q(t)\, \dot{E}_q(t) = \left[-\frac{1}{2} + i\, \left(\delta \omega_q\, \tau_p + \delta D_q\right)\right] \left|E_q(t)\right|^2 + E^\ast_q(t)\, F_q(t)\, ,
\end{equation}
giving
\begin{align}
  \Re\left[ \frac{\dot{E}_q(t)}{E_q(t)} \right] &= -\frac{1}{2} + \Re\left[ \frac{F_q(t)}{E_q(t)} \right]\, , \text{ and} \\
  \Im\left[ \frac{\dot{E}_q(t)}{E_q(t)} \right] &= \delta \omega_q\, \tau_p + \delta D_q + \Im\left[ \frac{F_q(t)}{E_q(t)} \right]\, .
\end{align}
We see in the top two plots that $\Re[\dot{E}_q(t) / E_q(t)] \longrightarrow 0$ as $t \longrightarrow t_f$, and that in the same limit $\Im[\dot{E}_q(t) / E_q(t)] \longrightarrow \delta \nu_q\, \tau_p$, where
\begin{equation}
  \delta \nu_q \equiv \delta \omega_q + \frac{\delta D_q}{\tau_p} + \frac{1}{\tau_p}\, \Im\left[ \frac{F_q(t)}{E_q(t)} \right] \equiv \text{constant}\, .
\end{equation}
So we can use as our FOM the equations
\begin{align}
  \Re\left[ \frac{2\, F_q(t_f)}{E_q(t_f)} \right] &= 1\, , \text{ and} \\
  \Im\left[ \frac{\ddot{E}_q(t_f)}{E_q(t_f)} \right] &= 0\, ;
\end{align}
but how do we estimate $\ddot{E}_q(t_f)$?

 \subsubsection{Power Spectral Density}
Suppose that we have a numerically stable (steady-state) solution to \eqn{mml_edot_temp}, and we wish to compute the frequency content of the output intensity, defined as the square of the absolute value of an output field given by one of \eqn{laser_resonator_1d_swl_out}. Neglecting the overall normalization constant, we have
 \begin{align}%\label{}
I_\text{out}(t) &= \left| \sum_{p} e^{-i\, 2\, p\, \pi\, t}\, E_p \right|^2 = \sum_{p, p^\prime} e^{-i\, 2\, (p - p^\prime)\, \pi\, t} E_p\, E^\ast_{p^\prime} \\
&\equiv \sum_q A_q\, e^{-i\, 2\, \pi\, q\, t} ,
 \end{align}
where
 \begin{equation}
A_q \equiv \sum_p E_p\, E^\ast_{p - q} .
 \end{equation}
If $p \in \{-p_\textrm{max}, \dots, +p_\textrm{max}\}$, then, since $I_\text{out}(t)$ is real,
 \begin{align} \label{eqn:mml_1d_iout_final}
I_\text{out}(t) &= A_0 + 2 \sum_{q = 1}^{2\, p_\textrm{max}} \Re\left[ A_q\, e^{-i\, 2\, \pi\, q\, t} \right] \\
&= A_0 + 2 \sum_{q = 1}^{2\, p_\textrm{max}} \left[ \Re(A_q) \cos(2\, \pi\, q\, t) + \Im(A_q) \sin(2\, \pi\, q\, t) \right] .
 \end{align}
Therefore, following standard practice\footnote{Although both the in-phase and quadrature components are included in the definition given by \eqn{mml_1d_psd_def}, the factor of 2 in the sum of \eqn{mml_1d_iout_final} is ignored for essentially the same reason we neglect the negative frequencies when plotting the digital Fourier transform of a real signal.}, we define the \emph{power spectral density} at each frequency as
 \begin{equation} \label{eqn:mml_1d_psd_def}
P_q \equiv \frac{\sqrt{\Re(A_q)^2 + \Im(A_q)^2}}{A_0} = \frac{|A_q|}{A_0}\, ,
 \end{equation}
valid for $q \in \{0, \dots, 2\, p_\textrm{max}\}$.

 \subsubsection{Chaotic Behavior}
 \subsubsection{Passive Temporal Mode-Locking with a Saturable Absorber}
 \subsubsection{Passive Frequency Mode-Locking}

\input{files/laser_dynamics_1d_mfl}


%%%%%%%%%%%%%%%%%%%%%%%%%%%%%%%%%%%%%%%%%%%%%%%%%%%%%%%%%%%%%%%%%%%%%%%%%%%%%%%
%
% Chapter file included in main project file using \input{}
%
% Assumes that LaTeX2e macros and packages defined in rgb_laser_physics.sty
%   are available
%
%%%%%%%%%%%%%%%%%%%%%%%%%%%%%%%%%%%%%%%%%%%%%%%%%%%%%%%%%%%%%%%%%%%%%%%%%%%%%%

 \chapter{One-dimensional Multi-Mode Laser Dynamics\label{chp:laser_dynamics_1d_mml}}

In this chapter, we describe the dynamics of one-dimensional laser amplifiers and oscillators by applying the quasi-normal mode expansions derived in \sct{laser_resonators_1d_qnm} to the wave equation given by \eqn{wave_eqn_1d} and the density matrix evolution equations defined by \eqn{fls_mbe_rwa_pol} and \eqn{fls_mbe_rwa_pop_diff}. Under the right experimental conditions, these multimode representations (approximate as they are) can provide remarkably illuminating descriptions of laser behavior, including optimum output coupling, frequency pulling, wave mixing, and mode-locking.

When multiple modes oscillate in a laser, they give rise to coherent modulations of the populations in the nonlinear gain medium that create interactions between those modes. The frequencies of these modulations are integer multiples of the free spectral range $\Delta \omega_\text{FSR}$ --- defined by \eqn{delta_w_fsr_def} --- between adjacent intracavity field modes. In \fig{multimode_gain_spectrum_1d}, we show a plot of a gain medium with a peak at frequency $\omega_0 = \omega_{a b}$ as a function of the frequency detuning. We have superimposed the frequency modal structure --- over several free-spectral ranges --- of a cavity containing that medium. In the following sections, we will find that for a particular frequency $\Delta \omega_q \equiv 2 q \pi$, fluctuations in the gain medium at frequency $2 (q - p) \pi$ couple the electric field amplitude with frequency $2 p \pi$ to the macroscopic polarization component at frequency $2 q \pi$.

 \begin{figure}
  \centering
  \includegraphics[width=4.5in]{figures/multimode_gain_spectrum_1d}
  \caption{\label{fig:multimode_gain_spectrum_1d} Plot of a gain medium with a peak at frequency $\omega_0 = \omega_{a b}$ as a function of the frequency detuning. We have superimposed the frequency modal structure --- over several free-spectral ranges --- of a cavity containing the medium. We will find that for a particular frequency $\Delta \omega_q \equiv 2 q \pi$, fluctuations in the gain medium at frequency $2 (q - p) \pi$ couple the electric field amplitude with frequency $2 p \pi$ to the macroscopic polarization component at frequency $2 q \pi$.}
 \end{figure}

\section{One-Dimensional Multi-Mode Laser Evolution Equations\label{sct:laser_dynamics_1d_mml_evol_eqns}}

We begin by developing evolution equations for the complex longitudinal modal amplitudes of unidirectional and standing-wave intracavity laser fields based on the four-level Maxwell-Bloch equations given by \eqn{laser_statics_1d_sml_scaled} and \eqn{cw_sml_ftz_scaled}. We have
\begin{subequations}\label{eqn:laser_dynamics_1d_mml_scaled}
  \begin{align}
    \label{eqn:mml_etz_scaled}
    \ppt E^\pm\zt \pm \ppz E^\pm\zt &= \left[ i\, \widehat{\mathcal{D}}_0 - \half\, \an \right] E^\pm\zt + F^\pm\zt\, , \\
    \label{eqn:mml_ftz_scaled} \ppt \widetilde{F}\zt &= -\frac{1}{\tau_\perp} \left[ \mathcal{B}\, \widetilde{F}\zt - \frac{\mathcal{A}}{2}\, \widetilde{G}\zt \widetilde{E}\zt \right]\, , \nd \\
    \label{eqn:mml_gtz_scaled} \ppt \widetilde{G}\zt &= -\frac{1}{\tau_\parallel} \left\{ \widetilde{G}\zt - \Gn\zt + 2 \Re \left[ \widetilde{E}^\ast\zt \widetilde{F}\zt \right] \right\}\, ,
  \end{align}
\end{subequations}
where $\widehat{\mathcal{D}}_0$ is the differential operator defined by \eqn{cw_sml_disp_op}. Here we will defer the effects of frequency dispersion to \sct{laser_dynamics_1d_mml_frq_dis} by setting $\widehat{\mathcal{D}}_0 = 0$ in \eqn{mml_etz_scaled}.

Our goal will be to develop a set of nonlinear ordinary differential equations representing the time evolution of modal amplitudes of the electromagnetic field. Let's follow an approach similar to that used in \sct{laser_statics_1d_approx} and use the results of \sct{laser_resonators_1d_qnm} to expand $E^\pm\zt$ in terms of the quasi-normal modes of the laser resonator. For example, in the case of the one-dimensional unidirectional ring laser shown in \fig{resonator_1d_ring_gain}, $E^{-}\zt = 0$, and we can write the slowly-varying forward-propagating electric field amplitude as
\begin{equation}
  \label{eqn:mml_e_1d_zt_url}
  E^{+}\zt \equiv \sum_{q = -\infty}^\infty u_q\z\, e^{-i\, \Delta \omega_q\, t}\, E_q(t)\, ,
\end{equation}
where $u_q\z$ and the corresponding biorthogonal eigenfunction $v_q\z$ in the range $0 < z < 1$ are given by \eqn{laser_resonator_1d_u_unnorm} and \eqn{laser_resonator_1d_v_unnorm} as
\begin{subequations}
  \begin{align}
    \label{eqn:sml_1d_uq_url} u_q\z &=\mathcal{C}_\mathrm{URL}\, \exp\left[ +\left( i\, 2 q \pi + \ln\frac{1}{\sqrt{R}} \right) z \right]\, , \\
    \label{eqn:sml_1d_vq_url} v_q\z &=\mathcal{C}^{-1}_\mathrm{URL}\, \exp\left[ -\left( i\, 2 q \pi + \ln\frac{1}{\sqrt{R}} \right) z \right]\, ,
  \end{align}
\end{subequations}
$\mathcal{C}_\mathrm{URL}$ is given by \eqn{laser_resonator_1d_u_norm_url}, and
\begin{equation}\label{eqn:mml_1d_delta_w_q_def}
  \Delta \omega_q = 2 q \pi + \delta \omega_q\, ,
\end{equation}
consistent with both \eqn{cw_sml_etz_scaled} and \eqn{delta_w_q_def}. We apply the biorthogonality relation given by \eqn{laser_resonator_1d_uv_biortho} to \eqn{mml_etz_scaled} by substituting \eqn{mml_e_1d_zt_url} (with $q \longrightarrow p$) and a similar expression for $F^{+}\zt$; multiplying both sides through by $e^{+i\ \Delta \omega_q\, t}\, v_q\z$; and then integrating the result from $z = 0$ to $z = 1$. We find
\begin{equation} \label{eqn:mml_edot_temp}
  \dot{E}_q(t) = \left(-\frac{1}{2 \tau_\lambda} + i\, \delta \omega_q\right) E_q(t) + F_q(t)\, ,
\end{equation}
where $\tau_\lambda \equiv 1/\ln[1 / R \exp(-\anb)]$ is the photon lifetime\index{Photon lifetime} given by \eqn{f_fwhm} and $\anb \equiv \int_0^1 dz\, \alpha_0(z)$.

We shouldn't apply the rate-equation approximation (REA) to \eqn{mml_ftz_scaled} just yet, because a laser operating with $q_\text{max}$ longitudinal modes such that $q_\text{max}\, \Delta \omega_\text{FSR} \gtrsim 1 / \tau_\perp$ will exhibit a significant dependence of the unsaturated gain on the value of $q$. Instead, we will first substitute

because we want to keep the macroscopic polarization term $F_q(t)$ general for now. In the unidirectional ring laser case, we define $F_q(t)$ as

For the one-dimensional standing-wave laser shown in \fig{resonator_1d_sw_gain}, the slowly-varying biorthogonal eigenfunctions are $\mathbf{u}_q\z$ and $\mathbf{v}_q\z$, given by \eqn{laser_resonator_1d_u_sw_vec}, \eqn{laser_resonator_1d_u_sw}, \eqn{laser_resonator_1d_v_sw_vec}, and \eqn{laser_resonator_1d_v_sw}, and the corresponding normalization constant $\mathcal{C}_\mathrm{SWL}$ is given by \eqn{laser_resonator_1d_u_norm_swl}. Then
\begin{subequations} \label{eqn:sml_1d_uvq_swl}
  \begin{align}
    \label{eqn:sml_1d_uq_swl}
    \mathbf{u}_q\z &\equiv \begin{bmatrix} u^{+}_q\z \\ u^{-}_q\z \end{bmatrix} = \mathcal{C}_\mathrm{SWL} \begin{bmatrix} e^{+\left[ i\, 2 q \pi + \ln\left(1/\sqrt{R_1 R_2}\right) \right] z} \\ -\frac{1}{\sqrt{R_1}}\, e^{-\left[ i\, 2 q \pi + \ln\left(1/\sqrt{R_1 R_2}\right) \right] z} \end{bmatrix}\, , \nd \\
    \label{eqn:sml_1d_vq_swl}
    \mathbf{v}_q\z &\equiv \begin{bmatrix} v^{+}_q\z \\ v^{-}_q\z \end{bmatrix} = \mathcal{C}^{-1}_\mathrm{SWL} \begin{bmatrix} e^{-\left[ i\, 2 q \pi + \ln\left(1/\sqrt{R_1 R_2}\right) \right] z} \\ -\sqrt{R_1}\, e^{+\left[ i\, 2 q \pi + \ln\left(1/\sqrt{R_1 R_2}\right) \right] z} \end{bmatrix}\, ,
  \end{align}
\end{subequations}
where $0 < z < 1/2$ In this case, we apply the biorthogonality relation given by \eqn{laser_resonator_1d_uv_biortho_sw} to \eqn{cw_sml_ez_scaled} by substituting $E^{\pm}\z = \sum_p u^{\pm}_p\z\, \, e^{-i\, \Delta \omega_p\, t}\, E_p(t)$ and $F^{\pm}\z = \sum_p u^{\pm}_p\z\, \, e^{-i\, \Delta \omega_p\, t}\, F_p(t)$; forming the inner product of both sides with $e^{i\, \Delta \omega_q\, t}\, \mathbf{v}_q\z$; and then integrating the result from $z = 0$ to $z = 1/2$. Therefore, \eqn{e0_temp} remains valid for the standing-wave case with $\tau_\lambda \equiv 1 / \ln[1 / R_1 R_2 \exp(-\anb)]$ and $\anb \equiv 2 \int_0^{1/2} dz\, \alpha_0(z)$.

As a general representation of the spatially rapidly-varying fields in both unidirectional ring and standing-wave resonator configurations, we follow \sct{laser_resonators_1d_swl} and represent the electric field amplitude function as
\begin{equation}
  \label{eqn:mml_e_1d_zt_rv}
  \widetilde{E}\zt \equiv \sum_{q = -\infty}^\infty \widetilde{u}_q\z\, e^{-i\, \Delta \omega_q\, t}\, E_q(t)\, .
\end{equation}
In the unidirectional ring laser case,
\begin{subequations}
  \label{eqn:mml_1d_uvq_url}
  \begin{align}
    \label{eqn:mml_1d_uq_url} \widetilde{u}_q\z &= u^{+}_q\z\, e^{+i k_0 z}\, , \nd \\
    \label{eqn:mml_1d_vq_url} \widetilde{v}_q\z &= v^{+}_q\z\, e^{-i k_0 z}\, ,
  \end{align}
\end{subequations}
where $k_0$ is the propagation constant associated with the carrier frequency $\omega_0$. For a standing-wave resonator,
\begin{subequations}
  \label{eqn:mml_1d_uvq_swl}
  \begin{align}
    \label{eqn:mml_1d_uq_swl} \widetilde{u}_q\z &= u^{+}_q\z\, e^{+i k_0 z} + u^{-}_q\z\, e^{-i k_0 z}\, , \nd \\
    \label{eqn:mml_1d_vq_swl} \widetilde{v}_q\z &= v^{+}_q\z\, e^{-i k_0 z} + v^{-}_q\z\, e^{+i k_0 z}\, .
  \end{align}
\end{subequations}
We use a similar approach to the expansion of the amplitude of the macroscopic polarization.

\begin{multline} \label{eqn:mml_ftz_expansion}
  \sum_p \left[ \dot{F}_p(t) - i\, \Delta \omega_p\, F_p(t) \right] \widetilde{u}_p\z\, e^{-i\, \Delta \omega_p\, t} \\
  = -\frac{1}{\tau_\perp} \sum_p \left[ \mathcal{B}\, F_p(t) - \half\, \mathcal{A}\, \widetilde{G}\zt\, E_p(t) \right] \widetilde{u}_p\z\, e^{-i\, \Delta \omega_p\, t}\, .
\end{multline}
Now, rather than the rate-equation approximation (REA), we will apply the \emph{slowly-varying envelope approximation}\index{Slowly-varying envelope approximation} (SVEA) to \eqn{mml_ftz_expansion} by assuming that $|\dot{F}_p(t)| \ll |\Delta \omega_p F_p(t)|$ and neglecting the terms $\dot{F}_p(t)$ on the \lhs. This is valid when the time scale for changes in the modal polarization amplitudes $F_p(t)$ is long compared to the polarization relaxation time $\tau_\perp$.

\begin{equation} \label{eqn:mml_zop_def}
  \int d z \equiv \begin{cases}
    \int_0^{1} d z & \mbox{(URL)}\, , \\
    \int_0^{1/2} d z\, \frac{k_0}{2 \pi} \int_{z - \pi/k_0}^{z + \pi/k_0} d z' & \mbox{(SWL or SHB)}\, ,
  \end{cases}
\end{equation}

\begin{equation} %\label{eqn:mml_fq_sol_temp}
  \sum_p \mathcal{N}_{q p}\, F_p(t) = \half\, \Lq\, \sum_p e^{i \left(\Delta \omega_q - \Delta \omega_p\right) t}\, G_{q p}(t)\, E_p(t)\, , 
\end{equation}
\begin{equation} \label{eqn:mml_fq_sol_temp}
  \sum_p \mathcal{N}_{q p}\, \left(\mathcal{B} - i\, \Omega_p\right) F_p(t) = \half\, \mathcal{A}\, \sum_p e^{i \left(\Delta \omega_q - \Delta \omega_p\right) t}\, G_{q p}(t)\, E_p(t)\, , 
\end{equation}
where $\Omega_p \equiv \Delta \omega_q\, \tau_\perp$,
\begin{equation} \label{eqn:mml_nqp_def}
  \mathcal{N}_{q p} \equiv \int d z\, \widetilde{v}_q\z\, \widetilde{u}_p\z\, \nd
\end{equation}
\begin{equation} \label{eqn:mml_gqp_def}  
  G_{q p}(t) \equiv \int d z\, \widetilde{v}_q\z\, \widetilde{u}_p\z\, \widetilde{G}\zt\, .
\end{equation}
Applying \eqn{mml_zop_def} to \eqn{mml_nqp_def}, we find that for both unidirectional ring and standing-wave resonators, $\mathcal{N}_{q p} = \delta_{q p}$, and \eqn{mml_fq_sol_temp} simplifies to
\begin{equation} \label{eqn:mml_fq_sol}
  F_q(t) = \half\, \Lq\, \sum_p e^{i \left(\Delta \omega_q - \Delta \omega_p\right) t}\, G_{q p}(t)\, E_p(t)\, ,
\end{equation}
where
\begin{equation} \label{eqn:mml_lq_def}
  \Lq \equiv \frac{\mathcal{A}}{\mathcal{B} - i\, \Omega_q}\, .
\end{equation}

Let's now determine the evolution equation for $G_{q p}(t)$ by applying \eqn{mml_gqp_def} and \eqn{mml_zop_def} to \eqn{mml_gtz_scaled}. Using \eqn{mml_e_1d_zt_rv} and the corresponding representation for the macroscopic polarization, the nonlinear term on the \rhs can be written as
\begin{equation*}
  \begin{split}
    2 \Re \left[ \widetilde{E}^\ast\zt\, \widetilde{F}\zt \right] &= \widetilde{E}^\ast\zt\, \widetilde{F}\zt + c.c. \\
    &= \sum_{m n}  e^{-i\, \left(\Delta \omega_m - \Delta \omega_n\right) t}\, \widetilde{u}_m\z\, \widetilde{u}_n^\ast\z \left[ E_m(t)\, F_n^\ast(t) + F_m(t)\, E_n^\ast(t) \right]\, .
  \end{split}
\end{equation*}
We obtain
\begin{equation} \label{eqn:mml_gtz_gqp_temp}
  \dot{G}_{q p}(t) = -\frac{1}{\tau_\parallel} \left\{ G_{q p}(t) - \overline{G}_{q p}(t)  + \sum_{m n} e^{-i\, \left(\Delta \omega_m - \Delta \omega_n\right) t} \kappa_{q p m n} \left[ E_m(t)\, F_n^\ast(t) + F_m(t)\, E_n^\ast(t) \right] \right\}\, ,
\end{equation}
where
\begin{equation} \label{eqn:mml_gqp_bar_def}
  \overline{G}_{q p}(t) \equiv \int d z\, \widetilde{v}_q\z\, \widetilde{u}_p\z\, \Gn\zt\, ,
\end{equation}
and
\begin{equation} \label{eqn:mml_kqp_mn_def}
  \kappa_{q p m n} \equiv \int d z\, \widetilde{v}_q\z\, \widetilde{u}_p\z\, \widetilde{u}_m\z\, \widetilde{u}_n^\ast\z\, .
\end{equation}

Let's evaluate $\overline{G}_{q p}(t)$ and $\kappa_{q p m n}$ for the unidirectional ring resonator. Using \eqn{mml_1d_uvq_url} and \eqn{mml_zop_def}, we find
\begin{equation} %\label{eqn:mll_url_zqp_spec}
    \overline{G}_{q p}(t) = \int_{0}^{1} d z\, e^{i\, 2 (p - q)\, \pi\, z}\, \Gn\zt\, ,
\end{equation}
and we see that $\overline{G}_{q p}(t)$ is the complex exponential Fourier series coefficient of order $p - q$ for $\Gn\zt$ in the resonator. Suppose that $\Gn\zt = \Gnb(t)/(z_2 - z_1)$ for $0 < z_1 \le z \le z_2 < 1$, and is zero otherwise. In this (common) special case, when $q \ne p$ we have
\begin{equation} %\label{eqn:mll_url_zqp_spec}
    \overline{G}_{q p}(t) = \frac{\Gnb(t)}{z_2 - z_1}\, \int_{z_1}^{z_2} d z\, e^{i\, 2 (p - q)\, \pi\, z} = \frac{\exp[i\, 2 \left(p - q\right) \pi\, z_2] - \exp[i\, 2 \left(p - q\right) \pi\, z_1]}{i\, 2 \left(p - q\right) \pi \left(z_2 - z_1\right)} \, \Gnb(t)\, ,
\end{equation}
and $\overline{G}_{q q}(t) = \Gnb(t)$. Note that when $\{z_1, z_2\} \longrightarrow \{0, 1\}$, $\overline{G}_{q p}(t) \longrightarrow \delta_{q p}\, \Gnb(t)$.
For the URL, the spatial mode coupling coefficient defined by \eqn{mml_kqp_mn_def} becomes
\begin{equation}
  \label{eqn:mml_1d_kqpmn_url}
  \kappa_{q p m n} = \mathcal{C}^2_\mathrm{URL} \int_0^1 d z\, e^{i\, 2\, (-q + p + m - n)\, \pi\, z} = \Delta_{-q + p + m - n}(R)\, .
\end{equation}

In the case of the standing-wave resonator, we use \eqn{mml_1d_uvq_swl} and \eqn{mml_zop_def} to find
\begin{equation} %\label{eqn:mll_swl_zqp_spec}
  \overline{G}_{q p}(t) = \int_0^{1/2} d z\, \mathbf{v}_q\z \dotp \mathbf{u}_p\z\, \Gn\zt = 2 \int_0^{1/2} d z\, \cos\left[ 2\, (q - p)\, \pi\, z \right]\, \Gn\zt\, , 
\end{equation}
showing that $\overline{G}_{q p}(t)$ is the cosine Fourier series coefficient of order $q - p$ for $\Gn\zt$ in the SWL resonator. As we did above for the URL case, let's suppose that $\Gn\zt = \Gnb(t)/2 (z_2 - z_1)$ for $0 < z_1 \le z \le z_2 < 1/2$, and is zero otherwise. Then for $p \ne q$
 \begin{equation} %\label{eqn:mml_1d_gqp_swl}
\overline{G}_{q p}(t) = \frac{\sin[2\, (q - p)\, \pi\, z_2] - \sin[2\, (q - p)\, \pi\, z_1]}{2\, (q - p)\, \pi\, (z_2 - z_1)}\, \Gnb(t)\, .
  \end{equation}
When $\{z_1, z_2\} \longrightarrow \{0, 1/2\}$, $\overline{G}_{q p}(t) \longrightarrow \delta_{q p}\, \Gnb(t)$.
For the SWL, the spatial mode coupling coefficient defined by \eqn{mml_kqp_mn_def} becomes
\begin{equation*}
  \begin{split}
    \kappa_{q p m n} &= \int_0^{1/2} d z\, \frac{k_0}{2 \pi} \int_{z - \pi/k_0}^{z + \pi/k_0} d z^\prime \widetilde{v}_q\zp\, \widetilde{u}_p\zp\, \widetilde{u}_m\zp\, \widetilde{u}_n^\ast\zp \\
    &= \int_0^{1/2} d z\,
      \left\{ \left[ v_q^+\z\, u_p^+\z\, u_m^+\z\, u_n^{+\ast}\z + v_q^-\z\, u_p^-\z\, u_m^-\z\, u_n^{-\ast}\z \right]\right. \\
      &\quad\quad\quad\quad\;\;\: + \left[ v_q^+\z\, u_p^+\z\, u_m^-\z\, u_n^{-\ast}\z + v_q^-\z\, u_p^-\z\, u_m^+\z\, u_n^{+\ast}\z \right] \\
      &\quad\quad\quad\quad\;\;\, + \left. \left[ v_q^+\z\, u_p^-\z\, u_m^+\z\, u_n^{-\ast}\z + v_q^-\z\, u_p^+\z\, u_m^-\z\, u_n^{+\ast}\z \right] \right\}\\
    &= \mathcal{C}^2_\mathrm{SWL} \int_0^{1/2} d z\,
      \left\{ \left[ e^{\left[i\, 2 (-q + p + m - n) \pi\, z + \ln(1/R_1 R_2)\right] z} +  \frac{1}{R_1}\, e^{-\left[i\, 2 (-q + p + m - n) \pi\, z + \ln(1/R_1 R_2)\right] z} \right]\right. \\
      &\quad\quad\quad\quad\quad\quad\quad\, + \left[ e^{\left[i\, 2 (q - p + m - n) \pi\, z + \ln(1/R_1 R_2)\right] z} +  \frac{1}{R_1}\, e^{-\left[i\, 2 (q - p + m - n) \pi\, z + \ln(1/R_1 R_2)\right] z} \right] \\
      &\quad\quad\quad\quad\quad\quad\quad\, + \left. \left[ e^{\left[i\, 2 (q + p - m - n) \pi\, z + \ln(1/R_1 R_2)\right] z} +  \frac{1}{R_1}\, e^{-\left[i\, 2 (q + p - m - n) \pi\, z + \ln(1/R_1 R_2)\right] z} \right] \right\}\, ,
  \end{split}
\end{equation*}
or
\begin{equation}
  \label{eqn:mml_1d_kqpmn_swl}
  \kappa_{q p m n} = \Delta^\prime_{-q + p + m - n}\left(R_1\, R_2\right) + \Delta^\prime_{q - p + m - n}\left(R_1\, R_2\right) + \Delta^\prime_{q + p - m - n}\left(R_1\, R_2\right)\, ,
\end{equation}
where $\Delta^\prime_{q}\left(R_1\, R_2\right)$ is defined by \eqn{laser_resonator_1d_Deltap_qR}. The first term on the right of this equation couples co-propagating spatial modes (and is identical to the URL coupling term given by \eqn{mml_1d_kqpmn_url} when $R_2 = 1$), the second term couples counter-propagating spatial modes neglecting interference, and the third term couples modes incorporating spatial interference.

% In the case of the one-dimensional unidirectional ring laser shown in \fig{resonator_1d_ring_gain}, we can write the spatially rapidly-varying electric field amplitude and macroscopic polarization --- assumed to be propagating in the $+\hatb{z}$ direction --- in terms of the corresponding slowly-varying fields as $\widetilde{E}\zt = E\zt e^{i k_0 z}$ and $\widetilde{F}\zt = F\zt e^{i k_0 z}$, respectively. This common factor of $\exp(i\, k_0\, z)$ has already been canceled from both sides of \eqn{cw_sml_etz_scaled}, which holds for the slowly-varying field amplitudes. Let's use \eqn{laser_resonator_1d_ezt_expansion} to write the slowly-varying electric field envelope function $\Ezt$ as
%  \begin{equation} \label{eqn:mml_e_field_1d_t}
% \Ezt \equiv \sum_{p = -\infty}^\infty u_{p}\z\, e^{-i\, \Delta \omega_p\, t}\, E_{p}(t)\, ,
%  \end{equation}
% consistent with both \eqn{cw_sml_etz_scaled} and \eqn{delta_w_q_def}. In \sct{laser_dynamics_1d_mml_frq}, we'll use $\delta \omega_p$ to represent the majority of the frequency shifts due to frequency pulling and dispersion, thereby reducing the magnitude and increasing the time scale of the phase fluctuations of the field envelope amplitude $E_p(t)$. We build our wave equation by substituting \eqn{mml_e_field_1d_t} into \eqn{cw_sml_etz_scaled}, and then applying \eqn{laser_resonator_1d_u_hlde} to obtain
%  \begin{equation}%\label{}
%    \sum_{p = -\infty}^\infty u_{p}\z\, e^{-i\, \Delta \omega_p\, t} \left[\dot{E}_{p}(t) + \left(\frac{1}{2 \tau_p} - i\, \delta \omega_p\right) E_{p}(t)\right] = F\zt\, ,
%  \end{equation}
% where $\dot{E}_{p}(t) \equiv d E_{p}(t)/d t$, and $\tau_p$ is the photon lifetime of the cavity defined by \eqn{f_fwhm} with $|\Gamma|^2 = R e^{-\alpha\wn L}$. Now we multiply both sides of this equation by $v_q\z$, and then integrate over $z$ from $0$ to $1$ to obtain
% where we have used \eqn{laser_resonator_1d_uv_biortho} and defined
%  \begin{equation} \label{eqn:mml_1d_fq_def_url}
% F_q(t) \equiv e^{+i\, \Delta \omega_q\, t} \int_0^1 d z\, v_q\z\, F\zt\, .
%  \end{equation}

% We use a similar approach to formulate the corresponding multimode field amplitude evolution equation for the one-dimensional standing-wave laser shown in \fig{resonator_1d_sw_gain}. In this case, we must write the spatially rapidly-varying electric field amplitude and macroscopic polarization in terms of the corresponding slowly-varying fields as $\widetilde{E}\zt = E^+\zt e^{+i k_0 z} + E^-\zt e^{-i k_0 z}$ and $\widetilde{F}\zt = F^+\zt e^{+i k_0 z} + F^-\zt e^{-i k_0 z}$, respectively. We will follow \sct{laser_resonators_1d_swl}, and write both $\mathbf{E}\zt$ and $\mathbf{F}\zt$ as column vectors, as we did in \eqn{laser_resonators_1d_e_sw_def}, with the electric field amplitude defined in \sct{laser_resonators_1d_swl} by \eqn{laser_resonator_1d_ezt_expansion_sw}:
%  \begin{equation*}
% \mathbf{E}\zt \equiv \sum_{p = -\infty}^\infty \mathbf{u}_{p}\z\, e^{-i\, \Delta \omega_p\, t}\, E_{p}(t)\, .
%  \end{equation*}
% We use \eqn{cw_sml_etz_scaled} to write the wave equation for the slowly-varying amplitudes as a vector operator equation, given by
%  \begin{equation}
%  \hat{\mathcal{L}}\, \mathbf{E}\zt = \mathbf{F}\zt ,
%  \end{equation}
% where
%  \begin{equation}
% \hat{\mathcal{L}} =  \begin{bmatrix} \ppt + \ppz + \half \alpha\wn L & 0  \\ 0 & \ppt - \ppz + \half \alpha\wn L \end{bmatrix} .
%  \end{equation}
% Applying this operator to \eqn{laser_resonator_1d_ezt_expansion_sw}, we find
%  \begin{equation}%\label{}
%    \sum_{p = -\infty}^\infty \mathbf{u}_{p}\z\, e^{-i\, \Delta \omega_p\, t} \left[\dot{E}_{p}(t) + \left(\frac{1}{2 \tau_p} - i\, \delta \omega_p\right) E_{p}(t)\right] = \mathbf{F}\zt\, .
%  \end{equation}
% Now we take the dot product of both sides of this equation with $\mathbf{v}_q\z$, and then integrate over $z$ from $0$ to $1/2$ to reproduce \eqn{mml_edot_temp} in the standing-wave case, but with $F_q(t)$ defined through \eqn{laser_resonator_1d_uv_biortho_sw} as
%  \begin{equation} \label{eqn:mml_1d_fq_def_swl}
% F_q(t) \equiv e^{+i\, \Delta \omega_p\, t} \int_0^{1/2} d z\, \mathbf{v}_q\z \dotp \mathbf{F}\zt .
%  \end{equation}

In the next two sections, we use these multimode evolution equations to study the dynamics of \emph{injection-seeded gain-switched} and \emph{passively mode-locked} lasers. In \sct{laser_dynamics_1d_mml_qsl}, we will apply the \emph{strong} rate-equation approximation (REA) to construct a non-perturbative theory of a high-intensity pulsed laser that is driven by a short-duration pump and ``primed'' by a slowly-varying input field. In \sct{laser_dynamics_1d_mml_mll}, we will relax the REA to allow rapid intermodal interactions and build a weak-field perturbative model of \emph{coherent population pulsations}\index{Coherent population pulsations}\cite{ref:sargent1974lp} that lead to passive \emph{mode-locking}\index{Mode-locking} in either the time or the frequency domain.


%%%%%%%%%%%%%%%%%%%%%%%%%%%%%%%%%%%%%%%%%%%%%%%%%%%%%%%%%%%%%%%%%%%%%%%%%%%%%%
%
% Section file included in main project file using \input{}
%
% Assumes that LaTeX2e macros and packages defined in notes_qdcl.sty are
%   available
%
%%%%%%%%%%%%%%%%%%%%%%%%%%%%%%%%%%%%%%%%%%%%%%%%%%%%%%%%%%%%%%%%%%%%%%%%%%%%%%

 \section{Frequency Shifts in One-Dimensional Multimode Lasers\label{sct:laser_dynamics_1d_mml_frq}}

 \subsection{Frequency Pulling\label{sct:laser_dynamics_1d_mml_frq_frp}}

As an example of a particular choice of $\delta \omega_q$, we will attempt to capture the bulk of the final frequency-pulling effects we expect in a multimode laser by applying the approximate single-longitudinal-mode laser theory developed in \sct{laser_statics_1d_approx}. We define the complex amplitude $E_q(t)$ in terms of a real amplitude $A_q(t)$ and a real phase $\phi_q(t)$ as
 \begin{equation}\label{eqn:mml_1d_aq_phiq_def}
E_q(t) \equiv A_q(t)\, e^{-i\, \phi_q(t)} .
 \end{equation}
Following \sct{laser_statics_1d_approx}, we assume that $\dot{A}_q(t) = 0$ and $\dot{\phi}_q(t) = 0$, so that application of \eqn{mml_edot_temp} gives
 \begin{equation} \label{eqn:mml_1d_dwq}
\delta \omega_q = -\frac{1}{2\, \tau_p}\, \frac{\Im[f_q]}{\Re[f_q]}\, .
 \end{equation}
where $f_q(t) \equiv e^{i\, \phi_q(t)} F_q(t)$.
%Substituting this expression into \eqn{mml_edot}, and then separating the result into real and imaginary parts, we find
% \begin{subequations}\label{eqn:mml_1d_cq_phiq_sep}
% \begin{align}
% \label{eqn:mml_1d_cq_dot} \dot{c}_q(t) &= -\frac{1}{2\, \tau_p}\, c_q(t) + \Re\left[f_q(t)\right] , \nd \\
% \label{eqn:mml_1d_phiq_dot} \dot{\phi}_q(t) &= \delta \omega_q +  \frac{\Im\left[f_q(t)\right]}{c_q(t)},
% \end{align}
% \end{subequations}
%Our goal is to solve \eqn{mml_edot} in the (approximately) steady state for each amplitude $E_q(t)$. In this case, we should obtain both $\dot{c}_q(t) \approx 0$ and a small --- and ideally constant --- value of $\dot{\phi}_q(t)$. In \sct{laser_statics_1d_approx}, we found that $\Im[f_q(t)] = \Omega_q\, \Re[f_q(t)]$ in the single-mode case, and
In \sct{laser_statics_1d_approx}, we found that $\Im[f_0]/\Re[f_0] = \Im[\mathcal{L}_0]/\Re[\mathcal{L}_0]$, where $\mathcal{L}_0$ is the lineshape function, and we assume that this condition is approximately valid in the multimode case. That is,
 \begin{equation}
\frac{\Im[f_q]}{\Re[f_q]} \approx \frac{\Im[\mathcal{L}_q]}{\Re[\mathcal{L}_q]} = \Omega_q\, ,
 \end{equation}
where in the Lorentzian case
 \begin{equation} \label{eqn:mml_lmc_q_def}
\mathcal{L}_q \equiv \frac{1}{1 - i\, \Omega_q}\, .
 \end{equation}
%We note that in single-frequency laser theory, using \eqn{mml_1d_fq} to obtain $f_q(t)$ for the single mode $q$, we find
% \begin{equation}
%f_q(t) = \frac{\left(1 + \Omega_q^2\right) g_q}{1 + \Omega_q^2 + 2 c_q^2(t)}\, c_q(t) ,
% \end{equation}
%so that $\Im\left[f_q(t)\right] = 0$.
%In steady-state, then, $c_q(t) \approx 2\, \tau_p\, \Re[f_q(t)]$, and \eqn{mml_1d_phiq_dot} becomes
% \begin{equation} \label{eqn:mml_1d_phiq_dot_approx}
%\dot{\phi}_q(t) \approx \delta \omega_q + \frac{\Omega_q}{2\, \tau_p}\, .
% \end{equation}
Therefore, choosing $\delta \omega_q \equiv -\Omega_q/2\, \tau_p$ should give $\dot{\phi}_q(t) \approx 0$ when $\dot{A}_q(t) = 0$. If we assume that $\omega_0 = \omega_{a b}$, then
 \begin{equation} \label{eqn:mml_1d_omega_q_def}
\Omega_q = \Delta \omega_q\, \tau_\perp = \left(2 q \pi + \delta \omega_q\right) \tau_\perp\, ,
 \end{equation}
and we can solve for $\delta \omega_q$ to obtain
 \begin{subequations}\label{eqn:mml_1d_freq_pull}
 \begin{align}
 \delta \omega_q &= -\frac{\tau_\perp}{2\, \tau_p}\, \frac{2 q \pi}{1 + \tau_\perp/2\, \tau_p}\, , \nd \\
 \Delta \omega_q &= \frac{2 q \pi}{1 + \tau_\perp/2\, \tau_p}\, .
 \end{align}
 \end{subequations}
Finally, substituting $\delta \omega_q = -\Omega_q/2\, \tau_p$ into \eqn{mml_edot_temp} and simplifying, we find
 \begin{equation} \label{eqn:mml_1d_deq_dt_fp}
\dot{E}_q(t) = -\frac{1}{2 \tau_p} \left( 1 + i\, \Omega_q \right) E_q(t) + F_q(t) .
 \end{equation}

 \subsection{Dispersion\label{sct:laser_dynamics_1d_mml_frq_dis}}

In \sct{laser_amp_1d_pdes}, we derived the normalized wave equation in the time domain given by \eqn{cw_sml_etz_scaled} for a particular transverse mode of the electromagnetic field. Our goal here is to update \eqn{mml_edot_temp} to include the effects of dispersion. The relevant term in \eqn{cw_sml_etz_scaled} is
\begin{equation*}
    i\, \sum_{l = 2}^\infty \frac{D_l\wn}{l!} \left(i\, \frac{\partial}{\partial t}\right)^l E^\pm\zt\, ,
\end{equation*}
where $D_l\wn$ is given by \eqn{cw_sml_disp_coeff}. Let's consider the unidirectional case --- the result for the standing-wave laser will be the same --- and use the one-dimensional single-mode expansion of the slowly-varying complex field envelope function given by \eqn{laser_resonator_1d_ezt_expansion}. We find
\begin{equation}
    \begin{split}
        i\, \sum_{l = 2}^\infty \frac{D_l\wn}{l!} \left(i\, \frac{\partial}{\partial t}\right)^l E\zt &= i\, \sum_{l = 2}^\infty \frac{D_l\wn}{l!} \left(i\, \frac{\partial}{\partial t}\right)^l \sum_p u_p\z\, e^{-i\, \Delta \omega_p\, t}\, E_p(t) \\
        &= i\, \sum_{l p} i^l\, \frac{D_l\wn}{l!}\, u_p\z\, e^{-i\, \Delta \omega_p\, t}\, \sum_{j = 0}^{l} \binom{l}{j}\, (-i\, \Delta \omega_p)^j\, \frac{\partial^{l - j}}{\partial t^{l - j}}\, E_p(t)\, . 
    \end{split}
\end{equation}

If we keep only those terms proportional to $E_p(t)$ and $\dot{E}_p(t)$, then
\begin{equation*}
    \sum_{j = 0}^{l} \binom{l}{j}\, (-i\, \Delta \omega_p)^j\, \frac{\partial^{l - j}}{\partial t^{l - j}}\, E_p(t) \approx (-i\, \Delta \omega_p)^l\, E_p(t) + l\, (-i\, \Delta \omega_p)^{l - 1}\, \dot{E}_p(t)\, ,
\end{equation*}
and then
\begin{equation} \label{eqn:mml_1d_disp_cont}
    i\, \sum_{l = 2}^\infty \frac{D_l\wn}{l!} \left(i\, \frac{\partial}{\partial t}\right)^l E\zt = i\, \sum_p u_p\z\, e^{-i\, \Delta \omega_p\, t} \left[ \delta D_p\wn\, E_p(t) + i\, \delta \tau_p\wn\, \dot{E}_p(t) \right]\, ,
\end{equation}
where
\begin{subequations} \label{eqn:mml_1d_delta_dt_q_def}
    \begin{align}
        \label{eqn:mml_1d_delta_d_q_def}
        \delta D_p\wn &\equiv \sum_{l = 2}^\infty \frac{(2 p \pi)^l}{l!}\, D_l\wn\, , \\
        \label{eqn:mml_1d_delta_tau_q_def}
        \delta \tau_p\wn &\equiv \sum_{l = 2}^\infty \frac{(2 p \pi)^{l - 1}}{(l - 1)!}\, D_l\wn\, ,
    \end{align}
\end{subequations}
and we've used \eqn{mml_1d_delta_w_q_def} to make the approximation $\Delta \omega_p \approx 2 p \pi$.

% In the one-dimensional case, after applying the normalization procedure outlined in \sct{laser_amp_1d_pdes}, this wave equation becomes
% \begin{equation} \label{eqn:mml_1d_weq_w_norm}
%     \pm \frac{\partial}{\partial z} E^\pm\zw - i\, \omega\, E^\pm\zw - i\, \mathcal{D}(\omega_0, \omega)\, E^\pm\zw + \half \alpha\wn\, E^\pm\zw = F^\pm\zw\, ,
% \end{equation}
% where, using \eqn{idm_dispersion_def}, the dispersion is now given by
%  \begin{equation} \label{eqn:mml_1d_disp_def}
% \mathcal{D}(\omega_0, \omega) = \sum_{m = 2}^\infty \frac{\omega^m}{m!}\, D_m\wn
%  \end{equation}
% and
%  \begin{equation} \label{eqn:mml_1d_disp_coeff_redeff}
% D_m\wn = \frac{L}{\tau_g^m} \frac{d^m}{d \omega_0^m} \Re\left[\beta\wn\right]\, .
%  \end{equation}
% Recall that $z$ is expressed in units of $L$ (the round-trip physical length of the laser resonator), and $\alpha\wn$ in terms of $L^{-1}$. Similarly, $t$ has units of $\tau_g$ (the group round-trip propagation time), and $\omega$ has units of $\tau_g^{-1}$.

% We'll apply the same approach we used in the beginning of \sct{laser_dynamics_1d_mml} to build an expression for the time derivative of the electric field coefficient $E_q(t)$. We'll focus on the case of the unidirectional ring laser here, but a similar analysis of a standing-wave laser will yield the same result. Taking the Fourier transform of \eqn{mml_e_field_1d_t}, and applying the Fourier Shift Theorem discussed in \sct{math_prelim_fourier_transforms}, yields
%  \begin{equation} \label{eqn:mml_1d_e_field_w}
% E\zw = \sum_{q = -\infty}^\infty u_{q}\z\, E_{q}\left(\omega - \Delta \omega_q\right)\, ,
%  \end{equation}
% where, as usual, Fourier transform pairs are distinguished by their arguments. Let's define the small angular frequency $\nu \equiv \omega - \Delta \omega_q$. To first order in $\nu$ (the slowly-varying envelope approximation in the frequency domain),
%  \begin{equation} \label{eqn:mml_1d_eq_svea_w}
%  \begin{split}
% \omega^m\, E_q\left(\omega - \Delta \omega_q\right) &= (\Delta \omega_q + \nu)^m\, E_q(\nu) \\
% &\approx (2 q \pi)^m\, E_q(\nu) + m (2 q \pi)^{m - 1}\, \nu\, E_q(\nu)\, ,
%  \end{split}
%  \end{equation}
% where we've used \eqn{mml_1d_delta_w_q_def} to make the approximation $\Delta \omega_q \approx 2 q \pi$ in the exponentiated coefficients. Substituting \eqn{mml_1d_e_field_w} and \eqn{mml_1d_eq_svea_w} into \eqn{mml_1d_weq_w_norm}, and then applying \eqn{laser_resonator_1d_u_hlde} in the form
%  \begin{equation} \label{eqn:mml_1d_u_hlde}
% \ddz u_q\z = i\, \left(\Delta \omega_q - \delta \omega_q\right) u_q\z + \left[\frac{1}{2\, \tau_p} - \half\, \alpha\wn\right] u_q\z\, ,
%  \end{equation}
% we obtain
% % \begin{multline}
% %\sum_l u_l\z \left[ 1 + \delta \tau_l\wn \right] (-i \nu)\, E_l(\nu) = \\ \sum_l u_l\z \left\{-\frac{1}{2\, \tau_p} + i \left[ \delta \omega_l + \sum_{m = 2}^\infty \frac{(2 l \pi)^m}{m!}\, D_m\wn \right] \right\} E_l(\nu) + F\zw\, ,
% % \end{multline}
%  \begin{multline} \label{eqn:mml_1d_nu_e}
% \sum_l u_l\z \left[ 1 + \delta \tau_l\wn \right] (-i \nu)\, E_l(\nu) = \\ \sum_l u_l\z \left\{-\frac{1}{2 \tau_p} \left( 1 + i\, \Omega_l \right) + i\, \delta D_l\wn \right\} E_l(\nu) + F\zw\, ,
%  \end{multline}
% where we have chosen $\delta \omega_l = -\Omega_l/2\, \tau_p$ to incorporate frequency pulling,

We follow the approach we used to derive \eqn{mml_edot_temp}, and multiply both sides of \eqn{mml_1d_disp_cont} by $v_q\z$ and then integrate the result over the cavity length. Collecting the resulting dispersion terms allow us to obtain the updated field coefficient equation of motion
\begin{equation} \label{eqn:mml_1d_deq_dt_final}
    \dot{E}_q(t) = \frac{1}{1 + \delta \tau_q\wn} \left\{ \left[-\frac{1}{2\, \tau_p} \left( 1 + i\, \Omega_q \right) + i\, \delta D_q\wn\right] E_q(t) + F_q(t) \right\}\, ,
\end{equation}
where $F_q(t)$ is again given by \eqn{mml_fq_sol}. We see two primary effects of dispersion. First, there is an additional frequency shift for each mode that increases (in magnitude) nonlinearly with mode number $q$. Second, the group round-trip time is slightly different for each mode, changing with $q$ by a factor of $1 + \delta \tau_q\wn$.
   
% Using the Fourier Shift Theorem, we note that
%  \begin{equation}
% \int_{-\infty}^{\infty} \frac{d \omega}{2 \pi}\, e^{-i \omega t}\, E_{q}\left(\omega - \Delta \omega_q\right) = e^{i \Delta \omega_q t}\, \int_{-\infty}^{\infty} \frac{d \nu}{2 \pi}\, e^{-i \nu t}\, E_{q}(\nu)\, ,
%  \end{equation}
% and we apply this transform to \eqn{mml_1d_nu_e} to obtain the updated field coefficient equation of motion
% \begin{equation}
%\left[ 1 + \delta \tau_q\wn \right] \dot{E}_q(t) = \\ \left\{-\frac{1}{2\, \tau_p} + i \left[ \delta \omega_q + \sum_{m = 2}^\infty \frac{(2 q \pi)^m}{m!}\, D_m\wn \right] \right\} E_q(t) + F_q(t)\, ,
% \end{equation}
% \begin{equation} \label{eqn:mml_1d_deqdt}
%\left[ 1 + \delta \tau_q\wn \right] \dot{E}_q(t) = \\ \left\{-\frac{1}{2 \tau_p} \left( 1 + i\, \Omega_q \right) + i\, \delta D_q\wn \right\} E_q(t) + F_q(t)\, ,
% \end{equation}

%Following the same procedure leading to \eqn{mml_1d_phiq_dot_approx}, we find
% \begin{equation}%\label{eqn}
%\delta \omega_q + \sum_{m = 2}^\infty \frac{D_m\wn}{m!}\, (2 q \pi)^m = -\frac{\Omega_q}{2\, \tau_p}\, ,
% \end{equation}
%which we can solve using \eqn{mml_1d_omega_q_def} for $\delta \omega_q$. Therefore, the total frequency shift and frequency displacement for mode $q$ due to mode pulling and dispersion is given respectively by
% \begin{align}
%\label{eqn:mml_1d_freq_shift} \delta \omega_q &= -\frac{1}{1 + \tau_\perp/2\, \tau_p} \left[\frac{\tau_\perp}{2\, \tau_p}\, (2 q \pi) + \sum_{m = 2}^\infty \frac{(2 q \pi)^m}{m!}\, D_m\wn \right]\, , \nd \\
%\label{eqn:mml_1d_freq_disp} \Delta \omega_q &= \frac{1}{1 + \tau_\perp/2\, \tau_p} \left[2 q \pi - \sum_{m = 2}^\infty \frac{(2 q \pi)^m}{m!}\, D_m\wn \right]\, .
% \end{align}
% Collecting results, we have updated our field equation of motion to read
%Although it is not obvious from the form of \eqn{mml_1d_deq_dt_final}, as the magnitudes of the dispersion coefficients increase, the primary effect will be to change the phase of the nonlinear coupling driving the evolution of each mode.

%. Applying this result to the field in \eqn{mml_e_field_1d_t}, we have to third order
% \begin{equation}
%\mathcal{L}(t)\, \mathbf{E}\zt = \left[ i\, \beta_1\wn \frac{d}{d t} - \frac{\beta_2\wn}{2} \frac{d^2}{d t^2} - i\, \frac{\beta_3\wn}{6} \frac{d^3}{d t^3} \right] \mathbf{E}\zt
% \end{equation}
%In \eqn{mml_1d_deq_dt}, we have scaled the time by $\tau_s \equiv 2\, \tau_p$ and divided out a common factor of $\beta_1\wn$. For the moment, let's suppress the factor of $\tau_s$, and for convenience scale the time by $\tau_g^\parallel$. Then, multiplying $\mathcal{L}(t)$ through by a factor of $-i \tau_g^\parallel/\beta_1 = -i \tau_g^\parallel v_g = -i L$ (where $L$ is the round-trip physical path length, or twice the physical length of the cavity), we have a scaled dispersion operator and operand given by
%
%
%%keeping only the new second and third order terms, and approximating $\Delta \omega_q t \approx (2 q \pi/\tau_g^\parallel) (\tau_g^\parallel t) = (2 q \pi) t$ in \eqn{mml_e_field_1d_t}
% \begin{equation}%\label{}
%\mathcal{L}^\prime(t)\, E_q(t)\, e^{-i\, 2 q \pi\, t} = \left[ \frac{i}{2}\, D_2\wn\, \frac{d^2}{d t^2} - \frac{1}{6}\, D_3\wn\, \frac{d^3}{d t^3} \right] E_q(t)\, e^{-i\, 2 q \pi\, t} ,
% \end{equation}
%The second and third time derivatives are given by
% \begin{subequations}
% \begin{align}
%\frac{d^2}{d t^2} E_q(t)\, e^{-i\, 2 q \pi\, t} &= \left[ -i\, 2\, (2 q \pi)\, \dot{E}_q(t) - (2 q \pi)^2\, E_q(t) \right] e^{-i\, 2 q \pi\, t} , \nd \\
%\frac{d^3}{d t^3} E_q(t)\, e^{-i\, 2 q \pi\, t} &= \left[ -3\, (2 q \pi)^2\, \dot{E}_q(t) + i\, (2 q \pi)^3\, E_q(t) \right] e^{-i\, 2 q \pi\, t} .
% \end{align}
% \end{subequations}
%Therefore, collecting results, and cancelling the common factor of $\exp\left(-i\, 2 q \pi\, t\right)$, we find
% \begin{equation}
% \begin{split}
%e^{+i\, 2 q \pi\, t}\, \mathcal{L}^\prime(t)\, E_q(t)\, e^{-i\, 2 q \pi\, t} &= \left[ D_2\wn\, (2 q \pi) + \frac{D_3\wn}{2}\, (2 q \pi)^2 \right] \dot{E}_q(t)\\ &-i \left[ \frac{D_2\wn}{2}\, (2 q \pi)^2 + \frac{D_3\wn}{6}\, (2 q \pi)^3 \right] E_q(t),
% \end{split}
% \end{equation}
%

%%%%%%%%%%%%%%%%%%%%%%%%%%%%%%%%%%%%%%%%%%%%%%%%%%%%%%%%%%%%%%%%%%%%%%%%%%%%%%
%
% Subsection file included in section file using \input{}
%
% Assumes that LaTeX2e macros and packages defined in rgb_laser_physics.sty
%   are available
%
%%%%%%%%%%%%%%%%%%%%%%%%%%%%%%%%%%%%%%%%%%%%%%%%%%%%%%%%%%%%%%%%%%%%%%%%%%%%%%
 \section{Injection-Seeded Gain-Switched Lasers\label{sct:laser_dynamics_1d_mml_qsl}}

We consider the ideal four-level laser dynamical equations developed in \sct{laser_amp_1d_pdes}, and we assume that $\gamma_\perp \longrightarrow \infty$, so that $\Omega = 0$. The formal integral of \eqn{cw_sml_ftz_scaled} becomes
 \begin{equation} \label{eqn:qsl_ftzt_formal}
\widetilde{F}\zt = \frac{\gamma_\perp}{2}\, e^{-\gamma_\perp t} \int_{-\infty}^{t} d t^\prime\, e^{\gamma_\perp t^\prime}\, \widetilde{G}\left(z, t^\prime\right) \widetilde{E}\left(z, t^\prime\right) \, .
 \end{equation}
We now apply the strong REA in the same limit, and assume that $|\partial \widetilde{E}\zt / \partial t| \ll \gamma_\perp |\widetilde{E}\zt|$, and $|\partial \widetilde{G}\zt / \partial t| \ll \gamma_\perp |\widetilde{G}\zt|$. In this case, both $\widetilde{E}\left(z, t^\prime\right)$ and $\widetilde{G}\left(z, t^\prime\right)$ can be moved outside of the integral, yielding
 \begin{equation} \label{eqn:qsl_ftzt_rea}
\widetilde{F}\zt = \frac{1}{2}\, \widetilde{G}\zt\, \widetilde{E}\zt \, .
 \end{equation}
Substituting this result into \eqn{cw_sml_gtz_scaled} gives
 \begin{equation} \label{eqn:qsl_dgdt_rea}
\ppt \widetilde{G}\zt = \frac{1}{\tau_\parallel} \left[ \overline{G}\zt - \widetilde{G}\zt - \widetilde{G}\zt \left| \widetilde{E}\zt\right|^2 \right]\, ,
 \end{equation}

Let's allow the intracavity field to be supplemented by a quantity $\widetilde{J}\zt$ arising from a very weak input $F_1(t)$ injected through the output coupler mirror $\mathcal{M}_1$, as shown in \fig{resonator_1d_smat}. This additional field will contribute to the total macroscopic polarization, so that
 \begin{equation} \label{eqn:qsl_ftzt_inj}
\widetilde{F}\zt = \frac{1}{2}\, \widetilde{G}\zt\, \left[\widetilde{E}\zt + \widetilde{J}\zt\right]\, .
 \end{equation}
In \sct{laser_resonators_1d_tcm}, we learned how to expand $\widetilde{J}\zt$ as a series of quasi-normal spatial modes in both the unidirectional ring and standing-wave resonator cases. Because the injected field is so weak, we do not need to include it in the saturation term in \eqn{qsl_dgdt_rea}.

 \subsection{Unidirectional Ring Lasers\label{sct:laser_dynamics_1d_mml_qsl_url}}
As discussed in the introduction to \chp{laser_dynamics_1d_mml}, in the case of the URL the rapidly-varying spatial function $\exp(+i k_0 z)$ is common to both $\widetilde{E}\zt$ and $\widetilde{F}\zt$, and can therefore be ignored in \eqn{qsl_ftzt_rea} and \eqn{qsl_dgdt_rea}. Therefore, we find $F_q(t)$ in \eqn{mml_edot_temp} by substituting \eqn{mml_e_field_1d_t} --- and the corresponding expression for $J\zt$ --- into \eqn{qsl_ftzt_inj}, and then the result into \eqn{mml_1d_fq_def_url}. We obtain
 \begin{equation} \label{eqn:qsl_url_fqt}
F_q(t) = \half\, \sum_p e^{i 2 ( q - p ) \pi t}\, G_{q - p}(t) \left[ E_p(t) + J_p(t) \right]\, ,
 \end{equation}
where
 \begin{equation} \label{eqn:qsl_gqp_def}
G_{q - p}(t) \equiv \int_0^1 d z\, v_q\z\, u_p\z\, G\zt = \int_0^1 d z\, e^{-i 2 (q - p) \pi z}\, G\zt\, ,
 \end{equation}
and $J_p(t)$ is given by \eqn{eqt_inj}. \Eqn{qsl_url_fqt} and the slowly-spatially-varying partial differential equation
 \begin{equation} \label{eqn:qsl_dgdt_url}
\ppt G\zt = \frac{1}{\tau_\parallel} \left[ \overline{G}\zt - G\zt - G\zt \left| E\zt\right|^2 \right]\,
 \end{equation}
are the only tools we'll need to solve numerically a wide variety of gain-switched URL problems.

 \subsection{Standing-Wave Lasers\label{sct:laser_dynamics_1d_mml_qsl_swl}}
The calculation of the macroscopic polarization for a multimode standing-wave laser requires that we pay attention to the interference between the counterpropagating fields. We'll follow a strategy similar to that of the continuous-wave case described in \sct{laser_statics_1d_shb}. We begin with an explicit expression for the spatially rapidly-varying polarization of \eqn{qsl_ftzt_inj}, written as
 \begin{multline} \label{eqn:qsl_1d_fzt_swl}
F^+\zt\, e^{+i k_0 z} + F^-\zt\, e^{-i k_0 z} = \\ \half\, \widetilde{G}\zt \left\{\left[E^+\zt + J^+\zt\right] e^{+i k_0 z} + \left[E^-\zt + J^-\zt\right] e^{-i k_0 z}\right\}\, .
 \end{multline}
The envelope functions $F^\pm\zt$, $E^\pm\zt$, and $J^\pm\zt$ are spatially slowly varying, but we will need to average $\widetilde{G}\zt$ over a physical wavelength. Following the procedure outlined in \eqn{ld1d_sw_shb_pzp_full}, we find
 \begin{subequations} \label{eqn:qsl_1d_fpmzt_swl}
 \begin{align}
F^+\zt &= \half\, \mathcal{G}^{[0]}\zt \left[E^+\zt + J^+\zt\right] + \half\, \mathcal{G}^{[-2]}\zt \left[E^-\zt + J^-\zt\right]\, , \nd \\
F^-\zt &= \half\, \mathcal{G}^{[+2]}\zt \left[E^+\zt + J^+\zt\right] + \half\, \mathcal{G}^{[0]}\zt \left[E^-\zt + J^-\zt\right]\, ,
 \end{align}
 \end{subequations}
where
 \begin{equation} \label{eqn:qsl_1d_gnzt}
\mathcal{G}^{[n]}\zt \equiv \frac{k_0}{2 \pi} \int_{z - \pi/k_0}^{z + \pi/k_0} d z^\prime\, e^{+i n k_0 z^\prime}\, \widetilde{G}(z^\prime, t)\, .
 \end{equation}
Substituting \eqn{qsl_1d_fpmzt_swl} into \eqn{mml_1d_fq_def_swl} yields
 \begin{multline}
F_q(t) = \half \sum_p e^{i 2 (q - p) \pi t} \left[E_p(t) + J_p(t)\right]
\int_0^{1/2} d z\, \left[ \mathbf{v}_q\z \dotp \mathbf{u}_p\z\, \mathcal{G}^{[0]}\zt \right. \\
\left. + v_q^+\z\, u_p^-\z\, \mathcal{G}^{[-2]}\zt + v_q^-\z\, u_p^+\z\, \mathcal{G}^{[+2]}\zt \right]\, .
 \end{multline}

We'll need to construct partial differential equations for $\mathcal{G}^{[0]}\zt$ and $\mathcal{G}^{[\pm 2]}\zt$. We expand $|\widetilde{E}\zt|^2$ as
 \begin{equation}
\left|\widetilde{E}\zt\right|^2 = \left|E^{+}\zt\right|^2 +  \left|E^{-}\zt\right|^2 + E^{+}\zt\, E^{- \ast}\zt\, e^{+i 2 k_0 z} + E^{-}\zt\, E^{+ \ast}\zt\, e^{-i 2 k_0 z}\, ,
 \end{equation}
substitute this expression into \eqn{qsl_dgdt_rea}, and then apply the average specified by \eqn{qsl_1d_gnzt} to obtain
 \begin{equation} \label{eqn:qsl_dgndt_swl}
 \begin{split}
\ppt \mathcal{G}^{[n]}\zt &= \frac{1}{\tau_\parallel} \left\{ \delta_{n, 0}\, \overline{G}\zt - \mathcal{G}^{[n]}\zt - \mathcal{G}^{[n]}\zt \left[ \left|E^{+}\zt\right|^2 +  \left|E^{-}\zt\right|^2 \right] \right. \\
&\qquad \left. -~\mathcal{G}^{[n + 2]}\zt\, E^{+}\zt\, E^{- \ast}\zt - \mathcal{G}^{[n - 2]}\zt\, E^{-}\zt\, E^{+ \ast}\zt \right\}\, .
 \end{split}
 \end{equation}
If the pump has a short duration compared to $\tau_\parallel$, then rapid temporal oscillations in $E^{\pm}\zt\, E^{\mp \ast}\zt$ will diminish the contributions of higher-order spatial averages and allow us to neglect $\mathcal{G}^{[\pm 4]}\zt$. 
%%%%%%%%%%%%%%%%%%%%%%%%%%%%%%%%%%%%%%%%%%%%%%%%%%%%%%%%%%%%%%%%%%%%%%%%%%%%%%
%
% Subsection file included in section file using \input{}
%
% Assumes that LaTeX2e macros and packages defined in rgb_laser_physics.sty
%   are available
%
%%%%%%%%%%%%%%%%%%%%%%%%%%%%%%%%%%%%%%%%%%%%%%%%%%%%%%%%%%%%%%%%%%%%%%%%%%%%%%
 \section{Passively Mode-Locked Lasers\label{sct:laser_dynamics_1d_mml_mll}}

% \subsection{New and Busted: Mode-Locked Lasers}
% \begin{subequations}
%     \begin{align}
%         \widetilde{E}\zt &= \sum_{m n} E_{m n}\, \widetilde{u}_m(z)\, e^{-i\, \Delta \omega_n\, t}\, , \nd \\
%         \widetilde{F}\zt &= \sum_{l q} F_{l q}\, \widetilde{u}_l\z\, e^{-i\, \Delta \omega_q\, t}\, .
%     \end{align}
% \end{subequations}
% where $\Delta \omega_n \equiv 2\, n\, \pi$, and
% \begin{subequations}
%     \begin{align}
%         E_{m n} &\equiv \int_{z_\mathrm{min}}^{z_\mathrm{max}} d z\, \int_{-\half}^{+\half}\, d t\, \widetilde{v}_m(z)\, e^{i\, \omega_n\, t}\, E\zt\, , \nd \\
%         F_{l q} &\equiv \int_{z_\mathrm{min}}^{z_\mathrm{max}} d z\, \int_{-\half}^{+\half}\, \widetilde{v}_l\z\, e^{i\, \Delta \omega_q\, t}\, \widetilde{F}\zt\, .
%     \end{align}
% \end{subequations}

% \begin{equation}
%     2\, \Re\left[\widetilde{E}^\ast\zt\, \widetilde{F}\zt\right] = \sum_{i j m n} \widetilde{u}_i\z\, \widetilde{u}^\ast_{j}\z\, e^{-i\, \Delta \omega_{m - n}\, t} \left( E_{i m}\, F^\ast_{j n} + F_{i m}\, E^\ast_{j n} \right)\, .
% \end{equation}

% \begin{equation}
%     \widetilde{G}\zt = \Gnz - \sum_{i j m n} \widetilde{u}_i\z\, \widetilde{u}^\ast_{j}\z\, e^{-i\, \Delta \omega_{m - n}\, t}\, \mathcal{C}_{m - n} \left( E_{i m}\, F^\ast_{j n} + F_{i m}\, E^\ast_{j n} \right)\, ,
% \end{equation}
% where
% \begin{equation}
%     \mathcal{C}_q \equiv \left(1 - i\, \Delta \omega_q\, \tau_\parallel\right)^{-1}\, .
% \end{equation}

% \begin{multline}
%     \widetilde{E}\zt\, \widetilde{G}\zt = \Gnz \sum_{k p} \widetilde{u}_k\z\, e^{-i\, \Delta \omega_p\, t} E_{l p} \\
%     - \sum_{i j k m n p} \widetilde{u}_i\z\, \widetilde{u}^\ast_j\z\, \widetilde{u}_k\z\, e^{-i\, \Delta \omega_{m - n + p}\, t}\, \mathcal{C}_{m - n} \left( E_{i m}\, F^\ast_{j n} + F_{i m}\, E^\ast_{j n} \right) E_{k p}\, .
% \end{multline}

% \begin{multline}
%     \widetilde{F}\zt = \half\, \Gnz \sum_{k p} \widetilde{u}_k\z\, e^{-i\, \Delta \omega_p\, t}\, \mathcal{L}\left(\Omega_p\right)\, E_{k p} \\
%     - \half \sum_{i j k m n p} \widetilde{u}_i\z\, \widetilde{u}^\ast_j\z\, \widetilde{u}_k\z\, e^{-i\, (\Delta \omega_{m - n + p})\, t}\, \mathcal{L}\left(\Omega_{m - n + p}\right)\,  \mathcal{C}_{m - n} \left( E_{i m}\, F^\ast_{j n} + F_{i m}\, E^\ast_{j n} \right) E_{k p}\, .
% \end{multline}

% Perform the time integral to obtain
% \begin{equation}
%     \begin{split}
%         F_{l q} &= \half\, \mathcal{L}\left(\Omega_q\right) \sum_k\, E_{k q} \int d z\, \widetilde{v}_l\z\, \widetilde{u}_k\z\, \Gnz \\
%         &- \half\, \mathcal{L}\left(\Omega_q\right) \sum_{i j k m n} \mathcal{C}_{m - n} \left( E_{i m}\, F^\ast_{j n} + F_{i m}\, E^\ast_{j n} \right) E_{k, m - n + q}\, \int d z\, \widetilde{v}_l\z\, \widetilde{u}_i\z\, \widetilde{u}^\ast_j\z\, \widetilde{u}_k\z \\
%         &\equiv \half\, \mathcal{L}\left(\Omega_q\right) \left[ \sum_k\, \overline{G}_{k l}\, E_{k q} - \sum_{i j k m n} \kappa_{i j k l}\, \mathcal{C}_{m - n} \left( E_{i m}\, F^\ast_{j n} + F_{i m}\, E^\ast_{j n} \right) E_{k, m - n + q}\right]\, ,
%     \end{split}
% \end{equation}
% where
% \begin{align}
%     \overline{G}_{k l} &\equiv \int d z\, \widetilde{v}_l\z\, \widetilde{u}_k\z\, \Gnz\, , \nd \\
%     \kappa_{i j k l} &\equiv \int d z\, \widetilde{v}_l\z\, \widetilde{u}_i\z\, \widetilde{u}^\ast_j\z\, \widetilde{u}_k\z\, .
% \end{align}

% \begin{equation}
%     \begin{split}
%         F_{q p} &= \half\, \mathcal{L}\left(\Omega_q\right) \sum_k\, E_{q k} \int d z\, \widetilde{u}_k\z\, \widetilde{v}_p\z\, \Gnz \\
%         &- \half\, \mathcal{L}\left(\Omega_q\right) \sum_{i j k m n} \mathcal{C}_{m - n} \left( E_{m j}\, F^\ast_{n k} + F_{m j}\, E^\ast_{n k} \right) E_{m - n + q,\, l}\, \int d z\, \widetilde{u}_j\z\, \widetilde{u}^\ast_k\z\, \widetilde{u}_l\z\, \widetilde{v}_p\z \\
%         &\equiv \half\, \mathcal{L}\left(\Omega_q\right) \left[ \sum_k\, E_{q k}\, \overline{G}_{k p} - \sum_{j k l m n} \mathcal{C}_{m - n}\, \kappa_{j k l p} \left( E_{m j}\, F^\ast_{n k} + F_{m j}\, E^\ast_{n k} \right) E_{m - n + q,\, l}\right]\, ,
%     \end{split}
% \end{equation}
% where
% \begin{align}
%     \overline{G}_{k p} &\equiv \int d z\, \widetilde{u}_k\z\, \widetilde{v}_p\z\, \Gnz\, , \nd \\
%     \kappa_{j k l p} &\equiv \int d z\, \widetilde{u}_j\z\, \widetilde{u}^\ast_k\z\, \widetilde{u}_l\z\, \widetilde{v}_p\z\, .
% \end{align}

% \subsubsection{Unidirectional Ring Lasers}
% In the case of a unidirectional ring laser, the rapidly-varying quasi-normal spatial modes are given by
% \begin{subequations} %\label{eqn:laser_resonator_1d_uv}
%     \begin{align}
%        \widetilde{u}_q\z &\equiv \mathcal{C}\, e^{+\left[ i 2 q \pi + \ln(1/\sqrt{R}) \right] z}\, e^{i\, k_0\, z} , \nd \\ %\label{eqn:laser_resonator_1d_u} \\
%        \widetilde{v}_q\z &\equiv \frac{1}{\mathcal{C}}\, e^{-\left[ i 2 q \pi + \ln(1/\sqrt{R}) \right] z}\, e^{-i\, k_0\, z}\, , %\label{eqn:laser_resonator_1d_v}
%     \end{align}
% \end{subequations}
% where $\mathcal{C}$ is given by \eqn{laser_resonator_1d_u_norm_url}. Therefore,
% \begin{equation} %\label{eqn:mll_url_zqp_spec}
%     \overline{G}_{k l} = \int_{z_1}^{z_2} d z\, e^{i\, 2 (k - l)\, \pi\, z} \Gnz\, ,
% \end{equation}
% and we see that $\overline{G}_{k l}$ is the Fourier series coefficient of order $k - l$ for $\Gnz$ in the resonator. \red{In practice, we can use this representation as a guide to the range of values of $l$ that we need to include to provide a numerically accurate computation of the intracavity gain.} Suppose that $\Gnz = \Gnb/(z_2 - z_1)$ for $0 < z_1 \le z \le z_2 < 1$, and is zero otherwise. In this (common) special case, when $k \ne l$ we have
% \begin{equation} %\label{eqn:mll_url_zqp_spec}
%     \overline{G}_{k l} = \frac{\Gnb}{z_2 - z_1}\, \int_{z_1}^{z_2} d z\, e^{i\, 2 (k - l)\, \pi\, z} = -i\, \Gnb\, \frac{\exp[i\, 2 \left(k - l\right) \pi\, z_2] - \exp[i\, 2 \left(k - l\right) \pi\, z_1]}{2 \left(k - l\right) \pi \left(z_2 - z_1\right)}\, ,
% \end{equation}
% and $\overline{G}_{l l} = \Gnb$. Note that when $\{z_1, z_2\} \longrightarrow \{0, 1\}$, $\overline{G}_{k l} \longrightarrow \delta_{k l}\, \Gnb$.

% \begin{equation}
%     \kappa_{i j k l} = \mathcal{C}^2 \int_0^1 d z\, e^{\left[i\, 2 \left(i - j + k - l\right) \pi + \ln(1/R)\right] z} = \Delta_{i - j + k - l}(R)\, .
% \end{equation}

% \subsection{Old Hotness: Mode-Locked Lasers}
As in the case of the $Q$-switched laser discussed in \sct{laser_dynamics_1d_mml_qsl}, intermodal coupling through nonlinearities in the macroscopic polarization $\widetilde{F}\zt$ add dynamics to the gain and saturation of each mode that can lead to novel dynamical behavior. In a mode-locked laser, the amplitude and phases of the longitudinal modes are fixed in such a way that the output of the laser has particularly desirable properties, such as very short pulses or very stable (quasi-continuous-wave) behavior. We can understand this behavior through formal expansions of $\widetilde{F}\zt$ and $\widetilde{G}\zt$ in the Fourier frequency domain, but we must relax the rate-equation approximation that led to \eqn{qsl_ftzt_rea} to allow fluctuations in the polarization and gain that occur at integer multiples of the cavity free-spectral range $2 \pi/\tau_g$.

In mode-locked lasers, the gain is generally constant in time, and the dynamic fields arise from the nonlinear coupling of the longitudinal modes through the macroscopic polarization. However, the coefficients of both the electric field and macroscopic polarization vary slowly over the round-trip propagation time $\tau_g$, and the time derivatives of both $E_q(t)$ and $F_q(t)$ tend to zero as the intracavity laser amplifier reaches equilibrium. Therefore, we begin with the formal solution of \eqn{cw_sml_gtz_scaled}, finding
\begin{equation}
  G_{q p}(t) = \overline{G}_{q p} - \sum_{m n} e^{-i\, \Delta \omega_{m - n}\, t} \kappa_{q p m n}\, \mathcal{C}_{m n} \left( E_m\, F_n^\ast + F_m\, E_n^\ast \right)\, ,
\end{equation}
where
\begin{equation}
  \mathcal{C}_{m n} \equiv \left(1 - i\, \Delta \omega_{m - n}\, \tau_\parallel\right)^{-1}\, .
\end{equation}
Note that we have assumed that the pump $\Gn\zt$ is constant in time, so that its Fourier coefficients $\overline{G}_{q p}$ are also constant in time. Substituting this expression into \eqn{mml_fq_sol}, we find
\begin{equation}
    \label{eqn:mml_fqt_mll}
    \begin{split}
        F_q &= \half\, \Lq \sum_p e^{i\, \Delta \omega_{q - p}\, t}\, \overline{G}_{q p}\, E_p \\
        &- \half\, \Lq \sum_{m n p} e^{i\, \Delta \omega_{q - p - m + n}\, t}\, \kappa_{q p m n}\, \mathcal{C}_{m n} \left( E_m\, F_n^\ast + F_m\, E_n^\ast \right) E_p\, ,
    \end{split}
\end{equation}
Both $E_q(t)$ and $F_q(t)$ vary slowly in time compared to the rapid oscillations of the exponential functions in \eqn{mml_fqt_mll}, so terms with nonzero frequencies will average out. Therefore, only terms with $p = q$ in the first sum and $p = q - m + n$ in the second sum will contribute significantly to the value of $F_q$. Thus, we obtain the simplified expression
\begin{equation}
    \label{eqn:mml_fq_mml}
    F_q = \half\, \Lq \Gnb\, E_q
    - \half\, \Lq \sum_{m n} \kappa_{q m n}\, \mathcal{C}_{m n} \left( E_m\, F_n^\ast + F_m\, E_n^\ast \right) E_{q - m + n}\, ,
\end{equation}
where the contracted three-index spatial coupling coefficient is given by
\begin{equation}
    \kappa_{q m n} = \begin{cases}
      \Delta_0(R) & \text{(URL)}\\
      \Delta_0(R_1\, R_2) + \Delta_{2(m - n)}(R_1\, R_2) & \text{(SWL)} \\
      \Delta_0(R_1\, R_2) + \Delta_{2(m - n)}(R_1\, R_2) + \Delta_{2(q - m)}(R_1\, R_2) & \text{(SHB)}
    \end{cases}
\end{equation}

Referring to \eqn{mml_1d_deq_dt_final}, as the mode-locked laser oscillator reaches equilibrium, $F_q(t)$ becomes a constant and $\dot{E}_q(t) \longrightarrow 0$. In this case, we find that $F_q$ must also satisfy the expression
\begin{equation}
    F_q = R_q\, E_q\, ,
\end{equation}
where
\begin{equation}
    R_q \equiv \frac{1}{2\, \tau_\lambda} \left(1 + i\, \Omega_q\right) - i\, \delta D_q\, .
\end{equation}
Therefore,
% \begin{equation}
%     \sum_p e^{i\, \Delta \omega_{q - p}\, t}\, \overline{G}_{q p}\, E_p - 2\, \mathcal{L}^{-1}\bigl(\Omega_q\bigr)\, B_q\, E_q
%     = \sum_{m n p} e^{i\, \Delta \omega_{q - p - m + n}\, t}\, \kappa_{q p m n}\, C_{m - n} \left( B_m + B_n^\ast \right) E_m\, E_n^\ast\, E_p
% \end{equation}
\begin{equation}
    \left[ \Gnb - 2\, R_q  / \Lq\right] E_q
    = \sum_{m n} \kappa_{q m n}\, B_{m n}\, C_{m n}\, E_m\, E_n^\ast\, E_{q - m + n}\, ,
\end{equation}
where
\begin{equation}
    B_{m n} \equiv R_m + R_n^\ast = \frac{1}{\tau_\lambda} + i\, \frac{\Omega_m - \Omega_n}{2\, \tau_\lambda} - i \left(\delta D_m - \delta D_n\right)\, .
\end{equation}
\begin{equation}
    B_{m n} \equiv R_m + R_n^\ast = \frac{1}{\tau_\lambda} + i\, \frac{\tau_\perp}{2\, \tau_\lambda}\, \left(\omega_m - \omega_n\right) - i \left(\delta D_m - \delta D_n\right)\, .
\end{equation}

% This expression must be true at all times, so for convenience, let's use it to calculate $E_q$ at $t = 0$. We obtain
% \begin{equation}
%     \sum_p \overline{G}_{q p}\, E_p - 2\, \mathcal{L}^{-1}\bigl(\Omega_q\bigr)\, B_q\, E_q = \sum_{m n p} \kappa_{q p m n}\, C_{m n} \left( B_m + B_n^\ast \right) E_m\, E_n^\ast\, E_p
% \end{equation}


% As a general representation of the spatially rapidly-varying fields in both unidirectional ring and standing-wave resonator configurations, we follow \sct{laser_resonators_1d_swl} and represent the electric field amplitude function as
%  \begin{equation} \label{eqn:mml_e_1d_t_rv}
% \widetilde{E}\zt \equiv \sum_{p = -\infty}^\infty \widetilde{u}_p\z\, e^{-i\, \Delta \omega_p\, t}\, E_p(t)\, .
%  \end{equation}
% In the unidirectional ring case,
%  \begin{equation}
% \widetilde{u}_q\z = u^{+}_q\z\, e^{+i k_0 z}\, ,
%  \end{equation}
% where $k_0$ is the propagation constant associated with the carrier frequency $\omega_0$. For a standing-wave resonator,
%  \begin{equation}
% \widetilde{u}_q\z = u^{+}_q\z\, e^{+i k_0 z} + u^{-}_q\z\, e^{-i k_0 z}\, .
%  \end{equation}
% We use a similar approach to the expansion of the amplitude of the macroscopic polarization, with a subtle difference:
%  \begin{equation} \label{eqn:mml_f_1d_t_rv}
% \widetilde{F}\zt \equiv \sum_{q = -\infty}^\infty \widetilde{w}_q\z\, e^{-i\, \Delta \omega_q\, t}\, F_q(t)\, .
%  \end{equation}
% Here $\widetilde{w}_q(z)$ represents the spatial dependence of each frequency component of the macroscopic polarization. If we look carefully at \eqn{cw_sml_gtz_scaled}, we see that to first order in $\widetilde{E}\zt$, the gain is given by the function $\overline{G}\zt$ that describes the pump. Suppose that the pump is constant in time, so that $\overline{G}\zt \equiv \overline{G}\z$. Then we can write the pump function as
% \begin{equation} \label{eqn:mml_1d_pump_sep}
%   \overline{G}\z \equiv \Gn\, \mathcal{Z}\z\, ,
% \end{equation}
% \begin{equation}
%   G_0\z \equiv \Gn\, \mathcal{Z}\z\, ,
% \end{equation}
% where $\mathcal{Z}\z$ is a real function of $z$ normalized such that $\int_0^1 d z\, \mathcal{Z}\z = 1$ in the URL case or $2 \int_0^{1/2} d z\, \mathcal{Z}\z = 1$ in the SWL case, and $\Gn$ represents the \emph{small-signal (unsaturated) round-trip intensity gain}. Then a comparison of both sides of \eqn{cw_sml_ftz_scaled} suggests that
%  \begin{equation}
% \widetilde{w}_q\z \approx \mathcal{Z}\z\, \widetilde{u}_q\z\, .
%  \end{equation}
% In the following analysis, we'll also make a simplifying assumption: as discussed in \sct{laser_dynamics_1d_mml_frq}, $\delta \omega_q$ and, therefore, $\Delta \omega_q$ are linear in $q$.

% In this section, our primary tools will be the formal solutions of \eqn{cw_sml_ftz_scaled} and \eqn{cw_sml_gtz_scaled}, obtained through Fourier transform expansions. For example, consider the ordinary differential equation
%  \begin{equation}
% \ddt y(t) = -\frac{1}{\tau} \left[y(t) + s(t)\right]\, ,
%  \end{equation}
% for some function $s(t)$. Applying the Fourier Transform and using \eqn{fourier_freq} and \eqn{fourier_shift_thm}, we find
%  \begin{equation}
% y(\omega) = \frac{s(\omega)}{1 - i\, \omega\, \tau}\, .
%  \end{equation}
% %We obtain
% % \begin{equation}
% %A(t) = \left(1 + \tau\, \ddt\right)^{-1} B(t)\, ,
% % \end{equation}
% % \begin{equation}
% %\left(1 + \tau\, \ddt\right)^{-1} \equiv \sum_{l = 0}^{\infty} \left( -\tau\, \ddt \right)^l \, .
% % \end{equation}
% Suppose that $s(t)$ can be written as
%  \begin{equation}
% s(t) = \sum_q e^{-i\, \Delta \omega_q\, t}\, s_q(t)\, ,
%  \end{equation}
% giving the transform
%  \begin{equation}
% s(\omega) = \sum_q s_q(\omega - \Delta \omega_q)\, .
%  \end{equation}
% %If we define $\nu_q \equiv \omega - \Delta \omega_q$, and
% % \begin{equation}
% %c_q \equiv \frac{1}{1 - i\, \Delta \omega_q\, \tau}\, .
% % \end{equation}
% If we define $\nu_q \equiv \omega - \Delta \omega_q$, then we can rewrite $y(\omega)$ as
%  \begin{equation}
% y(\omega) = \sum_q \frac{1}{1 - i\, \Delta \omega_q\, \tau - i\, \nu_q\, \tau}\, s_q(\nu_q)\, .
%  \end{equation}
% With our usual casual indifference to mathematical rigor, we expand the denominator of this equation as a power series, and then apply the inverse Fourier transform over the frequency $\omega$ to each term separately. We obtain
%  \begin{equation}
% y(t) = \sum_q e^{-i\, \Delta \omega_q\, t}\, \left(1 - i\, \Delta \omega_q\, \tau + \tau\, \ddt\right)^{-1} s_q(t)\, ,
%  \end{equation}
% where for convenience we have defined the differential operator
% % \begin{equation}% \label{eqn:mll_diff_oper}
% %   \begin{split}
% %     \left(1 - i\, \Delta \omega_q\, \tau + \tau\, \ddt\right)^{-1} &= \sum_{l = 0}^{\infty} \left( i\, \Delta \omega_q\, \tau - \tau\, \ddt \right)^l \\
% %     &= \sum_{l = 0}^{\infty} \sum_{j = 0}^{l} \binom{l}{j} \left( i\, \Delta \omega_q\, \tau\right)^{l - j} (-\tau)^j\, \frac{d^j}{d t^j} \\
% %     &= \sum_{j = 0}^{\infty} \frac{(-\tau)^j}{\left(1 - i\, \Delta \omega_q\, \tau\right)^{j + 1}}\, \frac{d^j}{d t^j}\, .
% %   \end{split}
% % \end{equation}
% \begin{equation} \label{eqn:mll_diff_oper}
%   \begin{split}
%     \left(1 - i\, \Delta \omega_q\, \tau + \tau\, \ddt\right)^{-1} &= \left(1 - i\, \Delta \omega_q\, \tau\right)^{-1} \left(1 + \frac{\tau}{1 - i\, \Delta \omega_q\, \tau}\, \ddt\right)^{-1} \\
%     &= \frac{1}{1 - i\, \Delta \omega_q\, \tau}\, \sum_{l = 0}^{\infty} \left(-\frac{\tau}{1 - i\, \Delta \omega_q\, \tau}\right)^l\, \frac{d^l}{d t^l}\, .
%   \end{split}
% \end{equation}

% Let's apply this technique to solve the evolution equation for $\widetilde{G}\zt$ given by \eqn{cw_sml_gtz_scaled}. Using \eqn{mml_e_1d_t_rv} and \eqn{mml_f_1d_t_rv}, the nonlinear term on the \rhs can be written as
%  \begin{equation*}
%  \begin{split}
%     2 \Re \left[ \widetilde{E}^\ast\zt\, \widetilde{F}\zt \right] &= \widetilde{E}^\ast\zt\, \widetilde{F}\zt + c.c. \\
%     &= \sum_{m n}  e^{-i\, \Delta \omega_{m - n}\, t}\, \widetilde{u}_m\z\, \widetilde{u}_n^\ast\z\, \mathcal{Z}\z \left[ E_m(t)\, F_n^\ast(t) + F_m(t)\, E_n^\ast(t) \right]\, ,
%  \end{split}
%  \end{equation*}
% Therefore, using the Fourier transform expansion described above, we quickly find the formal solution
%  \begin{equation}  \label{eqn:mll_gzt_formal}
%  \begin{split}
% \widetilde{G}\zt &= \Gn\, \mathcal{Z}\z - \sum_{m n}  e^{-i\, \Delta \omega_{m - n}\, t}\, \widetilde{u}_m\z\, \widetilde{u}_n^\ast\z\, \mathcal{Z}\z \\
% &\qquad \times \left(1 - i\, \Delta \omega_{m - n} \, \tau_\parallel + \tau_\parallel\, \ddt\right)^{-1} \left[ E_m(t)\, F_n^\ast(t) + F_m(t)\, E_n^\ast(t) \right]\, .
%  \end{split}
%  \end{equation}
% %where
% % \begin{equation} \label{eqn:mll_1d_c_def}
% %C_q \equiv \frac{1}{1 - i\, \Delta \omega_q\, \tau_\parallel}\, .
% % \end{equation}
% We see that $\widetilde{G}\zt$ is rapidly-varying in space, and oscillates in time at a collection of frequencies that are approximately integer multiples of the free spectral range of the resonator. The Fourier transform approach to \eqn{cw_sml_ftz_scaled} is equally straightforward, giving the formal solution
%  \begin{equation}  \label{eqn:mll_fzt_formal}
%  \begin{split}
% \widetilde{F}\zt &= \frac{\Gn}{2}\, \sum_p e^{-i\, \Delta \omega_p\, t}\, \widetilde{u}_p\z\, \mathcal{Z}\z \left(1 - i\, \Omega_p + \tau_\perp\, \ddt\right)^{-1} E_p(t) \\
% &\quad - \half\, \sum_{m n p}  e^{-i\, \Delta \omega_{m - n + p}\, t}\, \widetilde{u}_m\z\, \widetilde{u}_n^\ast\z\, \widetilde{u}_p\z\, \mathcal{Z}\z \left(1 - i\, \Omega_{m - n + p} + \tau_\perp\, \ddt\right)^{-1} E_p(t) \\
% &\qquad \times \left(1 - i\, \Delta \omega_{m - n} \, \tau_\parallel + \tau_\parallel\, \ddt\right)^{-1} \left[ E_m(t)\, F_n^\ast(t) + F_m(t)\, E_n^\ast(t) \right]\, ,
%  \end{split}
%  \end{equation}
% where $\Omega_q \equiv \Omega_0 + \Delta \omega_q\, \tau_\perp$, and $\Omega_0$ is given by \eqn{tls_omega_0_def}.

% \begin{subequations} \label{eqn:mll_fgtzt_formal}
% \begin{align}
% \label{eqn:mll_ftzt_formal} \widetilde{F}\zt &= \frac{\gamma_\perp}{2}\, e^{-\gamma_\perp ( 1 - i\, \Omega_0 ) t} \int_{-\infty}^{t} d t^\prime\, e^{\gamma_\perp ( 1 - i\, \Omega_0 ) t^\prime}\, \widetilde{G}\left(z, t^\prime\right) \widetilde{E}\left(z, t^\prime\right) \, , \nd \\
% \label{eqn:mll_gtzt_formal} \widetilde{G}\zt &= \gamma_\parallel\, e^{-\gamma_\parallel t} \int_{-\infty}^{t} d t^\prime\, e^{\gamma_\parallel t^\prime}\, \left\{ \overline{G}\left(z, t^\prime\right) - 2 \Re \left[ \widetilde{E}^\ast\left(z, t^\prime\right) \widetilde{F}\left(z, t^\prime\right) \right] \right\} \, ,
% \end{align}
% \end{subequations}
%and our goal will be an expression for $\widetilde{F}\zt$ that is accurate to third order in $\widetilde{E}\zt$ \cite{ref:sargent1974lp}. Following the assumptions leading to \eqn{mml_1d_omega_q_def} and \eqn{mml_1d_freq_shift}, we'll take $\Omega_0 = 0$ in the remainder of this discussion.
%
%Suppose that the pump $\overline{G}\zt$ applied to the laser amplifier changes very slowly compared to the upper laser level lifetime $\tau_\parallel = \gamma_\parallel^{-1}$. (In most cases of practical interest, this constraint is equivalent to assuming that the pump is constant in time.) Then, to zeroth-order in the electric field amplitude, \eqn{mll_gtzt_formal} predicts that
% \begin{equation} \label{eqn:mll_gzt0}
%\widetilde{G}^{(0)}\zt = \gamma_\parallel\, e^{-\gamma_\parallel t} \int_{-\infty}^{t} d t^\prime\, e^{\gamma_\parallel t^\prime} \, \overline{G}\left(z, t^\prime\right) \cong \overline{G}\zt\, \gamma_\parallel\, e^{-\gamma_\parallel t} \int_{-\infty}^{t} d t^\prime\, e^{\gamma_\parallel t^\prime} = \overline{G}\zt\, .
% \end{equation}
%Next, we substitute this expression and \eqn{mml_e_field_1d_t} into \eqn{mll_ftzt_formal} to obtain an expression for $\widetilde{F}\zt$ that is accurate to first-order in $E_p(t)$. We find
% \begin{equation}
%\widetilde{F}^{(1)}\zt = \half\, \sum_p  \widetilde{u}_p\z\, \gamma_\perp\, e^{-\gamma_\perp\, t} \int_{-\infty}^{t} d t^\prime\, e^{\gamma_\perp ( 1 - i\, \Omega_p ) t^\prime}\, \overline{G}\left(z, t^\prime\right)\, E_p\left(t^\prime\right)\, ,
% \end{equation}
%where $\Omega_p = \Delta \omega_p\, \tau_\perp = (2 p \pi + \delta \omega_p) / \gamma_\perp$. Let's refine the rate equation approximation in this multimode case to assume that neither $\overline{G}\zt$ nor $E_p(t)$ change significantly during a time duration $\tau_\perp = \gamma_\perp^{-1}$. Moving both $\overline{G}\left(z, t^\prime\right)$ and $E_p\left(t^\prime\right)$ outside the time integral yields
% \begin{equation} \label{eqn:mll_fzt1}
%\widetilde{F}^{(1)}\zt = \frac{\overline{G}\zt}{2}\, \sum_p \frac{e^{-i\, \Delta \omega_p\, t}}{1 - i\, \Omega_p}\, E_p(t)\, \widetilde{u}_p\z\, .
% \end{equation}
%
%Our next assignment is to use \eqn{mml_e_field_1d_t}, \eqn{mll_gtzt_formal} and \eqn{mll_fzt1} to determine $\widetilde{G}^{(2)}\zt$. To second order in $E_q(t)$, the nonlinear term in \eqn{mll_gtzt_formal} becomes
% \begin{equation*}
% \begin{split}
%    2 \Re \left[ \widetilde{E}^\ast\zt\, \widetilde{F}^{(1)}\zt \right] &= \widetilde{E}^\ast\zt\, \widetilde{F}^{(1)}\zt + c.c. \\
%    &\equiv \overline{G}\zt\, \sum_{m n}  e^{-i (\Delta \omega_m - \Delta \omega_n) t}\, B_{m n}\, \widetilde{u}_m\z\, \widetilde{u}_n^\ast\z\, E_m(t)\, E_n^\ast(t)\, ,
% \end{split}
% \end{equation*}
%where
% \begin{equation} \label{eqn:mll_1d_b_def}
%B_{m n} \equiv \half\, \left( \frac{1}{1 - i\, \Omega_m} + \frac{1}{1 + i\, \Omega_n} \right)\, .
% \end{equation}
%Substituting this result into \eqn{mll_gtzt_formal} yields
% \begin{equation}
%\widetilde{G}^{(2)}\zt = -\overline{G}\zt\, \sum_{m n} B_{m n}\, \widetilde{u}_m\z\, \widetilde{u}_n^\ast\z\, \gamma_\parallel\, e^{-\gamma_\parallel t} \int_{-\infty}^{t} d t^\prime\, e^{[\gamma_\parallel - i (\Delta \omega_m - \Delta \omega_n)] t^\prime}\, E_m\left(t^\prime\right)\, E_n^\ast\left(t^\prime\right)\, .
% \end{equation}
%In practice, it may be difficult to claim that $E_q(t)$ will vary slowly relative to the timescale $\tau_\parallel = \gamma_\parallel^{-1}$. However, we can make the much more reasonable assumption that the dynamical variables will not change significantly during the group round-trip time $\tau_g$, so that $e^{- i (\Delta \omega_m - \Delta \omega_n) t}$ varies rapidly compared to $E_q(t)$. In this case, we have
% \begin{equation} \label{eqn:mll_gzt2}
%\widetilde{G}^{(2)}\zt \equiv -\overline{G}\zt\, \sum_{m n} e^{-i (\Delta \omega_m - \Delta \omega_n) t}\, B_{m n}\, C_{m n}\, \widetilde{u}_m\z\, \widetilde{u}_n^\ast\z\, E_m(t)\, E_n^\ast(t)\, .
% \end{equation}
%Finally, substitution of \eqn{mml_e_field_1d_t} and \eqn{mll_gzt2} into \eqn{mll_ftzt_formal}, followed by application of the rate-equation approximation,  yields for the third-order macroscopic polarization
% \begin{equation} \label{eqn:mll_fzt3}
% \begin{split}
%\widetilde{F}^{(3)}\zt &= -\frac{\overline{G}\zt}{2}\, \sum_{p m n}  \frac{e^{-i (\Delta \omega_p + \Delta \omega_m - \Delta \omega_n) t}}{1 - i (\Omega_p + \Omega_m - \Omega_n)}\, B_{m n}\, C_{m n} \\
%&\qquad \times \widetilde{u}_p\z\, \widetilde{u}_m\z\, \widetilde{u}_n^\ast\z\, E_p(t)\, E_m(t)\, E_n^\ast(t)\, .
% \end{split}
% \end{equation}
%The total macroscopic polarization, valid to third order in $E_q(t)$, is given by the sum of \eqn{mll_fzt1} and \eqn{mll_fzt3}.

%  \subsection{Unidirectional Ring Lasers\label{sct:laser_dynamics_1d_mml_mll_url}}
% As discussed above, in the case of the URL, the rapidly-varying spatial function $\exp(+i k_0 z)$ is common to both $\widetilde{E}\zt$ and $\widetilde{F}\zt$, and can therefore be ignored in \eqn{mll_fzt_formal}. If we substitute \eqn{mll_fzt_formal} into \eqn{mml_fq_sol}, we obtain
% % \begin{equation}
% % \begin{split}
% %F_q(t) &\cong \half\, \sum_p \frac{e^{i (\Delta \omega_q - \Delta \omega_p) t}}{1 - i\, \Omega_p}\, \overline{G}_{q - p}(t)\, E_p(t) \\
% %&\qquad - \half\, \sum_{p m n} \frac{e^{i (\Delta \omega_q - \Delta \omega_p - \Delta \omega_m + \Delta \omega_n) t}}{1 - i (\Omega_p + \Omega_m - \Omega_n)}\, B_{m n}\, C_{m n}\, \overline{G}_{q - p, m - n}(t)\, E_p(t)\, E_m(t)\, E_n^\ast(t)\,  ,
% % \end{split}
% % \end{equation}
%  \begin{equation} \label{eqn:mll_url_fqt_g}
%  \begin{split}
% F_q(t) &= \frac{\Gn}{2}\, \sum_p e^{i\, \Delta \omega_{q - p}\, t}\, \mathcal{Z}_{q - p}\, \left(1 - i\, \Omega_p + \tau_\perp\, \ddt\right)^{-1} E_p(t) \\
% &\quad - \half\, \sum_{m n p}  e^{i\, \Delta \omega_{q - m + n - p}\, t}\, \kappa_{q m n p}\, \left(1 - i\, \Omega_{m - n + p} + \tau_\perp\, \ddt\right)^{-1} E_p(t) \\
% &\qquad \times \left(1 - i\, \Delta \omega_{m - n} \, \tau_\parallel + \tau_\parallel\, \ddt\right)^{-1} \left[ E_m(t)\, F_n^\ast(t) + F_m(t)\, E_n^\ast(t) \right]\, ,
%  \end{split}
%  \end{equation}
% where
%  \begin{equation} \label{eqn:mll_url_zqp_def}
% \mathcal{Z}_{q - p} \equiv \int_0^1 d z\, v_q\z\, u_p\z\, \mathcal{Z}\z = \int_0^1 d z\, e^{-i 2 (q - p) \pi z}\, \mathcal{Z}\z
%  \end{equation}
% is essentially a one-dimensional discrete spatial Fourier transform of $\mathcal{Z}\z$, and
%  \begin{equation}
% \kappa_{q m n p} \equiv \int_0^1 d z\, v_q\z\, u_p\z\, u_m\z\, u_n^\ast\z\, \mathcal{Z}\z\, .
%  \end{equation}
%
% Then
% \begin{equation} \label{eqn:mll_url_gbqp_def}
%\overline{G}_{q - p} = \Gn \int_0^1 d z\, e^{-i 2 (q - p) \pi z}\, \mathcal{Z}\z \equiv \Gn\, \mathcal{Z}_{q - p}\, .
% \end{equation}
% \begin{equation} \label{eqn:mll_url_fqt1}
%F_q^{(1)}(t) = \frac{\Gnt}{2}\, \sum_p \frac{e^{i 2 ( q - p ) \pi t}}{1 - i\, \Omega_p}\, \mathcal{Z}_{q - p}\, E_p(t)\, .
% \end{equation}
% Suppose that $\mathcal{Z}\z = 1/(z_2 - z_1)$ for $0 < z_1 \le z \le z_2 < 1$, and is zero otherwise. In this (common) special case, when $q \ne p$ we have
%  \begin{equation} \label{eqn:mll_url_zqp_spec}
% \mathcal{Z}_{q - p} = i\, \frac{\exp[-i 2 (q - p) \pi z_2] - \exp[-i 2 (q - p) \pi z_1]}{2 (q - p) \pi (z_2 - z_1)}\, ,
%  \end{equation}
% and $\mathcal{Z}_0 = 1$ due to the normalization of $\mathcal{Z}\z$. Note that when $\{z_1, z_2\} \longrightarrow \{0, 1\}$, $\mathcal{Z}_{q - p} \longrightarrow \delta_{q p}$. However, if the intracavity laser amplifier does not fill the resonator, then \emph{in general} the quasi-normal spatial modes of the cavity will couple at first order through the Fourier transform included in \eqn{mll_url_zqp_def}.

% But we now make a crucial observation that will simplify our numerical analysis of mode-locked URLs. Since we have assumed that $|\dot{E}_q(t)| \ll |E_q(t)|/\tau_g$, terms in the first sum on the \rhs of \eqn{mll_url_fqt_g} with $p \ne q$, as well as terms in the second sum with $p \ne q - m + n$, are rapidly varying and will average out after the laser has reached stable operation. If we neglect these terms, then the polarization becomes
% \begin{equation}
%F_q(t) \cong \frac{\Gnt}{2 \left(1 - i\, \Omega_q\right)} \left[ E_q(t) - \kappa\, \sum_{m n} e^{-i\, \delta \phi_{q m n}(t)}\, B_{m n}\, C_{m n}\, E_{q - m + n}(t)\, E_m(t)\, E_n^\ast(t) \right]\,  ,
% \end{equation}
%  \begin{equation} \label{eqn:mll_url_fqt}
%  \begin{split}
% F_q(t) &= \half\, \Gn\, \left(1 - i\, \Omega_q + \tau_\perp\, \ddt\right)^{-1} E_q(t) - \frac{\kappa}{2}\, \left(1 - i\, \Omega_q + \tau_\perp\, \ddt\right)^{-1} \\
% &\qquad \times \sum_{m n} E_{q - m + n}(t)\, \left(1 - i\, \Delta \omega_{m - n} \, \tau_\parallel + \tau_\parallel\, \ddt\right)^{-1} \left[ E_m(t)\, F_n^\ast(t) + F_m(t)\, E_n^\ast(t) \right]\, ,
%  \end{split}
%  \end{equation}
% where
% \begin{equation}
%\delta \phi_{q m n}(t) \equiv (\delta \omega_{q - m + n} - \delta \omega_q + \delta \omega_m - \delta \omega_n)\, t\, , \nd
% \end{equation}
%  \begin{equation}
% \kappa \equiv \mathcal{C}^2 \int_0^1 d z\, e^{\ln(1/R)\, z}\, \mathcal{Z}\z\, .
%  \end{equation}
% In the case where $\mathcal{Z}\z = 1/(z_2 - z_1)$ for $0 < z_1 \le z \le z_2 < 1$, and is zero otherwise, we find
%  \begin{equation}
% \kappa = \frac{R}{1 - R}\, \frac{e^{\ln(1/R)\, z_2} - e^{\ln(1/R)\, z_1}}{z_2 - z_1} .
%  \end{equation}
% If $\{z_1, z_2\} \longrightarrow \{0, 1\}$, then $\kappa \longrightarrow 1$.

%In the third-order term of \eqn{mll_url_fqt}, we have made the approximation
% \begin{equation}
%\Omega_{q - m + n} + \Omega_m - \Omega_n = \Omega_q + \delta \phi_{q m n}(\tau_\perp) \approx \Omega_q
% \end{equation}
%because we have assumed throughout this analysis that $\delta \omega_q \ll 2 q \pi$. We can make a similar approximation for the coefficients $B_{m n}$ and $C_{m n}$ under the same assumption, but not for $\delta \phi_{q m n}(t)$. Because the shift caused by frequency pulling is linear in $q$ by \eqn{mml_1d_freq_shift}, if we can neglect dispersion then $\delta \phi_{q m n}(t) = 0$. However, if we include the effects of dispersion, then (to third order)
% \begin{equation}
%\delta \phi_{q m n}(t) = \frac{(q - m)(m - n)(2 \pi)^2}{1 + \tau_\perp/2\, \tau_p} \left[ D_2\wn + (q + n)\, \pi\, D_3\wn \right] t\, .
% \end{equation}
%Including this phase shift allows us to reduce the temporal fluctuations in the amplitudes $E_q(t)$, and in practice allows numerical solutions of \eqn{mml_1d_deq_dt_final} to converge more rapidly.

%  \subsubsection{Standing-Wave Lasers\label{sct:laser_dynamics_1d_mml_mll_swl}}

% The calculation of the macroscopic polarization for a standing-wave laser proceeds in essentially the same fashion as the unidirectional ring laser, but interference between the counterpropagating fields will complicate our calculations of the spatial coupling between electric field modes contributing to the nonlinear terms in the macroscopic polarization. Our strategy is straightforward, if a bit tedious. Following our approach in both \sct{laser_statics_1d_shb} and \sct{laser_dynamics_1d_mml_qsl}, we begin with the spatially rapidly-varying polarization given by \eqn{mll_fzt_formal}, now written explicitly as
% \begin{multline}
%F^{+}\zt\, e^{+i k_0 z} + F^{-}\zt\, e^{-i k_0 z} = \\ \frac{\overline{G}\zt}{2}\, \sum_p \frac{e^{-i\, \Delta \omega_p\, t}}{1 - i\, \Omega_p}\, \left[u^+_p\z\, e^{+i k_0 z} + u^-_p\z\, e^{-i k_0 z}\right] E_p(t)\, .
% \end{multline}
%  \begin{equation*}%  \label{eqn:mll_fzt_formal}
%  \begin{split}
% F^{+}\zt\, e^{+i k_0 z} + F^{-}\zt\, e^{-i k_0 z} &= \frac{\Gn}{2}\, \sum_p e^{-i\, \Delta \omega_p\, t}\, \left[u^+_p\z\, e^{+i k_0 z} + u^-_p\z\, e^{-i k_0 z}\right] \mathcal{Z}\z \\
% &\quad \times \left(1 - i\, \Omega_p + \tau_\perp\, \ddt\right)^{-1} E_p(t) \\
% &\quad - \half\, \sum_{m n p}  e^{-i\, \Delta \omega_{m - n + p}\, t}\, \widetilde{u}_m\z\, \widetilde{u}_n^\ast\z\, \widetilde{u}_p\z\, \mathcal{Z}\z \\
% &\quad \times \left(1 - i\, \Omega_{m - n + p} + \tau_\perp\, \ddt\right)^{-1} E_p(t) \\
% &\quad \times \left(1 - i\, \Delta \omega_{m - n} \, \tau_\parallel + \tau_\parallel\, \ddt\right)^{-1} \left[ E_m(t)\, F_n^\ast(t) + F_m(t)\, E_n^\ast(t) \right]\, .
%  \end{split}
%  \end{equation*}
% Let's start with the linear (first) term on the \rhs of this expression. If we make the reasonable assumption that $\mathcal{Z}\z$ is spatially slowly-varying on the scale of a wavelength, then the counterpropagating components of the polarization cleanly separate, and
%  \begin{equation}
% \mathbf{F}^{(1)}\zt = \frac{\Gn}{2}\, \sum_p e^{-i\, \Delta \omega_p\, t}\, \mathbf{u}_p\z\, \mathcal{Z}\z \left(1 - i\, \Omega_p + \tau_\perp\, \ddt\right)^{-1} E_p(t)\, ,
%  \end{equation}
% where we have used \eqn{laser_resonator_1d_u_sw_vec}. We substitute this result into \eqn{mml_fq_sol} to reproduce the first term on the \rhs of \eqn{mll_url_fqt_g}, where now
%  \begin{equation} \label{eqn:mll_swl_zqp_def}
% \mathcal{Z}_{q - p} \equiv \int_0^{1/2} d z\, \mathbf{v}_q\z \dotp \mathbf{u}_p\z\, \mathcal{Z}\z = 2 \int_0^{1/2} d z\, \cos\left[ 2\, (q - p)\, \pi\, z \right]\, \mathcal{Z}\z\, .
%  \end{equation}
% Let's suppose that $\mathcal{Z}\z = 1/2 (z_2 - z_1)$ for $0 <  z_1 \le z \le z_2 < 1/2$, and is zero otherwise. In this (common) special case, for $p \ne q$
%  \begin{equation} \label{eqn:mml_1d_zeta_12_swl}
% \mathcal{Z}_{q - p} = \frac{\sin[2\, (q - p)\, \pi\, z_2] - \sin[2\, (q - p)\, \pi\, z_1]}{2\, (q - p)\, \pi\, (z_2 - z_1)}\, .
%  \end{equation}
% When $\{z_1, z_2\} \longrightarrow \{0, 1/2\}$, $\mathcal{Z}_{q - p} \longrightarrow \delta_{q p}$.

% We must go through the same exercise with the nonlinear contribution to the macroscopic polarization given by \eqn{mll_fzt_formal}. First we expand the rapidly-varying spatial functions to explicitly show their net dependence on $e^{\pm i k_0 z}$. We find
%  \begin{equation}
%  \begin{split}
% \widetilde{U}_{m n p}\z & \equiv \widetilde{u}_p\z\, \widetilde{u}_m\z\, \widetilde{u}_n^\ast\z \\
% &= \left[u^+_p\z\, e^{+i k_0 z} + u^-_p\z\, e^{-i k_0 z}\right] \Big[u_m^{+}\z\, u_n^{+ \ast}\z + u_m^{-}\z\, u_n^{- \ast}\z \\
%     & \qquad \left. + u_m^{+}\z\, u_n^{- \ast}\z\, e^{+i 2 k_0 z} + u_m^{-}\z\, u_n^{+ \ast}\z\, e^{-i 2 k_0 z}\right]\, ,
%  \end{split}
%  \end{equation}
% or, neglecting terms proportional to $e^{\pm i 3 k_0 z}$, we follow \eqn{laser_resonator_1d_u_sw_vec} and write
%  \begin{equation}
% \mathbf{U}_{m n p}\z = \begin{bmatrix}
% u_p^+\z\, \left[u_m^{+}\z\, u_n^{+ \ast}\z + u_m^{-}\z\, u_n^{- \ast}\z\right] + u_p^-\z\, u_m^{+}\z\, u_n^{- \ast}\z \\
% u_p^-\z\, \left[u_m^{+}\z\, u_n^{+ \ast}\z + u_m^{-}\z\, u_n^{- \ast}\z\right] + u_p^+\z\, u_m^{-}\z\, u_n^{+ \ast}\z
%                    \end{bmatrix}
%  \end{equation}
% Let us again assume that we can neglect all terms in the macroscopic polarization that vary on timescales greater than or equal to $\tau_g$, leading to $p = q$ in the first-order contribution, and $p = q - m + n$ for the nonlinear term. Then in order to derive $F_q(t)$ for the standing-wave laser, we need to calculate the integral
%  \begin{equation}
% \kappa_{q m n} \equiv \int_0^{1/2} d z\, \mathbf{v}_q\z \dotp \mathbf{U}_{q - m + n, m, n}\z\, \mathcal{Z}\z\, .
%  \end{equation}
% Once this result is in hand, coefficient $q$ of the modal macroscopic polarization expansion becomes
%  \begin{equation} \label{eqn:mll_swl_fqt}
%  \begin{split}
% F_q(t) &= \half\, \Gn\, \left(1 - i\, \Omega_q + \tau_\perp\, \ddt\right)^{-1} E_q(t) - \half\, \left(1 - i\, \Omega_q + \tau_\perp\, \ddt\right)^{-1} \\
% &\qquad \times \sum_{m n} \kappa_{q m n}\, E_{q - m + n}(t)\, \left(1 - i\, \Delta \omega_{m - n} \, \tau_\parallel + \tau_\parallel\, \ddt\right)^{-1} \left[ E_m(t)\, F_n^\ast(t) + F_m(t)\, E_n^\ast(t) \right]\, ,
%  \end{split}
%  \end{equation}

% If $\mathcal{Z}\z = 1/2 (z_2 - z_1)$ for $0 < z_1 \le z \le z_2 < 1/2$, and is zero otherwise, then
%  \begin{multline} \label{eqn:mll_1d_kappa_def_swl}
% \kappa_{q m n} \equiv \frac{1}{2 \left(z_2 - z_1\right)}\, \int_{z_1}^{z_2} d z\, \left\{ \mathbf{v}_q\z \dotp \mathbf{u}_{q - m + n}\z \left[u_m^{+}\z\, u_n^{+ \ast}\z + u_m^{-}\z\, u_n^{- \ast}\z\right] \right. \\ \left. + v_q^{+}\z\, u_{q - m + n}^{-}\z\, u_m^+\z\, u_n^{- \ast}\z + v_q^{-}\z\, u_{q - m + n}^{+}\z\, u_m^-\z\, u_n^{+ \ast}\z \right\}\, .
%  \end{multline}
% Using \eqn{laser_resonator_1d_u_sw} and \eqn{laser_resonator_1d_v_sw}, this spatial coupling constant becomes
%  \begin{equation} \label{eqn:mll_1d_kappa_swl}
%  \begin{split}
% \kappa_{q m n} &= \frac{\mathcal{C}^{-1} \mathcal{C}^3}{2 \left(z_2 - z_1\right)}\, \int_{z_1}^{z_2} d z\, \left\{ \left[e^{i 2 (m - n) \pi z} + e^{-i 2 (m - n) \pi z}\right] \right. \\
% &\qquad\qquad \times \left[e^{i 2 (m - n) \pi z} e^{\ln(1/R_1 R_2) z} + \frac{1}{R_1}\, e^{-i 2 (m - n) \pi z} e^{-\ln(1/R_1 R_2) z}\right] \\
% &\qquad \left. + e^{i 4 (q -m) \pi z} e^{\ln(1/R_1 R_2) z} + \frac{1}{R_1}\, e^{-i 4 (q - m) \pi z} e^{-\ln(1/R_1 R_2) z} \right\} \\
% & \equiv \Delta^\prime_{0}\left(R_1, R_2\right) + \Delta^\prime_{2 (m - n)}\left(R_1, R_2\right) + \Delta^\prime_{2 (q - m)}\left(R_1, R_2\right)\, ,
%  \end{split}
%  \end{equation}
% where
%  \begin{equation}
%  \begin{split}
% \Delta^\prime_{2 q}\left(R_1, R_2\right) &\equiv \frac{\mathcal{C}^2}{2 \left(z_2 - z_1\right)}\, \int_{z_1}^{z_2} d z\, \left\{ e^{\left[ i 4 q \pi + \ln(1/R_1 R_2)\right] z} + \frac{1}{R_1}\, e^{-\left[ i 4 q \pi + \ln(1/R_1 R_2)\right] z} \right\} \\
% &= \Delta_{2 q}(R_1 R_2)\, \frac{\mathcal{C}^2}{2 \left(z_2 - z_1\right) \ln(1/R_1 R_2)} \left\{ \left[e^{\left[i 4 q \pi + \ln(1/R_1 R_2)\right] z_2} - e^{\left[i 4 q \pi + \ln(1/R_1 R_2)\right] z_1}\right] \right. \\
% &\qquad \left.- R_1^{-1} \left[e^{-\left[i 4 q \pi + \ln(1/R_1 R_2)\right] z_2} - e^{-\left[i 4 q \pi + \ln(1/R_1 R_2)\right] z_1}\right] \right\}\,
% \, ,
%  \end{split}
%  \end{equation}
% and $\Delta_q(R)$ is defined by \eqn{laser_resonator_1d_Delta_qR}. If $\{z_1, z_2\} \longrightarrow \{0, 1/2\}$, then $\Delta^\prime_{2 q}(R_1, R_2) \longrightarrow \Delta_{2 q}(R_1 R_2)$. In this case, $\kappa_{q m n}$ becomes
%  \begin{equation} \label{eqn:mll_1d_kappa_swl_smpl}
% \kappa_{q m n} = 1 + \Delta_{2 (m - n)}(R_1 R_2) + \Delta_{2 (q - m)}(R_1 R_2)\, ,
%  \end{equation}
% The first term on the \rhs of \eqn{mll_1d_kappa_swl} and \eqn{mll_1d_kappa_swl_smpl} is simply the nonlinear coupling for the unidirectional ring laser. The second term arises from cross-saturation \emph{neglecting} interference between the counterpropagating fields, while the third term adds these interference effects. Note that \eqn{mll_1d_kappa_swl_smpl} predicts that $\kappa_{qqq} = 3$, which is the saturation constant appropriate for weak fields as described toward the end of \sct{laser_statics_1d_shb}.

 \subsection{Numerics}
Let's assume that in all cases of practical interest the transverse coherence time $\tau_\perp$ (which has been scaled by the group round-trip time $\tau_g$) is small enough that we can ignore the corresponding differential operators on the \rhs of a former equation, giving
 \begin{equation} \label{eqn:mll_swl_fqt_prac}
 \begin{split}
F_q(t) &= \half\, \mathcal{L}_q\, \Gn\, E_q(t) - \half\, \mathcal{L}_q\, \sum_{m n} \kappa_{q m n}\, E_{q - m + n}(t) \\
&\qquad \times \left(1 - i\, \Delta \omega_{m - n} \, \tau_\parallel + \tau_\parallel\, \ddt\right)^{-1} \left[ E_m(t)\, F_n^\ast(t) + F_m(t)\, E_n^\ast(t) \right]\, ,
 \end{split}
 \end{equation}
where $\mathcal{L}_{q}$ is defined by \eqn{mml_lmc_q_def}.
% \begin{equation} \label{eqn:mll_1d_l_def}
%\mathcal{L}_{q} \equiv \frac{1}{1 - i\, \Omega_q}\, .
% \end{equation}
In general, we can't make assumptions about the scaled value of $\tau_\parallel$; it could be smaller or larger than unity. In the case of a single-mode laser with no dispersion, our incorporation of frequency pulling into \eqn{mml_1d_deq_dt_fp} means that a constant pump will eventually result in $\dot{E}_q(t) = 0$. One approach to estimating the impact of the differential operator on the \rhs of \eqn{mll_swl_fqt_prac} to a multimode laser is to expand the nonlinear contribution to $F_q(t)$ to third order in the electric field coefficients. Using
 \begin{equation}
F^{(1)}_q(t) = \half\, \mathcal{L}_q\, \Gn\, E_q(t)\, ,
 \end{equation}
we obtain
 \begin{equation} \label{eqn:mll_swl_fqt_fwm_prac}
 \begin{split}
F_q(t) &\cong \half\, \mathcal{L}_q\, \Gn E_q(t) \\
&\quad - \half\, \mathcal{L}_q\, \Gn \sum_{m n} \kappa_{q m n}\, B_{m n}\, E_{q - m + n}(t) \left(1 - i\, \Delta \omega_{m - n} \, \tau_\parallel + \tau_\parallel\, \ddt\right)^{-1} E_m(t)\, E_n^\ast(t)\, ,
 \end{split}
 \end{equation}
where
 \begin{equation} \label{eqn:mll_1d_b_def}
B_{m n} \equiv \half\, \left( \mathcal{L}_m + \mathcal{L}^\ast_n \right)\, .
 \end{equation}
We note that
 \begin{equation} %\label{eqn:mll_diff_oper}
\left(1 - i\, \Delta \omega_{m - n} \, \tau_\parallel + \tau_\parallel\, \ddt\right)^{-1} \left[E_m(t)\, E_n^\ast(t)\right] = \sum_{l = 0}^{\infty} \left( i\, \Delta \omega_{m - n} \, \tau_\parallel - \tau_\parallel\, \ddt \right)^l \left[ E_m(t)\, E_n^\ast(t)\right]\, .
 \end{equation}
The $l = 1$ term of the sum on the \rhs has the form
 \begin{equation}
 \begin{split}
\left( i\, \Delta \omega_{m - n} \, \tau_\parallel - \tau_\parallel\, \ddt \right) \left[ E_m(t)\, E_n^\ast(t)\right] &= i\, \Delta \omega_{m - n} \, \tau_\parallel \left[ E_m(t)\, E_n^\ast(t)\right] \\
&\quad - \tau_\parallel \left[E_n^\ast(t)\, \dot{E}_m(t) + E_m(t)\, \dot{E}_n^\ast(t)\right]\, .
 \end{split}
 \end{equation}
Consistent with our third-order expansion of $F_q(t)$, we use \eqn{mml_1d_deq_dt_final} to estimate $\dot{E}_q(t)$ to first order in $E_q(t)$. We obtain
 \begin{equation} %\label{eqn:mml_1d_deq_dt_final}
 \dot{E}_q(t) \approx \gamma_q\, E_q(t)\, ,
 \end{equation}
where
 \begin{equation} \label{eqn:mml_1d_gamma_q_def}
 \gamma_q \equiv \frac{1}{1 + \delta \tau_q\wn} \left[ \half \left( 1 + i\, \Omega_q \right) \left( \frac{\Gn}{1 + \Omega_q^2} - \frac{1}{\tau_p} \right) + i\, \delta D_q\wn \right] .
 \end{equation}
Therefore
 \begin{equation}
\left( i\, \Delta \omega_{m - n} \, \tau_\parallel - \tau_\parallel\, \ddt \right) \left[ E_m(t)\, E_n^\ast(t)\right] \approx \left[ i\, \Delta \omega_{m - n} - \left(\gamma_m + \gamma_n^\ast\right) \right] \tau_\parallel\, \, E_m(t)\, E_n^\ast(t)\, ,
 \end{equation}
and
 \begin{equation}
 \begin{split}
\left(1 - i\, \Delta \omega_{m - n} \, \tau_\parallel + \tau_\parallel\, \ddt\right)^{-1} E_m(t)\, E_n^\ast(t) &= \sum_{l = 0}^{\infty} \left\{\left[ i\, \Delta \omega_{m - n} - \left(\gamma_m + \gamma_n^\ast\right) \right] \tau_\parallel\right\}^l\, \left[ E_m(t)\, E_n^\ast(t)\right] \\
&\equiv C_{m n}\, E_m(t)\, E_n^\ast(t)\, ,
 \end{split}
 \end{equation}
where
 \begin{equation} \label{eqn:mll_1d_cp_def}
C_{m n} \equiv \frac{1}{1 + \left(\gamma_m + \gamma_n^\ast - i\, \Delta \omega_{m - n}\right)\, \tau_\parallel}\, .
 \end{equation}
Suppose that $\tau_\parallel \lesssim 1$, and that $\Gn$ is only moderately above threshold, so that $\gamma_0 < 1$. Then $\gamma_m + \gamma_n^\ast$ can be neglected in favor of $\Delta \omega_{m - n}$. If $\tau_\parallel \gg 1$, then even at moderate gains $C_{m n}$ will be strongly suppressed; we have
 \begin{equation}
C_{m n} \approx \frac{\delta_{m n}}{1 + 2 \Re (\gamma_n)\, \tau_\parallel}\, .
 \end{equation}
%where
% \begin{equation}
%2 \Re (\gamma_n)\, \tau_\parallel = \frac{\tau_\parallel}{1 + \delta \tau_n\wn} \left( \frac{\Gn}{1 + \Omega_n^2} - \frac{1}{\tau_p} \right)\, .
% \end{equation}
This effect is even more pronounced for multimode systems well above threshold. Note that modes \emph{below} threshold should have $C_{n n} = 1$.

Let's now investigate numerical solutions of \eqn{mml_1d_deq_dt_final} after replacing the differential operator in \eqn{mll_swl_fqt_prac} with $C_{m n}$ in our computations of $F_q(t)$. First, in the ``all-wave-mixing'' (AWM) case, we choose
 \begin{equation} \label{eqn:mll_swl_fqt_awm}
 F_q(t) \cong \half\, \mathcal{L}_q\, \left\{ \Gn\, E_q(t) - \sum_{m n} \kappa_{q m n}\, C_{m n}\, E_{q - m + n}(t) \left[ E_m(t)\, F_n^\ast(t) + F_m(t)\, E_n^\ast(t) \right] \right\}\, .
 \end{equation}
This equation can be rewritten as a matrix equation for $F_q(t)$ in the form
 \begin{equation}\label{eqn:mll_1d_fqt_swl_awm}
\sum_m \left[ A_{q m}(t)\, F_m(t) + B_{q m}(t)\, F^\ast_m(t) \right] = H_q(t)\, ,
 \end{equation}
where
 \begin{align*}
A_{q m}(t) &\equiv \delta_{q, m} + \sum_n \mathcal{K}_{qmn}\, E_{q - m + n}(t)\, E_n^\ast(t)\, , \\
B_{q m}(t) &\equiv \sum_n \mathcal{K}_{qnm}\, E_{q - n + m}(t)\, E_n(t)\, , \\
\mathcal{K}_{qmn} &\equiv \half\, \mathcal{L}_q\, \kappa_{q m n}\, C_{m n}\, , \nd \\
H_q(t) &\equiv \half\, \Gn\, \mathcal{L}_q\, E_q(t)\, .
 \end{align*}
Suppose that the total number of modes in our simulation is $\mathcal{N} \equiv 2 q_\text{max} + 1$. Then we can think of $A_{q m}(t)$ and $B_{q m}(t)$ as $\mathcal{N} \times \mathcal{N}$ complex square matrices, and $F_q(t)$ and $H_q(t)$ as $\mathcal{N} \times 1$ complex column vectors. Separating all of these variables into their real and imaginary parts, we can rewrite \eqn{mll_1d_fqt_swl_awm} as the $(2 \mathcal{N} \times 2 \mathcal{N}) \cdot (2 \mathcal{N} \times 1)$ real matrix equation
 \begin{equation}\label{eqn:mml_1d_fqt_sw_mat}
\begin{bmatrix}
  \Re[A(t) + B(t)] & -\Im[A(t) - B(t)] \\
  \Im[A(t) + B(t)] & \Re[A(t) - B(t)]
\end{bmatrix} \begin{bmatrix}
                  \Re[\mathbf{F}(t)] \\
                  \Im[\mathbf{F}(t)]
                \end{bmatrix}
                 = \begin{bmatrix}
                  \Re[\mathbf{H}(t)] \\
                  \Im[\mathbf{H}(t)]
                \end{bmatrix}\, ,
 \end{equation}
which can be solved using standard numerical linear algebra techniques.% However, as written, this approach --- with $\mathcal{K}_{qmn}(t)$ incorporating $e^{-i\, \delta \phi_{q m n}(t)}$ --- can be numerically inefficient. In practice, a better algorithm would be:
% \begin{enumerate}
% \item replace $E_q(t)$ in $H(t)$ with $E_q(t) e^{-i \delta \omega_q t}$;
% \item drop $e^{-i\, \delta \phi_{q m n}(t)}$ from $\mathcal{K}_{qmn}(t)$;
% \item solve \eqn{mml_1d_fqt_sw_mat};
% \item multiply $F_q(t)$ by $e^{+i \delta \omega_q t}$; and
% \item substitute the result into \eqn{mml_1d_deq_dt_final}.
% \end{enumerate}
%It is easy to see that precisely the same approach can be applied to $F_q(t)$ in \eqn{mll_swl_fqt}.

In the low-gain, weak-field case, we can use the expansion of $F_q(t)$ to third-order in the electric field amplitude --- the ``four-wave mixing'' (FWM) case:
 \begin{equation} \label{eqn:mll_swl_fqt_fwm}
F_q(t) \cong \half\, \mathcal{L}_q\, \Gn \left[ E_q(t) - \sum_{m n} \kappa_{q m n}\, B_{m n}\, C_{m n}\, E_{q - m + n}(t)\, E_m(t)\, E_n^\ast(t) \right]\, .
 \end{equation}
In principle, \eqn{mml_1d_deq_dt_final} can be solved much more efficiently with $F_q(t)$ obtained from \eqn{mll_swl_fqt_fwm} than with \eqn{mml_1d_fqt_sw_mat}.
%
%As the unsaturated gain increases, numerical solutions of \eqn{mml_edot} relying on the third-order expansion of the macroscopic polarization given by \eqn{mll_swl_fqt} can become unstable. We can improve this stability --- at the expense of some loss of accuracy at high gains --- by treating \eqn{mll_swl_fqt} as a geometric series, and (indirectly) ``re-summing'' the terms. In this case, we find
% \begin{multline}\label{eqn:mll_swl_fqt_num}
%F_q(t) \approx \frac{1}{2 \left( 1 - i\, \Omega_q \right)} \Bigg\{ \Gnt\, E_q(t) - \\ \sum_{m n}  e^{-i\, \delta \phi_{q m n}(t)}\, \kappa_{q m n}\, C_{m n}\, E_{q - m + n}(t)  \left[ F_m(t)\, E_n^\ast(t) + E_m(t)\, F_n^\ast(t) \right] \Bigg\}\, .
% \end{multline}
%
%
%It is straightforward to show that a perturbative expansion of this expression reproduces \eqn{mll_swl_fqt}.

\subsubsection{Preliminary Solver}
\begin{equation}
  \left|E_q(t)\right|^2 -2 \Re\left[E_q^\ast(t)\, F_q(t)\right] = 0\, .
\end{equation}

In our code, we scale the time variable by the photon lifetime $\tau_p$, and compute the derivative using
\begin{equation}
  \dot{E}_q(t) = \left[-\half + i\, \left(\delta \omega_q\, \tau_p + \delta D_q\right)\right] E_q(t) + F_q(t)\, ,
\end{equation}
where $\delta \omega_q$ and $\delta D_q$ are given by \eqn{mml_1d_freq_pull} and \eqn{mml_1d_delta_d_q_def}, respectively.
Therefore,
\begin{equation}
  E^\ast_q(t)\, \dot{E}_q(t) = \left[-\frac{1}{2} + i\, \left(\delta \omega_q\, \tau_p + \delta D_q\right)\right] \left|E_q(t)\right|^2 + E^\ast_q(t)\, F_q(t)\, ,
\end{equation}
giving
\begin{align}
  \Re\left[ \frac{\dot{E}_q(t)}{E_q(t)} \right] &= -\frac{1}{2} + \Re\left[ \frac{F_q(t)}{E_q(t)} \right]\, , \text{ and} \\
  \Im\left[ \frac{\dot{E}_q(t)}{E_q(t)} \right] &= \delta \omega_q\, \tau_p + \delta D_q + \Im\left[ \frac{F_q(t)}{E_q(t)} \right]\, .
\end{align}
We see in the top two plots that $\Re[\dot{E}_q(t) / E_q(t)] \longrightarrow 0$ as $t \longrightarrow t_f$, and that in the same limit $\Im[\dot{E}_q(t) / E_q(t)] \longrightarrow \delta \nu_q\, \tau_p$, where
\begin{equation}
  \delta \nu_q \equiv \delta \omega_q + \frac{\delta D_q}{\tau_p} + \frac{1}{\tau_p}\, \Im\left[ \frac{F_q(t)}{E_q(t)} \right] \equiv \text{constant}\, .
\end{equation}
So we can use as our FOM the equations
\begin{align}
  \Re\left[ \frac{2\, F_q(t_f)}{E_q(t_f)} \right] &= 1\, , \text{ and} \\
  \Im\left[ \frac{\ddot{E}_q(t_f)}{E_q(t_f)} \right] &= 0\, ;
\end{align}
but how do we estimate $\ddot{E}_q(t_f)$?

 \subsubsection{Power Spectral Density}
Suppose that we have a numerically stable (steady-state) solution to \eqn{mml_edot_temp}, and we wish to compute the frequency content of the output intensity, defined as the square of the absolute value of an output field given by one of \eqn{laser_resonator_1d_swl_out}. Neglecting the overall normalization constant, we have
 \begin{align}%\label{}
I_\text{out}(t) &= \left| \sum_{p} e^{-i\, 2\, p\, \pi\, t}\, E_p \right|^2 = \sum_{p, p^\prime} e^{-i\, 2\, (p - p^\prime)\, \pi\, t} E_p\, E^\ast_{p^\prime} \\
&\equiv \sum_q A_q\, e^{-i\, 2\, \pi\, q\, t} ,
 \end{align}
where
 \begin{equation}
A_q \equiv \sum_p E_p\, E^\ast_{p - q} .
 \end{equation}
If $p \in \{-p_\textrm{max}, \dots, +p_\textrm{max}\}$, then, since $I_\text{out}(t)$ is real,
 \begin{align} \label{eqn:mml_1d_iout_final}
I_\text{out}(t) &= A_0 + 2 \sum_{q = 1}^{2\, p_\textrm{max}} \Re\left[ A_q\, e^{-i\, 2\, \pi\, q\, t} \right] \\
&= A_0 + 2 \sum_{q = 1}^{2\, p_\textrm{max}} \left[ \Re(A_q) \cos(2\, \pi\, q\, t) + \Im(A_q) \sin(2\, \pi\, q\, t) \right] .
 \end{align}
Therefore, following standard practice\footnote{Although both the in-phase and quadrature components are included in the definition given by \eqn{mml_1d_psd_def}, the factor of 2 in the sum of \eqn{mml_1d_iout_final} is ignored for essentially the same reason we neglect the negative frequencies when plotting the digital Fourier transform of a real signal.}, we define the \emph{power spectral density} at each frequency as
 \begin{equation} \label{eqn:mml_1d_psd_def}
P_q \equiv \frac{\sqrt{\Re(A_q)^2 + \Im(A_q)^2}}{A_0} = \frac{|A_q|}{A_0}\, ,
 \end{equation}
valid for $q \in \{0, \dots, 2\, p_\textrm{max}\}$.

 \subsubsection{Chaotic Behavior}
 \subsubsection{Passive Temporal Mode-Locking with a Saturable Absorber}
 \subsubsection{Passive Frequency Mode-Locking}

\input{files/laser_dynamics_1d_mfl}


%%%%%%%%%%%%%%%%%%%%%%%%%%%%%%%%%%%%%%%%%%%%%%%%%%%%%%%%%%%%%%%%%%%%%%%%%%%%%%%
%
% Chapter file included in main project file using \input{}
%
% Assumes that LaTeX2e macros and packages defined in rgb_laser_physics.sty
%   are available
%
%%%%%%%%%%%%%%%%%%%%%%%%%%%%%%%%%%%%%%%%%%%%%%%%%%%%%%%%%%%%%%%%%%%%%%%%%%%%%%

 \chapter{One-dimensional Multi-Mode Laser Dynamics\label{chp:laser_dynamics_1d_mml}}

In this chapter, we describe the dynamics of one-dimensional laser amplifiers and oscillators by applying the quasi-normal mode expansions derived in \sct{laser_resonators_1d_qnm} to the wave equation given by \eqn{wave_eqn_1d} and the density matrix evolution equations defined by \eqn{fls_mbe_rwa_pol} and \eqn{fls_mbe_rwa_pop_diff}. Under the right experimental conditions, these multimode representations (approximate as they are) can provide remarkably illuminating descriptions of laser behavior, including optimum output coupling, frequency pulling, wave mixing, and mode-locking.

When multiple modes oscillate in a laser, they give rise to coherent modulations of the populations in the nonlinear gain medium that create interactions between those modes. The frequencies of these modulations are integer multiples of the free spectral range $\Delta \omega_\text{FSR}$ --- defined by \eqn{delta_w_fsr_def} --- between adjacent intracavity field modes. In \fig{multimode_gain_spectrum_1d}, we show a plot of a gain medium with a peak at frequency $\omega_0 = \omega_{a b}$ as a function of the frequency detuning. We have superimposed the frequency modal structure --- over several free-spectral ranges --- of a cavity containing that medium. In the following sections, we will find that for a particular frequency $\Delta \omega_q \equiv 2 q \pi$, fluctuations in the gain medium at frequency $2 (q - p) \pi$ couple the electric field amplitude with frequency $2 p \pi$ to the macroscopic polarization component at frequency $2 q \pi$.

 \begin{figure}
  \centering
  \includegraphics[width=4.5in]{figures/multimode_gain_spectrum_1d}
  \caption{\label{fig:multimode_gain_spectrum_1d} Plot of a gain medium with a peak at frequency $\omega_0 = \omega_{a b}$ as a function of the frequency detuning. We have superimposed the frequency modal structure --- over several free-spectral ranges --- of a cavity containing the medium. We will find that for a particular frequency $\Delta \omega_q \equiv 2 q \pi$, fluctuations in the gain medium at frequency $2 (q - p) \pi$ couple the electric field amplitude with frequency $2 p \pi$ to the macroscopic polarization component at frequency $2 q \pi$.}
 \end{figure}

\section{One-Dimensional Multi-Mode Laser Evolution Equations\label{sct:laser_dynamics_1d_mml_evol_eqns}}

We begin by developing evolution equations for the complex longitudinal modal amplitudes of unidirectional and standing-wave intracavity laser fields based on the four-level Maxwell-Bloch equations given by \eqn{laser_statics_1d_sml_scaled} and \eqn{cw_sml_ftz_scaled}. We have
\begin{subequations}\label{eqn:laser_dynamics_1d_mml_scaled}
  \begin{align}
    \label{eqn:mml_etz_scaled}
    \ppt E^\pm\zt \pm \ppz E^\pm\zt &= \left[ i\, \widehat{\mathcal{D}}_0 - \half\, \an \right] E^\pm\zt + F^\pm\zt\, , \\
    \label{eqn:mml_ftz_scaled} \ppt \widetilde{F}\zt &= -\frac{1}{\tau_\perp} \left[ \mathcal{B}\, \widetilde{F}\zt - \frac{\mathcal{A}}{2}\, \widetilde{G}\zt \widetilde{E}\zt \right]\, , \nd \\
    \label{eqn:mml_gtz_scaled} \ppt \widetilde{G}\zt &= -\frac{1}{\tau_\parallel} \left\{ \widetilde{G}\zt - \Gn\zt + 2 \Re \left[ \widetilde{E}^\ast\zt \widetilde{F}\zt \right] \right\}\, ,
  \end{align}
\end{subequations}
where $\widehat{\mathcal{D}}_0$ is the differential operator defined by \eqn{cw_sml_disp_op}. Here we will defer the effects of frequency dispersion to \sct{laser_dynamics_1d_mml_frq_dis} by setting $\widehat{\mathcal{D}}_0 = 0$ in \eqn{mml_etz_scaled}.

Our goal will be to develop a set of nonlinear ordinary differential equations representing the time evolution of modal amplitudes of the electromagnetic field. Let's follow an approach similar to that used in \sct{laser_statics_1d_approx} and use the results of \sct{laser_resonators_1d_qnm} to expand $E^\pm\zt$ in terms of the quasi-normal modes of the laser resonator. For example, in the case of the one-dimensional unidirectional ring laser shown in \fig{resonator_1d_ring_gain}, $E^{-}\zt = 0$, and we can write the slowly-varying forward-propagating electric field amplitude as
\begin{equation}
  \label{eqn:mml_e_1d_zt_url}
  E^{+}\zt \equiv \sum_{q = -\infty}^\infty u_q\z\, e^{-i\, \Delta \omega_q\, t}\, E_q(t)\, ,
\end{equation}
where $u_q\z$ and the corresponding biorthogonal eigenfunction $v_q\z$ in the range $0 < z < 1$ are given by \eqn{laser_resonator_1d_u_unnorm} and \eqn{laser_resonator_1d_v_unnorm} as
\begin{subequations}
  \begin{align}
    \label{eqn:sml_1d_uq_url} u_q\z &=\mathcal{C}_\mathrm{URL}\, \exp\left[ +\left( i\, 2 q \pi + \ln\frac{1}{\sqrt{R}} \right) z \right]\, , \\
    \label{eqn:sml_1d_vq_url} v_q\z &=\mathcal{C}^{-1}_\mathrm{URL}\, \exp\left[ -\left( i\, 2 q \pi + \ln\frac{1}{\sqrt{R}} \right) z \right]\, ,
  \end{align}
\end{subequations}
$\mathcal{C}_\mathrm{URL}$ is given by \eqn{laser_resonator_1d_u_norm_url}, and
\begin{equation}\label{eqn:mml_1d_delta_w_q_def}
  \Delta \omega_q = 2 q \pi + \delta \omega_q\, ,
\end{equation}
consistent with both \eqn{cw_sml_etz_scaled} and \eqn{delta_w_q_def}. We apply the biorthogonality relation given by \eqn{laser_resonator_1d_uv_biortho} to \eqn{mml_etz_scaled} by substituting \eqn{mml_e_1d_zt_url} (with $q \longrightarrow p$) and a similar expression for $F^{+}\zt$; multiplying both sides through by $e^{+i\ \Delta \omega_q\, t}\, v_q\z$; and then integrating the result from $z = 0$ to $z = 1$. We find
\begin{equation} \label{eqn:mml_edot_temp}
  \dot{E}_q(t) = \left(-\frac{1}{2 \tau_\lambda} + i\, \delta \omega_q\right) E_q(t) + F_q(t)\, ,
\end{equation}
where $\tau_\lambda \equiv 1/\ln[1 / R \exp(-\anb)]$ is the photon lifetime\index{Photon lifetime} given by \eqn{f_fwhm} and $\anb \equiv \int_0^1 dz\, \alpha_0(z)$.

We shouldn't apply the rate-equation approximation (REA) to \eqn{mml_ftz_scaled} just yet, because a laser operating with $q_\text{max}$ longitudinal modes such that $q_\text{max}\, \Delta \omega_\text{FSR} \gtrsim 1 / \tau_\perp$ will exhibit a significant dependence of the unsaturated gain on the value of $q$. Instead, we will first substitute

because we want to keep the macroscopic polarization term $F_q(t)$ general for now. In the unidirectional ring laser case, we define $F_q(t)$ as

For the one-dimensional standing-wave laser shown in \fig{resonator_1d_sw_gain}, the slowly-varying biorthogonal eigenfunctions are $\mathbf{u}_q\z$ and $\mathbf{v}_q\z$, given by \eqn{laser_resonator_1d_u_sw_vec}, \eqn{laser_resonator_1d_u_sw}, \eqn{laser_resonator_1d_v_sw_vec}, and \eqn{laser_resonator_1d_v_sw}, and the corresponding normalization constant $\mathcal{C}_\mathrm{SWL}$ is given by \eqn{laser_resonator_1d_u_norm_swl}. Then
\begin{subequations} \label{eqn:sml_1d_uvq_swl}
  \begin{align}
    \label{eqn:sml_1d_uq_swl}
    \mathbf{u}_q\z &\equiv \begin{bmatrix} u^{+}_q\z \\ u^{-}_q\z \end{bmatrix} = \mathcal{C}_\mathrm{SWL} \begin{bmatrix} e^{+\left[ i\, 2 q \pi + \ln\left(1/\sqrt{R_1 R_2}\right) \right] z} \\ -\frac{1}{\sqrt{R_1}}\, e^{-\left[ i\, 2 q \pi + \ln\left(1/\sqrt{R_1 R_2}\right) \right] z} \end{bmatrix}\, , \nd \\
    \label{eqn:sml_1d_vq_swl}
    \mathbf{v}_q\z &\equiv \begin{bmatrix} v^{+}_q\z \\ v^{-}_q\z \end{bmatrix} = \mathcal{C}^{-1}_\mathrm{SWL} \begin{bmatrix} e^{-\left[ i\, 2 q \pi + \ln\left(1/\sqrt{R_1 R_2}\right) \right] z} \\ -\sqrt{R_1}\, e^{+\left[ i\, 2 q \pi + \ln\left(1/\sqrt{R_1 R_2}\right) \right] z} \end{bmatrix}\, ,
  \end{align}
\end{subequations}
where $0 < z < 1/2$ In this case, we apply the biorthogonality relation given by \eqn{laser_resonator_1d_uv_biortho_sw} to \eqn{cw_sml_ez_scaled} by substituting $E^{\pm}\z = \sum_p u^{\pm}_p\z\, \, e^{-i\, \Delta \omega_p\, t}\, E_p(t)$ and $F^{\pm}\z = \sum_p u^{\pm}_p\z\, \, e^{-i\, \Delta \omega_p\, t}\, F_p(t)$; forming the inner product of both sides with $e^{i\, \Delta \omega_q\, t}\, \mathbf{v}_q\z$; and then integrating the result from $z = 0$ to $z = 1/2$. Therefore, \eqn{e0_temp} remains valid for the standing-wave case with $\tau_\lambda \equiv 1 / \ln[1 / R_1 R_2 \exp(-\anb)]$ and $\anb \equiv 2 \int_0^{1/2} dz\, \alpha_0(z)$.

As a general representation of the spatially rapidly-varying fields in both unidirectional ring and standing-wave resonator configurations, we follow \sct{laser_resonators_1d_swl} and represent the electric field amplitude function as
\begin{equation}
  \label{eqn:mml_e_1d_zt_rv}
  \widetilde{E}\zt \equiv \sum_{q = -\infty}^\infty \widetilde{u}_q\z\, e^{-i\, \Delta \omega_q\, t}\, E_q(t)\, .
\end{equation}
In the unidirectional ring laser case,
\begin{subequations}
  \label{eqn:mml_1d_uvq_url}
  \begin{align}
    \label{eqn:mml_1d_uq_url} \widetilde{u}_q\z &= u^{+}_q\z\, e^{+i k_0 z}\, , \nd \\
    \label{eqn:mml_1d_vq_url} \widetilde{v}_q\z &= v^{+}_q\z\, e^{-i k_0 z}\, ,
  \end{align}
\end{subequations}
where $k_0$ is the propagation constant associated with the carrier frequency $\omega_0$. For a standing-wave resonator,
\begin{subequations}
  \label{eqn:mml_1d_uvq_swl}
  \begin{align}
    \label{eqn:mml_1d_uq_swl} \widetilde{u}_q\z &= u^{+}_q\z\, e^{+i k_0 z} + u^{-}_q\z\, e^{-i k_0 z}\, , \nd \\
    \label{eqn:mml_1d_vq_swl} \widetilde{v}_q\z &= v^{+}_q\z\, e^{-i k_0 z} + v^{-}_q\z\, e^{+i k_0 z}\, .
  \end{align}
\end{subequations}
We use a similar approach to the expansion of the amplitude of the macroscopic polarization.

\begin{multline} \label{eqn:mml_ftz_expansion}
  \sum_p \left[ \dot{F}_p(t) - i\, \Delta \omega_p\, F_p(t) \right] \widetilde{u}_p\z\, e^{-i\, \Delta \omega_p\, t} \\
  = -\frac{1}{\tau_\perp} \sum_p \left[ \mathcal{B}\, F_p(t) - \half\, \mathcal{A}\, \widetilde{G}\zt\, E_p(t) \right] \widetilde{u}_p\z\, e^{-i\, \Delta \omega_p\, t}\, .
\end{multline}
Now, rather than the rate-equation approximation (REA), we will apply the \emph{slowly-varying envelope approximation}\index{Slowly-varying envelope approximation} (SVEA) to \eqn{mml_ftz_expansion} by assuming that $|\dot{F}_p(t)| \ll |\Delta \omega_p F_p(t)|$ and neglecting the terms $\dot{F}_p(t)$ on the \lhs. This is valid when the time scale for changes in the modal polarization amplitudes $F_p(t)$ is long compared to the polarization relaxation time $\tau_\perp$.

\begin{equation} \label{eqn:mml_zop_def}
  \int d z \equiv \begin{cases}
    \int_0^{1} d z & \mbox{(URL)}\, , \\
    \int_0^{1/2} d z\, \frac{k_0}{2 \pi} \int_{z - \pi/k_0}^{z + \pi/k_0} d z' & \mbox{(SWL or SHB)}\, ,
  \end{cases}
\end{equation}

\begin{equation} %\label{eqn:mml_fq_sol_temp}
  \sum_p \mathcal{N}_{q p}\, F_p(t) = \half\, \Lq\, \sum_p e^{i \left(\Delta \omega_q - \Delta \omega_p\right) t}\, G_{q p}(t)\, E_p(t)\, , 
\end{equation}
\begin{equation} \label{eqn:mml_fq_sol_temp}
  \sum_p \mathcal{N}_{q p}\, \left(\mathcal{B} - i\, \Omega_p\right) F_p(t) = \half\, \mathcal{A}\, \sum_p e^{i \left(\Delta \omega_q - \Delta \omega_p\right) t}\, G_{q p}(t)\, E_p(t)\, , 
\end{equation}
where $\Omega_p \equiv \Delta \omega_q\, \tau_\perp$,
\begin{equation} \label{eqn:mml_nqp_def}
  \mathcal{N}_{q p} \equiv \int d z\, \widetilde{v}_q\z\, \widetilde{u}_p\z\, \nd
\end{equation}
\begin{equation} \label{eqn:mml_gqp_def}  
  G_{q p}(t) \equiv \int d z\, \widetilde{v}_q\z\, \widetilde{u}_p\z\, \widetilde{G}\zt\, .
\end{equation}
Applying \eqn{mml_zop_def} to \eqn{mml_nqp_def}, we find that for both unidirectional ring and standing-wave resonators, $\mathcal{N}_{q p} = \delta_{q p}$, and \eqn{mml_fq_sol_temp} simplifies to
\begin{equation} \label{eqn:mml_fq_sol}
  F_q(t) = \half\, \Lq\, \sum_p e^{i \left(\Delta \omega_q - \Delta \omega_p\right) t}\, G_{q p}(t)\, E_p(t)\, ,
\end{equation}
where
\begin{equation} \label{eqn:mml_lq_def}
  \Lq \equiv \frac{\mathcal{A}}{\mathcal{B} - i\, \Omega_q}\, .
\end{equation}

Let's now determine the evolution equation for $G_{q p}(t)$ by applying \eqn{mml_gqp_def} and \eqn{mml_zop_def} to \eqn{mml_gtz_scaled}. Using \eqn{mml_e_1d_zt_rv} and the corresponding representation for the macroscopic polarization, the nonlinear term on the \rhs can be written as
\begin{equation*}
  \begin{split}
    2 \Re \left[ \widetilde{E}^\ast\zt\, \widetilde{F}\zt \right] &= \widetilde{E}^\ast\zt\, \widetilde{F}\zt + c.c. \\
    &= \sum_{m n}  e^{-i\, \left(\Delta \omega_m - \Delta \omega_n\right) t}\, \widetilde{u}_m\z\, \widetilde{u}_n^\ast\z \left[ E_m(t)\, F_n^\ast(t) + F_m(t)\, E_n^\ast(t) \right]\, .
  \end{split}
\end{equation*}
We obtain
\begin{equation} \label{eqn:mml_gtz_gqp_temp}
  \dot{G}_{q p}(t) = -\frac{1}{\tau_\parallel} \left\{ G_{q p}(t) - \overline{G}_{q p}(t)  + \sum_{m n} e^{-i\, \left(\Delta \omega_m - \Delta \omega_n\right) t} \kappa_{q p m n} \left[ E_m(t)\, F_n^\ast(t) + F_m(t)\, E_n^\ast(t) \right] \right\}\, ,
\end{equation}
where
\begin{equation} \label{eqn:mml_gqp_bar_def}
  \overline{G}_{q p}(t) \equiv \int d z\, \widetilde{v}_q\z\, \widetilde{u}_p\z\, \Gn\zt\, ,
\end{equation}
and
\begin{equation} \label{eqn:mml_kqp_mn_def}
  \kappa_{q p m n} \equiv \int d z\, \widetilde{v}_q\z\, \widetilde{u}_p\z\, \widetilde{u}_m\z\, \widetilde{u}_n^\ast\z\, .
\end{equation}

Let's evaluate $\overline{G}_{q p}(t)$ and $\kappa_{q p m n}$ for the unidirectional ring resonator. Using \eqn{mml_1d_uvq_url} and \eqn{mml_zop_def}, we find
\begin{equation} %\label{eqn:mll_url_zqp_spec}
    \overline{G}_{q p}(t) = \int_{0}^{1} d z\, e^{i\, 2 (p - q)\, \pi\, z}\, \Gn\zt\, ,
\end{equation}
and we see that $\overline{G}_{q p}(t)$ is the complex exponential Fourier series coefficient of order $p - q$ for $\Gn\zt$ in the resonator. Suppose that $\Gn\zt = \Gnb(t)/(z_2 - z_1)$ for $0 < z_1 \le z \le z_2 < 1$, and is zero otherwise. In this (common) special case, when $q \ne p$ we have
\begin{equation} %\label{eqn:mll_url_zqp_spec}
    \overline{G}_{q p}(t) = \frac{\Gnb(t)}{z_2 - z_1}\, \int_{z_1}^{z_2} d z\, e^{i\, 2 (p - q)\, \pi\, z} = \frac{\exp[i\, 2 \left(p - q\right) \pi\, z_2] - \exp[i\, 2 \left(p - q\right) \pi\, z_1]}{i\, 2 \left(p - q\right) \pi \left(z_2 - z_1\right)} \, \Gnb(t)\, ,
\end{equation}
and $\overline{G}_{q q}(t) = \Gnb(t)$. Note that when $\{z_1, z_2\} \longrightarrow \{0, 1\}$, $\overline{G}_{q p}(t) \longrightarrow \delta_{q p}\, \Gnb(t)$.
For the URL, the spatial mode coupling coefficient defined by \eqn{mml_kqp_mn_def} becomes
\begin{equation}
  \label{eqn:mml_1d_kqpmn_url}
  \kappa_{q p m n} = \mathcal{C}^2_\mathrm{URL} \int_0^1 d z\, e^{i\, 2\, (-q + p + m - n)\, \pi\, z} = \Delta_{-q + p + m - n}(R)\, .
\end{equation}

In the case of the standing-wave resonator, we use \eqn{mml_1d_uvq_swl} and \eqn{mml_zop_def} to find
\begin{equation} %\label{eqn:mll_swl_zqp_spec}
  \overline{G}_{q p}(t) = \int_0^{1/2} d z\, \mathbf{v}_q\z \dotp \mathbf{u}_p\z\, \Gn\zt = 2 \int_0^{1/2} d z\, \cos\left[ 2\, (q - p)\, \pi\, z \right]\, \Gn\zt\, , 
\end{equation}
showing that $\overline{G}_{q p}(t)$ is the cosine Fourier series coefficient of order $q - p$ for $\Gn\zt$ in the SWL resonator. As we did above for the URL case, let's suppose that $\Gn\zt = \Gnb(t)/2 (z_2 - z_1)$ for $0 < z_1 \le z \le z_2 < 1/2$, and is zero otherwise. Then for $p \ne q$
 \begin{equation} %\label{eqn:mml_1d_gqp_swl}
\overline{G}_{q p}(t) = \frac{\sin[2\, (q - p)\, \pi\, z_2] - \sin[2\, (q - p)\, \pi\, z_1]}{2\, (q - p)\, \pi\, (z_2 - z_1)}\, \Gnb(t)\, .
  \end{equation}
When $\{z_1, z_2\} \longrightarrow \{0, 1/2\}$, $\overline{G}_{q p}(t) \longrightarrow \delta_{q p}\, \Gnb(t)$.
For the SWL, the spatial mode coupling coefficient defined by \eqn{mml_kqp_mn_def} becomes
\begin{equation*}
  \begin{split}
    \kappa_{q p m n} &= \int_0^{1/2} d z\, \frac{k_0}{2 \pi} \int_{z - \pi/k_0}^{z + \pi/k_0} d z^\prime \widetilde{v}_q\zp\, \widetilde{u}_p\zp\, \widetilde{u}_m\zp\, \widetilde{u}_n^\ast\zp \\
    &= \int_0^{1/2} d z\,
      \left\{ \left[ v_q^+\z\, u_p^+\z\, u_m^+\z\, u_n^{+\ast}\z + v_q^-\z\, u_p^-\z\, u_m^-\z\, u_n^{-\ast}\z \right]\right. \\
      &\quad\quad\quad\quad\;\;\: + \left[ v_q^+\z\, u_p^+\z\, u_m^-\z\, u_n^{-\ast}\z + v_q^-\z\, u_p^-\z\, u_m^+\z\, u_n^{+\ast}\z \right] \\
      &\quad\quad\quad\quad\;\;\, + \left. \left[ v_q^+\z\, u_p^-\z\, u_m^+\z\, u_n^{-\ast}\z + v_q^-\z\, u_p^+\z\, u_m^-\z\, u_n^{+\ast}\z \right] \right\}\\
    &= \mathcal{C}^2_\mathrm{SWL} \int_0^{1/2} d z\,
      \left\{ \left[ e^{\left[i\, 2 (-q + p + m - n) \pi\, z + \ln(1/R_1 R_2)\right] z} +  \frac{1}{R_1}\, e^{-\left[i\, 2 (-q + p + m - n) \pi\, z + \ln(1/R_1 R_2)\right] z} \right]\right. \\
      &\quad\quad\quad\quad\quad\quad\quad\, + \left[ e^{\left[i\, 2 (q - p + m - n) \pi\, z + \ln(1/R_1 R_2)\right] z} +  \frac{1}{R_1}\, e^{-\left[i\, 2 (q - p + m - n) \pi\, z + \ln(1/R_1 R_2)\right] z} \right] \\
      &\quad\quad\quad\quad\quad\quad\quad\, + \left. \left[ e^{\left[i\, 2 (q + p - m - n) \pi\, z + \ln(1/R_1 R_2)\right] z} +  \frac{1}{R_1}\, e^{-\left[i\, 2 (q + p - m - n) \pi\, z + \ln(1/R_1 R_2)\right] z} \right] \right\}\, ,
  \end{split}
\end{equation*}
or
\begin{equation}
  \label{eqn:mml_1d_kqpmn_swl}
  \kappa_{q p m n} = \Delta^\prime_{-q + p + m - n}\left(R_1\, R_2\right) + \Delta^\prime_{q - p + m - n}\left(R_1\, R_2\right) + \Delta^\prime_{q + p - m - n}\left(R_1\, R_2\right)\, ,
\end{equation}
where $\Delta^\prime_{q}\left(R_1\, R_2\right)$ is defined by \eqn{laser_resonator_1d_Deltap_qR}. The first term on the right of this equation couples co-propagating spatial modes (and is identical to the URL coupling term given by \eqn{mml_1d_kqpmn_url} when $R_2 = 1$), the second term couples counter-propagating spatial modes neglecting interference, and the third term couples modes incorporating spatial interference.

% In the case of the one-dimensional unidirectional ring laser shown in \fig{resonator_1d_ring_gain}, we can write the spatially rapidly-varying electric field amplitude and macroscopic polarization --- assumed to be propagating in the $+\hatb{z}$ direction --- in terms of the corresponding slowly-varying fields as $\widetilde{E}\zt = E\zt e^{i k_0 z}$ and $\widetilde{F}\zt = F\zt e^{i k_0 z}$, respectively. This common factor of $\exp(i\, k_0\, z)$ has already been canceled from both sides of \eqn{cw_sml_etz_scaled}, which holds for the slowly-varying field amplitudes. Let's use \eqn{laser_resonator_1d_ezt_expansion} to write the slowly-varying electric field envelope function $\Ezt$ as
%  \begin{equation} \label{eqn:mml_e_field_1d_t}
% \Ezt \equiv \sum_{p = -\infty}^\infty u_{p}\z\, e^{-i\, \Delta \omega_p\, t}\, E_{p}(t)\, ,
%  \end{equation}
% consistent with both \eqn{cw_sml_etz_scaled} and \eqn{delta_w_q_def}. In \sct{laser_dynamics_1d_mml_frq}, we'll use $\delta \omega_p$ to represent the majority of the frequency shifts due to frequency pulling and dispersion, thereby reducing the magnitude and increasing the time scale of the phase fluctuations of the field envelope amplitude $E_p(t)$. We build our wave equation by substituting \eqn{mml_e_field_1d_t} into \eqn{cw_sml_etz_scaled}, and then applying \eqn{laser_resonator_1d_u_hlde} to obtain
%  \begin{equation}%\label{}
%    \sum_{p = -\infty}^\infty u_{p}\z\, e^{-i\, \Delta \omega_p\, t} \left[\dot{E}_{p}(t) + \left(\frac{1}{2 \tau_p} - i\, \delta \omega_p\right) E_{p}(t)\right] = F\zt\, ,
%  \end{equation}
% where $\dot{E}_{p}(t) \equiv d E_{p}(t)/d t$, and $\tau_p$ is the photon lifetime of the cavity defined by \eqn{f_fwhm} with $|\Gamma|^2 = R e^{-\alpha\wn L}$. Now we multiply both sides of this equation by $v_q\z$, and then integrate over $z$ from $0$ to $1$ to obtain
% where we have used \eqn{laser_resonator_1d_uv_biortho} and defined
%  \begin{equation} \label{eqn:mml_1d_fq_def_url}
% F_q(t) \equiv e^{+i\, \Delta \omega_q\, t} \int_0^1 d z\, v_q\z\, F\zt\, .
%  \end{equation}

% We use a similar approach to formulate the corresponding multimode field amplitude evolution equation for the one-dimensional standing-wave laser shown in \fig{resonator_1d_sw_gain}. In this case, we must write the spatially rapidly-varying electric field amplitude and macroscopic polarization in terms of the corresponding slowly-varying fields as $\widetilde{E}\zt = E^+\zt e^{+i k_0 z} + E^-\zt e^{-i k_0 z}$ and $\widetilde{F}\zt = F^+\zt e^{+i k_0 z} + F^-\zt e^{-i k_0 z}$, respectively. We will follow \sct{laser_resonators_1d_swl}, and write both $\mathbf{E}\zt$ and $\mathbf{F}\zt$ as column vectors, as we did in \eqn{laser_resonators_1d_e_sw_def}, with the electric field amplitude defined in \sct{laser_resonators_1d_swl} by \eqn{laser_resonator_1d_ezt_expansion_sw}:
%  \begin{equation*}
% \mathbf{E}\zt \equiv \sum_{p = -\infty}^\infty \mathbf{u}_{p}\z\, e^{-i\, \Delta \omega_p\, t}\, E_{p}(t)\, .
%  \end{equation*}
% We use \eqn{cw_sml_etz_scaled} to write the wave equation for the slowly-varying amplitudes as a vector operator equation, given by
%  \begin{equation}
%  \hat{\mathcal{L}}\, \mathbf{E}\zt = \mathbf{F}\zt ,
%  \end{equation}
% where
%  \begin{equation}
% \hat{\mathcal{L}} =  \begin{bmatrix} \ppt + \ppz + \half \alpha\wn L & 0  \\ 0 & \ppt - \ppz + \half \alpha\wn L \end{bmatrix} .
%  \end{equation}
% Applying this operator to \eqn{laser_resonator_1d_ezt_expansion_sw}, we find
%  \begin{equation}%\label{}
%    \sum_{p = -\infty}^\infty \mathbf{u}_{p}\z\, e^{-i\, \Delta \omega_p\, t} \left[\dot{E}_{p}(t) + \left(\frac{1}{2 \tau_p} - i\, \delta \omega_p\right) E_{p}(t)\right] = \mathbf{F}\zt\, .
%  \end{equation}
% Now we take the dot product of both sides of this equation with $\mathbf{v}_q\z$, and then integrate over $z$ from $0$ to $1/2$ to reproduce \eqn{mml_edot_temp} in the standing-wave case, but with $F_q(t)$ defined through \eqn{laser_resonator_1d_uv_biortho_sw} as
%  \begin{equation} \label{eqn:mml_1d_fq_def_swl}
% F_q(t) \equiv e^{+i\, \Delta \omega_p\, t} \int_0^{1/2} d z\, \mathbf{v}_q\z \dotp \mathbf{F}\zt .
%  \end{equation}

In the next two sections, we use these multimode evolution equations to study the dynamics of \emph{injection-seeded gain-switched} and \emph{passively mode-locked} lasers. In \sct{laser_dynamics_1d_mml_qsl}, we will apply the \emph{strong} rate-equation approximation (REA) to construct a non-perturbative theory of a high-intensity pulsed laser that is driven by a short-duration pump and ``primed'' by a slowly-varying input field. In \sct{laser_dynamics_1d_mml_mll}, we will relax the REA to allow rapid intermodal interactions and build a weak-field perturbative model of \emph{coherent population pulsations}\index{Coherent population pulsations}\cite{ref:sargent1974lp} that lead to passive \emph{mode-locking}\index{Mode-locking} in either the time or the frequency domain.


%%%%%%%%%%%%%%%%%%%%%%%%%%%%%%%%%%%%%%%%%%%%%%%%%%%%%%%%%%%%%%%%%%%%%%%%%%%%%%
%
% Section file included in main project file using \input{}
%
% Assumes that LaTeX2e macros and packages defined in notes_qdcl.sty are
%   available
%
%%%%%%%%%%%%%%%%%%%%%%%%%%%%%%%%%%%%%%%%%%%%%%%%%%%%%%%%%%%%%%%%%%%%%%%%%%%%%%

 \section{Frequency Shifts in One-Dimensional Multimode Lasers\label{sct:laser_dynamics_1d_mml_frq}}

 \subsection{Frequency Pulling\label{sct:laser_dynamics_1d_mml_frq_frp}}

As an example of a particular choice of $\delta \omega_q$, we will attempt to capture the bulk of the final frequency-pulling effects we expect in a multimode laser by applying the approximate single-longitudinal-mode laser theory developed in \sct{laser_statics_1d_approx}. We define the complex amplitude $E_q(t)$ in terms of a real amplitude $A_q(t)$ and a real phase $\phi_q(t)$ as
 \begin{equation}\label{eqn:mml_1d_aq_phiq_def}
E_q(t) \equiv A_q(t)\, e^{-i\, \phi_q(t)} .
 \end{equation}
Following \sct{laser_statics_1d_approx}, we assume that $\dot{A}_q(t) = 0$ and $\dot{\phi}_q(t) = 0$, so that application of \eqn{mml_edot_temp} gives
 \begin{equation} \label{eqn:mml_1d_dwq}
\delta \omega_q = -\frac{1}{2\, \tau_p}\, \frac{\Im[f_q]}{\Re[f_q]}\, .
 \end{equation}
where $f_q(t) \equiv e^{i\, \phi_q(t)} F_q(t)$.
%Substituting this expression into \eqn{mml_edot}, and then separating the result into real and imaginary parts, we find
% \begin{subequations}\label{eqn:mml_1d_cq_phiq_sep}
% \begin{align}
% \label{eqn:mml_1d_cq_dot} \dot{c}_q(t) &= -\frac{1}{2\, \tau_p}\, c_q(t) + \Re\left[f_q(t)\right] , \nd \\
% \label{eqn:mml_1d_phiq_dot} \dot{\phi}_q(t) &= \delta \omega_q +  \frac{\Im\left[f_q(t)\right]}{c_q(t)},
% \end{align}
% \end{subequations}
%Our goal is to solve \eqn{mml_edot} in the (approximately) steady state for each amplitude $E_q(t)$. In this case, we should obtain both $\dot{c}_q(t) \approx 0$ and a small --- and ideally constant --- value of $\dot{\phi}_q(t)$. In \sct{laser_statics_1d_approx}, we found that $\Im[f_q(t)] = \Omega_q\, \Re[f_q(t)]$ in the single-mode case, and
In \sct{laser_statics_1d_approx}, we found that $\Im[f_0]/\Re[f_0] = \Im[\mathcal{L}_0]/\Re[\mathcal{L}_0]$, where $\mathcal{L}_0$ is the lineshape function, and we assume that this condition is approximately valid in the multimode case. That is,
 \begin{equation}
\frac{\Im[f_q]}{\Re[f_q]} \approx \frac{\Im[\mathcal{L}_q]}{\Re[\mathcal{L}_q]} = \Omega_q\, ,
 \end{equation}
where in the Lorentzian case
 \begin{equation} \label{eqn:mml_lmc_q_def}
\mathcal{L}_q \equiv \frac{1}{1 - i\, \Omega_q}\, .
 \end{equation}
%We note that in single-frequency laser theory, using \eqn{mml_1d_fq} to obtain $f_q(t)$ for the single mode $q$, we find
% \begin{equation}
%f_q(t) = \frac{\left(1 + \Omega_q^2\right) g_q}{1 + \Omega_q^2 + 2 c_q^2(t)}\, c_q(t) ,
% \end{equation}
%so that $\Im\left[f_q(t)\right] = 0$.
%In steady-state, then, $c_q(t) \approx 2\, \tau_p\, \Re[f_q(t)]$, and \eqn{mml_1d_phiq_dot} becomes
% \begin{equation} \label{eqn:mml_1d_phiq_dot_approx}
%\dot{\phi}_q(t) \approx \delta \omega_q + \frac{\Omega_q}{2\, \tau_p}\, .
% \end{equation}
Therefore, choosing $\delta \omega_q \equiv -\Omega_q/2\, \tau_p$ should give $\dot{\phi}_q(t) \approx 0$ when $\dot{A}_q(t) = 0$. If we assume that $\omega_0 = \omega_{a b}$, then
 \begin{equation} \label{eqn:mml_1d_omega_q_def}
\Omega_q = \Delta \omega_q\, \tau_\perp = \left(2 q \pi + \delta \omega_q\right) \tau_\perp\, ,
 \end{equation}
and we can solve for $\delta \omega_q$ to obtain
 \begin{subequations}\label{eqn:mml_1d_freq_pull}
 \begin{align}
 \delta \omega_q &= -\frac{\tau_\perp}{2\, \tau_p}\, \frac{2 q \pi}{1 + \tau_\perp/2\, \tau_p}\, , \nd \\
 \Delta \omega_q &= \frac{2 q \pi}{1 + \tau_\perp/2\, \tau_p}\, .
 \end{align}
 \end{subequations}
Finally, substituting $\delta \omega_q = -\Omega_q/2\, \tau_p$ into \eqn{mml_edot_temp} and simplifying, we find
 \begin{equation} \label{eqn:mml_1d_deq_dt_fp}
\dot{E}_q(t) = -\frac{1}{2 \tau_p} \left( 1 + i\, \Omega_q \right) E_q(t) + F_q(t) .
 \end{equation}

 \subsection{Dispersion\label{sct:laser_dynamics_1d_mml_frq_dis}}

In \sct{laser_amp_1d_pdes}, we derived the normalized wave equation in the time domain given by \eqn{cw_sml_etz_scaled} for a particular transverse mode of the electromagnetic field. Our goal here is to update \eqn{mml_edot_temp} to include the effects of dispersion. The relevant term in \eqn{cw_sml_etz_scaled} is
\begin{equation*}
    i\, \sum_{l = 2}^\infty \frac{D_l\wn}{l!} \left(i\, \frac{\partial}{\partial t}\right)^l E^\pm\zt\, ,
\end{equation*}
where $D_l\wn$ is given by \eqn{cw_sml_disp_coeff}. Let's consider the unidirectional case --- the result for the standing-wave laser will be the same --- and use the one-dimensional single-mode expansion of the slowly-varying complex field envelope function given by \eqn{laser_resonator_1d_ezt_expansion}. We find
\begin{equation}
    \begin{split}
        i\, \sum_{l = 2}^\infty \frac{D_l\wn}{l!} \left(i\, \frac{\partial}{\partial t}\right)^l E\zt &= i\, \sum_{l = 2}^\infty \frac{D_l\wn}{l!} \left(i\, \frac{\partial}{\partial t}\right)^l \sum_p u_p\z\, e^{-i\, \Delta \omega_p\, t}\, E_p(t) \\
        &= i\, \sum_{l p} i^l\, \frac{D_l\wn}{l!}\, u_p\z\, e^{-i\, \Delta \omega_p\, t}\, \sum_{j = 0}^{l} \binom{l}{j}\, (-i\, \Delta \omega_p)^j\, \frac{\partial^{l - j}}{\partial t^{l - j}}\, E_p(t)\, . 
    \end{split}
\end{equation}

If we keep only those terms proportional to $E_p(t)$ and $\dot{E}_p(t)$, then
\begin{equation*}
    \sum_{j = 0}^{l} \binom{l}{j}\, (-i\, \Delta \omega_p)^j\, \frac{\partial^{l - j}}{\partial t^{l - j}}\, E_p(t) \approx (-i\, \Delta \omega_p)^l\, E_p(t) + l\, (-i\, \Delta \omega_p)^{l - 1}\, \dot{E}_p(t)\, ,
\end{equation*}
and then
\begin{equation} \label{eqn:mml_1d_disp_cont}
    i\, \sum_{l = 2}^\infty \frac{D_l\wn}{l!} \left(i\, \frac{\partial}{\partial t}\right)^l E\zt = i\, \sum_p u_p\z\, e^{-i\, \Delta \omega_p\, t} \left[ \delta D_p\wn\, E_p(t) + i\, \delta \tau_p\wn\, \dot{E}_p(t) \right]\, ,
\end{equation}
where
\begin{subequations} \label{eqn:mml_1d_delta_dt_q_def}
    \begin{align}
        \label{eqn:mml_1d_delta_d_q_def}
        \delta D_p\wn &\equiv \sum_{l = 2}^\infty \frac{(2 p \pi)^l}{l!}\, D_l\wn\, , \\
        \label{eqn:mml_1d_delta_tau_q_def}
        \delta \tau_p\wn &\equiv \sum_{l = 2}^\infty \frac{(2 p \pi)^{l - 1}}{(l - 1)!}\, D_l\wn\, ,
    \end{align}
\end{subequations}
and we've used \eqn{mml_1d_delta_w_q_def} to make the approximation $\Delta \omega_p \approx 2 p \pi$.

% In the one-dimensional case, after applying the normalization procedure outlined in \sct{laser_amp_1d_pdes}, this wave equation becomes
% \begin{equation} \label{eqn:mml_1d_weq_w_norm}
%     \pm \frac{\partial}{\partial z} E^\pm\zw - i\, \omega\, E^\pm\zw - i\, \mathcal{D}(\omega_0, \omega)\, E^\pm\zw + \half \alpha\wn\, E^\pm\zw = F^\pm\zw\, ,
% \end{equation}
% where, using \eqn{idm_dispersion_def}, the dispersion is now given by
%  \begin{equation} \label{eqn:mml_1d_disp_def}
% \mathcal{D}(\omega_0, \omega) = \sum_{m = 2}^\infty \frac{\omega^m}{m!}\, D_m\wn
%  \end{equation}
% and
%  \begin{equation} \label{eqn:mml_1d_disp_coeff_redeff}
% D_m\wn = \frac{L}{\tau_g^m} \frac{d^m}{d \omega_0^m} \Re\left[\beta\wn\right]\, .
%  \end{equation}
% Recall that $z$ is expressed in units of $L$ (the round-trip physical length of the laser resonator), and $\alpha\wn$ in terms of $L^{-1}$. Similarly, $t$ has units of $\tau_g$ (the group round-trip propagation time), and $\omega$ has units of $\tau_g^{-1}$.

% We'll apply the same approach we used in the beginning of \sct{laser_dynamics_1d_mml} to build an expression for the time derivative of the electric field coefficient $E_q(t)$. We'll focus on the case of the unidirectional ring laser here, but a similar analysis of a standing-wave laser will yield the same result. Taking the Fourier transform of \eqn{mml_e_field_1d_t}, and applying the Fourier Shift Theorem discussed in \sct{math_prelim_fourier_transforms}, yields
%  \begin{equation} \label{eqn:mml_1d_e_field_w}
% E\zw = \sum_{q = -\infty}^\infty u_{q}\z\, E_{q}\left(\omega - \Delta \omega_q\right)\, ,
%  \end{equation}
% where, as usual, Fourier transform pairs are distinguished by their arguments. Let's define the small angular frequency $\nu \equiv \omega - \Delta \omega_q$. To first order in $\nu$ (the slowly-varying envelope approximation in the frequency domain),
%  \begin{equation} \label{eqn:mml_1d_eq_svea_w}
%  \begin{split}
% \omega^m\, E_q\left(\omega - \Delta \omega_q\right) &= (\Delta \omega_q + \nu)^m\, E_q(\nu) \\
% &\approx (2 q \pi)^m\, E_q(\nu) + m (2 q \pi)^{m - 1}\, \nu\, E_q(\nu)\, ,
%  \end{split}
%  \end{equation}
% where we've used \eqn{mml_1d_delta_w_q_def} to make the approximation $\Delta \omega_q \approx 2 q \pi$ in the exponentiated coefficients. Substituting \eqn{mml_1d_e_field_w} and \eqn{mml_1d_eq_svea_w} into \eqn{mml_1d_weq_w_norm}, and then applying \eqn{laser_resonator_1d_u_hlde} in the form
%  \begin{equation} \label{eqn:mml_1d_u_hlde}
% \ddz u_q\z = i\, \left(\Delta \omega_q - \delta \omega_q\right) u_q\z + \left[\frac{1}{2\, \tau_p} - \half\, \alpha\wn\right] u_q\z\, ,
%  \end{equation}
% we obtain
% % \begin{multline}
% %\sum_l u_l\z \left[ 1 + \delta \tau_l\wn \right] (-i \nu)\, E_l(\nu) = \\ \sum_l u_l\z \left\{-\frac{1}{2\, \tau_p} + i \left[ \delta \omega_l + \sum_{m = 2}^\infty \frac{(2 l \pi)^m}{m!}\, D_m\wn \right] \right\} E_l(\nu) + F\zw\, ,
% % \end{multline}
%  \begin{multline} \label{eqn:mml_1d_nu_e}
% \sum_l u_l\z \left[ 1 + \delta \tau_l\wn \right] (-i \nu)\, E_l(\nu) = \\ \sum_l u_l\z \left\{-\frac{1}{2 \tau_p} \left( 1 + i\, \Omega_l \right) + i\, \delta D_l\wn \right\} E_l(\nu) + F\zw\, ,
%  \end{multline}
% where we have chosen $\delta \omega_l = -\Omega_l/2\, \tau_p$ to incorporate frequency pulling,

We follow the approach we used to derive \eqn{mml_edot_temp}, and multiply both sides of \eqn{mml_1d_disp_cont} by $v_q\z$ and then integrate the result over the cavity length. Collecting the resulting dispersion terms allow us to obtain the updated field coefficient equation of motion
\begin{equation} \label{eqn:mml_1d_deq_dt_final}
    \dot{E}_q(t) = \frac{1}{1 + \delta \tau_q\wn} \left\{ \left[-\frac{1}{2\, \tau_p} \left( 1 + i\, \Omega_q \right) + i\, \delta D_q\wn\right] E_q(t) + F_q(t) \right\}\, ,
\end{equation}
where $F_q(t)$ is again given by \eqn{mml_fq_sol}. We see two primary effects of dispersion. First, there is an additional frequency shift for each mode that increases (in magnitude) nonlinearly with mode number $q$. Second, the group round-trip time is slightly different for each mode, changing with $q$ by a factor of $1 + \delta \tau_q\wn$.
   
% Using the Fourier Shift Theorem, we note that
%  \begin{equation}
% \int_{-\infty}^{\infty} \frac{d \omega}{2 \pi}\, e^{-i \omega t}\, E_{q}\left(\omega - \Delta \omega_q\right) = e^{i \Delta \omega_q t}\, \int_{-\infty}^{\infty} \frac{d \nu}{2 \pi}\, e^{-i \nu t}\, E_{q}(\nu)\, ,
%  \end{equation}
% and we apply this transform to \eqn{mml_1d_nu_e} to obtain the updated field coefficient equation of motion
% \begin{equation}
%\left[ 1 + \delta \tau_q\wn \right] \dot{E}_q(t) = \\ \left\{-\frac{1}{2\, \tau_p} + i \left[ \delta \omega_q + \sum_{m = 2}^\infty \frac{(2 q \pi)^m}{m!}\, D_m\wn \right] \right\} E_q(t) + F_q(t)\, ,
% \end{equation}
% \begin{equation} \label{eqn:mml_1d_deqdt}
%\left[ 1 + \delta \tau_q\wn \right] \dot{E}_q(t) = \\ \left\{-\frac{1}{2 \tau_p} \left( 1 + i\, \Omega_q \right) + i\, \delta D_q\wn \right\} E_q(t) + F_q(t)\, ,
% \end{equation}

%Following the same procedure leading to \eqn{mml_1d_phiq_dot_approx}, we find
% \begin{equation}%\label{eqn}
%\delta \omega_q + \sum_{m = 2}^\infty \frac{D_m\wn}{m!}\, (2 q \pi)^m = -\frac{\Omega_q}{2\, \tau_p}\, ,
% \end{equation}
%which we can solve using \eqn{mml_1d_omega_q_def} for $\delta \omega_q$. Therefore, the total frequency shift and frequency displacement for mode $q$ due to mode pulling and dispersion is given respectively by
% \begin{align}
%\label{eqn:mml_1d_freq_shift} \delta \omega_q &= -\frac{1}{1 + \tau_\perp/2\, \tau_p} \left[\frac{\tau_\perp}{2\, \tau_p}\, (2 q \pi) + \sum_{m = 2}^\infty \frac{(2 q \pi)^m}{m!}\, D_m\wn \right]\, , \nd \\
%\label{eqn:mml_1d_freq_disp} \Delta \omega_q &= \frac{1}{1 + \tau_\perp/2\, \tau_p} \left[2 q \pi - \sum_{m = 2}^\infty \frac{(2 q \pi)^m}{m!}\, D_m\wn \right]\, .
% \end{align}
% Collecting results, we have updated our field equation of motion to read
%Although it is not obvious from the form of \eqn{mml_1d_deq_dt_final}, as the magnitudes of the dispersion coefficients increase, the primary effect will be to change the phase of the nonlinear coupling driving the evolution of each mode.

%. Applying this result to the field in \eqn{mml_e_field_1d_t}, we have to third order
% \begin{equation}
%\mathcal{L}(t)\, \mathbf{E}\zt = \left[ i\, \beta_1\wn \frac{d}{d t} - \frac{\beta_2\wn}{2} \frac{d^2}{d t^2} - i\, \frac{\beta_3\wn}{6} \frac{d^3}{d t^3} \right] \mathbf{E}\zt
% \end{equation}
%In \eqn{mml_1d_deq_dt}, we have scaled the time by $\tau_s \equiv 2\, \tau_p$ and divided out a common factor of $\beta_1\wn$. For the moment, let's suppress the factor of $\tau_s$, and for convenience scale the time by $\tau_g^\parallel$. Then, multiplying $\mathcal{L}(t)$ through by a factor of $-i \tau_g^\parallel/\beta_1 = -i \tau_g^\parallel v_g = -i L$ (where $L$ is the round-trip physical path length, or twice the physical length of the cavity), we have a scaled dispersion operator and operand given by
%
%
%%keeping only the new second and third order terms, and approximating $\Delta \omega_q t \approx (2 q \pi/\tau_g^\parallel) (\tau_g^\parallel t) = (2 q \pi) t$ in \eqn{mml_e_field_1d_t}
% \begin{equation}%\label{}
%\mathcal{L}^\prime(t)\, E_q(t)\, e^{-i\, 2 q \pi\, t} = \left[ \frac{i}{2}\, D_2\wn\, \frac{d^2}{d t^2} - \frac{1}{6}\, D_3\wn\, \frac{d^3}{d t^3} \right] E_q(t)\, e^{-i\, 2 q \pi\, t} ,
% \end{equation}
%The second and third time derivatives are given by
% \begin{subequations}
% \begin{align}
%\frac{d^2}{d t^2} E_q(t)\, e^{-i\, 2 q \pi\, t} &= \left[ -i\, 2\, (2 q \pi)\, \dot{E}_q(t) - (2 q \pi)^2\, E_q(t) \right] e^{-i\, 2 q \pi\, t} , \nd \\
%\frac{d^3}{d t^3} E_q(t)\, e^{-i\, 2 q \pi\, t} &= \left[ -3\, (2 q \pi)^2\, \dot{E}_q(t) + i\, (2 q \pi)^3\, E_q(t) \right] e^{-i\, 2 q \pi\, t} .
% \end{align}
% \end{subequations}
%Therefore, collecting results, and cancelling the common factor of $\exp\left(-i\, 2 q \pi\, t\right)$, we find
% \begin{equation}
% \begin{split}
%e^{+i\, 2 q \pi\, t}\, \mathcal{L}^\prime(t)\, E_q(t)\, e^{-i\, 2 q \pi\, t} &= \left[ D_2\wn\, (2 q \pi) + \frac{D_3\wn}{2}\, (2 q \pi)^2 \right] \dot{E}_q(t)\\ &-i \left[ \frac{D_2\wn}{2}\, (2 q \pi)^2 + \frac{D_3\wn}{6}\, (2 q \pi)^3 \right] E_q(t),
% \end{split}
% \end{equation}
%

%%%%%%%%%%%%%%%%%%%%%%%%%%%%%%%%%%%%%%%%%%%%%%%%%%%%%%%%%%%%%%%%%%%%%%%%%%%%%%
%
% Subsection file included in section file using \input{}
%
% Assumes that LaTeX2e macros and packages defined in rgb_laser_physics.sty
%   are available
%
%%%%%%%%%%%%%%%%%%%%%%%%%%%%%%%%%%%%%%%%%%%%%%%%%%%%%%%%%%%%%%%%%%%%%%%%%%%%%%
 \section{Injection-Seeded Gain-Switched Lasers\label{sct:laser_dynamics_1d_mml_qsl}}

We consider the ideal four-level laser dynamical equations developed in \sct{laser_amp_1d_pdes}, and we assume that $\gamma_\perp \longrightarrow \infty$, so that $\Omega = 0$. The formal integral of \eqn{cw_sml_ftz_scaled} becomes
 \begin{equation} \label{eqn:qsl_ftzt_formal}
\widetilde{F}\zt = \frac{\gamma_\perp}{2}\, e^{-\gamma_\perp t} \int_{-\infty}^{t} d t^\prime\, e^{\gamma_\perp t^\prime}\, \widetilde{G}\left(z, t^\prime\right) \widetilde{E}\left(z, t^\prime\right) \, .
 \end{equation}
We now apply the strong REA in the same limit, and assume that $|\partial \widetilde{E}\zt / \partial t| \ll \gamma_\perp |\widetilde{E}\zt|$, and $|\partial \widetilde{G}\zt / \partial t| \ll \gamma_\perp |\widetilde{G}\zt|$. In this case, both $\widetilde{E}\left(z, t^\prime\right)$ and $\widetilde{G}\left(z, t^\prime\right)$ can be moved outside of the integral, yielding
 \begin{equation} \label{eqn:qsl_ftzt_rea}
\widetilde{F}\zt = \frac{1}{2}\, \widetilde{G}\zt\, \widetilde{E}\zt \, .
 \end{equation}
Substituting this result into \eqn{cw_sml_gtz_scaled} gives
 \begin{equation} \label{eqn:qsl_dgdt_rea}
\ppt \widetilde{G}\zt = \frac{1}{\tau_\parallel} \left[ \overline{G}\zt - \widetilde{G}\zt - \widetilde{G}\zt \left| \widetilde{E}\zt\right|^2 \right]\, ,
 \end{equation}

Let's allow the intracavity field to be supplemented by a quantity $\widetilde{J}\zt$ arising from a very weak input $F_1(t)$ injected through the output coupler mirror $\mathcal{M}_1$, as shown in \fig{resonator_1d_smat}. This additional field will contribute to the total macroscopic polarization, so that
 \begin{equation} \label{eqn:qsl_ftzt_inj}
\widetilde{F}\zt = \frac{1}{2}\, \widetilde{G}\zt\, \left[\widetilde{E}\zt + \widetilde{J}\zt\right]\, .
 \end{equation}
In \sct{laser_resonators_1d_tcm}, we learned how to expand $\widetilde{J}\zt$ as a series of quasi-normal spatial modes in both the unidirectional ring and standing-wave resonator cases. Because the injected field is so weak, we do not need to include it in the saturation term in \eqn{qsl_dgdt_rea}.

 \subsection{Unidirectional Ring Lasers\label{sct:laser_dynamics_1d_mml_qsl_url}}
As discussed in the introduction to \chp{laser_dynamics_1d_mml}, in the case of the URL the rapidly-varying spatial function $\exp(+i k_0 z)$ is common to both $\widetilde{E}\zt$ and $\widetilde{F}\zt$, and can therefore be ignored in \eqn{qsl_ftzt_rea} and \eqn{qsl_dgdt_rea}. Therefore, we find $F_q(t)$ in \eqn{mml_edot_temp} by substituting \eqn{mml_e_field_1d_t} --- and the corresponding expression for $J\zt$ --- into \eqn{qsl_ftzt_inj}, and then the result into \eqn{mml_1d_fq_def_url}. We obtain
 \begin{equation} \label{eqn:qsl_url_fqt}
F_q(t) = \half\, \sum_p e^{i 2 ( q - p ) \pi t}\, G_{q - p}(t) \left[ E_p(t) + J_p(t) \right]\, ,
 \end{equation}
where
 \begin{equation} \label{eqn:qsl_gqp_def}
G_{q - p}(t) \equiv \int_0^1 d z\, v_q\z\, u_p\z\, G\zt = \int_0^1 d z\, e^{-i 2 (q - p) \pi z}\, G\zt\, ,
 \end{equation}
and $J_p(t)$ is given by \eqn{eqt_inj}. \Eqn{qsl_url_fqt} and the slowly-spatially-varying partial differential equation
 \begin{equation} \label{eqn:qsl_dgdt_url}
\ppt G\zt = \frac{1}{\tau_\parallel} \left[ \overline{G}\zt - G\zt - G\zt \left| E\zt\right|^2 \right]\,
 \end{equation}
are the only tools we'll need to solve numerically a wide variety of gain-switched URL problems.

 \subsection{Standing-Wave Lasers\label{sct:laser_dynamics_1d_mml_qsl_swl}}
The calculation of the macroscopic polarization for a multimode standing-wave laser requires that we pay attention to the interference between the counterpropagating fields. We'll follow a strategy similar to that of the continuous-wave case described in \sct{laser_statics_1d_shb}. We begin with an explicit expression for the spatially rapidly-varying polarization of \eqn{qsl_ftzt_inj}, written as
 \begin{multline} \label{eqn:qsl_1d_fzt_swl}
F^+\zt\, e^{+i k_0 z} + F^-\zt\, e^{-i k_0 z} = \\ \half\, \widetilde{G}\zt \left\{\left[E^+\zt + J^+\zt\right] e^{+i k_0 z} + \left[E^-\zt + J^-\zt\right] e^{-i k_0 z}\right\}\, .
 \end{multline}
The envelope functions $F^\pm\zt$, $E^\pm\zt$, and $J^\pm\zt$ are spatially slowly varying, but we will need to average $\widetilde{G}\zt$ over a physical wavelength. Following the procedure outlined in \eqn{ld1d_sw_shb_pzp_full}, we find
 \begin{subequations} \label{eqn:qsl_1d_fpmzt_swl}
 \begin{align}
F^+\zt &= \half\, \mathcal{G}^{[0]}\zt \left[E^+\zt + J^+\zt\right] + \half\, \mathcal{G}^{[-2]}\zt \left[E^-\zt + J^-\zt\right]\, , \nd \\
F^-\zt &= \half\, \mathcal{G}^{[+2]}\zt \left[E^+\zt + J^+\zt\right] + \half\, \mathcal{G}^{[0]}\zt \left[E^-\zt + J^-\zt\right]\, ,
 \end{align}
 \end{subequations}
where
 \begin{equation} \label{eqn:qsl_1d_gnzt}
\mathcal{G}^{[n]}\zt \equiv \frac{k_0}{2 \pi} \int_{z - \pi/k_0}^{z + \pi/k_0} d z^\prime\, e^{+i n k_0 z^\prime}\, \widetilde{G}(z^\prime, t)\, .
 \end{equation}
Substituting \eqn{qsl_1d_fpmzt_swl} into \eqn{mml_1d_fq_def_swl} yields
 \begin{multline}
F_q(t) = \half \sum_p e^{i 2 (q - p) \pi t} \left[E_p(t) + J_p(t)\right]
\int_0^{1/2} d z\, \left[ \mathbf{v}_q\z \dotp \mathbf{u}_p\z\, \mathcal{G}^{[0]}\zt \right. \\
\left. + v_q^+\z\, u_p^-\z\, \mathcal{G}^{[-2]}\zt + v_q^-\z\, u_p^+\z\, \mathcal{G}^{[+2]}\zt \right]\, .
 \end{multline}

We'll need to construct partial differential equations for $\mathcal{G}^{[0]}\zt$ and $\mathcal{G}^{[\pm 2]}\zt$. We expand $|\widetilde{E}\zt|^2$ as
 \begin{equation}
\left|\widetilde{E}\zt\right|^2 = \left|E^{+}\zt\right|^2 +  \left|E^{-}\zt\right|^2 + E^{+}\zt\, E^{- \ast}\zt\, e^{+i 2 k_0 z} + E^{-}\zt\, E^{+ \ast}\zt\, e^{-i 2 k_0 z}\, ,
 \end{equation}
substitute this expression into \eqn{qsl_dgdt_rea}, and then apply the average specified by \eqn{qsl_1d_gnzt} to obtain
 \begin{equation} \label{eqn:qsl_dgndt_swl}
 \begin{split}
\ppt \mathcal{G}^{[n]}\zt &= \frac{1}{\tau_\parallel} \left\{ \delta_{n, 0}\, \overline{G}\zt - \mathcal{G}^{[n]}\zt - \mathcal{G}^{[n]}\zt \left[ \left|E^{+}\zt\right|^2 +  \left|E^{-}\zt\right|^2 \right] \right. \\
&\qquad \left. -~\mathcal{G}^{[n + 2]}\zt\, E^{+}\zt\, E^{- \ast}\zt - \mathcal{G}^{[n - 2]}\zt\, E^{-}\zt\, E^{+ \ast}\zt \right\}\, .
 \end{split}
 \end{equation}
If the pump has a short duration compared to $\tau_\parallel$, then rapid temporal oscillations in $E^{\pm}\zt\, E^{\mp \ast}\zt$ will diminish the contributions of higher-order spatial averages and allow us to neglect $\mathcal{G}^{[\pm 4]}\zt$. 
%%%%%%%%%%%%%%%%%%%%%%%%%%%%%%%%%%%%%%%%%%%%%%%%%%%%%%%%%%%%%%%%%%%%%%%%%%%%%%
%
% Subsection file included in section file using \input{}
%
% Assumes that LaTeX2e macros and packages defined in rgb_laser_physics.sty
%   are available
%
%%%%%%%%%%%%%%%%%%%%%%%%%%%%%%%%%%%%%%%%%%%%%%%%%%%%%%%%%%%%%%%%%%%%%%%%%%%%%%
 \section{Passively Mode-Locked Lasers\label{sct:laser_dynamics_1d_mml_mll}}

% \subsection{New and Busted: Mode-Locked Lasers}
% \begin{subequations}
%     \begin{align}
%         \widetilde{E}\zt &= \sum_{m n} E_{m n}\, \widetilde{u}_m(z)\, e^{-i\, \Delta \omega_n\, t}\, , \nd \\
%         \widetilde{F}\zt &= \sum_{l q} F_{l q}\, \widetilde{u}_l\z\, e^{-i\, \Delta \omega_q\, t}\, .
%     \end{align}
% \end{subequations}
% where $\Delta \omega_n \equiv 2\, n\, \pi$, and
% \begin{subequations}
%     \begin{align}
%         E_{m n} &\equiv \int_{z_\mathrm{min}}^{z_\mathrm{max}} d z\, \int_{-\half}^{+\half}\, d t\, \widetilde{v}_m(z)\, e^{i\, \omega_n\, t}\, E\zt\, , \nd \\
%         F_{l q} &\equiv \int_{z_\mathrm{min}}^{z_\mathrm{max}} d z\, \int_{-\half}^{+\half}\, \widetilde{v}_l\z\, e^{i\, \Delta \omega_q\, t}\, \widetilde{F}\zt\, .
%     \end{align}
% \end{subequations}

% \begin{equation}
%     2\, \Re\left[\widetilde{E}^\ast\zt\, \widetilde{F}\zt\right] = \sum_{i j m n} \widetilde{u}_i\z\, \widetilde{u}^\ast_{j}\z\, e^{-i\, \Delta \omega_{m - n}\, t} \left( E_{i m}\, F^\ast_{j n} + F_{i m}\, E^\ast_{j n} \right)\, .
% \end{equation}

% \begin{equation}
%     \widetilde{G}\zt = \Gnz - \sum_{i j m n} \widetilde{u}_i\z\, \widetilde{u}^\ast_{j}\z\, e^{-i\, \Delta \omega_{m - n}\, t}\, \mathcal{C}_{m - n} \left( E_{i m}\, F^\ast_{j n} + F_{i m}\, E^\ast_{j n} \right)\, ,
% \end{equation}
% where
% \begin{equation}
%     \mathcal{C}_q \equiv \left(1 - i\, \Delta \omega_q\, \tau_\parallel\right)^{-1}\, .
% \end{equation}

% \begin{multline}
%     \widetilde{E}\zt\, \widetilde{G}\zt = \Gnz \sum_{k p} \widetilde{u}_k\z\, e^{-i\, \Delta \omega_p\, t} E_{l p} \\
%     - \sum_{i j k m n p} \widetilde{u}_i\z\, \widetilde{u}^\ast_j\z\, \widetilde{u}_k\z\, e^{-i\, \Delta \omega_{m - n + p}\, t}\, \mathcal{C}_{m - n} \left( E_{i m}\, F^\ast_{j n} + F_{i m}\, E^\ast_{j n} \right) E_{k p}\, .
% \end{multline}

% \begin{multline}
%     \widetilde{F}\zt = \half\, \Gnz \sum_{k p} \widetilde{u}_k\z\, e^{-i\, \Delta \omega_p\, t}\, \mathcal{L}\left(\Omega_p\right)\, E_{k p} \\
%     - \half \sum_{i j k m n p} \widetilde{u}_i\z\, \widetilde{u}^\ast_j\z\, \widetilde{u}_k\z\, e^{-i\, (\Delta \omega_{m - n + p})\, t}\, \mathcal{L}\left(\Omega_{m - n + p}\right)\,  \mathcal{C}_{m - n} \left( E_{i m}\, F^\ast_{j n} + F_{i m}\, E^\ast_{j n} \right) E_{k p}\, .
% \end{multline}

% Perform the time integral to obtain
% \begin{equation}
%     \begin{split}
%         F_{l q} &= \half\, \mathcal{L}\left(\Omega_q\right) \sum_k\, E_{k q} \int d z\, \widetilde{v}_l\z\, \widetilde{u}_k\z\, \Gnz \\
%         &- \half\, \mathcal{L}\left(\Omega_q\right) \sum_{i j k m n} \mathcal{C}_{m - n} \left( E_{i m}\, F^\ast_{j n} + F_{i m}\, E^\ast_{j n} \right) E_{k, m - n + q}\, \int d z\, \widetilde{v}_l\z\, \widetilde{u}_i\z\, \widetilde{u}^\ast_j\z\, \widetilde{u}_k\z \\
%         &\equiv \half\, \mathcal{L}\left(\Omega_q\right) \left[ \sum_k\, \overline{G}_{k l}\, E_{k q} - \sum_{i j k m n} \kappa_{i j k l}\, \mathcal{C}_{m - n} \left( E_{i m}\, F^\ast_{j n} + F_{i m}\, E^\ast_{j n} \right) E_{k, m - n + q}\right]\, ,
%     \end{split}
% \end{equation}
% where
% \begin{align}
%     \overline{G}_{k l} &\equiv \int d z\, \widetilde{v}_l\z\, \widetilde{u}_k\z\, \Gnz\, , \nd \\
%     \kappa_{i j k l} &\equiv \int d z\, \widetilde{v}_l\z\, \widetilde{u}_i\z\, \widetilde{u}^\ast_j\z\, \widetilde{u}_k\z\, .
% \end{align}

% \begin{equation}
%     \begin{split}
%         F_{q p} &= \half\, \mathcal{L}\left(\Omega_q\right) \sum_k\, E_{q k} \int d z\, \widetilde{u}_k\z\, \widetilde{v}_p\z\, \Gnz \\
%         &- \half\, \mathcal{L}\left(\Omega_q\right) \sum_{i j k m n} \mathcal{C}_{m - n} \left( E_{m j}\, F^\ast_{n k} + F_{m j}\, E^\ast_{n k} \right) E_{m - n + q,\, l}\, \int d z\, \widetilde{u}_j\z\, \widetilde{u}^\ast_k\z\, \widetilde{u}_l\z\, \widetilde{v}_p\z \\
%         &\equiv \half\, \mathcal{L}\left(\Omega_q\right) \left[ \sum_k\, E_{q k}\, \overline{G}_{k p} - \sum_{j k l m n} \mathcal{C}_{m - n}\, \kappa_{j k l p} \left( E_{m j}\, F^\ast_{n k} + F_{m j}\, E^\ast_{n k} \right) E_{m - n + q,\, l}\right]\, ,
%     \end{split}
% \end{equation}
% where
% \begin{align}
%     \overline{G}_{k p} &\equiv \int d z\, \widetilde{u}_k\z\, \widetilde{v}_p\z\, \Gnz\, , \nd \\
%     \kappa_{j k l p} &\equiv \int d z\, \widetilde{u}_j\z\, \widetilde{u}^\ast_k\z\, \widetilde{u}_l\z\, \widetilde{v}_p\z\, .
% \end{align}

% \subsubsection{Unidirectional Ring Lasers}
% In the case of a unidirectional ring laser, the rapidly-varying quasi-normal spatial modes are given by
% \begin{subequations} %\label{eqn:laser_resonator_1d_uv}
%     \begin{align}
%        \widetilde{u}_q\z &\equiv \mathcal{C}\, e^{+\left[ i 2 q \pi + \ln(1/\sqrt{R}) \right] z}\, e^{i\, k_0\, z} , \nd \\ %\label{eqn:laser_resonator_1d_u} \\
%        \widetilde{v}_q\z &\equiv \frac{1}{\mathcal{C}}\, e^{-\left[ i 2 q \pi + \ln(1/\sqrt{R}) \right] z}\, e^{-i\, k_0\, z}\, , %\label{eqn:laser_resonator_1d_v}
%     \end{align}
% \end{subequations}
% where $\mathcal{C}$ is given by \eqn{laser_resonator_1d_u_norm_url}. Therefore,
% \begin{equation} %\label{eqn:mll_url_zqp_spec}
%     \overline{G}_{k l} = \int_{z_1}^{z_2} d z\, e^{i\, 2 (k - l)\, \pi\, z} \Gnz\, ,
% \end{equation}
% and we see that $\overline{G}_{k l}$ is the Fourier series coefficient of order $k - l$ for $\Gnz$ in the resonator. \red{In practice, we can use this representation as a guide to the range of values of $l$ that we need to include to provide a numerically accurate computation of the intracavity gain.} Suppose that $\Gnz = \Gnb/(z_2 - z_1)$ for $0 < z_1 \le z \le z_2 < 1$, and is zero otherwise. In this (common) special case, when $k \ne l$ we have
% \begin{equation} %\label{eqn:mll_url_zqp_spec}
%     \overline{G}_{k l} = \frac{\Gnb}{z_2 - z_1}\, \int_{z_1}^{z_2} d z\, e^{i\, 2 (k - l)\, \pi\, z} = -i\, \Gnb\, \frac{\exp[i\, 2 \left(k - l\right) \pi\, z_2] - \exp[i\, 2 \left(k - l\right) \pi\, z_1]}{2 \left(k - l\right) \pi \left(z_2 - z_1\right)}\, ,
% \end{equation}
% and $\overline{G}_{l l} = \Gnb$. Note that when $\{z_1, z_2\} \longrightarrow \{0, 1\}$, $\overline{G}_{k l} \longrightarrow \delta_{k l}\, \Gnb$.

% \begin{equation}
%     \kappa_{i j k l} = \mathcal{C}^2 \int_0^1 d z\, e^{\left[i\, 2 \left(i - j + k - l\right) \pi + \ln(1/R)\right] z} = \Delta_{i - j + k - l}(R)\, .
% \end{equation}

% \subsection{Old Hotness: Mode-Locked Lasers}
As in the case of the $Q$-switched laser discussed in \sct{laser_dynamics_1d_mml_qsl}, intermodal coupling through nonlinearities in the macroscopic polarization $\widetilde{F}\zt$ add dynamics to the gain and saturation of each mode that can lead to novel dynamical behavior. In a mode-locked laser, the amplitude and phases of the longitudinal modes are fixed in such a way that the output of the laser has particularly desirable properties, such as very short pulses or very stable (quasi-continuous-wave) behavior. We can understand this behavior through formal expansions of $\widetilde{F}\zt$ and $\widetilde{G}\zt$ in the Fourier frequency domain, but we must relax the rate-equation approximation that led to \eqn{qsl_ftzt_rea} to allow fluctuations in the polarization and gain that occur at integer multiples of the cavity free-spectral range $2 \pi/\tau_g$.

In mode-locked lasers, the gain is generally constant in time, and the dynamic fields arise from the nonlinear coupling of the longitudinal modes through the macroscopic polarization. However, the coefficients of both the electric field and macroscopic polarization vary slowly over the round-trip propagation time $\tau_g$, and the time derivatives of both $E_q(t)$ and $F_q(t)$ tend to zero as the intracavity laser amplifier reaches equilibrium. Therefore, we begin with the formal solution of \eqn{cw_sml_gtz_scaled}, finding
\begin{equation}
  G_{q p}(t) = \overline{G}_{q p} - \sum_{m n} e^{-i\, \Delta \omega_{m - n}\, t} \kappa_{q p m n}\, \mathcal{C}_{m n} \left( E_m\, F_n^\ast + F_m\, E_n^\ast \right)\, ,
\end{equation}
where
\begin{equation}
  \mathcal{C}_{m n} \equiv \left(1 - i\, \Delta \omega_{m - n}\, \tau_\parallel\right)^{-1}\, .
\end{equation}
Note that we have assumed that the pump $\Gn\zt$ is constant in time, so that its Fourier coefficients $\overline{G}_{q p}$ are also constant in time. Substituting this expression into \eqn{mml_fq_sol}, we find
\begin{equation}
    \label{eqn:mml_fqt_mll}
    \begin{split}
        F_q &= \half\, \Lq \sum_p e^{i\, \Delta \omega_{q - p}\, t}\, \overline{G}_{q p}\, E_p \\
        &- \half\, \Lq \sum_{m n p} e^{i\, \Delta \omega_{q - p - m + n}\, t}\, \kappa_{q p m n}\, \mathcal{C}_{m n} \left( E_m\, F_n^\ast + F_m\, E_n^\ast \right) E_p\, ,
    \end{split}
\end{equation}
Both $E_q(t)$ and $F_q(t)$ vary slowly in time compared to the rapid oscillations of the exponential functions in \eqn{mml_fqt_mll}, so terms with nonzero frequencies will average out. Therefore, only terms with $p = q$ in the first sum and $p = q - m + n$ in the second sum will contribute significantly to the value of $F_q$. Thus, we obtain the simplified expression
\begin{equation}
    \label{eqn:mml_fq_mml}
    F_q = \half\, \Lq \Gnb\, E_q
    - \half\, \Lq \sum_{m n} \kappa_{q m n}\, \mathcal{C}_{m n} \left( E_m\, F_n^\ast + F_m\, E_n^\ast \right) E_{q - m + n}\, ,
\end{equation}
where the contracted three-index spatial coupling coefficient is given by
\begin{equation}
    \kappa_{q m n} = \begin{cases}
      \Delta_0(R) & \text{(URL)}\\
      \Delta_0(R_1\, R_2) + \Delta_{2(m - n)}(R_1\, R_2) & \text{(SWL)} \\
      \Delta_0(R_1\, R_2) + \Delta_{2(m - n)}(R_1\, R_2) + \Delta_{2(q - m)}(R_1\, R_2) & \text{(SHB)}
    \end{cases}
\end{equation}

Referring to \eqn{mml_1d_deq_dt_final}, as the mode-locked laser oscillator reaches equilibrium, $F_q(t)$ becomes a constant and $\dot{E}_q(t) \longrightarrow 0$. In this case, we find that $F_q$ must also satisfy the expression
\begin{equation}
    F_q = R_q\, E_q\, ,
\end{equation}
where
\begin{equation}
    R_q \equiv \frac{1}{2\, \tau_\lambda} \left(1 + i\, \Omega_q\right) - i\, \delta D_q\, .
\end{equation}
Therefore,
% \begin{equation}
%     \sum_p e^{i\, \Delta \omega_{q - p}\, t}\, \overline{G}_{q p}\, E_p - 2\, \mathcal{L}^{-1}\bigl(\Omega_q\bigr)\, B_q\, E_q
%     = \sum_{m n p} e^{i\, \Delta \omega_{q - p - m + n}\, t}\, \kappa_{q p m n}\, C_{m - n} \left( B_m + B_n^\ast \right) E_m\, E_n^\ast\, E_p
% \end{equation}
\begin{equation}
    \left[ \Gnb - 2\, R_q  / \Lq\right] E_q
    = \sum_{m n} \kappa_{q m n}\, B_{m n}\, C_{m n}\, E_m\, E_n^\ast\, E_{q - m + n}\, ,
\end{equation}
where
\begin{equation}
    B_{m n} \equiv R_m + R_n^\ast = \frac{1}{\tau_\lambda} + i\, \frac{\Omega_m - \Omega_n}{2\, \tau_\lambda} - i \left(\delta D_m - \delta D_n\right)\, .
\end{equation}
\begin{equation}
    B_{m n} \equiv R_m + R_n^\ast = \frac{1}{\tau_\lambda} + i\, \frac{\tau_\perp}{2\, \tau_\lambda}\, \left(\omega_m - \omega_n\right) - i \left(\delta D_m - \delta D_n\right)\, .
\end{equation}

% This expression must be true at all times, so for convenience, let's use it to calculate $E_q$ at $t = 0$. We obtain
% \begin{equation}
%     \sum_p \overline{G}_{q p}\, E_p - 2\, \mathcal{L}^{-1}\bigl(\Omega_q\bigr)\, B_q\, E_q = \sum_{m n p} \kappa_{q p m n}\, C_{m n} \left( B_m + B_n^\ast \right) E_m\, E_n^\ast\, E_p
% \end{equation}


% As a general representation of the spatially rapidly-varying fields in both unidirectional ring and standing-wave resonator configurations, we follow \sct{laser_resonators_1d_swl} and represent the electric field amplitude function as
%  \begin{equation} \label{eqn:mml_e_1d_t_rv}
% \widetilde{E}\zt \equiv \sum_{p = -\infty}^\infty \widetilde{u}_p\z\, e^{-i\, \Delta \omega_p\, t}\, E_p(t)\, .
%  \end{equation}
% In the unidirectional ring case,
%  \begin{equation}
% \widetilde{u}_q\z = u^{+}_q\z\, e^{+i k_0 z}\, ,
%  \end{equation}
% where $k_0$ is the propagation constant associated with the carrier frequency $\omega_0$. For a standing-wave resonator,
%  \begin{equation}
% \widetilde{u}_q\z = u^{+}_q\z\, e^{+i k_0 z} + u^{-}_q\z\, e^{-i k_0 z}\, .
%  \end{equation}
% We use a similar approach to the expansion of the amplitude of the macroscopic polarization, with a subtle difference:
%  \begin{equation} \label{eqn:mml_f_1d_t_rv}
% \widetilde{F}\zt \equiv \sum_{q = -\infty}^\infty \widetilde{w}_q\z\, e^{-i\, \Delta \omega_q\, t}\, F_q(t)\, .
%  \end{equation}
% Here $\widetilde{w}_q(z)$ represents the spatial dependence of each frequency component of the macroscopic polarization. If we look carefully at \eqn{cw_sml_gtz_scaled}, we see that to first order in $\widetilde{E}\zt$, the gain is given by the function $\overline{G}\zt$ that describes the pump. Suppose that the pump is constant in time, so that $\overline{G}\zt \equiv \overline{G}\z$. Then we can write the pump function as
% \begin{equation} \label{eqn:mml_1d_pump_sep}
%   \overline{G}\z \equiv \Gn\, \mathcal{Z}\z\, ,
% \end{equation}
% \begin{equation}
%   G_0\z \equiv \Gn\, \mathcal{Z}\z\, ,
% \end{equation}
% where $\mathcal{Z}\z$ is a real function of $z$ normalized such that $\int_0^1 d z\, \mathcal{Z}\z = 1$ in the URL case or $2 \int_0^{1/2} d z\, \mathcal{Z}\z = 1$ in the SWL case, and $\Gn$ represents the \emph{small-signal (unsaturated) round-trip intensity gain}. Then a comparison of both sides of \eqn{cw_sml_ftz_scaled} suggests that
%  \begin{equation}
% \widetilde{w}_q\z \approx \mathcal{Z}\z\, \widetilde{u}_q\z\, .
%  \end{equation}
% In the following analysis, we'll also make a simplifying assumption: as discussed in \sct{laser_dynamics_1d_mml_frq}, $\delta \omega_q$ and, therefore, $\Delta \omega_q$ are linear in $q$.

% In this section, our primary tools will be the formal solutions of \eqn{cw_sml_ftz_scaled} and \eqn{cw_sml_gtz_scaled}, obtained through Fourier transform expansions. For example, consider the ordinary differential equation
%  \begin{equation}
% \ddt y(t) = -\frac{1}{\tau} \left[y(t) + s(t)\right]\, ,
%  \end{equation}
% for some function $s(t)$. Applying the Fourier Transform and using \eqn{fourier_freq} and \eqn{fourier_shift_thm}, we find
%  \begin{equation}
% y(\omega) = \frac{s(\omega)}{1 - i\, \omega\, \tau}\, .
%  \end{equation}
% %We obtain
% % \begin{equation}
% %A(t) = \left(1 + \tau\, \ddt\right)^{-1} B(t)\, ,
% % \end{equation}
% % \begin{equation}
% %\left(1 + \tau\, \ddt\right)^{-1} \equiv \sum_{l = 0}^{\infty} \left( -\tau\, \ddt \right)^l \, .
% % \end{equation}
% Suppose that $s(t)$ can be written as
%  \begin{equation}
% s(t) = \sum_q e^{-i\, \Delta \omega_q\, t}\, s_q(t)\, ,
%  \end{equation}
% giving the transform
%  \begin{equation}
% s(\omega) = \sum_q s_q(\omega - \Delta \omega_q)\, .
%  \end{equation}
% %If we define $\nu_q \equiv \omega - \Delta \omega_q$, and
% % \begin{equation}
% %c_q \equiv \frac{1}{1 - i\, \Delta \omega_q\, \tau}\, .
% % \end{equation}
% If we define $\nu_q \equiv \omega - \Delta \omega_q$, then we can rewrite $y(\omega)$ as
%  \begin{equation}
% y(\omega) = \sum_q \frac{1}{1 - i\, \Delta \omega_q\, \tau - i\, \nu_q\, \tau}\, s_q(\nu_q)\, .
%  \end{equation}
% With our usual casual indifference to mathematical rigor, we expand the denominator of this equation as a power series, and then apply the inverse Fourier transform over the frequency $\omega$ to each term separately. We obtain
%  \begin{equation}
% y(t) = \sum_q e^{-i\, \Delta \omega_q\, t}\, \left(1 - i\, \Delta \omega_q\, \tau + \tau\, \ddt\right)^{-1} s_q(t)\, ,
%  \end{equation}
% where for convenience we have defined the differential operator
% % \begin{equation}% \label{eqn:mll_diff_oper}
% %   \begin{split}
% %     \left(1 - i\, \Delta \omega_q\, \tau + \tau\, \ddt\right)^{-1} &= \sum_{l = 0}^{\infty} \left( i\, \Delta \omega_q\, \tau - \tau\, \ddt \right)^l \\
% %     &= \sum_{l = 0}^{\infty} \sum_{j = 0}^{l} \binom{l}{j} \left( i\, \Delta \omega_q\, \tau\right)^{l - j} (-\tau)^j\, \frac{d^j}{d t^j} \\
% %     &= \sum_{j = 0}^{\infty} \frac{(-\tau)^j}{\left(1 - i\, \Delta \omega_q\, \tau\right)^{j + 1}}\, \frac{d^j}{d t^j}\, .
% %   \end{split}
% % \end{equation}
% \begin{equation} \label{eqn:mll_diff_oper}
%   \begin{split}
%     \left(1 - i\, \Delta \omega_q\, \tau + \tau\, \ddt\right)^{-1} &= \left(1 - i\, \Delta \omega_q\, \tau\right)^{-1} \left(1 + \frac{\tau}{1 - i\, \Delta \omega_q\, \tau}\, \ddt\right)^{-1} \\
%     &= \frac{1}{1 - i\, \Delta \omega_q\, \tau}\, \sum_{l = 0}^{\infty} \left(-\frac{\tau}{1 - i\, \Delta \omega_q\, \tau}\right)^l\, \frac{d^l}{d t^l}\, .
%   \end{split}
% \end{equation}

% Let's apply this technique to solve the evolution equation for $\widetilde{G}\zt$ given by \eqn{cw_sml_gtz_scaled}. Using \eqn{mml_e_1d_t_rv} and \eqn{mml_f_1d_t_rv}, the nonlinear term on the \rhs can be written as
%  \begin{equation*}
%  \begin{split}
%     2 \Re \left[ \widetilde{E}^\ast\zt\, \widetilde{F}\zt \right] &= \widetilde{E}^\ast\zt\, \widetilde{F}\zt + c.c. \\
%     &= \sum_{m n}  e^{-i\, \Delta \omega_{m - n}\, t}\, \widetilde{u}_m\z\, \widetilde{u}_n^\ast\z\, \mathcal{Z}\z \left[ E_m(t)\, F_n^\ast(t) + F_m(t)\, E_n^\ast(t) \right]\, ,
%  \end{split}
%  \end{equation*}
% Therefore, using the Fourier transform expansion described above, we quickly find the formal solution
%  \begin{equation}  \label{eqn:mll_gzt_formal}
%  \begin{split}
% \widetilde{G}\zt &= \Gn\, \mathcal{Z}\z - \sum_{m n}  e^{-i\, \Delta \omega_{m - n}\, t}\, \widetilde{u}_m\z\, \widetilde{u}_n^\ast\z\, \mathcal{Z}\z \\
% &\qquad \times \left(1 - i\, \Delta \omega_{m - n} \, \tau_\parallel + \tau_\parallel\, \ddt\right)^{-1} \left[ E_m(t)\, F_n^\ast(t) + F_m(t)\, E_n^\ast(t) \right]\, .
%  \end{split}
%  \end{equation}
% %where
% % \begin{equation} \label{eqn:mll_1d_c_def}
% %C_q \equiv \frac{1}{1 - i\, \Delta \omega_q\, \tau_\parallel}\, .
% % \end{equation}
% We see that $\widetilde{G}\zt$ is rapidly-varying in space, and oscillates in time at a collection of frequencies that are approximately integer multiples of the free spectral range of the resonator. The Fourier transform approach to \eqn{cw_sml_ftz_scaled} is equally straightforward, giving the formal solution
%  \begin{equation}  \label{eqn:mll_fzt_formal}
%  \begin{split}
% \widetilde{F}\zt &= \frac{\Gn}{2}\, \sum_p e^{-i\, \Delta \omega_p\, t}\, \widetilde{u}_p\z\, \mathcal{Z}\z \left(1 - i\, \Omega_p + \tau_\perp\, \ddt\right)^{-1} E_p(t) \\
% &\quad - \half\, \sum_{m n p}  e^{-i\, \Delta \omega_{m - n + p}\, t}\, \widetilde{u}_m\z\, \widetilde{u}_n^\ast\z\, \widetilde{u}_p\z\, \mathcal{Z}\z \left(1 - i\, \Omega_{m - n + p} + \tau_\perp\, \ddt\right)^{-1} E_p(t) \\
% &\qquad \times \left(1 - i\, \Delta \omega_{m - n} \, \tau_\parallel + \tau_\parallel\, \ddt\right)^{-1} \left[ E_m(t)\, F_n^\ast(t) + F_m(t)\, E_n^\ast(t) \right]\, ,
%  \end{split}
%  \end{equation}
% where $\Omega_q \equiv \Omega_0 + \Delta \omega_q\, \tau_\perp$, and $\Omega_0$ is given by \eqn{tls_omega_0_def}.

% \begin{subequations} \label{eqn:mll_fgtzt_formal}
% \begin{align}
% \label{eqn:mll_ftzt_formal} \widetilde{F}\zt &= \frac{\gamma_\perp}{2}\, e^{-\gamma_\perp ( 1 - i\, \Omega_0 ) t} \int_{-\infty}^{t} d t^\prime\, e^{\gamma_\perp ( 1 - i\, \Omega_0 ) t^\prime}\, \widetilde{G}\left(z, t^\prime\right) \widetilde{E}\left(z, t^\prime\right) \, , \nd \\
% \label{eqn:mll_gtzt_formal} \widetilde{G}\zt &= \gamma_\parallel\, e^{-\gamma_\parallel t} \int_{-\infty}^{t} d t^\prime\, e^{\gamma_\parallel t^\prime}\, \left\{ \overline{G}\left(z, t^\prime\right) - 2 \Re \left[ \widetilde{E}^\ast\left(z, t^\prime\right) \widetilde{F}\left(z, t^\prime\right) \right] \right\} \, ,
% \end{align}
% \end{subequations}
%and our goal will be an expression for $\widetilde{F}\zt$ that is accurate to third order in $\widetilde{E}\zt$ \cite{ref:sargent1974lp}. Following the assumptions leading to \eqn{mml_1d_omega_q_def} and \eqn{mml_1d_freq_shift}, we'll take $\Omega_0 = 0$ in the remainder of this discussion.
%
%Suppose that the pump $\overline{G}\zt$ applied to the laser amplifier changes very slowly compared to the upper laser level lifetime $\tau_\parallel = \gamma_\parallel^{-1}$. (In most cases of practical interest, this constraint is equivalent to assuming that the pump is constant in time.) Then, to zeroth-order in the electric field amplitude, \eqn{mll_gtzt_formal} predicts that
% \begin{equation} \label{eqn:mll_gzt0}
%\widetilde{G}^{(0)}\zt = \gamma_\parallel\, e^{-\gamma_\parallel t} \int_{-\infty}^{t} d t^\prime\, e^{\gamma_\parallel t^\prime} \, \overline{G}\left(z, t^\prime\right) \cong \overline{G}\zt\, \gamma_\parallel\, e^{-\gamma_\parallel t} \int_{-\infty}^{t} d t^\prime\, e^{\gamma_\parallel t^\prime} = \overline{G}\zt\, .
% \end{equation}
%Next, we substitute this expression and \eqn{mml_e_field_1d_t} into \eqn{mll_ftzt_formal} to obtain an expression for $\widetilde{F}\zt$ that is accurate to first-order in $E_p(t)$. We find
% \begin{equation}
%\widetilde{F}^{(1)}\zt = \half\, \sum_p  \widetilde{u}_p\z\, \gamma_\perp\, e^{-\gamma_\perp\, t} \int_{-\infty}^{t} d t^\prime\, e^{\gamma_\perp ( 1 - i\, \Omega_p ) t^\prime}\, \overline{G}\left(z, t^\prime\right)\, E_p\left(t^\prime\right)\, ,
% \end{equation}
%where $\Omega_p = \Delta \omega_p\, \tau_\perp = (2 p \pi + \delta \omega_p) / \gamma_\perp$. Let's refine the rate equation approximation in this multimode case to assume that neither $\overline{G}\zt$ nor $E_p(t)$ change significantly during a time duration $\tau_\perp = \gamma_\perp^{-1}$. Moving both $\overline{G}\left(z, t^\prime\right)$ and $E_p\left(t^\prime\right)$ outside the time integral yields
% \begin{equation} \label{eqn:mll_fzt1}
%\widetilde{F}^{(1)}\zt = \frac{\overline{G}\zt}{2}\, \sum_p \frac{e^{-i\, \Delta \omega_p\, t}}{1 - i\, \Omega_p}\, E_p(t)\, \widetilde{u}_p\z\, .
% \end{equation}
%
%Our next assignment is to use \eqn{mml_e_field_1d_t}, \eqn{mll_gtzt_formal} and \eqn{mll_fzt1} to determine $\widetilde{G}^{(2)}\zt$. To second order in $E_q(t)$, the nonlinear term in \eqn{mll_gtzt_formal} becomes
% \begin{equation*}
% \begin{split}
%    2 \Re \left[ \widetilde{E}^\ast\zt\, \widetilde{F}^{(1)}\zt \right] &= \widetilde{E}^\ast\zt\, \widetilde{F}^{(1)}\zt + c.c. \\
%    &\equiv \overline{G}\zt\, \sum_{m n}  e^{-i (\Delta \omega_m - \Delta \omega_n) t}\, B_{m n}\, \widetilde{u}_m\z\, \widetilde{u}_n^\ast\z\, E_m(t)\, E_n^\ast(t)\, ,
% \end{split}
% \end{equation*}
%where
% \begin{equation} \label{eqn:mll_1d_b_def}
%B_{m n} \equiv \half\, \left( \frac{1}{1 - i\, \Omega_m} + \frac{1}{1 + i\, \Omega_n} \right)\, .
% \end{equation}
%Substituting this result into \eqn{mll_gtzt_formal} yields
% \begin{equation}
%\widetilde{G}^{(2)}\zt = -\overline{G}\zt\, \sum_{m n} B_{m n}\, \widetilde{u}_m\z\, \widetilde{u}_n^\ast\z\, \gamma_\parallel\, e^{-\gamma_\parallel t} \int_{-\infty}^{t} d t^\prime\, e^{[\gamma_\parallel - i (\Delta \omega_m - \Delta \omega_n)] t^\prime}\, E_m\left(t^\prime\right)\, E_n^\ast\left(t^\prime\right)\, .
% \end{equation}
%In practice, it may be difficult to claim that $E_q(t)$ will vary slowly relative to the timescale $\tau_\parallel = \gamma_\parallel^{-1}$. However, we can make the much more reasonable assumption that the dynamical variables will not change significantly during the group round-trip time $\tau_g$, so that $e^{- i (\Delta \omega_m - \Delta \omega_n) t}$ varies rapidly compared to $E_q(t)$. In this case, we have
% \begin{equation} \label{eqn:mll_gzt2}
%\widetilde{G}^{(2)}\zt \equiv -\overline{G}\zt\, \sum_{m n} e^{-i (\Delta \omega_m - \Delta \omega_n) t}\, B_{m n}\, C_{m n}\, \widetilde{u}_m\z\, \widetilde{u}_n^\ast\z\, E_m(t)\, E_n^\ast(t)\, .
% \end{equation}
%Finally, substitution of \eqn{mml_e_field_1d_t} and \eqn{mll_gzt2} into \eqn{mll_ftzt_formal}, followed by application of the rate-equation approximation,  yields for the third-order macroscopic polarization
% \begin{equation} \label{eqn:mll_fzt3}
% \begin{split}
%\widetilde{F}^{(3)}\zt &= -\frac{\overline{G}\zt}{2}\, \sum_{p m n}  \frac{e^{-i (\Delta \omega_p + \Delta \omega_m - \Delta \omega_n) t}}{1 - i (\Omega_p + \Omega_m - \Omega_n)}\, B_{m n}\, C_{m n} \\
%&\qquad \times \widetilde{u}_p\z\, \widetilde{u}_m\z\, \widetilde{u}_n^\ast\z\, E_p(t)\, E_m(t)\, E_n^\ast(t)\, .
% \end{split}
% \end{equation}
%The total macroscopic polarization, valid to third order in $E_q(t)$, is given by the sum of \eqn{mll_fzt1} and \eqn{mll_fzt3}.

%  \subsection{Unidirectional Ring Lasers\label{sct:laser_dynamics_1d_mml_mll_url}}
% As discussed above, in the case of the URL, the rapidly-varying spatial function $\exp(+i k_0 z)$ is common to both $\widetilde{E}\zt$ and $\widetilde{F}\zt$, and can therefore be ignored in \eqn{mll_fzt_formal}. If we substitute \eqn{mll_fzt_formal} into \eqn{mml_fq_sol}, we obtain
% % \begin{equation}
% % \begin{split}
% %F_q(t) &\cong \half\, \sum_p \frac{e^{i (\Delta \omega_q - \Delta \omega_p) t}}{1 - i\, \Omega_p}\, \overline{G}_{q - p}(t)\, E_p(t) \\
% %&\qquad - \half\, \sum_{p m n} \frac{e^{i (\Delta \omega_q - \Delta \omega_p - \Delta \omega_m + \Delta \omega_n) t}}{1 - i (\Omega_p + \Omega_m - \Omega_n)}\, B_{m n}\, C_{m n}\, \overline{G}_{q - p, m - n}(t)\, E_p(t)\, E_m(t)\, E_n^\ast(t)\,  ,
% % \end{split}
% % \end{equation}
%  \begin{equation} \label{eqn:mll_url_fqt_g}
%  \begin{split}
% F_q(t) &= \frac{\Gn}{2}\, \sum_p e^{i\, \Delta \omega_{q - p}\, t}\, \mathcal{Z}_{q - p}\, \left(1 - i\, \Omega_p + \tau_\perp\, \ddt\right)^{-1} E_p(t) \\
% &\quad - \half\, \sum_{m n p}  e^{i\, \Delta \omega_{q - m + n - p}\, t}\, \kappa_{q m n p}\, \left(1 - i\, \Omega_{m - n + p} + \tau_\perp\, \ddt\right)^{-1} E_p(t) \\
% &\qquad \times \left(1 - i\, \Delta \omega_{m - n} \, \tau_\parallel + \tau_\parallel\, \ddt\right)^{-1} \left[ E_m(t)\, F_n^\ast(t) + F_m(t)\, E_n^\ast(t) \right]\, ,
%  \end{split}
%  \end{equation}
% where
%  \begin{equation} \label{eqn:mll_url_zqp_def}
% \mathcal{Z}_{q - p} \equiv \int_0^1 d z\, v_q\z\, u_p\z\, \mathcal{Z}\z = \int_0^1 d z\, e^{-i 2 (q - p) \pi z}\, \mathcal{Z}\z
%  \end{equation}
% is essentially a one-dimensional discrete spatial Fourier transform of $\mathcal{Z}\z$, and
%  \begin{equation}
% \kappa_{q m n p} \equiv \int_0^1 d z\, v_q\z\, u_p\z\, u_m\z\, u_n^\ast\z\, \mathcal{Z}\z\, .
%  \end{equation}
%
% Then
% \begin{equation} \label{eqn:mll_url_gbqp_def}
%\overline{G}_{q - p} = \Gn \int_0^1 d z\, e^{-i 2 (q - p) \pi z}\, \mathcal{Z}\z \equiv \Gn\, \mathcal{Z}_{q - p}\, .
% \end{equation}
% \begin{equation} \label{eqn:mll_url_fqt1}
%F_q^{(1)}(t) = \frac{\Gnt}{2}\, \sum_p \frac{e^{i 2 ( q - p ) \pi t}}{1 - i\, \Omega_p}\, \mathcal{Z}_{q - p}\, E_p(t)\, .
% \end{equation}
% Suppose that $\mathcal{Z}\z = 1/(z_2 - z_1)$ for $0 < z_1 \le z \le z_2 < 1$, and is zero otherwise. In this (common) special case, when $q \ne p$ we have
%  \begin{equation} \label{eqn:mll_url_zqp_spec}
% \mathcal{Z}_{q - p} = i\, \frac{\exp[-i 2 (q - p) \pi z_2] - \exp[-i 2 (q - p) \pi z_1]}{2 (q - p) \pi (z_2 - z_1)}\, ,
%  \end{equation}
% and $\mathcal{Z}_0 = 1$ due to the normalization of $\mathcal{Z}\z$. Note that when $\{z_1, z_2\} \longrightarrow \{0, 1\}$, $\mathcal{Z}_{q - p} \longrightarrow \delta_{q p}$. However, if the intracavity laser amplifier does not fill the resonator, then \emph{in general} the quasi-normal spatial modes of the cavity will couple at first order through the Fourier transform included in \eqn{mll_url_zqp_def}.

% But we now make a crucial observation that will simplify our numerical analysis of mode-locked URLs. Since we have assumed that $|\dot{E}_q(t)| \ll |E_q(t)|/\tau_g$, terms in the first sum on the \rhs of \eqn{mll_url_fqt_g} with $p \ne q$, as well as terms in the second sum with $p \ne q - m + n$, are rapidly varying and will average out after the laser has reached stable operation. If we neglect these terms, then the polarization becomes
% \begin{equation}
%F_q(t) \cong \frac{\Gnt}{2 \left(1 - i\, \Omega_q\right)} \left[ E_q(t) - \kappa\, \sum_{m n} e^{-i\, \delta \phi_{q m n}(t)}\, B_{m n}\, C_{m n}\, E_{q - m + n}(t)\, E_m(t)\, E_n^\ast(t) \right]\,  ,
% \end{equation}
%  \begin{equation} \label{eqn:mll_url_fqt}
%  \begin{split}
% F_q(t) &= \half\, \Gn\, \left(1 - i\, \Omega_q + \tau_\perp\, \ddt\right)^{-1} E_q(t) - \frac{\kappa}{2}\, \left(1 - i\, \Omega_q + \tau_\perp\, \ddt\right)^{-1} \\
% &\qquad \times \sum_{m n} E_{q - m + n}(t)\, \left(1 - i\, \Delta \omega_{m - n} \, \tau_\parallel + \tau_\parallel\, \ddt\right)^{-1} \left[ E_m(t)\, F_n^\ast(t) + F_m(t)\, E_n^\ast(t) \right]\, ,
%  \end{split}
%  \end{equation}
% where
% \begin{equation}
%\delta \phi_{q m n}(t) \equiv (\delta \omega_{q - m + n} - \delta \omega_q + \delta \omega_m - \delta \omega_n)\, t\, , \nd
% \end{equation}
%  \begin{equation}
% \kappa \equiv \mathcal{C}^2 \int_0^1 d z\, e^{\ln(1/R)\, z}\, \mathcal{Z}\z\, .
%  \end{equation}
% In the case where $\mathcal{Z}\z = 1/(z_2 - z_1)$ for $0 < z_1 \le z \le z_2 < 1$, and is zero otherwise, we find
%  \begin{equation}
% \kappa = \frac{R}{1 - R}\, \frac{e^{\ln(1/R)\, z_2} - e^{\ln(1/R)\, z_1}}{z_2 - z_1} .
%  \end{equation}
% If $\{z_1, z_2\} \longrightarrow \{0, 1\}$, then $\kappa \longrightarrow 1$.

%In the third-order term of \eqn{mll_url_fqt}, we have made the approximation
% \begin{equation}
%\Omega_{q - m + n} + \Omega_m - \Omega_n = \Omega_q + \delta \phi_{q m n}(\tau_\perp) \approx \Omega_q
% \end{equation}
%because we have assumed throughout this analysis that $\delta \omega_q \ll 2 q \pi$. We can make a similar approximation for the coefficients $B_{m n}$ and $C_{m n}$ under the same assumption, but not for $\delta \phi_{q m n}(t)$. Because the shift caused by frequency pulling is linear in $q$ by \eqn{mml_1d_freq_shift}, if we can neglect dispersion then $\delta \phi_{q m n}(t) = 0$. However, if we include the effects of dispersion, then (to third order)
% \begin{equation}
%\delta \phi_{q m n}(t) = \frac{(q - m)(m - n)(2 \pi)^2}{1 + \tau_\perp/2\, \tau_p} \left[ D_2\wn + (q + n)\, \pi\, D_3\wn \right] t\, .
% \end{equation}
%Including this phase shift allows us to reduce the temporal fluctuations in the amplitudes $E_q(t)$, and in practice allows numerical solutions of \eqn{mml_1d_deq_dt_final} to converge more rapidly.

%  \subsubsection{Standing-Wave Lasers\label{sct:laser_dynamics_1d_mml_mll_swl}}

% The calculation of the macroscopic polarization for a standing-wave laser proceeds in essentially the same fashion as the unidirectional ring laser, but interference between the counterpropagating fields will complicate our calculations of the spatial coupling between electric field modes contributing to the nonlinear terms in the macroscopic polarization. Our strategy is straightforward, if a bit tedious. Following our approach in both \sct{laser_statics_1d_shb} and \sct{laser_dynamics_1d_mml_qsl}, we begin with the spatially rapidly-varying polarization given by \eqn{mll_fzt_formal}, now written explicitly as
% \begin{multline}
%F^{+}\zt\, e^{+i k_0 z} + F^{-}\zt\, e^{-i k_0 z} = \\ \frac{\overline{G}\zt}{2}\, \sum_p \frac{e^{-i\, \Delta \omega_p\, t}}{1 - i\, \Omega_p}\, \left[u^+_p\z\, e^{+i k_0 z} + u^-_p\z\, e^{-i k_0 z}\right] E_p(t)\, .
% \end{multline}
%  \begin{equation*}%  \label{eqn:mll_fzt_formal}
%  \begin{split}
% F^{+}\zt\, e^{+i k_0 z} + F^{-}\zt\, e^{-i k_0 z} &= \frac{\Gn}{2}\, \sum_p e^{-i\, \Delta \omega_p\, t}\, \left[u^+_p\z\, e^{+i k_0 z} + u^-_p\z\, e^{-i k_0 z}\right] \mathcal{Z}\z \\
% &\quad \times \left(1 - i\, \Omega_p + \tau_\perp\, \ddt\right)^{-1} E_p(t) \\
% &\quad - \half\, \sum_{m n p}  e^{-i\, \Delta \omega_{m - n + p}\, t}\, \widetilde{u}_m\z\, \widetilde{u}_n^\ast\z\, \widetilde{u}_p\z\, \mathcal{Z}\z \\
% &\quad \times \left(1 - i\, \Omega_{m - n + p} + \tau_\perp\, \ddt\right)^{-1} E_p(t) \\
% &\quad \times \left(1 - i\, \Delta \omega_{m - n} \, \tau_\parallel + \tau_\parallel\, \ddt\right)^{-1} \left[ E_m(t)\, F_n^\ast(t) + F_m(t)\, E_n^\ast(t) \right]\, .
%  \end{split}
%  \end{equation*}
% Let's start with the linear (first) term on the \rhs of this expression. If we make the reasonable assumption that $\mathcal{Z}\z$ is spatially slowly-varying on the scale of a wavelength, then the counterpropagating components of the polarization cleanly separate, and
%  \begin{equation}
% \mathbf{F}^{(1)}\zt = \frac{\Gn}{2}\, \sum_p e^{-i\, \Delta \omega_p\, t}\, \mathbf{u}_p\z\, \mathcal{Z}\z \left(1 - i\, \Omega_p + \tau_\perp\, \ddt\right)^{-1} E_p(t)\, ,
%  \end{equation}
% where we have used \eqn{laser_resonator_1d_u_sw_vec}. We substitute this result into \eqn{mml_fq_sol} to reproduce the first term on the \rhs of \eqn{mll_url_fqt_g}, where now
%  \begin{equation} \label{eqn:mll_swl_zqp_def}
% \mathcal{Z}_{q - p} \equiv \int_0^{1/2} d z\, \mathbf{v}_q\z \dotp \mathbf{u}_p\z\, \mathcal{Z}\z = 2 \int_0^{1/2} d z\, \cos\left[ 2\, (q - p)\, \pi\, z \right]\, \mathcal{Z}\z\, .
%  \end{equation}
% Let's suppose that $\mathcal{Z}\z = 1/2 (z_2 - z_1)$ for $0 <  z_1 \le z \le z_2 < 1/2$, and is zero otherwise. In this (common) special case, for $p \ne q$
%  \begin{equation} \label{eqn:mml_1d_zeta_12_swl}
% \mathcal{Z}_{q - p} = \frac{\sin[2\, (q - p)\, \pi\, z_2] - \sin[2\, (q - p)\, \pi\, z_1]}{2\, (q - p)\, \pi\, (z_2 - z_1)}\, .
%  \end{equation}
% When $\{z_1, z_2\} \longrightarrow \{0, 1/2\}$, $\mathcal{Z}_{q - p} \longrightarrow \delta_{q p}$.

% We must go through the same exercise with the nonlinear contribution to the macroscopic polarization given by \eqn{mll_fzt_formal}. First we expand the rapidly-varying spatial functions to explicitly show their net dependence on $e^{\pm i k_0 z}$. We find
%  \begin{equation}
%  \begin{split}
% \widetilde{U}_{m n p}\z & \equiv \widetilde{u}_p\z\, \widetilde{u}_m\z\, \widetilde{u}_n^\ast\z \\
% &= \left[u^+_p\z\, e^{+i k_0 z} + u^-_p\z\, e^{-i k_0 z}\right] \Big[u_m^{+}\z\, u_n^{+ \ast}\z + u_m^{-}\z\, u_n^{- \ast}\z \\
%     & \qquad \left. + u_m^{+}\z\, u_n^{- \ast}\z\, e^{+i 2 k_0 z} + u_m^{-}\z\, u_n^{+ \ast}\z\, e^{-i 2 k_0 z}\right]\, ,
%  \end{split}
%  \end{equation}
% or, neglecting terms proportional to $e^{\pm i 3 k_0 z}$, we follow \eqn{laser_resonator_1d_u_sw_vec} and write
%  \begin{equation}
% \mathbf{U}_{m n p}\z = \begin{bmatrix}
% u_p^+\z\, \left[u_m^{+}\z\, u_n^{+ \ast}\z + u_m^{-}\z\, u_n^{- \ast}\z\right] + u_p^-\z\, u_m^{+}\z\, u_n^{- \ast}\z \\
% u_p^-\z\, \left[u_m^{+}\z\, u_n^{+ \ast}\z + u_m^{-}\z\, u_n^{- \ast}\z\right] + u_p^+\z\, u_m^{-}\z\, u_n^{+ \ast}\z
%                    \end{bmatrix}
%  \end{equation}
% Let us again assume that we can neglect all terms in the macroscopic polarization that vary on timescales greater than or equal to $\tau_g$, leading to $p = q$ in the first-order contribution, and $p = q - m + n$ for the nonlinear term. Then in order to derive $F_q(t)$ for the standing-wave laser, we need to calculate the integral
%  \begin{equation}
% \kappa_{q m n} \equiv \int_0^{1/2} d z\, \mathbf{v}_q\z \dotp \mathbf{U}_{q - m + n, m, n}\z\, \mathcal{Z}\z\, .
%  \end{equation}
% Once this result is in hand, coefficient $q$ of the modal macroscopic polarization expansion becomes
%  \begin{equation} \label{eqn:mll_swl_fqt}
%  \begin{split}
% F_q(t) &= \half\, \Gn\, \left(1 - i\, \Omega_q + \tau_\perp\, \ddt\right)^{-1} E_q(t) - \half\, \left(1 - i\, \Omega_q + \tau_\perp\, \ddt\right)^{-1} \\
% &\qquad \times \sum_{m n} \kappa_{q m n}\, E_{q - m + n}(t)\, \left(1 - i\, \Delta \omega_{m - n} \, \tau_\parallel + \tau_\parallel\, \ddt\right)^{-1} \left[ E_m(t)\, F_n^\ast(t) + F_m(t)\, E_n^\ast(t) \right]\, ,
%  \end{split}
%  \end{equation}

% If $\mathcal{Z}\z = 1/2 (z_2 - z_1)$ for $0 < z_1 \le z \le z_2 < 1/2$, and is zero otherwise, then
%  \begin{multline} \label{eqn:mll_1d_kappa_def_swl}
% \kappa_{q m n} \equiv \frac{1}{2 \left(z_2 - z_1\right)}\, \int_{z_1}^{z_2} d z\, \left\{ \mathbf{v}_q\z \dotp \mathbf{u}_{q - m + n}\z \left[u_m^{+}\z\, u_n^{+ \ast}\z + u_m^{-}\z\, u_n^{- \ast}\z\right] \right. \\ \left. + v_q^{+}\z\, u_{q - m + n}^{-}\z\, u_m^+\z\, u_n^{- \ast}\z + v_q^{-}\z\, u_{q - m + n}^{+}\z\, u_m^-\z\, u_n^{+ \ast}\z \right\}\, .
%  \end{multline}
% Using \eqn{laser_resonator_1d_u_sw} and \eqn{laser_resonator_1d_v_sw}, this spatial coupling constant becomes
%  \begin{equation} \label{eqn:mll_1d_kappa_swl}
%  \begin{split}
% \kappa_{q m n} &= \frac{\mathcal{C}^{-1} \mathcal{C}^3}{2 \left(z_2 - z_1\right)}\, \int_{z_1}^{z_2} d z\, \left\{ \left[e^{i 2 (m - n) \pi z} + e^{-i 2 (m - n) \pi z}\right] \right. \\
% &\qquad\qquad \times \left[e^{i 2 (m - n) \pi z} e^{\ln(1/R_1 R_2) z} + \frac{1}{R_1}\, e^{-i 2 (m - n) \pi z} e^{-\ln(1/R_1 R_2) z}\right] \\
% &\qquad \left. + e^{i 4 (q -m) \pi z} e^{\ln(1/R_1 R_2) z} + \frac{1}{R_1}\, e^{-i 4 (q - m) \pi z} e^{-\ln(1/R_1 R_2) z} \right\} \\
% & \equiv \Delta^\prime_{0}\left(R_1, R_2\right) + \Delta^\prime_{2 (m - n)}\left(R_1, R_2\right) + \Delta^\prime_{2 (q - m)}\left(R_1, R_2\right)\, ,
%  \end{split}
%  \end{equation}
% where
%  \begin{equation}
%  \begin{split}
% \Delta^\prime_{2 q}\left(R_1, R_2\right) &\equiv \frac{\mathcal{C}^2}{2 \left(z_2 - z_1\right)}\, \int_{z_1}^{z_2} d z\, \left\{ e^{\left[ i 4 q \pi + \ln(1/R_1 R_2)\right] z} + \frac{1}{R_1}\, e^{-\left[ i 4 q \pi + \ln(1/R_1 R_2)\right] z} \right\} \\
% &= \Delta_{2 q}(R_1 R_2)\, \frac{\mathcal{C}^2}{2 \left(z_2 - z_1\right) \ln(1/R_1 R_2)} \left\{ \left[e^{\left[i 4 q \pi + \ln(1/R_1 R_2)\right] z_2} - e^{\left[i 4 q \pi + \ln(1/R_1 R_2)\right] z_1}\right] \right. \\
% &\qquad \left.- R_1^{-1} \left[e^{-\left[i 4 q \pi + \ln(1/R_1 R_2)\right] z_2} - e^{-\left[i 4 q \pi + \ln(1/R_1 R_2)\right] z_1}\right] \right\}\,
% \, ,
%  \end{split}
%  \end{equation}
% and $\Delta_q(R)$ is defined by \eqn{laser_resonator_1d_Delta_qR}. If $\{z_1, z_2\} \longrightarrow \{0, 1/2\}$, then $\Delta^\prime_{2 q}(R_1, R_2) \longrightarrow \Delta_{2 q}(R_1 R_2)$. In this case, $\kappa_{q m n}$ becomes
%  \begin{equation} \label{eqn:mll_1d_kappa_swl_smpl}
% \kappa_{q m n} = 1 + \Delta_{2 (m - n)}(R_1 R_2) + \Delta_{2 (q - m)}(R_1 R_2)\, ,
%  \end{equation}
% The first term on the \rhs of \eqn{mll_1d_kappa_swl} and \eqn{mll_1d_kappa_swl_smpl} is simply the nonlinear coupling for the unidirectional ring laser. The second term arises from cross-saturation \emph{neglecting} interference between the counterpropagating fields, while the third term adds these interference effects. Note that \eqn{mll_1d_kappa_swl_smpl} predicts that $\kappa_{qqq} = 3$, which is the saturation constant appropriate for weak fields as described toward the end of \sct{laser_statics_1d_shb}.

 \subsection{Numerics}
Let's assume that in all cases of practical interest the transverse coherence time $\tau_\perp$ (which has been scaled by the group round-trip time $\tau_g$) is small enough that we can ignore the corresponding differential operators on the \rhs of a former equation, giving
 \begin{equation} \label{eqn:mll_swl_fqt_prac}
 \begin{split}
F_q(t) &= \half\, \mathcal{L}_q\, \Gn\, E_q(t) - \half\, \mathcal{L}_q\, \sum_{m n} \kappa_{q m n}\, E_{q - m + n}(t) \\
&\qquad \times \left(1 - i\, \Delta \omega_{m - n} \, \tau_\parallel + \tau_\parallel\, \ddt\right)^{-1} \left[ E_m(t)\, F_n^\ast(t) + F_m(t)\, E_n^\ast(t) \right]\, ,
 \end{split}
 \end{equation}
where $\mathcal{L}_{q}$ is defined by \eqn{mml_lmc_q_def}.
% \begin{equation} \label{eqn:mll_1d_l_def}
%\mathcal{L}_{q} \equiv \frac{1}{1 - i\, \Omega_q}\, .
% \end{equation}
In general, we can't make assumptions about the scaled value of $\tau_\parallel$; it could be smaller or larger than unity. In the case of a single-mode laser with no dispersion, our incorporation of frequency pulling into \eqn{mml_1d_deq_dt_fp} means that a constant pump will eventually result in $\dot{E}_q(t) = 0$. One approach to estimating the impact of the differential operator on the \rhs of \eqn{mll_swl_fqt_prac} to a multimode laser is to expand the nonlinear contribution to $F_q(t)$ to third order in the electric field coefficients. Using
 \begin{equation}
F^{(1)}_q(t) = \half\, \mathcal{L}_q\, \Gn\, E_q(t)\, ,
 \end{equation}
we obtain
 \begin{equation} \label{eqn:mll_swl_fqt_fwm_prac}
 \begin{split}
F_q(t) &\cong \half\, \mathcal{L}_q\, \Gn E_q(t) \\
&\quad - \half\, \mathcal{L}_q\, \Gn \sum_{m n} \kappa_{q m n}\, B_{m n}\, E_{q - m + n}(t) \left(1 - i\, \Delta \omega_{m - n} \, \tau_\parallel + \tau_\parallel\, \ddt\right)^{-1} E_m(t)\, E_n^\ast(t)\, ,
 \end{split}
 \end{equation}
where
 \begin{equation} \label{eqn:mll_1d_b_def}
B_{m n} \equiv \half\, \left( \mathcal{L}_m + \mathcal{L}^\ast_n \right)\, .
 \end{equation}
We note that
 \begin{equation} %\label{eqn:mll_diff_oper}
\left(1 - i\, \Delta \omega_{m - n} \, \tau_\parallel + \tau_\parallel\, \ddt\right)^{-1} \left[E_m(t)\, E_n^\ast(t)\right] = \sum_{l = 0}^{\infty} \left( i\, \Delta \omega_{m - n} \, \tau_\parallel - \tau_\parallel\, \ddt \right)^l \left[ E_m(t)\, E_n^\ast(t)\right]\, .
 \end{equation}
The $l = 1$ term of the sum on the \rhs has the form
 \begin{equation}
 \begin{split}
\left( i\, \Delta \omega_{m - n} \, \tau_\parallel - \tau_\parallel\, \ddt \right) \left[ E_m(t)\, E_n^\ast(t)\right] &= i\, \Delta \omega_{m - n} \, \tau_\parallel \left[ E_m(t)\, E_n^\ast(t)\right] \\
&\quad - \tau_\parallel \left[E_n^\ast(t)\, \dot{E}_m(t) + E_m(t)\, \dot{E}_n^\ast(t)\right]\, .
 \end{split}
 \end{equation}
Consistent with our third-order expansion of $F_q(t)$, we use \eqn{mml_1d_deq_dt_final} to estimate $\dot{E}_q(t)$ to first order in $E_q(t)$. We obtain
 \begin{equation} %\label{eqn:mml_1d_deq_dt_final}
 \dot{E}_q(t) \approx \gamma_q\, E_q(t)\, ,
 \end{equation}
where
 \begin{equation} \label{eqn:mml_1d_gamma_q_def}
 \gamma_q \equiv \frac{1}{1 + \delta \tau_q\wn} \left[ \half \left( 1 + i\, \Omega_q \right) \left( \frac{\Gn}{1 + \Omega_q^2} - \frac{1}{\tau_p} \right) + i\, \delta D_q\wn \right] .
 \end{equation}
Therefore
 \begin{equation}
\left( i\, \Delta \omega_{m - n} \, \tau_\parallel - \tau_\parallel\, \ddt \right) \left[ E_m(t)\, E_n^\ast(t)\right] \approx \left[ i\, \Delta \omega_{m - n} - \left(\gamma_m + \gamma_n^\ast\right) \right] \tau_\parallel\, \, E_m(t)\, E_n^\ast(t)\, ,
 \end{equation}
and
 \begin{equation}
 \begin{split}
\left(1 - i\, \Delta \omega_{m - n} \, \tau_\parallel + \tau_\parallel\, \ddt\right)^{-1} E_m(t)\, E_n^\ast(t) &= \sum_{l = 0}^{\infty} \left\{\left[ i\, \Delta \omega_{m - n} - \left(\gamma_m + \gamma_n^\ast\right) \right] \tau_\parallel\right\}^l\, \left[ E_m(t)\, E_n^\ast(t)\right] \\
&\equiv C_{m n}\, E_m(t)\, E_n^\ast(t)\, ,
 \end{split}
 \end{equation}
where
 \begin{equation} \label{eqn:mll_1d_cp_def}
C_{m n} \equiv \frac{1}{1 + \left(\gamma_m + \gamma_n^\ast - i\, \Delta \omega_{m - n}\right)\, \tau_\parallel}\, .
 \end{equation}
Suppose that $\tau_\parallel \lesssim 1$, and that $\Gn$ is only moderately above threshold, so that $\gamma_0 < 1$. Then $\gamma_m + \gamma_n^\ast$ can be neglected in favor of $\Delta \omega_{m - n}$. If $\tau_\parallel \gg 1$, then even at moderate gains $C_{m n}$ will be strongly suppressed; we have
 \begin{equation}
C_{m n} \approx \frac{\delta_{m n}}{1 + 2 \Re (\gamma_n)\, \tau_\parallel}\, .
 \end{equation}
%where
% \begin{equation}
%2 \Re (\gamma_n)\, \tau_\parallel = \frac{\tau_\parallel}{1 + \delta \tau_n\wn} \left( \frac{\Gn}{1 + \Omega_n^2} - \frac{1}{\tau_p} \right)\, .
% \end{equation}
This effect is even more pronounced for multimode systems well above threshold. Note that modes \emph{below} threshold should have $C_{n n} = 1$.

Let's now investigate numerical solutions of \eqn{mml_1d_deq_dt_final} after replacing the differential operator in \eqn{mll_swl_fqt_prac} with $C_{m n}$ in our computations of $F_q(t)$. First, in the ``all-wave-mixing'' (AWM) case, we choose
 \begin{equation} \label{eqn:mll_swl_fqt_awm}
 F_q(t) \cong \half\, \mathcal{L}_q\, \left\{ \Gn\, E_q(t) - \sum_{m n} \kappa_{q m n}\, C_{m n}\, E_{q - m + n}(t) \left[ E_m(t)\, F_n^\ast(t) + F_m(t)\, E_n^\ast(t) \right] \right\}\, .
 \end{equation}
This equation can be rewritten as a matrix equation for $F_q(t)$ in the form
 \begin{equation}\label{eqn:mll_1d_fqt_swl_awm}
\sum_m \left[ A_{q m}(t)\, F_m(t) + B_{q m}(t)\, F^\ast_m(t) \right] = H_q(t)\, ,
 \end{equation}
where
 \begin{align*}
A_{q m}(t) &\equiv \delta_{q, m} + \sum_n \mathcal{K}_{qmn}\, E_{q - m + n}(t)\, E_n^\ast(t)\, , \\
B_{q m}(t) &\equiv \sum_n \mathcal{K}_{qnm}\, E_{q - n + m}(t)\, E_n(t)\, , \\
\mathcal{K}_{qmn} &\equiv \half\, \mathcal{L}_q\, \kappa_{q m n}\, C_{m n}\, , \nd \\
H_q(t) &\equiv \half\, \Gn\, \mathcal{L}_q\, E_q(t)\, .
 \end{align*}
Suppose that the total number of modes in our simulation is $\mathcal{N} \equiv 2 q_\text{max} + 1$. Then we can think of $A_{q m}(t)$ and $B_{q m}(t)$ as $\mathcal{N} \times \mathcal{N}$ complex square matrices, and $F_q(t)$ and $H_q(t)$ as $\mathcal{N} \times 1$ complex column vectors. Separating all of these variables into their real and imaginary parts, we can rewrite \eqn{mll_1d_fqt_swl_awm} as the $(2 \mathcal{N} \times 2 \mathcal{N}) \cdot (2 \mathcal{N} \times 1)$ real matrix equation
 \begin{equation}\label{eqn:mml_1d_fqt_sw_mat}
\begin{bmatrix}
  \Re[A(t) + B(t)] & -\Im[A(t) - B(t)] \\
  \Im[A(t) + B(t)] & \Re[A(t) - B(t)]
\end{bmatrix} \begin{bmatrix}
                  \Re[\mathbf{F}(t)] \\
                  \Im[\mathbf{F}(t)]
                \end{bmatrix}
                 = \begin{bmatrix}
                  \Re[\mathbf{H}(t)] \\
                  \Im[\mathbf{H}(t)]
                \end{bmatrix}\, ,
 \end{equation}
which can be solved using standard numerical linear algebra techniques.% However, as written, this approach --- with $\mathcal{K}_{qmn}(t)$ incorporating $e^{-i\, \delta \phi_{q m n}(t)}$ --- can be numerically inefficient. In practice, a better algorithm would be:
% \begin{enumerate}
% \item replace $E_q(t)$ in $H(t)$ with $E_q(t) e^{-i \delta \omega_q t}$;
% \item drop $e^{-i\, \delta \phi_{q m n}(t)}$ from $\mathcal{K}_{qmn}(t)$;
% \item solve \eqn{mml_1d_fqt_sw_mat};
% \item multiply $F_q(t)$ by $e^{+i \delta \omega_q t}$; and
% \item substitute the result into \eqn{mml_1d_deq_dt_final}.
% \end{enumerate}
%It is easy to see that precisely the same approach can be applied to $F_q(t)$ in \eqn{mll_swl_fqt}.

In the low-gain, weak-field case, we can use the expansion of $F_q(t)$ to third-order in the electric field amplitude --- the ``four-wave mixing'' (FWM) case:
 \begin{equation} \label{eqn:mll_swl_fqt_fwm}
F_q(t) \cong \half\, \mathcal{L}_q\, \Gn \left[ E_q(t) - \sum_{m n} \kappa_{q m n}\, B_{m n}\, C_{m n}\, E_{q - m + n}(t)\, E_m(t)\, E_n^\ast(t) \right]\, .
 \end{equation}
In principle, \eqn{mml_1d_deq_dt_final} can be solved much more efficiently with $F_q(t)$ obtained from \eqn{mll_swl_fqt_fwm} than with \eqn{mml_1d_fqt_sw_mat}.
%
%As the unsaturated gain increases, numerical solutions of \eqn{mml_edot} relying on the third-order expansion of the macroscopic polarization given by \eqn{mll_swl_fqt} can become unstable. We can improve this stability --- at the expense of some loss of accuracy at high gains --- by treating \eqn{mll_swl_fqt} as a geometric series, and (indirectly) ``re-summing'' the terms. In this case, we find
% \begin{multline}\label{eqn:mll_swl_fqt_num}
%F_q(t) \approx \frac{1}{2 \left( 1 - i\, \Omega_q \right)} \Bigg\{ \Gnt\, E_q(t) - \\ \sum_{m n}  e^{-i\, \delta \phi_{q m n}(t)}\, \kappa_{q m n}\, C_{m n}\, E_{q - m + n}(t)  \left[ F_m(t)\, E_n^\ast(t) + E_m(t)\, F_n^\ast(t) \right] \Bigg\}\, .
% \end{multline}
%
%
%It is straightforward to show that a perturbative expansion of this expression reproduces \eqn{mll_swl_fqt}.

\subsubsection{Preliminary Solver}
\begin{equation}
  \left|E_q(t)\right|^2 -2 \Re\left[E_q^\ast(t)\, F_q(t)\right] = 0\, .
\end{equation}

In our code, we scale the time variable by the photon lifetime $\tau_p$, and compute the derivative using
\begin{equation}
  \dot{E}_q(t) = \left[-\half + i\, \left(\delta \omega_q\, \tau_p + \delta D_q\right)\right] E_q(t) + F_q(t)\, ,
\end{equation}
where $\delta \omega_q$ and $\delta D_q$ are given by \eqn{mml_1d_freq_pull} and \eqn{mml_1d_delta_d_q_def}, respectively.
Therefore,
\begin{equation}
  E^\ast_q(t)\, \dot{E}_q(t) = \left[-\frac{1}{2} + i\, \left(\delta \omega_q\, \tau_p + \delta D_q\right)\right] \left|E_q(t)\right|^2 + E^\ast_q(t)\, F_q(t)\, ,
\end{equation}
giving
\begin{align}
  \Re\left[ \frac{\dot{E}_q(t)}{E_q(t)} \right] &= -\frac{1}{2} + \Re\left[ \frac{F_q(t)}{E_q(t)} \right]\, , \text{ and} \\
  \Im\left[ \frac{\dot{E}_q(t)}{E_q(t)} \right] &= \delta \omega_q\, \tau_p + \delta D_q + \Im\left[ \frac{F_q(t)}{E_q(t)} \right]\, .
\end{align}
We see in the top two plots that $\Re[\dot{E}_q(t) / E_q(t)] \longrightarrow 0$ as $t \longrightarrow t_f$, and that in the same limit $\Im[\dot{E}_q(t) / E_q(t)] \longrightarrow \delta \nu_q\, \tau_p$, where
\begin{equation}
  \delta \nu_q \equiv \delta \omega_q + \frac{\delta D_q}{\tau_p} + \frac{1}{\tau_p}\, \Im\left[ \frac{F_q(t)}{E_q(t)} \right] \equiv \text{constant}\, .
\end{equation}
So we can use as our FOM the equations
\begin{align}
  \Re\left[ \frac{2\, F_q(t_f)}{E_q(t_f)} \right] &= 1\, , \text{ and} \\
  \Im\left[ \frac{\ddot{E}_q(t_f)}{E_q(t_f)} \right] &= 0\, ;
\end{align}
but how do we estimate $\ddot{E}_q(t_f)$?

 \subsubsection{Power Spectral Density}
Suppose that we have a numerically stable (steady-state) solution to \eqn{mml_edot_temp}, and we wish to compute the frequency content of the output intensity, defined as the square of the absolute value of an output field given by one of \eqn{laser_resonator_1d_swl_out}. Neglecting the overall normalization constant, we have
 \begin{align}%\label{}
I_\text{out}(t) &= \left| \sum_{p} e^{-i\, 2\, p\, \pi\, t}\, E_p \right|^2 = \sum_{p, p^\prime} e^{-i\, 2\, (p - p^\prime)\, \pi\, t} E_p\, E^\ast_{p^\prime} \\
&\equiv \sum_q A_q\, e^{-i\, 2\, \pi\, q\, t} ,
 \end{align}
where
 \begin{equation}
A_q \equiv \sum_p E_p\, E^\ast_{p - q} .
 \end{equation}
If $p \in \{-p_\textrm{max}, \dots, +p_\textrm{max}\}$, then, since $I_\text{out}(t)$ is real,
 \begin{align} \label{eqn:mml_1d_iout_final}
I_\text{out}(t) &= A_0 + 2 \sum_{q = 1}^{2\, p_\textrm{max}} \Re\left[ A_q\, e^{-i\, 2\, \pi\, q\, t} \right] \\
&= A_0 + 2 \sum_{q = 1}^{2\, p_\textrm{max}} \left[ \Re(A_q) \cos(2\, \pi\, q\, t) + \Im(A_q) \sin(2\, \pi\, q\, t) \right] .
 \end{align}
Therefore, following standard practice\footnote{Although both the in-phase and quadrature components are included in the definition given by \eqn{mml_1d_psd_def}, the factor of 2 in the sum of \eqn{mml_1d_iout_final} is ignored for essentially the same reason we neglect the negative frequencies when plotting the digital Fourier transform of a real signal.}, we define the \emph{power spectral density} at each frequency as
 \begin{equation} \label{eqn:mml_1d_psd_def}
P_q \equiv \frac{\sqrt{\Re(A_q)^2 + \Im(A_q)^2}}{A_0} = \frac{|A_q|}{A_0}\, ,
 \end{equation}
valid for $q \in \{0, \dots, 2\, p_\textrm{max}\}$.

 \subsubsection{Chaotic Behavior}
 \subsubsection{Passive Temporal Mode-Locking with a Saturable Absorber}
 \subsubsection{Passive Frequency Mode-Locking}

\input{files/laser_dynamics_1d_mfl}


%%%%%%%%%%%%%%%%%%%%%%%%%%%%%%%%%%%%%%%%%%%%%%%%%%%%%%%%%%%%%%%%%%%%%%%%%%%%%%%
%
% Chapter file included in main project file using \input{}
%
% Assumes that LaTeX2e macros and packages defined in rgb_laser_physics.sty
%   are available
%
%%%%%%%%%%%%%%%%%%%%%%%%%%%%%%%%%%%%%%%%%%%%%%%%%%%%%%%%%%%%%%%%%%%%%%%%%%%%%%

 \chapter{One-dimensional Multi-Mode Laser Dynamics\label{chp:laser_dynamics_1d_mml}}

In this chapter, we describe the dynamics of one-dimensional laser amplifiers and oscillators by applying the quasi-normal mode expansions derived in \sct{laser_resonators_1d_qnm} to the wave equation given by \eqn{wave_eqn_1d} and the density matrix evolution equations defined by \eqn{fls_mbe_rwa_pol} and \eqn{fls_mbe_rwa_pop_diff}. Under the right experimental conditions, these multimode representations (approximate as they are) can provide remarkably illuminating descriptions of laser behavior, including optimum output coupling, frequency pulling, wave mixing, and mode-locking.

When multiple modes oscillate in a laser, they give rise to coherent modulations of the populations in the nonlinear gain medium that create interactions between those modes. The frequencies of these modulations are integer multiples of the free spectral range $\Delta \omega_\text{FSR}$ --- defined by \eqn{delta_w_fsr_def} --- between adjacent intracavity field modes. In \fig{multimode_gain_spectrum_1d}, we show a plot of a gain medium with a peak at frequency $\omega_0 = \omega_{a b}$ as a function of the frequency detuning. We have superimposed the frequency modal structure --- over several free-spectral ranges --- of a cavity containing that medium. In the following sections, we will find that for a particular frequency $\Delta \omega_q \equiv 2 q \pi$, fluctuations in the gain medium at frequency $2 (q - p) \pi$ couple the electric field amplitude with frequency $2 p \pi$ to the macroscopic polarization component at frequency $2 q \pi$.

 \begin{figure}
  \centering
  \includegraphics[width=4.5in]{figures/multimode_gain_spectrum_1d}
  \caption{\label{fig:multimode_gain_spectrum_1d} Plot of a gain medium with a peak at frequency $\omega_0 = \omega_{a b}$ as a function of the frequency detuning. We have superimposed the frequency modal structure --- over several free-spectral ranges --- of a cavity containing the medium. We will find that for a particular frequency $\Delta \omega_q \equiv 2 q \pi$, fluctuations in the gain medium at frequency $2 (q - p) \pi$ couple the electric field amplitude with frequency $2 p \pi$ to the macroscopic polarization component at frequency $2 q \pi$.}
 \end{figure}

\section{One-Dimensional Multi-Mode Laser Evolution Equations\label{sct:laser_dynamics_1d_mml_evol_eqns}}

We begin by developing evolution equations for the complex longitudinal modal amplitudes of unidirectional and standing-wave intracavity laser fields based on the four-level Maxwell-Bloch equations given by \eqn{laser_statics_1d_sml_scaled} and \eqn{cw_sml_ftz_scaled}. We have
\begin{subequations}\label{eqn:laser_dynamics_1d_mml_scaled}
  \begin{align}
    \label{eqn:mml_etz_scaled}
    \ppt E^\pm\zt \pm \ppz E^\pm\zt &= \left[ i\, \widehat{\mathcal{D}}_0 - \half\, \an \right] E^\pm\zt + F^\pm\zt\, , \\
    \label{eqn:mml_ftz_scaled} \ppt \widetilde{F}\zt &= -\frac{1}{\tau_\perp} \left[ \mathcal{B}\, \widetilde{F}\zt - \frac{\mathcal{A}}{2}\, \widetilde{G}\zt \widetilde{E}\zt \right]\, , \nd \\
    \label{eqn:mml_gtz_scaled} \ppt \widetilde{G}\zt &= -\frac{1}{\tau_\parallel} \left\{ \widetilde{G}\zt - \Gn\zt + 2 \Re \left[ \widetilde{E}^\ast\zt \widetilde{F}\zt \right] \right\}\, ,
  \end{align}
\end{subequations}
where $\widehat{\mathcal{D}}_0$ is the differential operator defined by \eqn{cw_sml_disp_op}. Here we will defer the effects of frequency dispersion to \sct{laser_dynamics_1d_mml_frq_dis} by setting $\widehat{\mathcal{D}}_0 = 0$ in \eqn{mml_etz_scaled}.

Our goal will be to develop a set of nonlinear ordinary differential equations representing the time evolution of modal amplitudes of the electromagnetic field. Let's follow an approach similar to that used in \sct{laser_statics_1d_approx} and use the results of \sct{laser_resonators_1d_qnm} to expand $E^\pm\zt$ in terms of the quasi-normal modes of the laser resonator. For example, in the case of the one-dimensional unidirectional ring laser shown in \fig{resonator_1d_ring_gain}, $E^{-}\zt = 0$, and we can write the slowly-varying forward-propagating electric field amplitude as
\begin{equation}
  \label{eqn:mml_e_1d_zt_url}
  E^{+}\zt \equiv \sum_{q = -\infty}^\infty u_q\z\, e^{-i\, \Delta \omega_q\, t}\, E_q(t)\, ,
\end{equation}
where $u_q\z$ and the corresponding biorthogonal eigenfunction $v_q\z$ in the range $0 < z < 1$ are given by \eqn{laser_resonator_1d_u_unnorm} and \eqn{laser_resonator_1d_v_unnorm} as
\begin{subequations}
  \begin{align}
    \label{eqn:sml_1d_uq_url} u_q\z &=\mathcal{C}_\mathrm{URL}\, \exp\left[ +\left( i\, 2 q \pi + \ln\frac{1}{\sqrt{R}} \right) z \right]\, , \\
    \label{eqn:sml_1d_vq_url} v_q\z &=\mathcal{C}^{-1}_\mathrm{URL}\, \exp\left[ -\left( i\, 2 q \pi + \ln\frac{1}{\sqrt{R}} \right) z \right]\, ,
  \end{align}
\end{subequations}
$\mathcal{C}_\mathrm{URL}$ is given by \eqn{laser_resonator_1d_u_norm_url}, and
\begin{equation}\label{eqn:mml_1d_delta_w_q_def}
  \Delta \omega_q = 2 q \pi + \delta \omega_q\, ,
\end{equation}
consistent with both \eqn{cw_sml_etz_scaled} and \eqn{delta_w_q_def}. We apply the biorthogonality relation given by \eqn{laser_resonator_1d_uv_biortho} to \eqn{mml_etz_scaled} by substituting \eqn{mml_e_1d_zt_url} (with $q \longrightarrow p$) and a similar expression for $F^{+}\zt$; multiplying both sides through by $e^{+i\ \Delta \omega_q\, t}\, v_q\z$; and then integrating the result from $z = 0$ to $z = 1$. We find
\begin{equation} \label{eqn:mml_edot_temp}
  \dot{E}_q(t) = \left(-\frac{1}{2 \tau_\lambda} + i\, \delta \omega_q\right) E_q(t) + F_q(t)\, ,
\end{equation}
where $\tau_\lambda \equiv 1/\ln[1 / R \exp(-\anb)]$ is the photon lifetime\index{Photon lifetime} given by \eqn{f_fwhm} and $\anb \equiv \int_0^1 dz\, \alpha_0(z)$.

We shouldn't apply the rate-equation approximation (REA) to \eqn{mml_ftz_scaled} just yet, because a laser operating with $q_\text{max}$ longitudinal modes such that $q_\text{max}\, \Delta \omega_\text{FSR} \gtrsim 1 / \tau_\perp$ will exhibit a significant dependence of the unsaturated gain on the value of $q$. Instead, we will first substitute

because we want to keep the macroscopic polarization term $F_q(t)$ general for now. In the unidirectional ring laser case, we define $F_q(t)$ as

For the one-dimensional standing-wave laser shown in \fig{resonator_1d_sw_gain}, the slowly-varying biorthogonal eigenfunctions are $\mathbf{u}_q\z$ and $\mathbf{v}_q\z$, given by \eqn{laser_resonator_1d_u_sw_vec}, \eqn{laser_resonator_1d_u_sw}, \eqn{laser_resonator_1d_v_sw_vec}, and \eqn{laser_resonator_1d_v_sw}, and the corresponding normalization constant $\mathcal{C}_\mathrm{SWL}$ is given by \eqn{laser_resonator_1d_u_norm_swl}. Then
\begin{subequations} \label{eqn:sml_1d_uvq_swl}
  \begin{align}
    \label{eqn:sml_1d_uq_swl}
    \mathbf{u}_q\z &\equiv \begin{bmatrix} u^{+}_q\z \\ u^{-}_q\z \end{bmatrix} = \mathcal{C}_\mathrm{SWL} \begin{bmatrix} e^{+\left[ i\, 2 q \pi + \ln\left(1/\sqrt{R_1 R_2}\right) \right] z} \\ -\frac{1}{\sqrt{R_1}}\, e^{-\left[ i\, 2 q \pi + \ln\left(1/\sqrt{R_1 R_2}\right) \right] z} \end{bmatrix}\, , \nd \\
    \label{eqn:sml_1d_vq_swl}
    \mathbf{v}_q\z &\equiv \begin{bmatrix} v^{+}_q\z \\ v^{-}_q\z \end{bmatrix} = \mathcal{C}^{-1}_\mathrm{SWL} \begin{bmatrix} e^{-\left[ i\, 2 q \pi + \ln\left(1/\sqrt{R_1 R_2}\right) \right] z} \\ -\sqrt{R_1}\, e^{+\left[ i\, 2 q \pi + \ln\left(1/\sqrt{R_1 R_2}\right) \right] z} \end{bmatrix}\, ,
  \end{align}
\end{subequations}
where $0 < z < 1/2$ In this case, we apply the biorthogonality relation given by \eqn{laser_resonator_1d_uv_biortho_sw} to \eqn{cw_sml_ez_scaled} by substituting $E^{\pm}\z = \sum_p u^{\pm}_p\z\, \, e^{-i\, \Delta \omega_p\, t}\, E_p(t)$ and $F^{\pm}\z = \sum_p u^{\pm}_p\z\, \, e^{-i\, \Delta \omega_p\, t}\, F_p(t)$; forming the inner product of both sides with $e^{i\, \Delta \omega_q\, t}\, \mathbf{v}_q\z$; and then integrating the result from $z = 0$ to $z = 1/2$. Therefore, \eqn{e0_temp} remains valid for the standing-wave case with $\tau_\lambda \equiv 1 / \ln[1 / R_1 R_2 \exp(-\anb)]$ and $\anb \equiv 2 \int_0^{1/2} dz\, \alpha_0(z)$.

As a general representation of the spatially rapidly-varying fields in both unidirectional ring and standing-wave resonator configurations, we follow \sct{laser_resonators_1d_swl} and represent the electric field amplitude function as
\begin{equation}
  \label{eqn:mml_e_1d_zt_rv}
  \widetilde{E}\zt \equiv \sum_{q = -\infty}^\infty \widetilde{u}_q\z\, e^{-i\, \Delta \omega_q\, t}\, E_q(t)\, .
\end{equation}
In the unidirectional ring laser case,
\begin{subequations}
  \label{eqn:mml_1d_uvq_url}
  \begin{align}
    \label{eqn:mml_1d_uq_url} \widetilde{u}_q\z &= u^{+}_q\z\, e^{+i k_0 z}\, , \nd \\
    \label{eqn:mml_1d_vq_url} \widetilde{v}_q\z &= v^{+}_q\z\, e^{-i k_0 z}\, ,
  \end{align}
\end{subequations}
where $k_0$ is the propagation constant associated with the carrier frequency $\omega_0$. For a standing-wave resonator,
\begin{subequations}
  \label{eqn:mml_1d_uvq_swl}
  \begin{align}
    \label{eqn:mml_1d_uq_swl} \widetilde{u}_q\z &= u^{+}_q\z\, e^{+i k_0 z} + u^{-}_q\z\, e^{-i k_0 z}\, , \nd \\
    \label{eqn:mml_1d_vq_swl} \widetilde{v}_q\z &= v^{+}_q\z\, e^{-i k_0 z} + v^{-}_q\z\, e^{+i k_0 z}\, .
  \end{align}
\end{subequations}
We use a similar approach to the expansion of the amplitude of the macroscopic polarization.

\begin{multline} \label{eqn:mml_ftz_expansion}
  \sum_p \left[ \dot{F}_p(t) - i\, \Delta \omega_p\, F_p(t) \right] \widetilde{u}_p\z\, e^{-i\, \Delta \omega_p\, t} \\
  = -\frac{1}{\tau_\perp} \sum_p \left[ \mathcal{B}\, F_p(t) - \half\, \mathcal{A}\, \widetilde{G}\zt\, E_p(t) \right] \widetilde{u}_p\z\, e^{-i\, \Delta \omega_p\, t}\, .
\end{multline}
Now, rather than the rate-equation approximation (REA), we will apply the \emph{slowly-varying envelope approximation}\index{Slowly-varying envelope approximation} (SVEA) to \eqn{mml_ftz_expansion} by assuming that $|\dot{F}_p(t)| \ll |\Delta \omega_p F_p(t)|$ and neglecting the terms $\dot{F}_p(t)$ on the \lhs. This is valid when the time scale for changes in the modal polarization amplitudes $F_p(t)$ is long compared to the polarization relaxation time $\tau_\perp$.

\begin{equation} \label{eqn:mml_zop_def}
  \int d z \equiv \begin{cases}
    \int_0^{1} d z & \mbox{(URL)}\, , \\
    \int_0^{1/2} d z\, \frac{k_0}{2 \pi} \int_{z - \pi/k_0}^{z + \pi/k_0} d z' & \mbox{(SWL or SHB)}\, ,
  \end{cases}
\end{equation}

\begin{equation} %\label{eqn:mml_fq_sol_temp}
  \sum_p \mathcal{N}_{q p}\, F_p(t) = \half\, \Lq\, \sum_p e^{i \left(\Delta \omega_q - \Delta \omega_p\right) t}\, G_{q p}(t)\, E_p(t)\, , 
\end{equation}
\begin{equation} \label{eqn:mml_fq_sol_temp}
  \sum_p \mathcal{N}_{q p}\, \left(\mathcal{B} - i\, \Omega_p\right) F_p(t) = \half\, \mathcal{A}\, \sum_p e^{i \left(\Delta \omega_q - \Delta \omega_p\right) t}\, G_{q p}(t)\, E_p(t)\, , 
\end{equation}
where $\Omega_p \equiv \Delta \omega_q\, \tau_\perp$,
\begin{equation} \label{eqn:mml_nqp_def}
  \mathcal{N}_{q p} \equiv \int d z\, \widetilde{v}_q\z\, \widetilde{u}_p\z\, \nd
\end{equation}
\begin{equation} \label{eqn:mml_gqp_def}  
  G_{q p}(t) \equiv \int d z\, \widetilde{v}_q\z\, \widetilde{u}_p\z\, \widetilde{G}\zt\, .
\end{equation}
Applying \eqn{mml_zop_def} to \eqn{mml_nqp_def}, we find that for both unidirectional ring and standing-wave resonators, $\mathcal{N}_{q p} = \delta_{q p}$, and \eqn{mml_fq_sol_temp} simplifies to
\begin{equation} \label{eqn:mml_fq_sol}
  F_q(t) = \half\, \Lq\, \sum_p e^{i \left(\Delta \omega_q - \Delta \omega_p\right) t}\, G_{q p}(t)\, E_p(t)\, ,
\end{equation}
where
\begin{equation} \label{eqn:mml_lq_def}
  \Lq \equiv \frac{\mathcal{A}}{\mathcal{B} - i\, \Omega_q}\, .
\end{equation}

Let's now determine the evolution equation for $G_{q p}(t)$ by applying \eqn{mml_gqp_def} and \eqn{mml_zop_def} to \eqn{mml_gtz_scaled}. Using \eqn{mml_e_1d_zt_rv} and the corresponding representation for the macroscopic polarization, the nonlinear term on the \rhs can be written as
\begin{equation*}
  \begin{split}
    2 \Re \left[ \widetilde{E}^\ast\zt\, \widetilde{F}\zt \right] &= \widetilde{E}^\ast\zt\, \widetilde{F}\zt + c.c. \\
    &= \sum_{m n}  e^{-i\, \left(\Delta \omega_m - \Delta \omega_n\right) t}\, \widetilde{u}_m\z\, \widetilde{u}_n^\ast\z \left[ E_m(t)\, F_n^\ast(t) + F_m(t)\, E_n^\ast(t) \right]\, .
  \end{split}
\end{equation*}
We obtain
\begin{equation} \label{eqn:mml_gtz_gqp_temp}
  \dot{G}_{q p}(t) = -\frac{1}{\tau_\parallel} \left\{ G_{q p}(t) - \overline{G}_{q p}(t)  + \sum_{m n} e^{-i\, \left(\Delta \omega_m - \Delta \omega_n\right) t} \kappa_{q p m n} \left[ E_m(t)\, F_n^\ast(t) + F_m(t)\, E_n^\ast(t) \right] \right\}\, ,
\end{equation}
where
\begin{equation} \label{eqn:mml_gqp_bar_def}
  \overline{G}_{q p}(t) \equiv \int d z\, \widetilde{v}_q\z\, \widetilde{u}_p\z\, \Gn\zt\, ,
\end{equation}
and
\begin{equation} \label{eqn:mml_kqp_mn_def}
  \kappa_{q p m n} \equiv \int d z\, \widetilde{v}_q\z\, \widetilde{u}_p\z\, \widetilde{u}_m\z\, \widetilde{u}_n^\ast\z\, .
\end{equation}

Let's evaluate $\overline{G}_{q p}(t)$ and $\kappa_{q p m n}$ for the unidirectional ring resonator. Using \eqn{mml_1d_uvq_url} and \eqn{mml_zop_def}, we find
\begin{equation} %\label{eqn:mll_url_zqp_spec}
    \overline{G}_{q p}(t) = \int_{0}^{1} d z\, e^{i\, 2 (p - q)\, \pi\, z}\, \Gn\zt\, ,
\end{equation}
and we see that $\overline{G}_{q p}(t)$ is the complex exponential Fourier series coefficient of order $p - q$ for $\Gn\zt$ in the resonator. Suppose that $\Gn\zt = \Gnb(t)/(z_2 - z_1)$ for $0 < z_1 \le z \le z_2 < 1$, and is zero otherwise. In this (common) special case, when $q \ne p$ we have
\begin{equation} %\label{eqn:mll_url_zqp_spec}
    \overline{G}_{q p}(t) = \frac{\Gnb(t)}{z_2 - z_1}\, \int_{z_1}^{z_2} d z\, e^{i\, 2 (p - q)\, \pi\, z} = \frac{\exp[i\, 2 \left(p - q\right) \pi\, z_2] - \exp[i\, 2 \left(p - q\right) \pi\, z_1]}{i\, 2 \left(p - q\right) \pi \left(z_2 - z_1\right)} \, \Gnb(t)\, ,
\end{equation}
and $\overline{G}_{q q}(t) = \Gnb(t)$. Note that when $\{z_1, z_2\} \longrightarrow \{0, 1\}$, $\overline{G}_{q p}(t) \longrightarrow \delta_{q p}\, \Gnb(t)$.
For the URL, the spatial mode coupling coefficient defined by \eqn{mml_kqp_mn_def} becomes
\begin{equation}
  \label{eqn:mml_1d_kqpmn_url}
  \kappa_{q p m n} = \mathcal{C}^2_\mathrm{URL} \int_0^1 d z\, e^{i\, 2\, (-q + p + m - n)\, \pi\, z} = \Delta_{-q + p + m - n}(R)\, .
\end{equation}

In the case of the standing-wave resonator, we use \eqn{mml_1d_uvq_swl} and \eqn{mml_zop_def} to find
\begin{equation} %\label{eqn:mll_swl_zqp_spec}
  \overline{G}_{q p}(t) = \int_0^{1/2} d z\, \mathbf{v}_q\z \dotp \mathbf{u}_p\z\, \Gn\zt = 2 \int_0^{1/2} d z\, \cos\left[ 2\, (q - p)\, \pi\, z \right]\, \Gn\zt\, , 
\end{equation}
showing that $\overline{G}_{q p}(t)$ is the cosine Fourier series coefficient of order $q - p$ for $\Gn\zt$ in the SWL resonator. As we did above for the URL case, let's suppose that $\Gn\zt = \Gnb(t)/2 (z_2 - z_1)$ for $0 < z_1 \le z \le z_2 < 1/2$, and is zero otherwise. Then for $p \ne q$
 \begin{equation} %\label{eqn:mml_1d_gqp_swl}
\overline{G}_{q p}(t) = \frac{\sin[2\, (q - p)\, \pi\, z_2] - \sin[2\, (q - p)\, \pi\, z_1]}{2\, (q - p)\, \pi\, (z_2 - z_1)}\, \Gnb(t)\, .
  \end{equation}
When $\{z_1, z_2\} \longrightarrow \{0, 1/2\}$, $\overline{G}_{q p}(t) \longrightarrow \delta_{q p}\, \Gnb(t)$.
For the SWL, the spatial mode coupling coefficient defined by \eqn{mml_kqp_mn_def} becomes
\begin{equation*}
  \begin{split}
    \kappa_{q p m n} &= \int_0^{1/2} d z\, \frac{k_0}{2 \pi} \int_{z - \pi/k_0}^{z + \pi/k_0} d z^\prime \widetilde{v}_q\zp\, \widetilde{u}_p\zp\, \widetilde{u}_m\zp\, \widetilde{u}_n^\ast\zp \\
    &= \int_0^{1/2} d z\,
      \left\{ \left[ v_q^+\z\, u_p^+\z\, u_m^+\z\, u_n^{+\ast}\z + v_q^-\z\, u_p^-\z\, u_m^-\z\, u_n^{-\ast}\z \right]\right. \\
      &\quad\quad\quad\quad\;\;\: + \left[ v_q^+\z\, u_p^+\z\, u_m^-\z\, u_n^{-\ast}\z + v_q^-\z\, u_p^-\z\, u_m^+\z\, u_n^{+\ast}\z \right] \\
      &\quad\quad\quad\quad\;\;\, + \left. \left[ v_q^+\z\, u_p^-\z\, u_m^+\z\, u_n^{-\ast}\z + v_q^-\z\, u_p^+\z\, u_m^-\z\, u_n^{+\ast}\z \right] \right\}\\
    &= \mathcal{C}^2_\mathrm{SWL} \int_0^{1/2} d z\,
      \left\{ \left[ e^{\left[i\, 2 (-q + p + m - n) \pi\, z + \ln(1/R_1 R_2)\right] z} +  \frac{1}{R_1}\, e^{-\left[i\, 2 (-q + p + m - n) \pi\, z + \ln(1/R_1 R_2)\right] z} \right]\right. \\
      &\quad\quad\quad\quad\quad\quad\quad\, + \left[ e^{\left[i\, 2 (q - p + m - n) \pi\, z + \ln(1/R_1 R_2)\right] z} +  \frac{1}{R_1}\, e^{-\left[i\, 2 (q - p + m - n) \pi\, z + \ln(1/R_1 R_2)\right] z} \right] \\
      &\quad\quad\quad\quad\quad\quad\quad\, + \left. \left[ e^{\left[i\, 2 (q + p - m - n) \pi\, z + \ln(1/R_1 R_2)\right] z} +  \frac{1}{R_1}\, e^{-\left[i\, 2 (q + p - m - n) \pi\, z + \ln(1/R_1 R_2)\right] z} \right] \right\}\, ,
  \end{split}
\end{equation*}
or
\begin{equation}
  \label{eqn:mml_1d_kqpmn_swl}
  \kappa_{q p m n} = \Delta^\prime_{-q + p + m - n}\left(R_1\, R_2\right) + \Delta^\prime_{q - p + m - n}\left(R_1\, R_2\right) + \Delta^\prime_{q + p - m - n}\left(R_1\, R_2\right)\, ,
\end{equation}
where $\Delta^\prime_{q}\left(R_1\, R_2\right)$ is defined by \eqn{laser_resonator_1d_Deltap_qR}. The first term on the right of this equation couples co-propagating spatial modes (and is identical to the URL coupling term given by \eqn{mml_1d_kqpmn_url} when $R_2 = 1$), the second term couples counter-propagating spatial modes neglecting interference, and the third term couples modes incorporating spatial interference.

% In the case of the one-dimensional unidirectional ring laser shown in \fig{resonator_1d_ring_gain}, we can write the spatially rapidly-varying electric field amplitude and macroscopic polarization --- assumed to be propagating in the $+\hatb{z}$ direction --- in terms of the corresponding slowly-varying fields as $\widetilde{E}\zt = E\zt e^{i k_0 z}$ and $\widetilde{F}\zt = F\zt e^{i k_0 z}$, respectively. This common factor of $\exp(i\, k_0\, z)$ has already been canceled from both sides of \eqn{cw_sml_etz_scaled}, which holds for the slowly-varying field amplitudes. Let's use \eqn{laser_resonator_1d_ezt_expansion} to write the slowly-varying electric field envelope function $\Ezt$ as
%  \begin{equation} \label{eqn:mml_e_field_1d_t}
% \Ezt \equiv \sum_{p = -\infty}^\infty u_{p}\z\, e^{-i\, \Delta \omega_p\, t}\, E_{p}(t)\, ,
%  \end{equation}
% consistent with both \eqn{cw_sml_etz_scaled} and \eqn{delta_w_q_def}. In \sct{laser_dynamics_1d_mml_frq}, we'll use $\delta \omega_p$ to represent the majority of the frequency shifts due to frequency pulling and dispersion, thereby reducing the magnitude and increasing the time scale of the phase fluctuations of the field envelope amplitude $E_p(t)$. We build our wave equation by substituting \eqn{mml_e_field_1d_t} into \eqn{cw_sml_etz_scaled}, and then applying \eqn{laser_resonator_1d_u_hlde} to obtain
%  \begin{equation}%\label{}
%    \sum_{p = -\infty}^\infty u_{p}\z\, e^{-i\, \Delta \omega_p\, t} \left[\dot{E}_{p}(t) + \left(\frac{1}{2 \tau_p} - i\, \delta \omega_p\right) E_{p}(t)\right] = F\zt\, ,
%  \end{equation}
% where $\dot{E}_{p}(t) \equiv d E_{p}(t)/d t$, and $\tau_p$ is the photon lifetime of the cavity defined by \eqn{f_fwhm} with $|\Gamma|^2 = R e^{-\alpha\wn L}$. Now we multiply both sides of this equation by $v_q\z$, and then integrate over $z$ from $0$ to $1$ to obtain
% where we have used \eqn{laser_resonator_1d_uv_biortho} and defined
%  \begin{equation} \label{eqn:mml_1d_fq_def_url}
% F_q(t) \equiv e^{+i\, \Delta \omega_q\, t} \int_0^1 d z\, v_q\z\, F\zt\, .
%  \end{equation}

% We use a similar approach to formulate the corresponding multimode field amplitude evolution equation for the one-dimensional standing-wave laser shown in \fig{resonator_1d_sw_gain}. In this case, we must write the spatially rapidly-varying electric field amplitude and macroscopic polarization in terms of the corresponding slowly-varying fields as $\widetilde{E}\zt = E^+\zt e^{+i k_0 z} + E^-\zt e^{-i k_0 z}$ and $\widetilde{F}\zt = F^+\zt e^{+i k_0 z} + F^-\zt e^{-i k_0 z}$, respectively. We will follow \sct{laser_resonators_1d_swl}, and write both $\mathbf{E}\zt$ and $\mathbf{F}\zt$ as column vectors, as we did in \eqn{laser_resonators_1d_e_sw_def}, with the electric field amplitude defined in \sct{laser_resonators_1d_swl} by \eqn{laser_resonator_1d_ezt_expansion_sw}:
%  \begin{equation*}
% \mathbf{E}\zt \equiv \sum_{p = -\infty}^\infty \mathbf{u}_{p}\z\, e^{-i\, \Delta \omega_p\, t}\, E_{p}(t)\, .
%  \end{equation*}
% We use \eqn{cw_sml_etz_scaled} to write the wave equation for the slowly-varying amplitudes as a vector operator equation, given by
%  \begin{equation}
%  \hat{\mathcal{L}}\, \mathbf{E}\zt = \mathbf{F}\zt ,
%  \end{equation}
% where
%  \begin{equation}
% \hat{\mathcal{L}} =  \begin{bmatrix} \ppt + \ppz + \half \alpha\wn L & 0  \\ 0 & \ppt - \ppz + \half \alpha\wn L \end{bmatrix} .
%  \end{equation}
% Applying this operator to \eqn{laser_resonator_1d_ezt_expansion_sw}, we find
%  \begin{equation}%\label{}
%    \sum_{p = -\infty}^\infty \mathbf{u}_{p}\z\, e^{-i\, \Delta \omega_p\, t} \left[\dot{E}_{p}(t) + \left(\frac{1}{2 \tau_p} - i\, \delta \omega_p\right) E_{p}(t)\right] = \mathbf{F}\zt\, .
%  \end{equation}
% Now we take the dot product of both sides of this equation with $\mathbf{v}_q\z$, and then integrate over $z$ from $0$ to $1/2$ to reproduce \eqn{mml_edot_temp} in the standing-wave case, but with $F_q(t)$ defined through \eqn{laser_resonator_1d_uv_biortho_sw} as
%  \begin{equation} \label{eqn:mml_1d_fq_def_swl}
% F_q(t) \equiv e^{+i\, \Delta \omega_p\, t} \int_0^{1/2} d z\, \mathbf{v}_q\z \dotp \mathbf{F}\zt .
%  \end{equation}

In the next two sections, we use these multimode evolution equations to study the dynamics of \emph{injection-seeded gain-switched} and \emph{passively mode-locked} lasers. In \sct{laser_dynamics_1d_mml_qsl}, we will apply the \emph{strong} rate-equation approximation (REA) to construct a non-perturbative theory of a high-intensity pulsed laser that is driven by a short-duration pump and ``primed'' by a slowly-varying input field. In \sct{laser_dynamics_1d_mml_mll}, we will relax the REA to allow rapid intermodal interactions and build a weak-field perturbative model of \emph{coherent population pulsations}\index{Coherent population pulsations}\cite{ref:sargent1974lp} that lead to passive \emph{mode-locking}\index{Mode-locking} in either the time or the frequency domain.


\input{files/laser_dynamics_1d_mml_frq}
\input{files/laser_dynamics_1d_mml_qsl}
\input{files/laser_dynamics_1d_mml_mll}

%\input{files/laser_dynamics_1d_mml}



