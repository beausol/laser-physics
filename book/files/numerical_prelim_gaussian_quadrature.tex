%%%%%%%%%%%%%%%%%%%%%%%%%%%%%%%%%%%%%%%%%%%%%%%%%%%%%%%%%%%%%%%%%%%%%%%%%%%%%%
%
% Section file included in chapter file using \input{}
%
% Assumes that LaTeX2e macros and packages defined in rgb_laser_physics.sty
%   are available
%
% $Id$
%
%%%%%%%%%%%%%%%%%%%%%%%%%%%%%%%%%%%%%%%%%%%%%%%%%%%%%%%%%%%%%%%%%%%%%%%%%%%%%%

 \section{Gaussian Quadrature\label{sct:numerical_prelim_gaussian_quadrature}}\index{Gaussian quadrature}
For a given weight function $W(x)$ specified on the interval $[a, b]$, we can define a set of polynomials with the orthogonality relation
 \begin{equation} \label{eqn:gq_ortho_rel}
 \int_a^b d x\, W(x) P_n(x) P_{n^\prime}(x) = 0 \quad \textrm{if} \quad n \ne n^\prime .
 \end{equation}
The polynomial $P_n(x)$ has $n$ real roots such that $a < x_1 < x_2 < ... < x_{n-1} < x_n < b$, and satisfies the three-term recurrence relation
 \begin{equation} \label{eqn:gq_rec_rel}
 P_{k+1}(x) = (A_k x + B_k) P_k(x) - C_k P_{k-1}(x),
 \end{equation}
with $P_{-1}(x) \equiv 0$ and $P_0(x) \equiv 1$. The essential idea of Gaussian Quadrature~\cite{ref:press2007nrc} is that armed with the $n$ zeros of the orthogonal polynomial $P_n(x)$, we can approximate an integral of a real function $f(x)$ over the domain $[a, b]$ by
 \begin{equation} \label{eqn:gq_weighted_quadrature_def}
 \int_a^b d x\, W(x) f(x) \cong \sum_{j = 1}^n w_j\, f(x_j) ,
 \end{equation}
where $w_j \propto -[P_{n+1}(x_j)\, P_n^\prime(x_j)]^{-1}$, and $P_n^\prime(x) \equiv d P_n(x)/d x$.

Golub and Welsch~\cite{ref:golub1969cgq} have shown that---given the weight function and three-term recurrence relation for a particular orthogonal polynomial---the abscissae and weights needed to estimate an integral using the quadrature rule given by \eqn{gq_weighted_quadrature_def} can be found from the eigenvalues and eigenvectors of the symmetric tridiagonal Jacobi matrix
 \begin{equation} \label{eqn:gq_jacobi_matrix}
 J = \begin{bmatrix}%
 \alpha_1 & \beta_1  & 0        & \cdots      & 0              & 0 \\
 \beta_1  & \alpha_2 & \beta_2  & \cdots      & 0              & 0 \\
 0        & \beta_2  & \alpha_3 & \ddots      & 0              & 0 \\
 \vdots   & \vdots   & \ddots   & \ddots      & \beta_{n-2}    & 0 \\
 0        & 0        & 0        & \beta_{n-2} & \alpha_{n - 1} & \beta_{n-1} \\
 0        & 0        & 0        & 0           & \beta_{n - 1}  & \alpha_{n}
 \end{bmatrix} ,
 \end{equation}
where the matrix elements $\alpha_j$ and $\beta_j$ are given by
 \begin{subequations} \label{eqn:gq_alpha_beta_def}
 \begin{align}
 \label{eqn:gq_alpha_def} \alpha_j &\equiv -\frac{B_{j-1}}{A_{j-1}}, \nd \\
 \label{eqn:gq_beta_def}  \beta_j  &\equiv \left( \frac{C_j}{A_j\, A_{j-1}} \right)^{\half} .
 \end{align}
 \end{subequations}
The abscissae $x_j$ are the eigenvalues of $J$, and the weights are found from the first elements of the corresponding eigenvector using\footnote{Here the notation $v_j(1)$ indicates the first element of the vector $v_j$, consistent with MATLAB's convention for array indices.}
 \begin{equation} \label{eqn:gq_weight_def}
 w_j = \mu_0\,  v_j^2(1) ,
 \end{equation}
where
 \begin{equation} \label{eqn:gq_mu0_def}
 \mu_0 \equiv \int_a^b d x\, W(x) .
 \end{equation}

 \subsection{Legendre Polynomials}\index{Gaussian quadrature!Gauss-Legendre quadrature}
The weight function of the Legendre polynomials is $W(x) = 1$ over the domain $x = [-1, +1]$, and the corresponding recursion relation is given by
 \begin{equation} \label{eqn:gq_rec_rel_legendre}
 P_{k+1}(x) = \left(\frac{2k+1}{k+1}\right) x\, P_k(x) - \left(\frac{k}{k+1}\right) P_{k-1}(x) .
 \end{equation}
Using \eqn{gq_alpha_beta_def}, we find
 \begin{subequations} \label{eqn:gq_alpha_beta_legendre}
 \begin{align}
 \alpha_j &= 0, \nd \\
 \beta_j  &= \frac{j}{\sqrt{4 j^2 - 1}} ,
 \end{align}
 \end{subequations}
and, from \eqn{gq_mu0_def},
 \begin{equation} \label{gq_mu0_legendre}
\mu_0 = 2 .
 \end{equation}
Since the more general integral
 \begin{equation} \label{eqn:gq_legendre_general}
 \int_a^b d x\, f(x) = \frac{b - a}{2} \int_{-1}^{+1} d x\, f\left(\frac{b-a}{2} x + \frac{a+b}{2}\right) ,
 \end{equation}
we can use Gauss-Legendre quadrature to estimate its value if we apply the transformations
 \begin{subequations} \label{eqn:gq_legendre_general_xforms}
 \begin{align}
 x &\longrightarrow \frac{b-a}{2} x + \frac{a+b}{2} , \nd \\
 w &\longrightarrow \frac{b-a}{2} w .
 \end{align}
 \end{subequations}

A MATLAB R2011b function that returns abscissae and weights for Gauss-Legendre quadrature is shown in \lst{numerical_prelim_gauss_legendre}.
 \lstinputlisting[language=Matlab,caption={MATLAB Function \texttt{gauss\_legendre}},label=lst:numerical_prelim_gauss_legendre]{files/numerical_prelim_gauss_legendre.m}

 \subsection{Hermite Polynomials}\index{Gaussian quadrature!Gauss-Hermite quadrature}
The weight function of the Hermite polynomials is $W(x) = e^{-x^2}$ over the domain $x = [-\infty, +\infty]$, and the corresponding recursion relation is given by
 \begin{equation} \label{eqn:gq_rec_rel_hermite}
 H_{k+1}(x) = 2\, x\, H_k(x) - 2\, n\, H_{k-1}(x) .
 \end{equation}
Using \eqn{gq_alpha_beta_def}, we find
 \begin{subequations} \label{eqn:gq_alpha_beta_hermite}
 \begin{align}
 \alpha_j &= 0, \nd \\
 \beta_j  &= \sqrt{\frac{j}{2}} ,
 \end{align}
 \end{subequations}
and, from \eqn{gq_mu0_def},
 \begin{equation} \label{gq_mu0_hermite}
\mu_0 = \sqrt{\pi} .
 \end{equation}
Since the more general integral
 \begin{equation} \label{eqn:gq_hermite_general}
 \int_{-\infty}^\infty d x\, e^{-(x - m)^2/\sigma^2} f(x) = \sigma \int_{-\infty}^\infty d x\, e^{-x^2} f\left(\sigma x + m\right) ,
 \end{equation}
we can use Gauss-Hermite quadrature to estimate its value if we apply the transformations
 \begin{subequations} \label{eqn:gq_hermite_general_xforms}
 \begin{align}
 x &\longrightarrow \sigma x + m , \nd \\
 w &\longrightarrow \sigma w .
 \end{align}
 \end{subequations}

A MATLAB R2011b function that returns abscissae and weights for Gauss-Hermite quadrature is shown in \lst{numerical_prelim_gauss_hermite}.
 \lstinputlisting[language=Matlab,caption={MATLAB Function \texttt{gauss\_hermite}},label=lst:numerical_prelim_gauss_hermite]{files/numerical_prelim_gauss_hermite.m}

 \subsection{Laguerre Polynomials}\index{Gaussian quadrature!Gauss-Laguerre quadrature}
The weight function of the generalized Laguerre polynomials is $W(x) = x^a e^{-x}$ over the domain $x = [0, \infty]$ for real $a > -1$, and the corresponding recursion relation is given by
 \begin{equation} \label{eqn:gq_rec_rel_laguerre}
 L_{k+1}^a(x) = \left(\frac{-x + 2 k + a + 1}{k+1}\right) L_k^a(x) - \left(\frac{k+a}{k+1}\right) L_{k-1}^a(x) .
 \end{equation}
Using \eqn{gq_alpha_beta_def}, we find
 \begin{subequations} \label{eqn:gq_alpha_beta_laguerre}
 \begin{align}
 \alpha_j &= 2 j + a - 1, \nd \\
 \beta_j  &= \sqrt{j (j + a)} ,
 \end{align}
 \end{subequations}
and, from \eqn{gq_mu0_def},
 \begin{equation} \label{gq_mu0_laguerre}
\mu_0 = \Gamma(a + 1) .
 \end{equation}
Since the more general integral
 \begin{equation} \label{eqn:gq_laguerre_general}
 \int_0^\infty d x\, x^a e^{-b x} f(x) = \frac{1}{b^{a+1}} \int_0^\infty d x\, x^a e^{-x} f\left(\frac{x}{b}\right) ,
 \end{equation}
we can use Gauss-Laguerre quadrature to estimate its value if we apply the transformations
 \begin{subequations} \label{eqn:gq_laguerre_general_xforms}
 \begin{align}
 x &\longrightarrow \frac{x}{b} , \nd \\
 w &\longrightarrow \frac{w}{b^{a+1}} .
 \end{align}
 \end{subequations}

A MATLAB R2011b function that returns abscissae and weights for Gauss-Laguerre quadrature is shown in \lst{numerical_prelim_gauss_laguerre}.
 \lstinputlisting[language=Matlab,caption={MATLAB Function \texttt{gauss\_laguerre}},label=lst:numerical_prelim_gauss_laguerre]{files/numerical_prelim_gauss_laguerre.m}
