%%%%%%%%%%%%%%%%%%%%%%%%%%%%%%%%%%%%%%%%%%%%%%%%%%%%%%%%%%%%%%%%%%%%%%%%%%%%%%
%
% Section file included in main project file using \input{}
%
% Assumes that LaTeX2e macros and packages defined in notes_qdcl.sty are
%   available
%
%%%%%%%%%%%%%%%%%%%%%%%%%%%%%%%%%%%%%%%%%%%%%%%%%%%%%%%%%%%%%%%%%%%%%%%%%%%%%%

 \section{Frequency Shifts in One-Dimensional Multimode Lasers\label{sct:laser_dynamics_1d_mml_frq}}

 \subsection{Frequency Pulling\label{sct:laser_dynamics_1d_mml_frq_frp}}

As an example of a particular choice of $\delta \omega_q$, we will attempt to capture the bulk of the final frequency-pulling effects we expect in a multimode laser by applying the approximate single-longitudinal-mode laser theory developed in \sct{laser_statics_1d_approx}. We define the complex amplitude $E_q(t)$ in terms of a real amplitude $A_q(t)$ and a real phase $\phi_q(t)$ as
 \begin{equation}\label{eqn:mml_1d_aq_phiq_def}
E_q(t) \equiv A_q(t)\, e^{-i\, \phi_q(t)} .
 \end{equation}
Following \sct{laser_statics_1d_approx}, we assume that $\dot{A}_q(t) = 0$ and $\dot{\phi}_q(t) = 0$, so that application of \eqn{mml_edot_temp} gives
 \begin{equation} \label{eqn:mml_1d_dwq}
\delta \omega_q = -\frac{1}{2\, \tau_p}\, \frac{\Im[f_q]}{\Re[f_q]}\, .
 \end{equation}
where $f_q(t) \equiv e^{i\, \phi_q(t)} F_q(t)$.
%Substituting this expression into \eqn{mml_edot}, and then separating the result into real and imaginary parts, we find
% \begin{subequations}\label{eqn:mml_1d_cq_phiq_sep}
% \begin{align}
% \label{eqn:mml_1d_cq_dot} \dot{c}_q(t) &= -\frac{1}{2\, \tau_p}\, c_q(t) + \Re\left[f_q(t)\right] , \nd \\
% \label{eqn:mml_1d_phiq_dot} \dot{\phi}_q(t) &= \delta \omega_q +  \frac{\Im\left[f_q(t)\right]}{c_q(t)},
% \end{align}
% \end{subequations}
%Our goal is to solve \eqn{mml_edot} in the (approximately) steady state for each amplitude $E_q(t)$. In this case, we should obtain both $\dot{c}_q(t) \approx 0$ and a small --- and ideally constant --- value of $\dot{\phi}_q(t)$. In \sct{laser_statics_1d_approx}, we found that $\Im[f_q(t)] = \Omega_q\, \Re[f_q(t)]$ in the single-mode case, and
In \sct{laser_statics_1d_approx}, we found that $\Im[f_0]/\Re[f_0] = \Im[\mathcal{L}_0]/\Re[\mathcal{L}_0]$, where $\mathcal{L}_0$ is the lineshape function, and we assume that this condition is approximately valid in the multimode case. That is,
 \begin{equation}
\frac{\Im[f_q]}{\Re[f_q]} \approx \frac{\Im[\mathcal{L}_q]}{\Re[\mathcal{L}_q]} = \Omega_q\, ,
 \end{equation}
where in the Lorentzian case
 \begin{equation} \label{eqn:mml_lmc_q_def}
\mathcal{L}_q \equiv \frac{1}{1 - i\, \Omega_q}\, .
 \end{equation}
%We note that in single-frequency laser theory, using \eqn{mml_1d_fq} to obtain $f_q(t)$ for the single mode $q$, we find
% \begin{equation}
%f_q(t) = \frac{\left(1 + \Omega_q^2\right) g_q}{1 + \Omega_q^2 + 2 c_q^2(t)}\, c_q(t) ,
% \end{equation}
%so that $\Im\left[f_q(t)\right] = 0$.
%In steady-state, then, $c_q(t) \approx 2\, \tau_p\, \Re[f_q(t)]$, and \eqn{mml_1d_phiq_dot} becomes
% \begin{equation} \label{eqn:mml_1d_phiq_dot_approx}
%\dot{\phi}_q(t) \approx \delta \omega_q + \frac{\Omega_q}{2\, \tau_p}\, .
% \end{equation}
Therefore, choosing $\delta \omega_q \equiv -\Omega_q/2\, \tau_p$ should give $\dot{\phi}_q(t) \approx 0$ when $\dot{A}_q(t) = 0$. If we assume that $\omega_0 = \omega_{a b}$, then
 \begin{equation} \label{eqn:mml_1d_omega_q_def}
\Omega_q = \Delta \omega_q\, \tau_\perp = \left(2 q \pi + \delta \omega_q\right) \tau_\perp\, ,
 \end{equation}
and we can solve for $\delta \omega_q$ to obtain
 \begin{subequations}\label{eqn:mml_1d_freq_pull}
 \begin{align}
 \delta \omega_q &= -\frac{\tau_\perp}{2\, \tau_p}\, \frac{2 q \pi}{1 + \tau_\perp/2\, \tau_p}\, , \nd \\
 \Delta \omega_q &= \frac{2 q \pi}{1 + \tau_\perp/2\, \tau_p}\, .
 \end{align}
 \end{subequations}
Finally, substituting $\delta \omega_q = -\Omega_q/2\, \tau_p$ into \eqn{mml_edot_temp} and simplifying, we find
 \begin{equation} \label{eqn:mml_1d_deq_dt_fp}
\dot{E}_q(t) = -\frac{1}{2 \tau_p} \left( 1 + i\, \Omega_q \right) E_q(t) + F_q(t) .
 \end{equation}

 \subsection{Dispersion\label{sct:laser_dynamics_1d_mml_frq_dis}}

In \sct{laser_amp_1d_pdes}, we derived the normalized wave equation in the time domain given by \eqn{cw_sml_etz_scaled} for a particular transverse mode of the electromagnetic field. Our goal here is to update \eqn{mml_edot_temp} to include the effects of dispersion. The relevant term in \eqn{cw_sml_etz_scaled} is
\begin{equation*}
    i\, \sum_{l = 2}^\infty \frac{D_l\wn}{l!} \left(i\, \frac{\partial}{\partial t}\right)^l E^\pm\zt\, ,
\end{equation*}
where $D_l\wn$ is given by \eqn{cw_sml_disp_coeff}. Let's consider the unidirectional case --- the result for the standing-wave laser will be the same --- and use the one-dimensional single-mode expansion of the slowly-varying complex field envelope function given by \eqn{laser_resonator_1d_ezt_expansion}. We find
\begin{equation}
    \begin{split}
        i\, \sum_{l = 2}^\infty \frac{D_l\wn}{l!} \left(i\, \frac{\partial}{\partial t}\right)^l E\zt &= i\, \sum_{l = 2}^\infty \frac{D_l\wn}{l!} \left(i\, \frac{\partial}{\partial t}\right)^l \sum_p u_p\z\, e^{-i\, \Delta \omega_p\, t}\, E_p(t) \\
        &= i\, \sum_{l p} i^l\, \frac{D_l\wn}{l!}\, u_p\z\, e^{-i\, \Delta \omega_p\, t}\, \sum_{j = 0}^{l} \binom{l}{j}\, (-i\, \Delta \omega_p)^j\, \frac{\partial^{l - j}}{\partial t^{l - j}}\, E_p(t)\, . 
    \end{split}
\end{equation}

If we keep only those terms proportional to $E_p(t)$ and $\dot{E}_p(t)$, then
\begin{equation*}
    \sum_{j = 0}^{l} \binom{l}{j}\, (-i\, \Delta \omega_p)^j\, \frac{\partial^{l - j}}{\partial t^{l - j}}\, E_p(t) \approx (-i\, \Delta \omega_p)^l\, E_p(t) + l\, (-i\, \Delta \omega_p)^{l - 1}\, \dot{E}_p(t)\, ,
\end{equation*}
and then
\begin{equation} \label{eqn:mml_1d_disp_cont}
    i\, \sum_{l = 2}^\infty \frac{D_l\wn}{l!} \left(i\, \frac{\partial}{\partial t}\right)^l E\zt = i\, \sum_p u_p\z\, e^{-i\, \Delta \omega_p\, t} \left[ \delta D_p\wn\, E_p(t) + i\, \delta \tau_p\wn\, \dot{E}_p(t) \right]\, ,
\end{equation}
where
\begin{subequations} \label{eqn:mml_1d_delta_dt_q_def}
    \begin{align}
        \label{eqn:mml_1d_delta_d_q_def}
        \delta D_p\wn &\equiv \sum_{l = 2}^\infty \frac{(2 p \pi)^l}{l!}\, D_l\wn\, , \\
        \label{eqn:mml_1d_delta_tau_q_def}
        \delta \tau_p\wn &\equiv \sum_{l = 2}^\infty \frac{(2 p \pi)^{l - 1}}{(l - 1)!}\, D_l\wn\, ,
    \end{align}
\end{subequations}
and we've used \eqn{mml_1d_delta_w_q_def} to make the approximation $\Delta \omega_p \approx 2 p \pi$.

% In the one-dimensional case, after applying the normalization procedure outlined in \sct{laser_amp_1d_pdes}, this wave equation becomes
% \begin{equation} \label{eqn:mml_1d_weq_w_norm}
%     \pm \frac{\partial}{\partial z} E^\pm\zw - i\, \omega\, E^\pm\zw - i\, \mathcal{D}(\omega_0, \omega)\, E^\pm\zw + \half \alpha\wn\, E^\pm\zw = F^\pm\zw\, ,
% \end{equation}
% where, using \eqn{idm_dispersion_def}, the dispersion is now given by
%  \begin{equation} \label{eqn:mml_1d_disp_def}
% \mathcal{D}(\omega_0, \omega) = \sum_{m = 2}^\infty \frac{\omega^m}{m!}\, D_m\wn
%  \end{equation}
% and
%  \begin{equation} \label{eqn:mml_1d_disp_coeff_redeff}
% D_m\wn = \frac{L}{\tau_g^m} \frac{d^m}{d \omega_0^m} \Re\left[\beta\wn\right]\, .
%  \end{equation}
% Recall that $z$ is expressed in units of $L$ (the round-trip physical length of the laser resonator), and $\alpha\wn$ in terms of $L^{-1}$. Similarly, $t$ has units of $\tau_g$ (the group round-trip propagation time), and $\omega$ has units of $\tau_g^{-1}$.

% We'll apply the same approach we used in the beginning of \sct{laser_dynamics_1d_mml} to build an expression for the time derivative of the electric field coefficient $E_q(t)$. We'll focus on the case of the unidirectional ring laser here, but a similar analysis of a standing-wave laser will yield the same result. Taking the Fourier transform of \eqn{mml_e_field_1d_t}, and applying the Fourier Shift Theorem discussed in \sct{math_prelim_fourier_transforms}, yields
%  \begin{equation} \label{eqn:mml_1d_e_field_w}
% E\zw = \sum_{q = -\infty}^\infty u_{q}\z\, E_{q}\left(\omega - \Delta \omega_q\right)\, ,
%  \end{equation}
% where, as usual, Fourier transform pairs are distinguished by their arguments. Let's define the small angular frequency $\nu \equiv \omega - \Delta \omega_q$. To first order in $\nu$ (the slowly-varying envelope approximation in the frequency domain),
%  \begin{equation} \label{eqn:mml_1d_eq_svea_w}
%  \begin{split}
% \omega^m\, E_q\left(\omega - \Delta \omega_q\right) &= (\Delta \omega_q + \nu)^m\, E_q(\nu) \\
% &\approx (2 q \pi)^m\, E_q(\nu) + m (2 q \pi)^{m - 1}\, \nu\, E_q(\nu)\, ,
%  \end{split}
%  \end{equation}
% where we've used \eqn{mml_1d_delta_w_q_def} to make the approximation $\Delta \omega_q \approx 2 q \pi$ in the exponentiated coefficients. Substituting \eqn{mml_1d_e_field_w} and \eqn{mml_1d_eq_svea_w} into \eqn{mml_1d_weq_w_norm}, and then applying \eqn{laser_resonator_1d_u_hlde} in the form
%  \begin{equation} \label{eqn:mml_1d_u_hlde}
% \ddz u_q\z = i\, \left(\Delta \omega_q - \delta \omega_q\right) u_q\z + \left[\frac{1}{2\, \tau_p} - \half\, \alpha\wn\right] u_q\z\, ,
%  \end{equation}
% we obtain
% % \begin{multline}
% %\sum_l u_l\z \left[ 1 + \delta \tau_l\wn \right] (-i \nu)\, E_l(\nu) = \\ \sum_l u_l\z \left\{-\frac{1}{2\, \tau_p} + i \left[ \delta \omega_l + \sum_{m = 2}^\infty \frac{(2 l \pi)^m}{m!}\, D_m\wn \right] \right\} E_l(\nu) + F\zw\, ,
% % \end{multline}
%  \begin{multline} \label{eqn:mml_1d_nu_e}
% \sum_l u_l\z \left[ 1 + \delta \tau_l\wn \right] (-i \nu)\, E_l(\nu) = \\ \sum_l u_l\z \left\{-\frac{1}{2 \tau_p} \left( 1 + i\, \Omega_l \right) + i\, \delta D_l\wn \right\} E_l(\nu) + F\zw\, ,
%  \end{multline}
% where we have chosen $\delta \omega_l = -\Omega_l/2\, \tau_p$ to incorporate frequency pulling,

We follow the approach we used to derive \eqn{mml_edot_temp}, and multiply both sides of \eqn{mml_1d_disp_cont} by $v_q\z$ and then integrate the result over the cavity length. Collecting the resulting dispersion terms allow us to obtain the updated field coefficient equation of motion
\begin{equation} \label{eqn:mml_1d_deq_dt_final}
    \dot{E}_q(t) = \frac{1}{1 + \delta \tau_q\wn} \left\{ \left[-\frac{1}{2\, \tau_p} \left( 1 + i\, \Omega_q \right) + i\, \delta D_q\wn\right] E_q(t) + F_q(t) \right\}\, ,
\end{equation}
where $F_q(t)$ is again given by \eqn{mml_fq_sol}. We see two primary effects of dispersion. First, there is an additional frequency shift for each mode that increases (in magnitude) nonlinearly with mode number $q$. Second, the group round-trip time is slightly different for each mode, changing with $q$ by a factor of $1 + \delta \tau_q\wn$.
   
% Using the Fourier Shift Theorem, we note that
%  \begin{equation}
% \int_{-\infty}^{\infty} \frac{d \omega}{2 \pi}\, e^{-i \omega t}\, E_{q}\left(\omega - \Delta \omega_q\right) = e^{i \Delta \omega_q t}\, \int_{-\infty}^{\infty} \frac{d \nu}{2 \pi}\, e^{-i \nu t}\, E_{q}(\nu)\, ,
%  \end{equation}
% and we apply this transform to \eqn{mml_1d_nu_e} to obtain the updated field coefficient equation of motion
% \begin{equation}
%\left[ 1 + \delta \tau_q\wn \right] \dot{E}_q(t) = \\ \left\{-\frac{1}{2\, \tau_p} + i \left[ \delta \omega_q + \sum_{m = 2}^\infty \frac{(2 q \pi)^m}{m!}\, D_m\wn \right] \right\} E_q(t) + F_q(t)\, ,
% \end{equation}
% \begin{equation} \label{eqn:mml_1d_deqdt}
%\left[ 1 + \delta \tau_q\wn \right] \dot{E}_q(t) = \\ \left\{-\frac{1}{2 \tau_p} \left( 1 + i\, \Omega_q \right) + i\, \delta D_q\wn \right\} E_q(t) + F_q(t)\, ,
% \end{equation}

%Following the same procedure leading to \eqn{mml_1d_phiq_dot_approx}, we find
% \begin{equation}%\label{eqn}
%\delta \omega_q + \sum_{m = 2}^\infty \frac{D_m\wn}{m!}\, (2 q \pi)^m = -\frac{\Omega_q}{2\, \tau_p}\, ,
% \end{equation}
%which we can solve using \eqn{mml_1d_omega_q_def} for $\delta \omega_q$. Therefore, the total frequency shift and frequency displacement for mode $q$ due to mode pulling and dispersion is given respectively by
% \begin{align}
%\label{eqn:mml_1d_freq_shift} \delta \omega_q &= -\frac{1}{1 + \tau_\perp/2\, \tau_p} \left[\frac{\tau_\perp}{2\, \tau_p}\, (2 q \pi) + \sum_{m = 2}^\infty \frac{(2 q \pi)^m}{m!}\, D_m\wn \right]\, , \nd \\
%\label{eqn:mml_1d_freq_disp} \Delta \omega_q &= \frac{1}{1 + \tau_\perp/2\, \tau_p} \left[2 q \pi - \sum_{m = 2}^\infty \frac{(2 q \pi)^m}{m!}\, D_m\wn \right]\, .
% \end{align}
% Collecting results, we have updated our field equation of motion to read
%Although it is not obvious from the form of \eqn{mml_1d_deq_dt_final}, as the magnitudes of the dispersion coefficients increase, the primary effect will be to change the phase of the nonlinear coupling driving the evolution of each mode.

%. Applying this result to the field in \eqn{mml_e_field_1d_t}, we have to third order
% \begin{equation}
%\mathcal{L}(t)\, \mathbf{E}\zt = \left[ i\, \beta_1\wn \frac{d}{d t} - \frac{\beta_2\wn}{2} \frac{d^2}{d t^2} - i\, \frac{\beta_3\wn}{6} \frac{d^3}{d t^3} \right] \mathbf{E}\zt
% \end{equation}
%In \eqn{mml_1d_deq_dt}, we have scaled the time by $\tau_s \equiv 2\, \tau_p$ and divided out a common factor of $\beta_1\wn$. For the moment, let's suppress the factor of $\tau_s$, and for convenience scale the time by $\tau_g^\parallel$. Then, multiplying $\mathcal{L}(t)$ through by a factor of $-i \tau_g^\parallel/\beta_1 = -i \tau_g^\parallel v_g = -i L$ (where $L$ is the round-trip physical path length, or twice the physical length of the cavity), we have a scaled dispersion operator and operand given by
%
%
%%keeping only the new second and third order terms, and approximating $\Delta \omega_q t \approx (2 q \pi/\tau_g^\parallel) (\tau_g^\parallel t) = (2 q \pi) t$ in \eqn{mml_e_field_1d_t}
% \begin{equation}%\label{}
%\mathcal{L}^\prime(t)\, E_q(t)\, e^{-i\, 2 q \pi\, t} = \left[ \frac{i}{2}\, D_2\wn\, \frac{d^2}{d t^2} - \frac{1}{6}\, D_3\wn\, \frac{d^3}{d t^3} \right] E_q(t)\, e^{-i\, 2 q \pi\, t} ,
% \end{equation}
%The second and third time derivatives are given by
% \begin{subequations}
% \begin{align}
%\frac{d^2}{d t^2} E_q(t)\, e^{-i\, 2 q \pi\, t} &= \left[ -i\, 2\, (2 q \pi)\, \dot{E}_q(t) - (2 q \pi)^2\, E_q(t) \right] e^{-i\, 2 q \pi\, t} , \nd \\
%\frac{d^3}{d t^3} E_q(t)\, e^{-i\, 2 q \pi\, t} &= \left[ -3\, (2 q \pi)^2\, \dot{E}_q(t) + i\, (2 q \pi)^3\, E_q(t) \right] e^{-i\, 2 q \pi\, t} .
% \end{align}
% \end{subequations}
%Therefore, collecting results, and cancelling the common factor of $\exp\left(-i\, 2 q \pi\, t\right)$, we find
% \begin{equation}
% \begin{split}
%e^{+i\, 2 q \pi\, t}\, \mathcal{L}^\prime(t)\, E_q(t)\, e^{-i\, 2 q \pi\, t} &= \left[ D_2\wn\, (2 q \pi) + \frac{D_3\wn}{2}\, (2 q \pi)^2 \right] \dot{E}_q(t)\\ &-i \left[ \frac{D_2\wn}{2}\, (2 q \pi)^2 + \frac{D_3\wn}{6}\, (2 q \pi)^3 \right] E_q(t),
% \end{split}
% \end{equation}
%
