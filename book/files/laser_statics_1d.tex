%%%%%%%%%%%%%%%%%%%%%%%%%%%%%%%%%%%%%%%%%%%%%%%%%%%%%%%%%%%%%%%%%%%%%%%%%%%%%%
%
% Chapter file included in main project file using \input{}
%
% Assumes that LaTeX2e macros and packages defined in rgb_laser_physics.sty
%   are available
%
% $Id:$
%
%%%%%%%%%%%%%%%%%%%%%%%%%%%%%%%%%%%%%%%%%%%%%%%%%%%%%%%%%%%%%%%%%%%%%%%%%%%%%%

 \chapter{One-dimensional Single-Mode Continuous-Wave Lasers\label{chp:laser_statics_1d}}

In this chapter, our goal will be to understand gain saturation in both unidirectional and standing-wave laser amplifiers, and how it affects continuous-wave (i.e., static) laser oscillator performance. We will work in one dimension only, and begin with \eqn{laser_statics_1d_sml_scaled} to describe a laser amplifier in the steady-state:
\begin{subequations}\label{eqn:laser_statics_1d_sml_cw_scaled}
    \begin{align}
        \label{eqn:cw_sml_ez_scaled} \pm \ddz E^\pm\z &= \left[ i\, \delta \omega_0 - \half\, \anz \right] E^\pm\z + F^\pm\z\, , \\
        \label{eqn:cw_sml_fz_scaled} \widetilde{F}\z &= \half\, \mathcal{L}(\Omega)\, \widetilde{G}\z\, \widetilde{E}\z\, , \nd \\
        \label{eqn:cw_sml_gz_scaled} \widetilde{G}\z &= \Gnz - 2 \Re \left[ \widetilde{E}^\ast\z\, \widetilde{F}\z \right]\, ,
    \end{align}
\end{subequations}
where $\an \equiv \alpha\wn$, and $\widetilde{E}\z$ and $\widetilde{P}\z \equiv -i\, 2\, \widetilde{F}\z$ are the rapidly-varying (in space) electric field and macroscopic polarization, respectively; $E^\pm\z$ and $F^\pm\z$ are the corresponding spatially slowly-varying amplitudes; and $\widetilde{G}\z$ is the rapidly-varying dimensionless gain. We have updated \eqn{cw_sml_fz_scaled} to include the generalized lineshape function $\mathcal{L}(\Omega)$ developed through \eqn{cw_sml_ftz_scaled}. 

If we substitute \eqn{cw_sml_fz_scaled} into \eqn{cw_sml_gz_scaled}, solve for $\widetilde{G}\z$, and then update \eqn{cw_sml_fz_scaled} with this result, we obtain
\begin{subequations}\label{eqn:laser_statics_1d_sml_cw}
    \begin{align}
        \label{eqn:laser_statics_1d_sml_cw_gz} \widetilde{G}\z &= \frac{\Gnz}{1 + \Re\left[\mathcal{L}(\Omega)\right] \left|\widetilde{E}\z\right|^2} \, ,  \nd \\
        \label{eqn:laser_statics_1d_sml_cw_fz} \widetilde{F}\z &= \half\, \frac{\mathcal{L}(\Omega)}{1 + \Re\left[\mathcal{L}(\Omega)\right] \left|\widetilde{E}\z\right|^2}\, \Gnz \, \widetilde{E}\z \, ,
    \end{align}
\end{subequations}
which --- in the case of a Lorentzian lineshape --- becomes
\begin{subequations}\label{eqn:laser_statics_1d_sml_cw_lor}
    \begin{align}
        \label{eqn:laser_statics_1d_sml_cw_gz_lor} \widetilde{G}\z &= \frac{\left(1 + \Omega^2\right) \Gnz}{1 + \Omega^2 + \left|\widetilde{E}\z\right|^2} \, ,  \nd \\
        \label{eqn:laser_statics_1d_sml_cw_fz_lor} \widetilde{F}\z &= \half\, \frac{1 + i\, \Omega}{1 + \Omega^2 + \left|\widetilde{E}\z\right|^2}\, \Gnz \, \widetilde{E}\z \, .
    \end{align}
\end{subequations}
We will rely primarily on \eqn{cw_sml_ez_scaled} and \eqn{laser_statics_1d_sml_cw_fz} for our descriptions of one-dimensional steady-state laser amplifiers and oscillators.


%%%%%%%%%%%%%%%%%%%%%%%%%%%%%%%%%%%%%%%%%%%%%%%%%%%%%%%%%%%%%%%%%%%%%%%%%%%%%%
%
% Section file included in chapter file using \input{}
%
% Assumes that LaTeX2e macros and packages defined in rgb_laser_physics.sty
%   are available
%
%%%%%%%%%%%%%%%%%%%%%%%%%%%%%%%%%%%%%%%%%%%%%%%%%%%%%%%%%%%%%%%%%%%%%%%%%%%%%%

\section{Nonlinear Gain and Phase Shifts in Unidirectional Laser Amplifiers\label{sct:laser_statics_1d_amp}}
%We can begin to understand the potential accuracy and limitations of the simple time-dependent one-dimensional laser models that we built in \sct{laser_statics_1d_sml} by comparing them in the steady-state domain with exact solutions obtained by assuming that the gain doesn't depend on time. Using the same scaling conventions as in \sct{la1d_url}, we can rewrite \eqn{fls_mbe_rwa_rea} and \eqn{fls_mbe_rwa_pop_diff_cw} in one dimension as
% \begin{equation} \label{eqn:ld1d_uni_amp_pz}
%\widetilde{P}\z = -i\, \frac{1 + i\, \Omega_0}{1 + \Omega_0^2}\, G\z \widetilde{E}\z
% \end{equation}
%and
% \begin{equation} \label{eqn:ld1d_uni_amp_gz}
%G\z = \frac{1 + \Omega_0^2}{1 + \Omega_0^2 + \left|\widetilde{E}\z\right|^2}\, \overline{G}\z\, .
% \end{equation}
%Substituting \eqn{ld1d_uni_amp_gz} into  \eqn{ld1d_uni_amp_pz} then gives the scaled rapidly-varying macroscopic polarization in one dimension as
% \begin{equation}
%\widetilde{P}\z = -i\, \frac{1 + i\, \Omega_0}{1 + \Omega_0^2 + \left|\widetilde{E}\z\right|^2}\, \overline{G}\z \widetilde{E}\z\, .
% \end{equation}
% \eqn{cw_sml_ez_scaled}, we have for a field propagating in the $+\hatb{z}$ direction the wave equation
% \begin{equation}
%\ddz E\z =  F\z\, , \textrm{or}
% \end{equation}
In the case of a unidirectional laser amplifier, $\widetilde{E}\z \equiv E^\pm\z e^{\pm i\, k_0\wn\, z}$, so that both $I\z \equiv |\widetilde{E}\z|^2 = |E^\pm\z|^2$ and $G\z$ are slowly-varying in space. Then \eqn{cw_sml_fz_scaled} shows that the rapid spatial variation of $\widetilde{F}\z \equiv F^\pm\z e^{\pm i\, k_0\wn\, z}$ is given entirely by that of $\widetilde{E}\z$, and the common factor of $e^{\pm i\, k_0\wn\, z}$ can be cancelled. Therefore, defining the slowly-varying field amplitude as
\begin{equation}
   E^\pm\z \equiv \sqrt{I^\pm\z}\, e^{- i \phi^\pm\z}
\end{equation}
we rewrite \eqn{cw_sml_ez_scaled} as
\begin{equation} \label{eqn:cw_sml_ez_reim}
   \pm \left[ \frac{1}{2 I^\pm\z}\, \ddz\, I^\pm\z - i\, \ddz \phi^\pm\z \right] = i\, \delta \omega_0 - \half\, \anz + \frac{F^\pm\z}{E^\pm\z}\, .
\end{equation}
Using \eqn{laser_statics_1d_sml_cw_fz}, the real part of this equation becomes
\begin{equation} \label{eqn:cw_sml_ez_re}
   % \pm \frac{1}{I^\pm\z}\, \ddz I^\pm\z = - \anz + \frac{1}{1 + \Omega_0^2 + I^\pm\z}\, \Gnz\, .
   \pm \frac{1}{I^\pm\z}\, \ddz I^\pm\z = - \anz + \frac{\rels}{1 + \rels\, I^\pm\z}\, \Gnz\, ,
\end{equation}
where
\begin{equation} \label{eqn:lineshape_re_rho_def}
   \rels \equiv \Re\left[\mathcal{L}(\Omega)\right]\, .
\end{equation}
Let's analyze the forward-propagation intensity $I^+\z \equiv I\z$ in the case of spatially independent unsaturated gain $\Gnz \equiv \Gnb$ and nonsaturable background loss $\anz \equiv \anb$; we find
\begin{equation} \label{eqn:ls1d_amp_didz}
   % \ddz I\z = \frac{\Gnb\, I\z}{1 + \Omega_0^2 + I\z} - \anb\, I\z = \frac{\left\{ \Gnb - \anb \left[1 + \Omega_0^2 + I\z \right] \right\} I\z}{1 + \Omega_0^2 + I\z}\, .
   \ddz I\z = \frac{\rels\, \Gnb\, I\z}{1 + \rels\, I\z} - \anb\, I\z = \frac{\left\{ \rels\, \Gnb - \anb \left[1 + \rels\, I\z \right] \right\} I\z}{1 + \rels\, I\z}\, .
\end{equation}
Noting that
\begin{equation*}
   % \frac{1 + \Omega_0^2 + I\z}{I\z \left\{ \Gnb - \anb \left[1 + \Omega_0^2 + I\z \right] \right\}} = \frac{1}{\Gnb - \anb \left(1 + \Omega_0^2\right)} \left\{ \frac{1 + \Omega_0^2}{I\z} + \frac{\Gnb}{\left\{ \Gnb - \anb \left[1 + \Omega_0^2 + I\z \right] \right\}}\right\}\, ,
   \frac{1 + \rels\, I\z}{\left\{ \rels\, \Gnb - \anb \left[1 + \rels\, I\z \right] \right\} I\z} = \frac{1}{\rels\, \Gnb - \anb} \left\{ \frac{1}{I\z} + \frac{\asdf\, \Gnb}{\left\{ \rels\, \Gnb - \anb \left[1 + \rels\, I\z \right] \right\}}\right\}\, ,
\end{equation*}
\begin{equation*}
\end{equation*}
we quickly obtain the transcendental solution for propagation from the reference plane $z = 0$ to the reference plane $z$, given by
\begin{equation} \label{eqn:ld1d_uni_amp_iz_loss}
   % \left( 1 + \Omega_0^2 \right) \ln \frac{I(z)}{I(0)} - \frac{\Gnb}{\anb}\, \ln \left\{\frac{\Gnb - \anb \left[1 + \Omega_0^2 + I(z) \right]}{\Gnb - \anb \left[1 + \Omega_0^2 + I(0) \right]}\right\} = \left[\Gnb - \anb \left(1 + \Omega_0^2\right)\right] z \, .
   \ln \frac{I(z)}{I(0)} - \frac{\rels\, \Gnb}{\anb}\, \ln \left\{\frac{\rels\, \Gnb - \anb \left[1 + \rels\, I(z) \right]}{\rels\, \Gnb - \anb \left[1 + \rels\, I(0) \right]}\right\} = \left[\rels\, \Gnb - \anb\right] z \, .
\end{equation}
         
 In the limit $\anb/\Gnb \ll 1$, \eqn{ld1d_uni_amp_iz_loss} becomes
% \begin{equation}
%     \begin{split}
%    \left( 1 + \Omega_0^2 \right) \ln \frac{I(z_2)}{I(z_1)} & + \left[I(z_2) - I(z_1)\right] \left\{1 + \frac{\an}{2 \Gn} \left[2 \left(1 + \Omega_0^2\right) + I(z_1) + I(z_2)\right]\right\} \\
%    & = \left[\Gn - \an \left(1 + \Omega_0^2\right)\right] \left(z_2 - z_1\right) \, .
%     \end{split}
% \end{equation}
\begin{equation}
   % \ln \frac{I(z)}{I(0)}  + \frac{I(z) - I(0)}{ 1 + \Omega_0^2} = \left(\frac{\Gnb}{ 1 + \Omega_0^2} - \anb\right) z \, .
   \ln \frac{I(z)}{I(0)}  + \rels \left[I(z) - I(0)\right] = \left[\rels\, \Gnb - \anb\right] z \, .
\end{equation}
If in fact $\anz = 0$, then we can relax the constraint that the unsaturated gain be spatially independent, and quickly find
 \begin{equation} \label{eqn:ld1d_uni_amp_iz}
\ln \frac{I(z)}{I(0)} + \rels \left[I(z) - I(0)\right] = \rels \int_0^z d z^\prime\, G_0\left(z^\prime\right)\, .
 \end{equation}
%where
%  \begin{equation} \label{eqn:ld1d_uni_amp_g0z}
% g_0 \equiv \int_{z_1}^{z_2} d z\, \overline{G}(z)\, .
%  \end{equation}
% \begin{equation} \label{eqn:laser_statics_1d_amp_hdef}
%    \Hnz \equiv \exp\left[\frac{1}{1 + \Omega_0^2}\, \int_0^z d z^\prime\, G_0\left(z^\prime\right)\right]\, .
% \end{equation}
Suppose that $I(z) \ll 1$ for all $z$. Then \eqn{cw_sml_ez_re} has the solution
\begin{equation} \label{eqn:ld1d_uni_amp_iz_small}
   % I(z) \cong I(0)\, \exp\left[\frac{1}{1 + \Omega_0^2}\, \int_0^z d z^\prime\, \Gn\left(z^\prime\right) - \int_0^z d z^\prime\, \an\left(z^\prime\right)\right]\, ,
   I(z) \cong I(0)\, \exp\left\{\int_0^z d z^\prime\, \left[\rels\, \Gn\left(z^\prime\right) - \an\left(z^\prime\right)\right]\right\}\, ,
\end{equation}
which is essentially just Beer's Law given by \eqn{beers_law}. On the other hand, if $I\z \gg 1$ for all $z$ (corresponding to a heavily saturated amplifier), \eqn{ls1d_amp_didz} gives
\begin{equation} \label{eqn:ld1d_uni_amp_iz_large}
%   I(z) \cong I(0) + g_0\, .
   I\z \cong e^{-\int_0^z d z^\prime\, \an\left(z^\prime\right)} \left[I(0) + \rels \int_0^z d z^\prime e^{\int_0^{z^\prime} d z^{\prime\prime}\, \an\left(z^{\prime\prime}\right)} G_0\left(z^\prime\right)\right]\, .
\end{equation}

In \fig{amplifier_1d_cw_igz}, we plot the effective gain $G_\mathrm{eff}\z \equiv I\z/I(0)$ and the saturated gain $G\z$ as a function of position in an amplifier with constant gain $\Gnb = 1.5$ and absorption $\anb = 0.5$, and a Lorentzian detuning $\Omega = 0.5$. For a relatively small input intensity, the effective gain is exponential in $z$, and the gain saturation is negligible. However, as the input intensity increases, $G_\mathrm{eff}\z$ becomes more linear, until at very high intensities the saturated gain is \emph{less} than the constant loss, and the net gain drops below 1. \Fig{amplifier_1d_cw_iz} was obtained using a numerical solution of \eqn{ld1d_uni_amp_iz_loss}, but a direct initial value integration of \eqn{ls1d_amp_didz} yields the same result.

\begin{figure}
   \centering
   \begin{subfigure}[b]{0.8\textwidth}
      \centering
      \includegraphics[width=5.0in]{figures/amplifier_1d_cw_iz.pdf}
      \caption{Effective net gain from \eqn{ld1d_uni_amp_iz_loss}}
      \label{fig:amplifier_1d_cw_iz}
   \end{subfigure}
   \par\vspace{0.25in}
   \begin{subfigure}[b]{0.8\textwidth}
      \centering
      \includegraphics[width=5.0in]{figures/amplifier_1d_cw_gz.pdf}
      \caption{Saturated gain from \eqn{laser_statics_1d_sml_cw_gz}}
      \label{fig:amplifier_1d_cw_gz}
   \end{subfigure}
   \caption{\label{fig:amplifier_1d_cw_igz} Plot of the effective gain $G_\mathrm{eff}\z$ and the saturated gain $G\z$ as a function of position in an amplifier with constant gain $\Gnb = 1.5$ and absorption $\anb = 0.5$, and a Lorentzian detuning $\Omega = 0.5$.}
\end{figure}
 

%We can check analytically our neglect of the linear loss coefficient in \eqn{wave_eqn_1d} in the case
%If the second term in the braces on the \lhs is not small compared to unity, then \eqn{ld1d_ur_i_out} may need to be modified.

We determine the corresponding phase shift in the amplifier using the imaginary part of \eqn{cw_sml_ez_reim}, which is
\begin{equation} \label{eqn:cw_sml_ez_im}
   \mp \ddz \phi^\pm\z = \delta \omega_0 + \Im\left[ \frac{F^\pm\z}{E^\pm\z} \right]\, .
\end{equation}
From \eqn{laser_statics_1d_sml_cw_fz}, we note that
\begin{equation*}
   \Im\left[ \frac{F^\pm\z}{E^\pm\z} \right] = \frac{\imls}{\rels}\, \Re\left[ \frac{F^\pm\z}{E^\pm\z} \right]\, ,
\end{equation*}
where
\begin{equation} \label{eqn:lineshape_im_iota_def}
   \imls \equiv \Im\left[\mathcal{L}(\Omega)\right]\, .
\end{equation}
%$\Im [F^\pm\z/E^\pm\z] = \Omega_0\, \Re [F^\pm\z/E^\pm\z]$,
Using \eqn{cw_sml_ez_reim}, we have
\begin{equation} \label{eqn:laser_statics_1d_dpdz}
   % \ddz \phi^\pm\z = \mp \delta \omega_0 - \frac{\Omega_0}{2} \left[ \pm \anz + \frac{1}{I^\pm\z}\, \ddz I^\pm\z \right]\, ,
   \ddz \phi^\pm\z = \mp \delta \omega_0 - \half\, \frac{\imls}{\rels} \left[ \pm \anz + \frac{1}{I^\pm\z}\, \ddz I^\pm\z \right]\, .
\end{equation}
Integrating from $0$ to $z$, we obtain the solution
\begin{equation} \label{eqn:laser_statics_1d_phase}
   % \phi^\pm(z) - \phi^\pm(0) = \mp \delta \omega_0\, z - \frac{\Omega_0}{2} \ln \left[ e^{\pm \int_0^z d z^\prime\, \an\left(z^\prime\right)} \frac{I^\pm(z)}{I^\pm(0)} \right]\, .
   \phi^\pm(z) - \phi^\pm(0) = \mp \delta \omega_0\, z - \half\, \frac{\imls}{\rels}\, \ln \left[ e^{\pm \int_0^z d z^\prime\, \an\left(z^\prime\right)} \frac{I^\pm(z)}{I^\pm(0)} \right]\, .
\end{equation}
We see that the phase shift depends linearly on the ratio of the imaginary part to the real part of the lineshape function $\mathcal{L}(\Omega)$; in the Lorentzian case, this ratio is the net fractional detuning of the laser frequency from the center of the gain distribution, given by $\Omega$. This result is completely general for steady-state one-dimensional laser amplifiers, and will allow us to easily calculate the magnitude of the ``frequency pulling''\index{Frequency pulling} that occurs within continuous-wave laser oscillators. We can also quickly recover the expression for the net phase shift that is commonly found in textbooks\cite{ref:siegman1986l}, valid in the unsaturated-gain forward-propagation case where $\phi^+\z \equiv \phi\z$, $\anz = 0$, $\Gnz \equiv \Gnb \ll 1$; from \eqn{ld1d_uni_amp_iz_small}, for a Lorentzian lineshape we find
 \begin{equation} \label{eqn:laser_statics_1d_small}
\phi(z) - \phi(0) = -\left[ \delta \omega_0 + \frac{\Omega}{2}\, \frac{\Gnb}{1 + \Omega^2} \right] z\, .
 \end{equation}
When the gain is low, the phase shift also depends linearly on the length of the laser amplifier.

There's a subtle effect of laser intensity on the gain lineshape known as \emph{power broadening}\index{Power Broadening}. From \eqn{cw_sml_ez_re}, the dependence of the saturated gain on the detuning arising from a Lorentzian lineshape is captured by the expression
\begin{equation} \label{eqn:amplifier_1d_cw_lsat}
   L_\mathrm{sat}(\Omega) \equiv \frac{1}{1 + I + \Omega^2}\, ,
\end{equation}
where $I$ is the unidirectional intensity at the point of interest within the amplifier. Since $L_\mathrm{sat}(0) = 1 / (1 + I)$, when $\Omega = \pm\sqrt{1 + I}$ we have $L_\mathrm{sat}(\Omega) = L_\mathrm{sat}(0) / 2$, giving the full width at half-maximum
\begin{equation}
   \Omega_\mathrm{FWHM} = 2\, \sqrt{1 + I}\, ,
\end{equation}
or, since $\Omega = \Delta \omega\, \tau_\perp$,
\begin{equation} \label{eqn:amplifier_1d_cw_fwhm}
   \Delta \omega_\mathrm{FWHM} = \sqrt{1 + I}\, \Delta \omega_g\, ,
\end{equation}
where $\Delta \omega_g = 2 / \tau_\perp$ is the FWHM when $I = 0$. This increase in the width of the gain lineshape is illustrated in \fig{amplifier_1d_cw_gom}. The effect is difficult to untangle visually from saturation, so in \fig{amplifier_1d_cw_gom_norm} we normalize each lineshape to unity at $\Omega = 0$ to make it clearer. Note that power broadening does \emph{not} increase gain at any frequency detuning; rather, it reduces the ``gain curvature''
\begin{equation}
   \frac{1}{2}\, \frac{\partial^2}{\partial \Omega^2}\, L_\mathrm{sat}(\Omega) = -\frac{\left(1 + I - 3\, \Omega^2\right)}{\left(1 + I + \Omega^2\right)^3} \approx -\frac{1}{\left(1 + I\right)^2} + \frac{6\, \Omega^2}{\left(1 + I\right)^3}
\end{equation}
by a factor of $(1 + I)^2$ near $\Omega = 0$.

\begin{figure}
   \centering
   \begin{subfigure}[b]{0.8\textwidth}
      \centering
      \includegraphics[width=5.0in]{figures/amplifier_1d_cw_gom.pdf}
      \caption{Saturated and broadened gain lineshape}
      \label{fig:amplifier_1d_cw_gom}
   \end{subfigure}
   \par\vspace{0.25in}
   \begin{subfigure}[b]{0.8\textwidth}
      \centering
      \includegraphics[width=5.0in]{figures/amplifier_1d_cw_gom_norm.pdf}
      \caption{Normalized and broadened gain lineshape}
      \label{fig:amplifier_1d_cw_gom_norm}
   \end{subfigure}
   \caption{\label{fig:amplifier_1d_cw_go} Plot of the effective gain lineshape function $L_\mathrm{sat}(\Omega)$ given by \eqn{amplifier_1d_cw_lsat}, including the effects of saturation and detuning.}
\end{figure}
 
%%%%%%%%%%%%%%%%%%%%%%%%%%%%%%%%%%%%%%%%%%%%%%%%%%%%%%%%%%%%%%%%%%%%%%%%%%%%%%
%
% Section file included in chapter file using \input{}
%
% Assumes that LaTeX2e macros and packages defined in rgb_laser_physics.sty
%   are available
%
% $Id$
%
%%%%%%%%%%%%%%%%%%%%%%%%%%%%%%%%%%%%%%%%%%%%%%%%%%%%%%%%%%%%%%%%%%%%%%%%%%%%%%

\section{Approximate Continuous-Wave Single-Mode Laser Models\label{sct:laser_statics_1d_approx}}

% \red{--- BEGIN IGNORE ---}
% \subsection{Old and Busted}

% In the steady-state ``continuous wave'' case, we can build a simple but quite useful model of a laser using the approach outlined in \sct{laser_resonators_1d_sml}. There we  developed a set of single-mode laser evolution equations by performing a separation of variables between $z$ and $t$. For a unidirectional ring laser, we specified $E\zt = u_0\z\, E(t)$, with the spatial function specified by the $q = 0$ eigenfunction
% \begin{equation} \label{eqn:sml_1d_u_url}
%   u_0\z = \mathcal{C}_\mathrm{URL}\, \exp\left[ \left( \ln\frac{1}{\sqrt{R}} \right) z \right]\, ,
% \end{equation}
% where $0 < z < 1$ and $\mathcal{C}_\mathrm{URL}$ is given by \eqn{laser_resonator_1d_u_norm_url}. For a standing-wave laser (neglecting interference between the counterpropagating fields in the amplifier), we used $E^\pm\zt = u_0^\pm\z\, E(t)$, using
% \begin{subequations} \label{eqn:sml_1d_u_swl}
%   \begin{align}
%     u^{+}_0\z &=\mathcal{C}_\mathrm{SWL}\, e^{+\left[ \ln\left(1/\sqrt{R_1 R_2}\right) \right] z} , \nd \\
%     u^{-}_0\z &=-\frac{\mathcal{C}_\mathrm{SWL}}{\sqrt{R_1}}\, e^{-\left[ \ln\left(1/\sqrt{R_1 R_2}\right) \right] z} \, ,
%   \end{align}
% \end{subequations}
% where $0 < z < 1/2$ and $\mathcal{C}_\mathrm{SWL}$ is given by \eqn{laser_resonator_1d_u_norm_swl}. In both cases, $E(t)$ is a complex function of time satisfying the ordinary differential equations given by \eqn{lr1d_epd_dot_sml}. Here we will use these equations to find a steady-state approximation of both URLs and SWLs that accurately predict the output characteristics of continuous-wave lasers.

% Let's define the complex amplitude $E(t)$ in terms of a real amplitude $A(t)$ and a real phase $\phi(t)$ as
% \begin{equation} \label{eqn:sml_1d_a_phi_def}
%   E(t) \equiv A(t)\, e^{i\, \phi(t)} .
% \end{equation}
% Substituting this expression into \eqn{lr1d_edot_sml}, and then separating the result into real and imaginary parts, we find
% \begin{subequations} \label{eqn:sml_1d_a_phi_sep}
%   \begin{align}
%     \label{eqn:sml_1d_a_dot} \dot{A}(t) &= -\frac{1}{2\, \tau_p}\, A(t) + \Re[f(t)] , \nd \\
%     \label{eqn:sml_1d_phi_dot} \dot{\phi}(t) &= \delta \omega_0 +  \frac{\Im[f(t)]}{A(t)},
%   \end{align}
% \end{subequations}
% where $f(t) \equiv e^{-i\, \phi(t)} F(t)$. In the steady state, we must have both $\dot{A}(t) = 0$ and $\dot{\phi}(t) = 0$, giving
% \begin{subequations}\label{eqn:sml_1d_a_phi_cw}
%   \begin{align}
%     \label{eqn:sml_1d_a_cw} A &= 2\, \tau_p\, \Re[f]\, , \nd \\
%     \label{eqn:sml_1d_phi_cw} \delta \omega_0 &= -\frac{1}{2\, \tau_p}\, \frac{\Im[f]}{\Re[f]}\, .
%   \end{align}
% \end{subequations}

% If we assume that the rate equation approximation is valid --- $\tau_\perp\, \dot{F}(t) \ll F(t)$ --- then we may solve \eqn{lr1d_fdot_sml} and \eqn{lr1d_gdot_sml} for $F$ and $G$ as
% \begin{subequations}
%   \begin{align}
%     \label{eqn:sml_1d_f_cw} F &= \half\, \mathcal{L}(\Omega)\, G\, E\, , \nd \\
%     \label{eqn:sml_1d_g_cw} G &= \Gnb - 2\, \kappa\, \Re\left[E^\ast F\right]\, ,
%   \end{align}
% \end{subequations}
% where again $\mathcal{L}(\Omega)$ is a general lineshape function, given in the Lorentzian case by \eqn{lineshape_lorentzian}.
% %  \begin{equation} \label{eqn:sml_lmc_0_def}
% % \mathcal{L}_0 \equiv \frac{1}{1 - i\, \Omega_0}\, .
% %  \end{equation}
% Substituting \eqn{sml_1d_f_cw} into \eqn{sml_1d_g_cw}, we find the gain
% \begin{equation} \label{eqn:sml_1d_gain}
%   G = \frac{\Gnb}{1 + \kappa\, \rels\, A^2}
% \end{equation}
% and then the continuous-wave polarization
% \begin{equation} \label{eqn:sml_1d_f_cw_final}
%   f = \half\, \frac{\mathcal{L}(\Omega)\, \Gnb\, A}{1 + \kappa\, \rels\, A^2}\, ,
% \end{equation}
% where $\rels$ is given by \eqn{lineshape_re_rho_def}

% \subsection{New Hotness}
% \red{--- END IGNORE ---}

Let's derive a simple single-mode model of continuous-wave laser oscillators based on the quasi-normal modes developed in \sct{laser_resonators_1d_qnm}, applied to \eqn{laser_statics_1d_sml_cw_scaled} in order. We begin with the unidirectional ring laser (URL) shown in \fig{resonator_1d_ring_gain}, which is the ring resonator of \fig{resonator_1d_ring_smat} with an incorporated laser amplifier. Note that this amplifier may or may not fill the entire resonator, and that only one pass is made through the amplifier every round trip. We define the spatially slowly-varying field as $E\z \equiv u_0\z\, E_0$, where $u_0\z$ and the corresponding biorthogonal eigenfunction $v_0\z$ in the range $0 < z < 1$ are given by \eqn{laser_resonator_1d_u_unnorm} and \eqn{laser_resonator_1d_v_unnorm} with $q = 0$, and the corresponding normalization constant $\mathcal{C}_\mathrm{URL}$ is given by \eqn{laser_resonator_1d_u_norm_url}.
Then
\begin{subequations}
  \begin{align}
    \label{eqn:sml_1d_u0_url} u_0\z &=\mathcal{C}_\mathrm{URL}\, \exp\left[ +\left( \ln\frac{1}{\sqrt{R}} \right) z \right]\, , \\
    \label{eqn:sml_1d_v0_url} v_0\z &=\mathcal{C}^{-1}_\mathrm{URL}\, \exp\left[ -\left( \ln\frac{1}{\sqrt{R}} \right) z \right]\, ,
  \end{align}
\end{subequations}
We apply the biorthogonality relation given by \eqn{laser_resonator_1d_uv_biortho} to \eqn{cw_sml_ez_scaled} by substituting $E^{+}\z = u_0\z\, E_0$ and $F^{+}\z = u_0\z\, F_0$; multiplying both sides through by $v_0\z$; and then integrating the result from $z = 0$ to $z = 1$. We find
\begin{equation} \label{eqn:e0_temp}
  \left( \frac{1}{2\, \tau_\lambda} - i\, \delta \omega_0 \right) E_0 = F_0\, ,
\end{equation}
where $\tau_\lambda \equiv 1/\ln[1 / R \exp(-\anb)]$ is the photon lifetime\index{Photon lifetime} given by \eqn{f_fwhm} and $\anb \equiv \int_0^1 dz\, \alpha_0(z)$.

The appropriate corresponding rapidly-varying biorthogonal eigenfunctions are
\begin{subequations}
  \begin{align}
    \label{eqn:sml_1d_u0_rv_url} \widetilde{u}_0\z &=\mathcal{C}_\mathrm{URL}\, \exp\left[ +\left( i\, k_0 + \ln\frac{1}{\sqrt{R}} \right) z \right]\, , \\
    \label{eqn:sml_1d_v0_rv_url} \widetilde{v}_0\z &=\mathcal{C}^{-1}_\mathrm{URL}\, \exp\left[ -\left( i\, k_0 + \ln\frac{1}{\sqrt{R}} \right) z \right]\, ,
  \end{align}
\end{subequations}
If we substitute $\widetilde{E}\z = E_0\, \widetilde{u}_0\z$ and $\widetilde{F}\z = F_0\, \widetilde{u}_0\z$ into \eqn{cw_sml_fz_scaled}, multiply through by $\widetilde{v}_0\z$, and integrate from $z = 0$ to $z = 1$, then we obtain
\begin{equation} \label{eqn:f0_temp}
  F_0 = \half\, \Lo\, \overline{G}\, E_0\, ,
\end{equation}
where $\overline{G} \equiv \int_{0}^{1} dz\, \widetilde{G}\z$ is the round-trip gain. Then, performing this integral directly on \eqn{cw_sml_gz_scaled}, we obtain
\begin{equation} \label{eqn:gbar_temp}
  \overline{G} = \Gnb - 2 \Re \left[\int_{0}^{1} dz\, E^{+\, \ast}\z\, F^+\z \right]\, ,
\end{equation}
where $\Gnb \equiv \int_{0}^{1} dz\, \overline{G}_0\z$ is the round-trip unsaturated gain. Applying the substitutions we've used above, the integral on the \rhs of this expression is
\begin{equation}
  \int_{0}^{1} dz\, E^{+\, \ast}\z\, F^+\z = E^\ast_0\, F_0\, \mathcal{C}^2_\mathrm{URL} \int_{0}^{1} dz\, e^{\ln(1/R)\, z} = \frac{E^\ast_0\, F_0\, \mathcal{C}^2_\mathrm{URL}}{\ln(1/R)} \left(\frac{1}{R} - 1\right) = \half\, \Lo\, \overline{G}\, \left|E_0\right|^2\, ,
\end{equation}
yielding
\begin{equation}
  \overline{G} = \frac{\Gnb}{1 + \rels\, \left|E_0\right|^2}\, .
\end{equation}
% To find $F_0$ in terms of $E_0$, we first substitute \eqn{cw_sml_gz_scaled} into \eqn{cw_sml_fz_scaled}, giving
% \begin{equation}
%   \widetilde{F}\z = \half\, \mathcal{L}(\Omega)\, \Gnz\, \widetilde{E}\z - \half\, \mathcal{L}(\Omega) \left[ \widetilde{E}^\ast\z\, \widetilde{F}\z + \widetilde{F}^\ast\z\, \widetilde{E}\z \right] \widetilde{E}\z\, .
% \end{equation}
% If we substitute $\widetilde{E}\z = E_0\, \widetilde{u}_0\z$ and $\widetilde{F}\z = F_0\, \widetilde{u}_0\z$ into this expression, multiply through by $\widetilde{v}\z$, and integrate from $z = 0$ to $z = 1$ to obtain


% \begin{equation} \label{eqn:cw_sml_fz_scaled_sub}
%   F^{+}\z = \half\, \mathcal{L}(\Omega)\, \Gnz\, E^{+}\z - \half\, \mathcal{L}(\Omega) \left[ E^{+\, \ast}\z\, F^{+}\z + F^{+\, \ast}\z\, E^{+}\z \right] E^{+}\z\, .
% \end{equation}


% After cancelling the common factor of $e^{i\, k_0\wn\, z}$ on both sides of \eqn{cw_sml_fz_scaled}, we can follow the same steps to find

% Next, we apply the biorthogonality relation given by \eqn{laser_resonator_1d_uv_biortho} by following a few simple steps:
%  \begin{enumerate}
%    \item substitute $E^{+}\zt = \mathcal{C}\, e^{\ln\left(1/\sqrt{R}\right) z}\, E(t)$ and $F^{+}\zt = \mathcal{C}\, e^{\ln\left(1/\sqrt{R}\right) z}\, F(t)$ into \eqn{cw_sml_etz_scaled};
%    \item multiply both sides through by $v_0 = \mathcal{C}^{-1}\, e^{-\ln\left(1/\sqrt{R}\right) z}$; and
%    \item integrate the result from $z = 0$ to $z = 1$.
%  \end{enumerate}
% We then obtain
%  \begin{equation} \label{eqn:edot_temp}
% \dot{E}(t) = \left[i\, \delta \omega_0 - \half \ln \left(\frac{1}{|\Gamma|^2} \right) \right] E(t) + F(t)\, ,
%  \end{equation}
% where $|\Gamma|^2 \equiv R\, e^{-\alpha\wn}$. After cancelling the common factor of $e^{i\, k_0\wn\, z}$ on both sides of \eqn{cw_sml_ftz_scaled}, we can follow the same steps to find
% \begin{equation} \label{eqn:fdot_temp}
%   \dot{F}(t) = -\frac{\mathcal{B}(\Omega)}{\tau_\perp} \left[ F(t) - \half\, \Lo\, G(t)\, E(t) \right]\, ,
% \end{equation}
% where $G(t) \equiv \int_{0}^{1} dz\, G\zt$. Then, performing this integral directly on \eqn{cw_sml_gtz_scaled}, we obtain
%  \begin{equation} \label{eqn:gdot_temp}
% \dot{G}(t) = -\frac{1}{\tau_\parallel} \left\{ G(t) - \Gnt + 2 \Re \left[\int_{0}^{1} dz\, E^{+\, \ast}\zt\, F^+\zt \right] \right\}\, ,
%  \end{equation}
% where $\Gnt \equiv \int_{0}^{1} dz\, \overline{G}\zt$. Given the substitutions in the enumerated list above, the integral on the \rhs of this expression is
%  \begin{equation}
% \int_{0}^{1} dz\, E^{+\, \ast}\zt\, F^+\zt = E^\ast(t) F(t)\, \mathcal{C}^2 \int_{0}^{1} dz\, e^{\ln(1/R)\, z} = \frac{E^\ast(t) F(t) \mathcal{C}^2}{\ln(1/R)} \left(\frac{1}{R} - 1\right) = E^\ast(t) F(t) .
%  \end{equation}

 \begin{figure}
   \centering
   \begin{subfigure}[b]{0.8\textwidth}
      \centering
      \includegraphics[width=5.0in]{figures/resonator_1d_ring_gain}
      \caption{Saturated and broadened gain lineshape}
      \label{fig:resonator_1d_ring_gain}
   \end{subfigure}
   \par\vspace{0.25in}
   \begin{subfigure}[b]{0.9\textwidth}
      \centering
      \includegraphics[width=6.0in]{figures/resonator_1d_sw_gain}
      \caption{Normalized and broadened gain lineshape}
      \label{fig:resonator_1d_sw_gain}
   \end{subfigure}
   \caption{\label{fig:resonator_1d_cw_gain} Schematics of laser resonators with an incorporated laser amplifier, which may or may not fill the entire resonator. (a) The unidirectional ring resonator of \fig{resonator_1d_ring_smat}, where only one pass is made through the amplifier every round trip. (b) The standing-wave resonator of \fig{resonator_1d_sw_smat}, where two passes are made through the amplifier every round trip, and counterpropagating fields are linked by boundary conditions at the two mirrors.}
\end{figure}

The derivation of the model for a standing-wave laser (SWL) proceeds similarly, provided that we consider interference between the counterpropagating fields within the amplifier. The slowly-varying biorthogonal eigenfunctions that we now use for the standing-wave case are $\mathbf{u}_0\z$ and $\mathbf{v}_0\z$, given by \eqn{laser_resonator_1d_u_sw_vec}, \eqn{laser_resonator_1d_u_sw}, \eqn{laser_resonator_1d_v_sw_vec}, and \eqn{laser_resonator_1d_v_sw} with $q = 0$, and the corresponding normalization constant $\mathcal{C}_\mathrm{SWL}$ is given by \eqn{laser_resonator_1d_u_norm_swl}. Then
\begin{subequations} \label{eqn:sml_1d_uv0_swl}
  \begin{align}
    \label{eqn:sml_1d_u0_swl}
    \mathbf{u}_0\z &\equiv \begin{bmatrix} u^{+}_0\z \\ u^{-}_0\z \end{bmatrix} = \mathcal{C}_\mathrm{SWL} \begin{bmatrix} e^{+\left[ \ln\left(1/\sqrt{R_1 R_2}\right) \right] z} \\ -\frac{1}{\sqrt{R_1}}\, e^{-\left[ \ln\left(1/\sqrt{R_1 R_2}\right) \right] z} \end{bmatrix}\, , \nd \\
    \label{eqn:sml_1d_v0_swl}
    \mathbf{v}_0\z &\equiv \begin{bmatrix} v^{+}_0\z \\ v^{-}_0\z \end{bmatrix} = \mathcal{C}^{-1}_\mathrm{SWL} \begin{bmatrix} e^{-\left[ \ln\left(1/\sqrt{R_1 R_2}\right) \right] z} \\ -\sqrt{R_1}\, e^{+\left[ \ln\left(1/\sqrt{R_1 R_2}\right) \right] z} \end{bmatrix}\, ,
  \end{align}
\end{subequations}
where $0 < z < 1/2$. In this case, we apply the biorthogonality relation given by \eqn{laser_resonator_1d_uv_biortho_sw} to \eqn{cw_sml_ez_scaled} by substituting $E^{\pm}\z = u^{\pm}_0\z\, E_0$ and $F^{\pm}\z = u_0^{\pm}\z\, F_0$; forming the inner product of both sides with $\mathbf{v}_0\z$; and then integrating the result from $z = 0$ to $z = 1/2$. We note that $\mathbf{v}_0\z \dotp \mathbf{u}_0\z = 2$ and $\int_{0}^{1/2} dz\, \mathbf{v}_0\z \dotp \mathbf{u}_0\z = 1$. Therefore, \eqn{e0_temp} remains valid for the standing-wave case with $\tau_\lambda \equiv 1 / \ln[1 / R_1 R_2 \exp(-\anb)]$ and $\anb \equiv 2 \int_0^{1/2} dz\, \alpha_0(z)$.

Because the gain in our amplifier can be sensitive to spatial interference between the counterpropagating fields, we must be careful in how we apply these eigenfunctions to \eqn{cw_sml_fz_scaled} and \eqn{cw_sml_gz_scaled}. We begin by substituting $\widetilde{E}\z = \widetilde{u}_0\z\, E_0$ and $\widetilde{F}\z = \widetilde{u}_0\z\, F_0$ into \eqn{cw_sml_fz_scaled}, where $\widetilde{u}_0\z$ and the corresponding rapidly-varying standing-wave biorthognal eigenfunction $\widetilde{v}_0\z$ are given by
\begin{subequations} \label{eqn:sml_1d_rv_uv0_swl}
  \begin{align}
    \label{eqn:sml_1d_rv_u0_swl}
    \widetilde{u}_0\z &= \mathcal{C}_\mathrm{SWL} \left\{ e^{+\left[ i\, k_0 + \ln\left(1/\sqrt{R_1 R_2}\right) \right] z} - \frac{1}{\sqrt{R_1}}\, e^{-\left[ i\, k_0 + \ln\left(1/\sqrt{R_1 R_2}\right) \right] z} \right\}\, , \nd \\
    \label{eqn:sml_1d_rv_v0_swl}
    \widetilde{v}_0\z & = \mathcal{C}^{-1}_\mathrm{SWL} \left\{ e^{-\left[ i\, k_0 + \ln\left(1/\sqrt{R_1 R_2}\right) \right] z} -\sqrt{R_1}\, e^{+\left[ i\, k_0 + \ln\left(1/\sqrt{R_1 R_2}\right) \right] z} \right\}\, .
  \end{align}
\end{subequations}
Then we multiply both sides of the resulting expression by $\widetilde{v}_0\z$, noting that
\begin{equation}
  \widetilde{v}_0\z\, \widetilde{u}_0\z = 2 - \sqrt{R_1}\, e^{+\left[ i\, 2\, k_0 + \ln\left(1/R_1 R_2\right) \right] z} - \frac{1}{\sqrt{R_1}}\, e^{-\left[ i\, 2\, k_0 + \ln\left(1/R_1 R_2\right) \right] z}\, .
\end{equation}
Using \eqn{idm_pol_m_def} as our guide, we first average this expression over one wavelength $\lambda_0 = 2 \pi / k_0$ to eliminate the rapidly-varying terms, and then integrate the result from $z = 0$ to $z = 1/2$. We find
\begin{equation}
  F_0\, \int_0^{1/2} d z\, \frac{k_0}{2 \pi} \int_{z - \pi/k_0}^{z + \pi/k_0} d z'\,\widetilde{v}_0\zp\, \widetilde{u}_0\zp = F_0 = \half\, \Lo\, \overline{G}\, E_0\, ,
\end{equation}
where
\begin{equation}
  \overline{G} = \int_{0}^{1/2} dz\, \frac{k_0}{2 \pi} \int_{z - \pi/k_0}^{z + \pi/k_0} d z'\, \widetilde{v}_0\zp\, \widetilde{u}_0\zp\, \widetilde{G}\zp\, ,
\end{equation}
so that we have recovered \eqn{f0_temp}. Performing this integral directly on \eqn{cw_sml_gz_scaled}, we obtain
\begin{equation} \label{eqn:gbar_swl_temp}
  \begin{split}
    \overline{G} &= \int_{0}^{1/2} dz\, \frac{k_0}{2 \pi} \int_{z - \pi/k_0}^{z + \pi/k_0} d z'\, \widetilde{v}_0\zp\, \widetilde{u}_0\zp \left\{ \Gnz - 2 \Re \left[E_0^\ast\, F_0\right] \left|\widetilde{u}_0\zp\right|^2 \right\} \\
    &= \Gnb - \rels \left|E_0\right|^2\, \overline{G}\, \int_{0}^{1/2} dz\, \frac{k_0}{2 \pi} \int_{z - \pi/k_0}^{z + \pi/k_0} d z'\, \widetilde{v}_0\zp\, \widetilde{u}_0\zp\, \left|\widetilde{u}_0\zp\right|^2\, ,
  \end{split}
\end{equation}
where $\Gnb \equiv 2 \int_{0}^{1/2} dz\, \Gnz$ is the round-trip unsaturated gain, and
\begin{equation}
  \left|\widetilde{u}_0\z\right|^2 = \mathcal{C}_\mathrm{SWL}^2 \left[ e^{+\ln\left(1/R_1 R_2\right)\, z} + \frac{1}{R_1}\, e^{-\ln\left(1/R_1 R_2\right)\, z} - \frac{1}{\sqrt{R_1}} \left( e^{+ i\, 2\, k_0 \, z} + e^{- i\, 2\, k_0\, z} \right)\right]\, .
\end{equation}

When we compute the average over one wavelength in \eqn{gbar_swl_temp}, we have the option of either neglecting or including spatial interference between the counterpropagating fields. When gain diffusion is high, and spatial hole-burning (SHB) can be neglected, we can treat the two fields as independent within the amplifier, and completely ignore the rapidly-varying terms in $\left|\widetilde{u}_0\z\right|^2$ when performing the average. However, when gain diffusion is low, and SHB effects are important, we must include the full expression for $\left|\widetilde{u}_0\z\right|^2$ when performing the average to describe the spatially periodic modulation of the gain within the amplifier. In either case, we find that
\begin{equation}
  \frac{k_0}{2 \pi} \int_{z - \pi/k_0}^{z + \pi/k_0} d z'\, \widetilde{v}_0\zp\, \widetilde{u}_0\zp\, \left|\widetilde{u}_0\zp\right|^2 = \widetilde{\kappa}\, \mathcal{C}_\mathrm{SWL}^2 \left[ e^{+\ln\left(1/R_1 R_2\right)\, z} + \frac{1}{R_1}\, e^{-\ln\left(1/R_1 R_2\right)\, z} \right]\, ,
\end{equation}
where we use $\widetilde{\kappa} = 2$ if we neglect spatial interference between the counterpropagating fields, and $\widetilde{\kappa} = 3$ if we include it. The integral over $z$ on the \rhs of \eqn{gbar_swl_temp} is then
\begin{equation} %\label{}
  \widetilde{\kappa}\, \mathcal{C}_\mathrm{SWL}^2 \int_{0}^{1/2} dz\, \left[ e^{+\ln\left(1/R_1 R_2\right)\, z} + \frac{1}{R_1}\, e^{-\ln\left(1/R_1 R_2\right)\, z} \right] = \widetilde{\kappa}\, .
\end{equation}

Collecting results, the approximate governing equations for continuous-wave single-mode one-dimensional lasers are
\begin{subequations} \label{eqn:laser_statics_1d_sml_eg}
\begin{align}
  \label{eqn:laser_statics_1d_sml_e} \left( \frac{1}{2\, \tau_\lambda} - i\, \delta \omega_0 \right) &= \half\, \Lo\, \overline{G}\, , \nd \\
  \label{eqn:laser_statics_1d_sml_g} \overline{G} &= \frac{\Gnb}{1 + \widetilde{\kappa}\, \rels\, I_0}\, ,
\end{align}
\end{subequations}
where $I_0 \equiv \left|E_0\right|^2$ is the average intracavity intensity in units of the saturation intensity $I_s$,
\begin{equation} \label{eqn:lr1d_gbar_sml}
  \Gnb \equiv \begin{cases}
    \int_0^{1} d z\, \Gnz & \mbox{(URL)}\, , \\
    2 \int_0^{1/2} d z\, \Gnz & \mbox{(SWL or SHB)}\, ,
  \end{cases}
\end{equation}
is the \emph{round-trip unsaturated intensity gain}, and
\begin{equation} \label{eqn:lr1d_kappa_sml}
  \widetilde{\kappa} \equiv \begin{cases}
    1 & \mbox{(URL)}\, , \\
    2 & \mbox{(SWL)}\, , \\
    3 & \mbox{(SHB)}\, ,
  \end{cases}
\end{equation}
is the \emph{effective saturation parameter}\index{Effective saturation parameter}. We will discover in \sct{laser_statics_shb_1d_al} that $\widetilde{\kappa} = 3$ is in fact an \emph{overestimate} of the effective saturation parameter, and we'll find an analytic approach that allows us to choose a value that is more accurate.

Applying \eqn{laser_statics_1d_sml_g} to the real part of \eqn{laser_statics_1d_sml_e}, we find
\begin{equation} \label{eqn:sml_1d_int}
  I_0 = \frac{\Hnb - 1}{\widetilde{\kappa}\, \rels}\, ,
\end{equation}
where
\begin{equation}
  \Hnb \equiv \rels\, \tau_\lambda\, \Gnb \equiv \frac{\Gnb}{\Gth}\, ,
\end{equation}
and the \emph{threshold}\index{Threshold} gain is
\begin{equation} \label{eqn:la1d_threshold}
  \Gth \equiv  \frac{1}{\rels\, \tau_\lambda}\, ,
\end{equation}
or
\begin{equation} \label{eqn:la1d_threshold_lor}
  \Gth = \frac{1 + \Omega^2}{\tau_\lambda}
\end{equation}
when the lineshape is Lorentzian.

%If we assume (without loss of generality) that we can write the electric field as
% \begin{equation} \label{eqn:la1d_e_ss}
%E(t) = E_0 ,%\, e^{-i\, \omega \tau_g\, t} ,
% \end{equation}
%where the amplitude $E_0$ is both constant in time and real, then we quickly obtain
% \begin{subequations}\label{eqn:laser_statics_1d_simple}
% \begin{align}
%\label{eqn:laser_statics_1d_sml_cw_e} F &= \frac{1}{2\, \tau_p} \left(1 - i\, 2 \delta \omega_0\, \tau_p \right) E_0\, , \\
%\label{eqn:laser_statics_1d_sml_cw_g} G &= \frac{1 + \Omega^2}{1 + \Omega^2 + \kappa\, E_0^2}\, \Gn \, ,  \nd \\
%\label{eqn:laser_statics_1d_sml_cw_f} F &= \half\, \frac{1 + i\, \Omega}{1 + \Omega^2 + \kappa\, E_0^2}\, \Gn \, E_0 \, .
% \end{align}
% \end{subequations}
%Therefore, substituting \eqn{laser_statics_1d_sml_cw_e} into \eqn{laser_statics_1d_sml_cw_f} and cancelling the common factor of $E_0$ gives
%%In this case, the dimensionless macroscopic polarization is given by
%% \begin{equation} \label{eqn:la1d_p_ss}
%%P = -i\, \frac{1 + i\, \Omega}{1 + \Omega^2}\, G\, E_0 . %E(t) .
%% \end{equation}
%%Substituting \eqn{la1d_p_ss} into \eqn{la1d_gdot_nodims}, we find the steady-state gain
%% \begin{equation}\label{eqn:la1d_g_ss}
%%G = \frac{1 + \Omega^2}{1 + \Omega^2 + \kappa E_0^2}\, \Gn ,
%% \end{equation}
%%and then \eqn{la1d_edot_nodims} gives
% \begin{equation}
%-i\, \delta \omega_0 + \frac{1}{2\, \tau_p} = \half\, \frac{1 + i\, \Omega}{1 + \Omega^2 + \kappa E_0^2}\, \Gn ,
% \end{equation}
%or, separately equating the real and imaginary parts of this expression,
% \begin{subequations} \label{eqn:la1d_ss}
% \begin{align}
% \label{eqn:la1d_ss_real} \frac{1}{\tau_p} &= \frac{\Gn}{1 + \Omega^2 + \kappa E_0^2} , \nd \\
% \label{eqn:la1d_ss_imag} \delta \omega_0 &= -\half\, \frac{\Omega\, \Gn}{1 + \Omega^2 + \kappa E_0^2} .
% \end{align}
% \end{subequations}
%Solving \eqn{la1d_ss_real} for the scaled intracavity intensity (in units of $I_s$), we find
% \begin{equation}\label{eqn:la1d_ss_e02}
% \begin{split}
%   E_0^2 &= \frac{1}{\kappa} \left[ \tau_p\, \Gn - \left(1 + \Omega^2\right) \right] \\
%   &= \frac{1}{\kappa} \left[ \frac{1}{\ln(1/|\Gamma|^2)}\, \Gn - \left(1 + \Omega^2\right) \right] .
% \end{split}
% \end{equation}
%where we have applied \eqn{f_fwhm}.
Note that $I_0 > 0$ only if the pump is strong enough that the round-trip unsaturated gain $\Gnb$ exceeds the threshold gain. If $\Gnb \leq \Gth$, then $I \longrightarrow 0$. Remarkably, if we substitute \eqn{sml_1d_int} into \eqn{laser_statics_1d_sml_g}, we find that --- when $\Gnb > \Gth$ --- the \emph{saturated} round-trip gain is given by
\begin{equation} \label{eqn:la1d_g_ss_clamp}
  \overline{G} = \Gth\, ,
\end{equation}
\emph{independent} of $\Gnb$. Above threshold, the steady-state gain is clamped at the threshold value regardless of the strength of the pump.

Suppose that each mirror has a small absorption/scattering of incident laser intensity $A$, so that the transmission is given by $T = 1 - A - R$. To find the intensity output from the URL's single mirror, we compute $I_\mathrm{out} = (1 - A - R)\, \left|u_0(1)\right|^2\, I_0$, giving
\begin{equation} \label{eqn:ls1d_i_out_approx}
  I_\textrm{out} = \frac{1 - A - R}{1 - R}\, \ln\left(\frac{1}{R}\right)\, \frac{\Hnb - 1}{\widetilde{\kappa}\, \rels}\, .
\end{equation}
For the SWL, the output fields are $I^+_\mathrm{out} = (1 - A_2 - R_2)\, \left|u^+_0(1/2)\right|^2\, I_0$ and $I^-_\mathrm{out} = (1 - A_1 - R_1)\, \left|u^-_0(0)\right|^2\, I_0$. If either one of the mirrors has an intensity reflectance of $1$, then the other mirror is the sole output coupler with reflectance $R$, and the intensity output through that mirror is also given by \eqn{ls1d_i_out_approx}. In both cases, the threshold gain is
% \begin{equation}
%   \Gth =\reils\, \ln\left(\frac{1}{R\, e^{-\anb}}\right)\, .
% \end{equation}
\begin{equation}
  \Gth =\reils\, \ln\left(\frac{1}{\left|\Gamma^\prime\right|^2 R}\right)\, ,
\end{equation}
where $|\Gamma^\prime|^2 = e^{-\anb}$. If necessary, we can generalize the definition of $\Gamma^\prime$ to include all round-trip boundary and background losses in the laser cavity \emph{except} the transmission and absorption of the output coupler.

If the resonator is strictly lossless, so that $\anb = A = 0$, then $\Gth = \reils\, \ln(1/R)$, and the output intensity becomes
\begin{equation}
  I_\textrm{out} = \frac{1}{\widetilde{\kappa}} \left( \Gnb - \Gth \right)\, .
\end{equation}
In this (unphysical) case, the output intensity increases monotonically with $R$ until the threshold gain vanishes as $R \longrightarrow 1$, while the intensity incident on the output coupler increases to $\Gnb / \widetilde{\kappa}\, (1 - R)$ in the same limit. The result would be a damaged mirror reflection coating. In \sct{laser_statics_1d_shb}, published models of intracavity intensity distributions in standing-wave lasers assume that $\anb = 0$, so we'll avoid the quandary in the limit $R \longrightarrow 1$ by assuming that the output coupler \emph{always} has a small finite loss and writing
\begin{equation} \label{eqn:ls1d_i_out_approx_lossless}
  I_\textrm{out} = \frac{1 - A - R}{1 - R}\,  \frac{ \Gnb - \Gth }{\widetilde{\kappa}}\, .
\end{equation}

In the practical case where losses are present, there's an optimum output coupling that maximizes $I_\mathrm{out}$. We can optimize \eqn{ls1d_i_out_approx} numerically to find $R_\mathrm{opt}$, but there's a simple trick we can use to obtain an analytic approximation that is quite reasonably accurate. We note in \fig{laser_statics_1d_approx} that the reflectance normalization of \eqn{ls1d_i_out_approx} is only slightly different from the function $\ln[(1 - A)/R]$ even when $A$ is very large. Hence, we rewrite $I_\mathrm{out}$ as
\begin{equation} \label{eqn:ls1d_i_out_approxx}
  I_\textrm{out} = \ln\left(\frac{1 - A}{R}\right)\, \frac{\Hnb - 1}{\widetilde{\kappa}\, \rels}\, ,
\end{equation}
differentiate the \rhs with respect to $R$, set the result to 0, and then solve for $R_\mathrm{opt}$. We find
\begin{equation} \label{eqn:la1d_r_opt}
  R_\text{opt} = \frac{1}{\left|\Gamma^\prime\right|^2} \exp\left\{ -\sqrt{\rels\, \Gnb\, \ln\left[\frac{1}{(1 - A) \left|\Gamma^\prime\right|^2}\right]} \right\}\, .
\end{equation}
The corresponding optimum output intensity is given by
\begin{equation} \label{eqn:la1d_i_opt}
  I_\text{opt} = \frac{1}{\widetilde{\kappa}\, \rels} \left\{ \sqrt{\rels\, \Gnb} - \sqrt{\ln\left[\frac{1}{(1 - A) \left|\Gamma^\prime\right|^2}\right]} \right\}^2 .
\end{equation}
Note that these equations make sensible predictions only when $(1 - A) \left|\Gamma^\prime\right|^2 < 1$. Also, the maximum value of the reflectance given by $R = 1 - A$ is the ``optimum'' output coupler reflectance when the gain $\Gnb$ has the minimum value
\begin{equation} \label{eqn:lald_g_min}
  G_\text{min} = \reils\, \ln\left[\frac{1}{(1 - A) \left|\Gamma^\prime\right|^2}\right]\, ,
\end{equation}
corresponding to $I_\text{opt} = 0$, as it must. %If $A \ll 1$ and $\anb \ll 1$, then

  
% From either \eqn{laser_resonator_1d_url_out} or \eqn{laser_resonator_1d_swl_out_1} --- subject to the discussion immediately following \eqn{laser_resonator_1d_swl_out} --- we obtain the output intensity (again, in units of $I_s$) at the output coupler with reflectance $R$ as
% \begin{equation} \label{eqn:la1d_i_out_1}
%   I_\text{out} = \frac{\tau_p}{\kappa}\, \ln\left(\frac{1}{R}\right) \left( \Gnb - \Gth \right) .
% \end{equation}
% Strictly speaking, we should have been calculating $T\, E_0^2$ as the output intensity, where $T \equiv 1 - A - R$, and $A$ represents a small absorbance in the mirror itself. We can incorporate the effect of mirror absorption in our simple single-mode model --- rigorously enforcing $I_\text{out} = 0$ when $T = 0$ and $R = 1 - A$  --- by replacing $R$ with $R/(1 - A)$ in \eqn{la1d_i_out_1} to obtain
% \begin{equation}\label{eqn:la1d_i_out}
%   I_\text{out} = \frac{\tau_p}{\kappa}\, \ln\left(\frac{1 - A}{R}\right) \left( \Gnb - \Gth \right) .
% \end{equation}

\begin{figure}
  \centering
  \includegraphics[width=5.0in]{figures/laser_statics_1d_approx}
  \caption{\label{fig:laser_statics_1d_approx} A comparison of the exact reflectance normalization given by \eqn{ls1d_i_out_approx} and the approximation used in \eqn{ls1d_i_out_approxx} for a relatively large value of the mirror intensity scattering and absorption coefficient. }
\end{figure}

What happens to the approximate model when the gain and loss are arbitrary functions of $z$ (e.g., they don't fill the resonator)? We can patch the function $u_0(z)$ to take this spatial nonuniformity into account while continuing to satisfy the boundary condition (e.g., for a URL) at $z = 1$:
\begin{equation} \label{eqn:sml_1d_u_url_nu}
    \left|u^\prime_0(z)\right|^2 = \mathcal{C}_\mathrm{URL}^2\, K(z)\, ,
\end{equation}
where
\begin{subequations}
  \begin{align}
    \label{eqn:sml_1d_u_url_nu_kz} K\z &\equiv \exp\left[ \beta \int_0^z d z^\prime\, G_0(z^\prime) - \int_0^z d z^\prime\, \alpha_0(z^\prime) \right]\, , \nd \\
    \label{eqn:sml_1d_u_url_nu_beta} \beta &\equiv \frac{\ln(1/R\, e^{-\overline{\alpha}_0})}{\overline{G}_0} = \frac{\rels}{\Hnb}\, .
  \end{align}
\end{subequations}
Similarly, we can extend $u^\pm_0(z)$ for the case of standing-wave lasers by defining
\begin{subequations} \label{eqn:sml_1d_u_swl_nu}
  \begin{align}
    \label{eqn:sml_1d_u_swl_nu_p} \left|u^{+ \prime}_0(z)\right|^2 &= \mathcal{C}_\mathrm{SWL}^2\, K\z\, , \nd \\
    \label{eqn:sml_1d_u_swl_nu_m} \left|u^{- \prime}_0(z)\right|^2 &= \frac{\mathcal{C}_\mathrm{SWL}^2}{R_1\, K\z}
  \end{align}
\end{subequations}
and using $R = R_1\, R_2$ in \eqn{sml_1d_u_url_nu_beta}. We'll see that this approach works surprisingly well for both unidirectional ring lasers (URLs) and standing-wave lasers (SWLs) when the gain and loss are not uniform.

% Let's define $\Gamma^\prime$ to include all round-trip boundary and background losses in the laser cavity \emph{except} the transmission and absorption of the output coupler, and then write $\ln |\Gamma|^2 \equiv \ln R + \ln |\Gamma^\prime|^2$. Temporarily defining $\xi \equiv \ln(1/R)$, and solving $\partial I_\text{out} / \partial \xi = 0$ for $\xi$, we find that $I_\text{out}$ reaches a maximum when
%  \begin{equation} \label{eqn:la1d_r_opt}
% R_\text{opt} = \frac{1}{\left|\Gamma^\prime\right|^2} \exp\left\{ -\sqrt{\frac{\Gnb}{1 + \Omega_0^2} \ln\left[\frac{1}{(1 - A) \left|\Gamma^\prime\right|^2}\right]} \right\}\, .
%  \end{equation}
% The corresponding optimum output intensity is given by
%  \begin{equation} \label{eqn:la1d_i_opt}
% I_\text{opt} = \frac{1 + \Omega_0^2}{\kappa} \left\{ \sqrt{\frac{\Gnb}{1 + \Omega_0^2}} - \sqrt{\ln\left[\frac{1}{(1 - A) \left|\Gamma^\prime\right|^2}\right]} \right\}^2 .
%  \end{equation}
% Note that these equations make sensible predictions only when $(1 - A) \left|\Gamma^\prime\right|^2 < 1$. Also, the maximum value of the reflectance given by $R = 1 - A$ is the ``optimum'' output coupler reflectance when the gain $\Gnb$ has the minimum value
%  \begin{equation} \label{eqn:lald_g_min}
% G_\text{min} = \left(1 + \Omega_0^2\right) \ln\left[\frac{1}{(1 - A) \left|\Gamma^\prime\right|^2}\right]\, ,
%  \end{equation}
% corresponding to $I_\text{opt} = 0$, as it must.

% \begin{equation}\label{eqn:la1d_r_opt}
%R_\text{opt} = \exp\left[ \ln\frac{1}{\left|\Gamma^\prime\right|^2} - \sqrt{\ln\left(\frac{1}{\left|\Gamma^\prime\right|^2}\right) \frac{\Gn}{1 + \Omega^2}}\, \right] = \frac{1}{\left|\Gamma^\prime\right|^2} \exp\left[ -\sqrt{\ln\left(\frac{1}{\left|\Gamma^\prime\right|^2}\right) \frac{\Gn}{1 + \Omega^2}}\, \right] ,
% \end{equation}
%with the value
% \begin{equation}\label{eqn:la1d_i_opt}
%I_\text{opt} = \frac{1}{\kappa} \left[ \sqrt{\Gn} - \sqrt{\left({1 + \Omega^2}\right) \ln\left(\frac{1}{\left|\Gamma^\prime\right|^2}\right)}\, \right]^2 .
% \end{equation}

Returning to the phase of the field, we take the ratio of the imaginary and real parts of \eqn{laser_statics_1d_sml_e} to find the steady-state frequency shift
\begin{equation} \label{eqn:la1d_dw0_def}
  \delta \omega_0 = -\frac{1}{2\, \tau_\lambda}\, \frac{\imls}{\rels}\, .
\end{equation}
Let's define $\Omega_0 \equiv (\omega_0 - \omega_{a b})\, \tau_\perp$. Then the total normalized detuning is
\begin{equation}
  \Omega = \Omega_0 + \delta \omega_0\, \tau_\perp,
\end{equation}
and we find $\delta \omega_0$ by substituting this expression into \eqn{la1d_dw0_def} and then solving. For example, if the lineshape is Lorentzian, then $\imls/\rels = \Omega$, and
% which, through $\Omega_0 \equiv (\omega_0 + \delta \omega_0 - \omega_{a b}) \tau_\perp$, can then be solved for the frequency shift $\delta \omega_0$ to obtain
\begin{equation} \label{eqn:la1d_ss_fp}
  % \delta \omega_0 = -\frac{\tau_\perp}{2\, \tau_p}\frac{\omega_0 - \omega_{a b}}{1 + \tau_\perp /2\, \tau_p} .
  \delta \omega_0 = -\frac{1}{2\, \tau_\lambda}\, \frac{\Omega_0}{1 + \tau_\perp /2\, \tau_\lambda} .
\end{equation}
Recall that the total angular frequency of the electric field is $\omega_0 + \delta \omega_0$, and that we have chosen the carrier frequency $\omega_0$ to be aligned with one of the modes of the unloaded cavity, such that $\exp(\pm i\, \omega_0 \tau) = 1$. If the frequency of that cavity mode does not coincide with the resonance frequency $\omega_{a b}$ of the gain medium, then \eqn{la1d_ss_fp} predicts that the total frequency of the laser will be \emph{pulled} away from that of the cavity mode toward the resonance of the medium. Note that the corresponding value of the total normalized frequency detuning $\Omega$ is
 \begin{equation} \label{eqn:la1d_ss_omega}
\Omega = \frac{\Omega_0}{1 + \tau_\perp/2\, \tau_\lambda}\, .
 \end{equation}
Therefore, the net result of this frequency-pulling\index{Frequency-pulling} effect is to \emph{reduce} the detuning, and \emph{increase} the unsaturated gain. But \eqn{la1d_ss_fp} indicates that this detuning depends on the total intracavity loss through $\tau_\lambda$, and \emph{not} the unsaturated gain at all. This apparent contradiction is resolved by \eqn{la1d_g_ss_clamp}: the gain is clamped at the threshold value, which is indeed independent of the pump.

%%%%%%%%%%%%%%%%%%%%%%%%%%%%%%%%%%%%%%%%%%%%%%%%%%%%%%%%%%%%%%%%%%%%%%%%%%%%%%
%
% Section file included in chapter file using \input{}
%
% Assumes that LaTeX2e macros and packages defined in rgb_laser_physics.sty
%   are available
%
%%%%%%%%%%%%%%%%%%%%%%%%%%%%%%%%%%%%%%%%%%%%%%%%%%%%%%%%%%%%%%%%%%%%%%%%%%%%%%

 \section{One-Dimensional Unidirectional Ring Lasers\label{sct:laser_statics_1d_url}}

%  \begin{figure}
%   \centering
%   \includegraphics[width=5.0in]{figures/unidirectional_ring_intensity_1d}
%   \caption{\label{fig:unidirectional_ring_intensity_1d} A comparison of the direct computation of the intracavity intensity given by \eqn{ld1d_uni_amp_iz} for a unidirectional ring laser with the simple single-mode model given by the first term in \eqn{laser_resonator_1d_ezt_expansion}. }
%  \end{figure}

In general, the simple one-dimensional unidirectional ring laser shown in \fig{resonator_1d_ring_gain} has an amplifier with position-dependent gain and loss distributed from $z = 0$ to $z = 1$. Let's begin to understand the performance characteristics of this laser by assuming that the gain and loss are constant, with the values $\Gnz \equiv \Gnb$ and $\anz \equiv \anb$, so that we can use \eqn{ld1d_uni_amp_iz_loss} to determine the intensity $I(1)$ incident on the output coupler. Substituting the boundary condition $I(0) = R\, I(1)$ into \eqn{ld1d_uni_amp_iz_loss}, we obtain
\begin{equation} \label{eqn:ls1d_i1_url}
    % I(1) = \left[ \frac{\Gnb}{\anb} - \left(1 + \Omega_0^2\right) \right] \frac{1 - e^{-\xi}}{1 - R\, e^{-\xi}}\, ,
    I(1) = \left[ \frac{\Gnb}{\anb} - \reils \right] \frac{1 - e^{-\xi}}{1 - R\, e^{-\xi}}\, ,
\end{equation}
where $\rels$ is defined by \eqn{lineshape_re_rho_def}, and
\begin{equation} \label{eqn:ls1d_url_xidef}
    \xi \equiv \frac{\anb}{\Gnb} \left( \Gnb - \Gth \right)\, .
\end{equation}
In the limit where $\anb/\Gnb \ll 1$,
\begin{equation}
    % I(1) \approx \frac{\Gnb - \Gth}{1 - R} \left\{ 1 - \left[ \frac{1 + R}{2 (1 - R)} \left(\Gnb - \Gth\right) + \left(1 + \Omega_0^2\right) \right] \frac{\anb}{\Gnb} \right\}\, .
    I(1) \approx \frac{\Gnb - \Gth}{1 - R} \left\{ 1 - \left[ \frac{1 + R}{2 (1 - R)} \left(\Gnb - \Gth\right) + \reils \right] \frac{\anb}{\Gnb} \right\}\, .
\end{equation}
(Often, when intracavity loss is nonzero, the analytic formulas that we use to drive our intuition tend to become fairly complicated.)

For a lossless cavity with $\anb = 0$, we have $I(1) = (\Gnb - \Gth) / (1 - R)$, and an output intensity given by
\begin{equation} \label{eqn:ls1d_i_out_url_lossless}
    I_\mathrm{out} = \frac{1 - A - R}{1 - R}\,  \left( \Gnb - \Gth \right)\, ,
\end{equation}
which is identical to \eqn{ls1d_i_out_approx_lossless} with $\kappa = 1$. In \fig{url_1d_izr_loss0}, we have assumed that $\Re[\mathcal{L}(0)] = 1$ and plotted the intracavity and output intensity for a lossless unidirectional ring laser in one dimension. In \fig{url_1d_iz_loss0}, the intracavity intensity is computed using three methods: the direct numerical integration of \eqn{ls1d_amp_didz} using the Scientific Python routine \href{https://docs.scipy.org/doc/scipy/reference/generated/scipy.integrate.solve\_bvp.html}{\texttt{scipy.integrate.solve\_bvp}}; the numerical solution of \eqn{ld1d_uni_amp_iz} using the Scientific Python routine \href{https://docs.scipy.org/doc/scipy/reference/generated/scipy.optimize.brentq.html}{\texttt{scipy.optimize.brentq}}; and the approximate solution provided by \eqn{sml_1d_int} in the form
\begin{equation} \label{eqn:ld1d_url_sol_approx}
  I\z \approx \frac{\Hnb - 1}{\kappa\, \rels}\, \left|u_0\z\right|^2
\end{equation}
with $\kappa = 1$. The approximate laser model described in \sct{laser_statics_1d_approx} essentially assumes that the intracavity gain profile will be exponential, which corresponds to the low-gain limit of \eqn{ld1d_uni_amp_iz_small}. But when the gain is significantly above threshold, the intensity becomes a more linear function of $z$, consistent with the approximation of \eqn{ld1d_uni_amp_iz_large}. Nevertheless, for a lossless laser the simple model accurately predicts the values of both $I(0)$ and $I(1)$, so that we can make reliable predictions of the laser output intensity. In \fig{url_1d_ir_loss0}, we compare the corresponding output intensity as a function of $R$ using \eqn{ls1d_i_out_approxx} (with $\anb = 0$) and \eqn{ls1d_i_out_url_lossless}. The results are virtually identical.

% Note that only one pass is made through the amplifier every round trip. If we define $\left|\Gamma^\prime\right|^2$ as the fraction of the light exiting the laser amplifier which reaches the output coupling mirror (\emph{including} the background loss $\exp[-\alpha\wn)$]\footnote{This approximation is equivalent to the assumption that the background loss is small enough that we can pretend that it is located at the exit facet of the amplifier.}, then the intensity at the input plane of the amplifier is related to that at the output plane by
%  \begin{equation} \label{eqn:ld1d_uni_amp_i_in}
% I(0) = \left|\Gamma\right|^2 I(1) = \left|\Gamma^\prime\right|^2 R\, I(1)\, ,
%  \end{equation}
% so that \eqn{ld1d_uni_amp_iz} gives
%  \begin{equation} \label{eqn:ld1d_uni_amp_i_out}
% I(1) = \frac{1}{1 - \left|\Gamma^\prime\right|^2 R} \left[ \Gn - \left(1 + \Omega^2\right) \ln \frac{1}{\left|\Gamma^\prime\right|^2 R}\right]\, ,
%  \end{equation}
% where $\Gn \equiv g_0 = \int_0^1 d z\, \overline{G}\z$. We can begin to see the difference between the single-spatial-mode dynamic model developed in \sct{laser_statics_1d_approx} and the exact case by comparing \eqn{ld1d_uni_amp_iz} with the first term in \eqn{laser_resonator_1d_ezt_expansion}, as shown in \fig{unidirectional_ring_intensity_1d}.

\begin{figure}
    \centering
    \begin{subfigure}[b]{0.8\textwidth}
        \centering
        \includegraphics[width=5.0in]{figures/url_1d_iz_loss0}
        \caption{Intracavity intensity}
        \label{fig:url_1d_iz_loss0}
    \end{subfigure}
    \par\vspace{0.25in}
    \begin{subfigure}[b]{0.8\textwidth}
        \centering
        \includegraphics[width=5.0in]{figures/url_1d_ir_loss0}
        \caption{Output intensity}
        \label{fig:url_1d_ir_loss0}
    \end{subfigure}
    \caption{\label{fig:url_1d_izr_loss0} Intracavity and output intensity for a lossless unidirectional ring laser in one dimension with $R = 0.5$, $\Omega = 0$, and $\Gnb = 4$. We have assumed that $\Re[\mathcal{L}(0)] = 1$. (a) The intracavity intensity computed using three methods: direct numerical integration of \eqn{ls1d_amp_didz}; numerical solution of \eqn{ld1d_uni_amp_iz}; and the approximate solution provided by \eqn{ld1d_url_sol_approx}. (b) Output intensity as a function of $R$ comparing \eqn{ls1d_i_out_approxx} (with $\anb = 0$) and \eqn{ls1d_i_out_url_lossless}. }
\end{figure}

In the general case of a one-dimensional unidirectional ring laser with intracavity absorption and nonzero scattering and loss in the output coupler, the output intensity is given by
\begin{equation} \label{eqn:ls1d_i_out_url}
    % I_\mathrm{out} = (1 - A - R)\, I(1) = (1 - A - R)\, \left[ \frac{\Gnb}{\anb} - \left(1 + \Omega_0^2\right) \right] \frac{1 - e^{-\xi}}{1 - R\, e^{-\xi}}\, ,
    I_\mathrm{out} = (1 - A - R)\, I(1) = (1 - A - R)\, \left[ \frac{\Gnb}{\anb} - \reils \right] \frac{1 - e^{-\xi}}{1 - R\, e^{-\xi}}\, ,
\end{equation}
where $\xi$ is defined by \eqn{ls1d_url_xidef}. In \fig{url_1d_izr_loss1}, we have plotted the intracavity and output intensity for a unidirectional ring laser in one dimension. Again, we have assumed that $\Re[\mathcal{L}(0)] = 1$. In \fig{url_1d_iz_loss1}, our approach is identical to that of \fig{url_1d_iz_loss0}. We see that our approximate model slightly overestimates the value of the intensity arriving at the output coupler when $\anb \ne 0$. In \fig{url_1d_ir_loss1}, we compare the corresponding output intensity as a function of $R$ using \eqn{ls1d_i_out_approxx} and \eqn{ls1d_i_out_url}. Even at relatively high gains, our simple model matches the exact analytic result reasonably well. This agreement is echoed by the computations of optimum output coupler reflectance and output intensity shown in \fig{url_1d_opt} for $A = 1\%$, $\Omega = 0$, and $\anb = \ln(1.25)$. In \fig{url_1d_opt_r}, we compare the optimum output coupler reflectance as a function of $\Gnb$ computed using \eqn{la1d_r_opt} and a numerical solution of \eqn{ls1d_i_out_url}. The close agreement of these two predictions is consistent with the corresponding output intensity computed using \eqn{la1d_i_opt} and \eqn{ls1d_i_out_url} shown in \fig{url_1d_opt_i}.

\begin{figure}
    \centering
    \begin{subfigure}[b]{0.8\textwidth}
        \centering
        \includegraphics[width=5.0in]{figures/url_1d_iz_loss1}
        \caption{Intracavity intensity}
        \label{fig:url_1d_iz_loss1}
    \end{subfigure}
    \par\vspace{0.25in}
    \begin{subfigure}[b]{0.8\textwidth}
        \centering
        \includegraphics[width=5.0in]{figures/url_1d_ir_loss1}
        \caption{Output intensity}
        \label{fig:url_1d_ir_loss1}
    \end{subfigure}
    \caption{\label{fig:url_1d_izr_loss1} Intracavity and output intensity for a  unidirectional ring laser in one dimension with $R = 0.5$, $A = 1\%$, $\Omega = 0$, $\anb = \ln(1.25)$, and $\Gnb = 4$. (a) The intracavity intensity computed using three methods: direct numerical integration of \eqn{ls1d_amp_didz}; numerical solution of \eqn{ld1d_uni_amp_iz_loss}; and the approximate solution provided by \eqn{ld1d_url_sol_approx}. (b) Output intensity as a function of $R$ comparing \eqn{ls1d_i_out_approxx} and \eqn{ls1d_i_out_url}. }
\end{figure}
    
\begin{figure}
    \centering
    \begin{subfigure}[b]{0.8\textwidth}
        \centering
        \includegraphics[width=5.0in]{figures/url_1d_opt_r}
        \caption{Intracavity intensity}
        \label{fig:url_1d_opt_r}
    \end{subfigure}
    \par\vspace{0.25in}
    \begin{subfigure}[b]{0.8\textwidth}
        \centering
        \includegraphics[width=5.0in]{figures/url_1d_opt_i}
        \caption{Output intensity}
        \label{fig:url_1d_opt_i}
    \end{subfigure}
    \caption{\label{fig:url_1d_opt} Optimum output coupler reflectance as a function of unsaturated round-trip gain for a unidirectional ring laser in one dimension with $A = 1\%$, $\Omega = 0$, and $\anb = \ln(1.25)$. (a) The optimum output coupler reflectance as a function of $\Gnb$ computed using \eqn{la1d_r_opt} and a numerical solution of \eqn{ls1d_i_out_url}. (b) The corresponding output intensity comparing \eqn{la1d_i_opt} and \eqn{ls1d_i_out_url}. }
\end{figure}

Suppose that the amplifier in the laser resonator is spatially nonuniform. For example, consider a gain region with constant $\Gnb$ extending from $z = z_1$ to $z = z_2$. Then $\Gnz$ is given by
\begin{equation}
    \Gnz = \frac{\Gnb}{z_2 - z_1}\,
\end{equation}
which clearly satisfies $\int_0^1 d z\, \Gnz = \Gnb$. For our approximate model of this unidirectional ring laser, we use $I(z) = I(0)\, |u_0\z|^2$, but we replace $|u_0\z|^2$ with $|u^\prime_0(z)|^2$ defined by \eqn{sml_1d_u_url_nu}. The result is shown in \fig{url_1d_iz_trap} for $z_1 = 0.25$ and $z_2 = 0.75$. The numerical solution of \eqn{ld1d_uni_amp_iz_loss} fails to predict the correct intensities everywhere except at the mirrors, but our approximate model using \eqn{sml_1d_u_url_nu} does a very reasonable job of matching the result of a direct numerical integration of \eqn{ls1d_amp_didz}.
\begin{figure}
    \centering
    \includegraphics[width=5.0in]{figures/url_1d_iz_trap}
    \caption{\label{fig:url_1d_iz_trap} Intracavity intensity as a function of $z$ for a constant gain region that extends from $z = 0.25$ to $z = 0.75$. The numerical solution of \eqn{ld1d_uni_amp_iz_loss} predicts the correct intensities at the mirrors, but (as it must) otherwise fails. On the other hand, an approximate model using \eqn{sml_1d_u_url_nu} does a credible job of matching the result of a direct numerical integration of \eqn{ls1d_amp_didz}. }
\end{figure}
    
Finally, we can use \eqn{laser_statics_1d_phase} to determine the small phase shift $\delta \omega_0$ that arises when the frequencies of the carrier ($\omega_0$), the center of the gain distribution ($\omega_\mathrm{a b}$), and the nearest cavity mode are not perfectly aligned. We require that the round-trip phase accumulated by the electric field must be zero, and we apply the boundary condition $I(0) = R\, I(1)$ to obtain
 \begin{equation} %\label{eqn:laser_statics_1d_phase}
0 = \phi(1) - \phi(0) = - \delta \omega_0 - \half\, \frac{\imls}{\rels}\,  \ln \left[ \frac{1}{e^{-\anb}\, R} \right] \, .
 \end{equation}
 Since $e^{-\alpha\wn} R \equiv |\Gamma|^2$, using \eqn{f_fwhm} we again have
 \begin{equation}
\delta \omega_0 = - \frac{1}{2\, \tau_p}\, \frac{\imls}{\rels}\, ,
 \end{equation}
in precise agreement with the frequency shift defined by \eqn{la1d_dw0_def}.

%  \begin{figure}
%   \centering
%   \includegraphics[width=5.0in]{figures/unidirectional_ring_output_1d}
%   \caption{\label{fig:unidirectional_ring_output_1d} A comparison of the direct computation of the output intensity given by \eqn{ld1d_uni_amp_i_out} for a constant gain-per-unit-length unidirectional ring laser with the simple single-mode model given by \eqn{la1d_i_out} with $\kappa = 1$. }
%  \end{figure}

% Given an output coupling mirror with transmittance $T = 1 - A - R$, where $A$ represents a small power absorbance in the mirror itself, the output intensity of the unidirectional ring laser is
%  \begin{equation} \label{eqn:ld1d_ur_i_out}
% I_\text{out} = T\, I(1) = \frac{1 - A - R}{1 - \left|\Gamma^\prime\right|^2 R} \left[ \Gn - \left(1 + \Omega^2\right) \ln \frac{1}{\left|\Gamma^\prime\right|^2 R}\right] .
%  \end{equation}
%For comparison, we can incorporate the effect of mirror absorption in our simple single-mode model --- rigorously enforcing $I_\text{out} = 0$ when $R = 1 - A$  --- by replacing $R$ with $R/(1 - A)$ in \eqn{la1d_i_out}, giving
% \begin{equation} \label{eqn:ld1d_ur_i_out_smpl}
%I_\text{out} \approx \frac{\ln\left[(1 - A)/R\right]}{\ln(1/\left|\Gamma^\prime\right|^2 R)} \left[ g_0 - \left(1 + \Omega^2\right) \ln \frac{1}{\left|\Gamma^\prime\right|^2 R}\right] .
% \end{equation}
% We have plotted both \eqn{ld1d_ur_i_out} and \eqn{la1d_i_out} with $\kappa = 1$ for a laser with relatively high intracavity loss, output coupler absorption, and gain in \fig{unidirectional_ring_output_1d}. Even in this extreme case, the agreement is quite reasonable, and it improves significantly as $\left|\Gamma^\prime\right|^2 \longrightarrow 1$.

% As is evident in \fig{unidirectional_ring_output_1d}, for given values of $A$, $\left|\Gamma^\prime\right|^2$, and $\Gn$, there is an output coupler reflectance $R$ that maximizes the output intensity $I_\text{out}$. Differentiating \eqn{ld1d_ur_i_out} with respect to $R$ and setting the result equal to zero, we obtain the constraint that the optimum reflectance must satisfy:
%  \begin{equation} \label{eqn:ld1d_ur_r_opt}
% \left(1 + \Omega^2\right) \left( 1 - \left|\Gamma^\prime\right|^2 R \right) \left( 1 - A - R \right) - R \left[ 1 - (1 - A) \left|\Gamma^\prime\right|^2 \right] \left[ \Gn - \left(1 + \Omega^2\right) \ln \frac{1}{\left|\Gamma^\prime\right|^2 R}\right] = 0
%  \end{equation}
% Once we have solved \eqn{ld1d_ur_r_opt} for $R$, we can substitute the result into \eqn{ld1d_ur_i_out} to determine the optimum output intensity. %Similarly, we can differentiate \eqn{ld1d_ur_i_out_smpl} with respect to $\ln(1/R)$, set the corresponding expression to zero, and then analytically solve for the optimum reflectance, given by
% \begin{equation} \label{eqn:ld1d_ur_r_opt_smpl}
%R_\text{opt} = \frac{1}{\left|\Gamma^\prime\right|^2} \exp\left\{ -\sqrt{\ln\left[\frac{1}{(1 - A) \left|\Gamma^\prime\right|^2}\right] \frac{g_0}{1 + \Omega^2}} \right\} ,
% \end{equation}
%and the optimum output intensity
% \begin{equation} \label{eqn:ld1d_ur_i_opt_smpl}
%I_\text{opt} = \left\{ \sqrt{g_0} - \sqrt{\left({1 + \Omega^2}\right) \ln\left[\frac{1}{(1 - A) \left|\Gamma^\prime\right|^2}\right]} \right\}^2 .
% \end{equation}
% We compare these exact results and the corresponding approximate results given by \eqn{la1d_r_opt} and \eqn{la1d_i_opt} with $\kappa = 1$ for $R_\text{opt}$ and $I_\text{opt}$ in \fig{unidirectional_ring_opt_1d} for the same extreme high-loss and high-gain case as \fig{unidirectional_ring_output_1d}. Note that \eqn{la1d_r_opt} is quite accurate, and \eqn{la1d_i_opt} slightly underestimates the optimum output intensity, as we'd expect from \fig{unidirectional_ring_output_1d}.

% \begin{figure}
% \centering
% \begin{subfigure}[b]{0.8\textwidth}
% \centering
% \includegraphics[width=5.0in]{figures/unidirectional_ring_r_opt_1d}
% \caption{Optimum output coupling}
% \label{fig:unidirectional_ring_r_opt_1d}
% \end{subfigure}
% \par\vspace{0.25in}
% \begin{subfigure}[b]{0.8\textwidth}
% \centering
% \includegraphics[width=5.0in]{figures/unidirectional_ring_i_opt_1d}
% \caption{Optimum output intensity}
% \label{fig:unidirectional_ring_i_opt_1d}
% \end{subfigure}
% \caption{\label{fig:unidirectional_ring_opt_1d} Optimum output coupling and intensity for a unidirectional ring laser in one dimension. The exact output coupling is found by solving \eqn{ld1d_ur_r_opt} for $R$, and then substituting the result into \eqn{ld1d_ur_i_out}. The approximate optimum reflectivity and output intensity are found using \eqn{la1d_r_opt} and \eqn{la1d_i_opt} with $\kappa = 1$, respectively.}
% \end{figure}

%%%%%%%%%%%%%%%%%%%%%%%%%%%%%%%%%%%%%%%%%%%%%%%%%%%%%%%%%%%%%%%%%%%%%%%%%%%%%%
%
% Section file included in chapter file using \input{}
%
% Assumes that LaTeX2e macros and packages defined in rgb_laser_physics.sty
%   are available
%
% $Id$
%
%%%%%%%%%%%%%%%%%%%%%%%%%%%%%%%%%%%%%%%%%%%%%%%%%%%%%%%%%%%%%%%%%%%%%%%%%%%%%%

 \section{Spatial Interference in One-Dimensional Standing-Wave Lasers\label{sct:laser_statics_1d_shb}}
%  becomes
%  \begin{equation} \label{eqn:ld1d_sw_shb_wave_eqn}
% \ddz E^\pm\z = \pm i\, \delta \omega_0\, E^\pm\z \pm F^\pm\z,
%  \end{equation}
Recall that the general time-independent wave propagation equation is given by \eqn{cw_sml_ez_scaled}, where, from \eqn{idm_pol_m_def} and \eqn{laser_statics_1d_sml_cw_fz}, $F^\pm(z)$ is given by
\begin{equation} \label{eqn:ld1d_sw_shb_pzp_init}
  F^\pm(z) = \half \Lo\, \Gnz\, \frac{k_0}{2 \pi} \int_{z - \pi/k_0}^{z + \pi/k_0} d z^\prime e^{\mp i\, k_0\, z^\prime} \frac{\widetilde{E}(z^\prime)}{1 + \rels \left|\widetilde{E}(z^\prime)\right|^2}\, ,
\end{equation}
where $\rels$ is defined by \eqn{lineshape_re_rho_def}, and the total spatially rapidly-varying electric field envelope function is given in the one-dimensional case by \eqn{idm_e_m_def} as
\begin{equation}
  \widetilde{E}(z) = E^+(z)\, e^{+ i\, k_0\, z} + E^-(z)\, e^{- i\, k_0\, z}\, ,
\end{equation}
Here we have assumed that $E^+(z)$, $E^-(z)$, and the pump, $\Gnz$, vary slowly over distance scales on the order of a wavelength or less. Again, let $E^{\pm}(z) \equiv \sqrt{I^{\pm}(z)}\, e^{-i\, \phi^{\pm}(z)}$, and $\phi(z) \equiv \phi^{-}(z) - \phi^{+}(z)$. Then, taking the real part of \eqn{cw_sml_ez_scaled}, we obtain
\begin{equation} \label{eqn:ls1d_didz}
  \ddz\, I^\pm\z = \mp \half\, \anz\, I^\pm\z \pm 2 \Re\left[\frac{F^\pm\z}{E^\pm\z}\right] I^\pm\z\, ,
\end{equation}
where now
\begin{multline} \label{eqn:ld1d_sw_shb_pzp_full}
  2 \Re\left[\frac{F^\pm\z}{E^\pm\z}\right] = \rels\, \Gnz \\ \times \frac{k_0}{2 \pi} \int_{z - \pi/k_0}^{z + \pi/k_0} d z^\prime \frac{1 + \sqrt{I^{\mp}(z)/I^{\pm}(z)}\, e^{- i\, [2\, k_0\, z^\prime + \phi(z)]}}{1 + \rels \left\{ I^{+}(z) + I^{-}(z) + 2\, \sqrt{I^{+}(z)\, I^{-}(z)}\, \cos[2\, k_0\, z^\prime + \phi(z)]\right\}}\, ,
\end{multline}
% For convenience, we temporarily remove the explicit dependence of the integrand on $1 + \Omega^2$ by rescaling $\overline{G}(z) \longrightarrow (1 + \Omega^2) \overline{G}(z)$, and $I^\pm(z) \longrightarrow (1 + \Omega^2) I^\pm(z)$. 
Since $\phi(z)$ is slowly-varying, we can shift the limits of integration by $\Delta z = -\phi(z)/2 k$, eliminating the contribution from $\sin[2\, k\, z^\prime + \phi(z)]$ in the numerator of the integrand, yielding
% \begin{equation} \label{eqn:ld1d_sw_shb_pzp}
%   F^{\pm}(z) = \half (1 + i\, \Omega) \overline{G}(z)\, E^{\pm}(z)\, \frac{1}{2 \pi} \int_{0}^{2 \pi} d \theta\, \frac{1 + c^{\pm 1}\, \cos \theta}{a + b\, \cos \theta}\, ,
% \end{equation}
\begin{equation} \label{eqn:ld1d_sw_shb_pzp}
  2 \Re\left[\frac{F^\pm\z}{E^\pm\z}\right] = \Gnz\, \frac{1}{2 \pi} \int_{0}^{2 \pi} d \theta\, \frac{1 + c^{\pm 1}\, \cos \theta}{a + b\, \cos \theta}\, ,
\end{equation}
where
\begin{subequations} \label{eqn:ld1d_sw_shb_abcdef}
  \begin{align}
    \label{eqn:ld1d_shb_sw_adef} a &= \reils + I^{+}(z) + I^{-}(z)\, , \\
    \label{eqn:ld1d_shb_sw_bdef} b &= 2\, \sqrt{I^{+}(z)\, I^{-}(z)}\, , \nd \\
    \label{eqn:ld1d_shb_sw_cdef} c & = \sqrt{I^{-}(z)/I^{+}(z)}\, ,
  \end{align}
\end{subequations}
and $\reils \equiv 1/\rels$. Then substituting \eqn{ld1d_sw_shb_pzp} into \eqn{ls1d_didz} yields
 \begin{subequations} \label{eqn:ld1d_sw_shb_dipmdz}
  \begin{align}
    \label{eqn:ld1d_sw_shb_dipdz} \ddz I^{+}\z &= \Gnz\, I^{+}\z \, \frac{1}{2 \pi} \int_{0}^{2 \pi} d \theta\, \frac{1 + c\, \cos \theta}{a + b\, \cos \theta} - \half\, \anz\, I^{+}\z\, , \nd \\
    \label{eqn:ld1d_sw_shb_dimdz} \ddz I^{-}\z &= -\Gnz\, I^{-}\z \, \frac{1}{2 \pi} \int_{0}^{2 \pi} d \theta\, \frac{1 + c^{-1}\, \cos \theta}{a + b\, \cos \theta} + \half\, \anz\, I^{-}\z\, .
  \end{align}
 \end{subequations}

Below we will apply \eqn{ld1d_sw_shb_dipmdz} to the standing-wave laser resonator in \fig{resonator_1d_sw_gain}, subject to the boundary conditions
 \begin{subequations} \label{eqn:ld1d_sw_shb_bcs}
 \begin{align}
\label{eqn:ld1d_sw_shb_bc0} I^{+}(0) &= R_1\, I^{-}(0)\, \nd \\
\label{eqn:ld1d_sw_shb_bcd} I^{-}(1/2) &= R_2\, I^{+}(1/2)
 \end{align}
 \end{subequations}
at the two mirrors, using two different approaches to the problem of spatial interference between the counterpropagating fields. However, we can follow the same approach as in \sct{laser_statics_1d_url} to find the amount of frequency pulling that occurs in a stable standing-wave laser. In this case, the total phase accumulated during one round trip is the sum of the phase acquired by the forward-propagating field as it travels from $R_1$ to $R_2$, and the phase accrued by the backward-propagating field as it travels from $R_2$ to $R_1$. Using \eqn{laser_statics_1d_phase}, we find
\begin{equation}
  \begin{split}
    0 &= \left[\phi^+\left(\half\right) - \phi^+\left(0\right)\right] + \left[\phi^-\left(0\right) - \phi^-\left(\half\right)\right] \\
    &= \delta \omega_0 \left[ -\left(\half - 0\right) + \left(0 - \half\right) \right] - \half\, \frac{\imls}{\rels}\, \ln \left[ e^{+\int_0^{1/2} d z\, \an\z - \int_{1/2}^0 d z\, \an\z} \frac{I^+\left(\half\right)}{I^+\left(0\right)}\, \frac{I^-\left(0\right)}{I^-\left(\half\right)} \right] \\
    &= -\delta \omega_0 - \half\, \frac{\imls}{\rels}\,  \ln \left( \frac{1}{e^{-\anb} R_1 R_2} \right)\, ,
  \end{split}
\end{equation}
where we have applied \eqn{ld1d_sw_shb_bcs}. Since $e^{-\alpha\wn} R_1 R_2 \equiv |\Gamma|^2$, using \eqn{f_fwhm} we again have
\begin{equation}
  \delta \omega_0 = -\frac{1}{2\, \tau_p}\, \frac{\imls}{\rels}\, ,% = -\frac{\Omega_0}{2\, \tau_p} \, ,
\end{equation}
in agreement with \eqn{la1d_dw0_def}. In other words, we have found that \eqn{la1d_dw0_def} gives a result for a frequency-pulling shift that is true for all single-mode one-dimensional continuous-wave lasers under a variety of approximations and assumptions.

 \subsection{Rigrod's Model\label{sct:laser_statics_shb_1d_rigrod}}
One of the first treatments of saturation effects in standing-wave lasers was reported by Rigrod in \cite{ref:rigrod1965seh}. He chose to ignore both intracavity absorption and spatial interference completely, which is equivalent to replacing $\cos \theta$ in \eqn{ld1d_sw_shb_dipmdz} with $0$, giving
\begin{subequations} \label{eqn:ld1d_sw_rigrod_dipmdz}
  \begin{align}
    \label{eqn:ld1d_sw_rigrod_dipdz} \ddz I^{+}\z &= \frac{\Gnz\, I^{+}\z}{\reils + I^{+}\z + I^{-}\z}\, , \nd \\
    \label{eqn:ld1d_sw_rigrod_dimdz} \ddz I^{-}\z &= -\frac{\Gnz\, I^{-}\z}{\reils + I^{+}\z + I^{-}\z}\, .
  \end{align}
\end{subequations}
In this simplified case, we see immediately that
\begin{equation*}
  I^{-}\z \ddz I^{+}\z + I^{+}\z \ddz I^{-}\z = \ddz I^{+}\z\, I^{-}\z = 0\, ,
\end{equation*}
so that
\begin{equation} \label{eqn:ld1d_sw_rigrod_ipim}
  I^{+}\z\, I^{-}\z \equiv \varphi_0^2\, ,
\end{equation}
where $\varphi_0$ is a constant of integration to be determined later. Applying \eqn{ld1d_sw_rigrod_ipim} to the boundary conditions given by \eqn{ld1d_sw_shb_bcs}, we find
\begin{subequations} \label{eqn:ld1d_sw_rigrod_bcs}
  \begin{align}
    I^{+}(0) &= \sqrt{R_1}\, \varphi_0\, , \\
    I^{+}(1/2) &= \frac{\varphi_0}{\sqrt{R_2}}\, , \\
    I^{-}(0) &= \frac{\varphi_0}{\sqrt{R_1}}\, , \nd \\
    I^{-}(1/2) &= \sqrt{R_2}\, \varphi_0\, .
  \end{align}
\end{subequations}
Hence, our task is essentially to determine $\varphi_0$.

Solving \eqn{ld1d_sw_rigrod_ipim} for $I^{-}\z$, and substituting the result into \eqn{ld1d_sw_rigrod_dipdz}, we have
\begin{equation}
  \ddz I^{+}\z = \frac{\Gnz\, I^{+}\z}{\reils + I^{+}\z + \varphi_0^2/I^{+}\z}\, ,
 \end{equation}
giving the solution
\begin{equation} \label{eqn:ld1d_sw_rigrod_ipz}
  % \ln \frac{I^{+}(1/2)}{I^{+}(0)} + \left[I^{+}(1/2) - I^{+}(0)\right] \left[ 1 + \frac{\varphi_0^2}{I^{+}(0)\, I^{+}(1/2)} \right] = \frac{\Gn}{2}\, ,
  \reils\, \ln \frac{I^{+}\z}{I^{+}(0)} + \left[I^{+}\z - I^{+}(0)\right] \left[ 1 + \frac{\varphi_0^2}{I^{+}(0)\, I^{+}\z} \right] = \int_0^z d z^\prime\, G_0(z^\prime)\, .
\end{equation}
Therefore, setting $z = 1/2$ using $I^{+}(0) = \sqrt{R_1}\, \varphi_0$ and $I^{+}(1/2) = \varphi_0/\sqrt{R_2}$, we find
\begin{equation} \label{eqn:ld1d_sw_rigrod_phi_def}
  \begin{split}
    \varphi_0 &= \frac{\sqrt{R_1 R_2}}{2 \left( \sqrt{R_1} + \sqrt{R_2} \right) \left( 1 - \sqrt{R_1 R_2} \right)}\, \left[ \Gnb - \reils\, \ln \frac{1}{R_1 R_2} \right] \\
    &= \frac{1}{2\, \sqrt{R_1}\, \rels}\, \mathcal{C}^2_\mathrm{SWL} \left( \Hnb - 1 \right)\, ,
  \end{split}
\end{equation}
where again $\Hnb = \Gnb / \Gth$, $\Gth = \reils\, \ln(1/R_1 R_2)$ when $\anb = 0$, and $\mathcal{C}_\mathrm{SWL}$ is given by \eqn{laser_resonator_1d_u_norm_swl}. In \fig{swl_1d_izs_loss0}, we have assumed that $\rho(0) = 1$ and plotted the intracavity and output intensity for a lossless standing-wave laser in one dimension. The intracavity intensity is computed using three methods: the direct numerical integration of \eqn{ld1d_sw_rigrod_dipmdz} using the Scientific Python routine \href{https://docs.scipy.org/doc/scipy/reference/generated/scipy.integrate.solve\_bvp.html}{\texttt{scipy.integrate.solve\_bvp}}; the numerical solution of \eqn{ld1d_sw_rigrod_ipz} --- with $\varphi_0$ computed with \eqn{ld1d_sw_rigrod_phi_def} --- using the Scientific Python routine \href{https://docs.scipy.org/doc/scipy/reference/generated/scipy.optimize.brentq.html}{\texttt{scipy.optimize.brentq}}; and the approximate solution provided by \eqn{sml_1d_int} in the form
\begin{equation} \label{eqn:ld1d_shb_sw_sol_approx}
  I^{\pm}\z \approx \frac{\Hnb - 1}{\kappa\, \rels}\, \left|u_0^{\pm}\z\right|^2
\end{equation}
with $\kappa = 2$. (We'll see soon that this expression is valid even when $\anb \ne 0$.) The agreement between the two numerical solvers is expected, but the accuracy of the simple approximation is remarkable and warrants further investigation.

\begin{figure}
  \centering
  \begin{subfigure}[b]{0.8\textwidth}
      \centering
      \includegraphics[width=5.0in]{figures/swl_1d_iz_loss0}
      \caption{Intracavity intensities}
      \label{fig:swl_1d_iz_loss0}
  \end{subfigure}
  \par\vspace{0.25in}
  \begin{subfigure}[b]{0.8\textwidth}
      \centering
      \includegraphics[width=5.0in]{figures/swl_1d_isum_loss0}
      \caption{Sum and average of the intracavity intensities}
      \label{fig:swl_1d_isum_loss0}
  \end{subfigure}
  \caption{\label{fig:swl_1d_izs_loss0} Intracavity and total intensities for a lossless standing-wave laser in one dimension with $R_1 = 0.4$, $R_2 = 0.8$, $\Omega = 0$, and $\Gnb = 4$. We have assumed that $\Re[\mathcal{L}(0)] = 1$. (a) The intracavity intensity computed using three methods: direct numerical integration of \eqn{ld1d_sw_rigrod_dipmdz}; numerical solution of \eqn{ld1d_sw_rigrod_ipz}; and the approximate solution provided by \eqn{ld1d_shb_sw_sol_approx}. (b) Comparison of the sum of the intracavity intensities using direct numerical integration and approximation. }
\end{figure}

A clue to the relevant physics is provided by the plot shown in \fig{swl_1d_isum_loss0}. We see that the sum of the counterpropagating intensities is very well approximated by the average value. Let's maintain the presence of the background loss and rewrite \eqn{ld1d_sw_shb_dipmdz} in the form
\begin{subequations} \label{eqn:ld1d_shb_sw_ode_approx}
  \begin{align}
    \ddz I^{+}\z &\cong \frac{\Gnz\, I^{+}\z}{\reils + \half\, \kappa \left[I^{+}\z + I^{-}\z\right]} - \anz\, I^{+}\z\, , \nd \\
    \ddz I^{-}\z &\cong -\frac{\Gnz\, I^{-}\z}{\reils + \half\, \kappa \left[I^{+}\z + I^{-}\z\right]} + \anz\, I^{-}\z\, ,
  \end{align}
\end{subequations}
where $\kappa = 2$ for the standing-wave laser in the Rigrod case, but we are anticipating the results of \sct{laser_statics_shb_1d_al}. We now assume that the gain and absorption functions are constants that fill the resonator, so that $\overline{G}\z \equiv \overline{G}_0$, and $\alpha_0\z \equiv \overline{\alpha}_0$. We then make the ansatz that
\begin{subequations}
  \begin{align}
    I^{+}\z &\approx I^{+}(0)\, e^{\ln\left(1/R_1\, R_2\right) z}\, , \nd \\
    I^{-}\z &\approx I^{-}(0)\, e^{-\ln\left(1/R_1\, R_2\right) z} = \frac{I^{+}(0)}{R_1}\, e^{-\ln\left(1/R_1\, R_2\right) z}\, .
  \end{align}
\end{subequations}
After substituting these trial functions into \eqn{ld1d_shb_sw_ode_approx} and canceling common factors, we obtain
\begin{equation}
  \frac{1}{\kappa\, \rels\, I^{+}(0)} \left(\frac{\overline{G}_0}{G_\mathrm{th}} - 1\right) \approx \half \left[ e^{\ln\left(1/R_1\, R_2\right) z} + \frac{1}{R_1}\, e^{-\ln\left(1/R_1\, R_2\right) z} \right]\, ,
\end{equation}
where now $G_\mathrm{th} = \reils\, \ln(1/R_1\, R_2\, e^{-\overline{\alpha}_0})$. For a suitable value of $I^{+}(0)$, this expression should become reasonably accurate since we've learned that the \rhs --- the sum of the counterpropagating intensities in a standing-wave laser cavity --- doesn't depend strongly on $z$. Let's estimate $I^{+}(0)$ by computing the mean value of the right-hand side of this expression over the single-pass length of the resonator. We find
\begin{equation}
  2\, \int_0^{1/2} d z\; \half \left[ e^{\ln\left(1/R_1\, R_2\right) z} + \frac{1}{R_1}\, e^{-\ln\left(1/R_1\, R_2\right) z} \right] = \frac{1}{\mathcal{C}_\mathrm{SWL}^2}\, .
\end{equation}
Therefore,
\begin{equation} \label{eqn:ld1d_shb_sw_ip0_approx}
  I^{+}(0) = \frac{\mathcal{C}_\mathrm{SWL}^2}{\kappa\, \rels} \left( \Hnb - 1 \right)\, ,
\end{equation}
and
% \begin{subequations}
%   \begin{align}
%     I^{+}\z &\approx \frac{\mathcal{C}_\mathrm{SWL}^2}{\kappa\, \rels} \left( \Hnb - 1 \right) e^{\ln\left(1/R_1\, R_2\right) z} = \frac{1}{\kappa\, \rels} \left( \Hnb - 1 \right)\, \left|u_0^+\z\right|^2, \nd \\
%     I^{-}\z &\approx \frac{\mathcal{C}_\mathrm{SWL}^2}{\kappa\, \rels\, R_1} \left( \Hnb - 1 \right) e^{-\ln\left(1/R_1\, R_2\right) z} = \frac{1}{\kappa\, \rels} \left( \Hnb - 1 \right)\, \left|u_0^-\z\right|^2\, ,
%   \end{align}
% \end{subequations}
\begin{align*}
  I^{+}\z &\approx \mathcal{C}_\mathrm{SWL}^2\, e^{\ln\left(1/R_1\, R_2\right) z}\, \frac{\Hnb - 1}{\kappa\, \rels} = \frac{\Hnb - 1}{\kappa\, \rels}\, \left|u_0^+\z\right|^2, \nd \\
  I^{-}\z &\approx \frac{\mathcal{C}_\mathrm{SWL}^2}{R_1}\, e^{-\ln\left(1/R_1\, R_2\right) z}\, \frac{\Hnb - 1}{\kappa\, \rels} = \frac{\Hnb - 1}{\kappa\, \rels}\, \left|u_0^-\z\right|^2\, ,
\end{align*}
where $u_0^\pm\z$ are given by \eqn{laser_resonator_1d_u_sw}. This result is identical to \eqn{ld1d_shb_sw_sol_approx}, with nonzero $\anb$ incorporated into $\Gth$.

Following the approach to output coupling described in \sct{laser_statics_1d_approx}, we define the transmittance of each mirror as $T_j = 1 - A_j - R_j$, $j \in \{1, 2\}$, where $A_j$ represents a small power absorption in mirror $j$. Using \eqn{ld1d_sw_rigrod_bcs}, the output intensity through each mirror is therefore
\begin{subequations} \label{eqn:ld1d_sw_rigrod_ipm_out}
  \begin{align}
    \label{eqn:ld1d_sw_rigrod_ip_out} I^{+}_\text{out} &= T_2\, I^{+}(1/2) = \left(1 - R_2 - A_2\right) \frac{\varphi_0}{\sqrt{R_2}}\, , \nd \\
    \label{eqn:ld1d_sw_rigrod_im_out} I^{-}_\text{out} &= T_1\, I^{-}(0) = \left(1 - R_1 - A_1\right) \frac{\varphi_0}{\sqrt{R_1}}\, .
  \end{align}
\end{subequations}
Suppose that $R_2 = 1$ and therefore $A_2 = 0$, so that mirror $\mathcal{M}_1$ is the only output coupler with $R_1 \equiv R$ and $A_1 \equiv A$. Then we find that
\begin{equation} \label{eqn:ls1d_i_out_swl_lossless}
  I_\mathrm{out} = \frac{1 - A - R}{2\, (1 - R)}\,  \left( \Gnb - \Gth \right)\, ,
\end{equation}
which is identical to \eqn{ls1d_i_out_approx_lossless} with $\kappa = 2$. As in the case of the lossless unidirectional ring laser in \fig{url_1d_ir_loss0}, it is no surprise that the same-gain curves in \fig{swl_1d_ir_loss0} are virtually identical.

\begin{figure}
  \centering
  \includegraphics[width=5.0in]{figures/swl_1d_ir_loss0}
  \caption{\label{fig:swl_1d_ir_loss0} Output intensity as a function of $R$ comparing \eqn{ls1d_i_out_approxx} and \eqn{ls1d_i_out_swl_lossless}. }
\end{figure}

In the general case of a one-dimensional standing-wave laser with intracavity scattering and absorption, when we neglect spatial hole-burning the counterpropagating intensities obey the differential equations
\begin{equation} \label{eqn:ld1d_sw_rigrod_dipmdz_a}
  \ddz I^{\pm}\z = \pm \frac{\Gnz\, I^{\pm}\z}{\reils + I^{+}\z + I^{-}\z} \mp \anz\, I^{\pm}\z\, ,
\end{equation}
subject to the boundary conditions given by \eqn{ld1d_sw_shb_bcs}.  In \fig{swl_1d_izr_loss1}, our approach is identical to that of \fig{swl_1d_iz_loss0} and \fig{swl_1d_ir_loss0}. We have plotted the intracavity intensities in \fig{swl_1d_iz_loss1} for the same laser as in \fig{swl_1d_iz_loss0}, but now with $\anb \ne 0$. We see that the numerical solution of \eqn{ld1d_sw_rigrod_ipz} is no longer useful, and that our approximate model does a remarkable job reproducing the values of the counterpropagating intensities computed everywhere in the cavity by direct integration of \eqn{ld1d_sw_rigrod_dipmdz_a}. Once again, this is the result of $I^{+}\z + I^{-}\z$ maintaining a nearly constant value throughout the cavity. In \fig{swl_1d_ir_loss1}, we set $R_2 = 1$, $R_1 \equiv R$, and $A_1 \equiv A \ne 0$, and compare the result of direct integration and \eqn{ls1d_i_out_approxx}. Even at relatively high gains, our simple model very closely matches the exact numerical result. This agreement is echoed by the computations of optimum output coupler reflectance and output intensity shown in \fig{swl_1d_opt}. In \fig{swl_1d_opt_r}, we compare the optimum output coupler reflectance as a function of $\Gnb$ computed using \eqn{la1d_r_opt} and a numerical solution of \eqn{ld1d_sw_rigrod_dipmdz_a}. The close agreement of these two predictions is consistent with the corresponding approximate output intensity computed using \eqn{la1d_i_opt} shown in \fig{swl_1d_opt_i}.

\begin{figure}
  \centering
  \begin{subfigure}[b]{0.8\textwidth}
      \centering
      \includegraphics[width=5.0in]{figures/swl_1d_iz_loss1}
      \caption{Intracavity intensity}
      \label{fig:swl_1d_iz_loss1}
  \end{subfigure}
  \par\vspace{0.25in}
  \begin{subfigure}[b]{0.8\textwidth}
      \centering
      \includegraphics[width=5.0in]{figures/swl_1d_ir_loss1}
      \caption{Output intensity}
      \label{fig:swl_1d_ir_loss1}
  \end{subfigure}
  \caption{\label{fig:swl_1d_izr_loss1} Intracavity and output intensity for a standing-wave laser in one dimension with both background and mirror absorption loss. (a) The intracavity intensity is computed using three methods: direct numerical integration of \eqn{ld1d_sw_rigrod_dipmdz_a}; numerical solution of \eqn{ld1d_sw_rigrod_ipz}; and the approximate solution provided by \eqn{ld1d_shb_sw_sol_approx}. (b) Output intensity as a function of $R$ comparing the result of direct numerical integration and \eqn{ls1d_i_out_approxx}. Here $R_2 = 1$, $R_1 \equiv R$ and $A_1 \equiv A$. }
\end{figure}
  
\begin{figure}
  \centering
  \begin{subfigure}[b]{0.8\textwidth}
      \centering
      \includegraphics[width=5.0in]{figures/swl_1d_opt_r}
      \caption{Optimum output coupler reflectance}
      \label{fig:swl_1d_opt_r}
  \end{subfigure}
  \par\vspace{0.25in}
  \begin{subfigure}[b]{0.8\textwidth}
      \centering
      \includegraphics[width=5.0in]{figures/swl_1d_opt_i}
      \caption{Optimum output intensity}
      \label{fig:swl_1d_opt_i}
  \end{subfigure}
  \caption{\label{fig:swl_1d_opt} Optimum output coupler reflectance as a function of unsaturated round-trip gain for a standing-wave laser in one dimension with both background and mirror absorption loss. (a) The optimum output coupler reflectance as a function of $\Gnb$ computed using direct numerical optimization of \eqn{ls1d_amp_didz} and \eqn{la1d_r_opt}. (b) The corresponding output intensity using \eqn{la1d_i_opt}. }
\end{figure}

Suppose that the amplifier in the laser resonator is spatially nonuniform. For example, consider a gain region with constant $\Gnb$ extending from $z = z_1$ to $z = z_2$. Then $\Gnz$ is given by
\begin{equation} \label{eqn:ld1d_sw_nonuniform_gnz}
    \Gnz = \frac{\Gnb}{2\, (z_2 - z_1)}\,
\end{equation}
which clearly satisfies $2 \int_0^{1/2} d z\, \Gnz = \Gnb$. For our approximate model of this standing-wave laser, we use $I^\pm(z) = I(0)\, |u^\pm_0\z|^2$, but we replace $|u^\pm_0\z|^2$ with $|u^{\pm \prime}_0(z)|^2$ defined by \eqn{sml_1d_u_swl_nu}. The result is shown in \fig{swl_1d_iz_trap} for $z_1 = 0.125$ and $z_2 = 0.375$. The numerical solution of \eqn{ld1d_sw_rigrod_ipz} fails to predict the correct intensities everywhere except at the mirrors, but our approximate model using \eqn{sml_1d_u_swl_nu} and \eqn{ld1d_shb_sw_sol_approx} does a very reasonable job of matching the result of a direct numerical integration of \eqn{ld1d_sw_rigrod_dipmdz_a}.

\begin{figure}
  \centering
  \includegraphics[width=5.0in]{figures/swl_1d_iz_trap}
  \caption{\label{fig:swl_1d_iz_trap} Intracavity intensity as a function of $z$ for a standing-wave laser with a constant gain region that extends from $z = 0.125$ to $z = 0.375$. The numerical solution of \eqn{ld1d_sw_rigrod_ipz} predicts incorrect intensities everywhere. On the other hand, an approximate model using \eqn{sml_1d_u_swl_nu} and \eqn{ld1d_shb_sw_sol_approx} does a credible job of matching the result of a direct numerical integration of \eqn{ld1d_sw_rigrod_dipmdz_a}. }
\end{figure}





%  \begin{figure}
%   \centering
%   \begin{subfigure}[b]{0.8\textwidth}
%    \centering
%    \includegraphics[width=5.0in]{figures/standing_wave_rigrod_1d}
%    \caption{Counterpropagating intensities}
%    \label{fig:standing_wave_rigrod_1d}
%   \end{subfigure}
%   \par\vspace{0.25in}
%   \begin{subfigure}[b]{0.8\textwidth}
%    \centering
%    \includegraphics[width=5.0in]{figures/standing_wave_rigrod_sum_1d}
%    \caption{Total intracavity intensity}
%    \label{fig:standing_wave_rigrod_sum_1d}
%   \end{subfigure}
%   \caption{\label{fig:standing_wave_rigrod_1d_all} (a) Counterpropagating intensities found by substituting \eqn{ld1d_sw_rigrod_phi_def} into \eqn{ld1d_sw_rigrod_ipz}, and then computing $I^{+}(z)$ and $I^{-}(z)$ for a very asymmetric laser cavity with large and constant gain per unit length in the amplifier. (b) The total intracavity intensity computed using Rigrod's model. Even in this case, $I^{+}(z) + I^{-}(z)$ does not depend strongly on position.}
%  \end{figure}

% We can build a \emph{very} simple model of the output intensity by generalizing \eqn{ls1d_i_out_approx} to the two-mirror standing-wave case with $\kappa = 2$. We quickly find
%  \begin{subequations} \label{eqn:ld1d_sw_rigrod_ipm_approx}
%  \begin{align}
% \label{eqn:ld1d_sw_rigrod_ip_approx} I^{+}_\text{out} &\cong \frac{\ln\left[(1 - A_2)/R_2\right]}{2 \ln(1/R_1 R_2)} \left[ \Gn - \left(1 + \Omega^2\right) \ln \frac{1}{R_1 R_2} \right]\, , \nd \\
% \label{eqn:ld1d_sw_rigrod_im_approx} I^{-}_\text{out} &\cong \frac{\ln\left[(1 - A_1)/R_1\right]}{2 \ln(1/R_1 R_2)} \left[ \Gn - \left(1 + \Omega^2\right) \ln \frac{1}{R_1 R_2} \right]\, .
%  \end{align}
%  \end{subequations}
% We compare the output intensity computed for a high-constant-gain single-sided laser resonator calculated using \eqn{ld1d_sw_rigrod_ip_out} and \eqn{ld1d_sw_rigrod_ip_approx} in \fig{standing_wave_simple_1d}. The relative accuracy of the simple factor-of-two saturation model seems surprising, but is a consequence of the fact that --- as shown in \fig{standing_wave_rigrod_sum_1d} --- $I^{+}(z) + I^{-}(z)$ is almost constant within the laser resonator. For example, at the reference planes of the mirrors, the total intensities are approximately
%  \begin{align*}
% I^{+}(0) + I^{-}(0) &= \frac{1 + R_1}{\sqrt{R_1}}\, \varphi_0 \approx 2 \left[1 + \frac{(1 - R_1)^2}{8}\right] \varphi_0\, , \nd \\
% I^{+}(1/2) + I^{-}(1/2) &= \frac{1 + R_2}{\sqrt{R_2}}\, \varphi_0 \approx 2 \left[1 + \frac{(1 - R_2)^2}{8}\right] \varphi_0\, .
%  \end{align*}
% Therefore, even in the highly asymmetric case considered here, the total intensity at either end of the cavity is about $2\, \varphi_0$.

%  \begin{figure}
%   \centering
%   \includegraphics[width=5.0in]{figures/standing_wave_simple_1d}
%   \caption{\label{fig:standing_wave_simple_1d} Comparison of the output intensity of a single-sided laser resonator calculated using \eqn{ld1d_sw_rigrod_ip_out} and \eqn{ld1d_sw_rigrod_ip_approx}. }
%  \end{figure}

 \subsection{Agrawal and Lax's Model\label{sct:laser_statics_shb_1d_al}}
A more comprehensive analysis of interference effects in standing-wave lasers was published by Agrawal and Lax in \cite{ref:agrawal1981aei}, but this paper seems to have been largely forgotten in the field. We begin by noting the result of an integral common in quantum optics,
 \begin{equation} \label{eqn:ld1d_sw_shb_qo_int}
\frac{1}{2 \pi} \int_{0}^{2 \pi} d \theta\, \frac{1}{a + b\, \cos \theta} = \frac{1}{\sqrt{a^2 - b^2}}\, ,
 \end{equation}
and then we find for the integrals on the \rhs of \eqn{ld1d_sw_shb_dipmdz}
 \begin{equation}
 \begin{split}
\frac{1}{2 \pi} \int_{0}^{2 \pi} d \theta\, \frac{1 + c\, \cos \theta}{a + b\, \cos \theta} &= \frac{1}{2 \pi} \int_{0}^{2 \pi} d \theta\, \frac{c}{b} \left(1 - \frac{a - b/c}{a + b\, \cos \theta} \right) \\
&= \frac{c}{b} \left(1 - \frac{a - b/c}{\sqrt{a^2 - b^2}} \right) \\
&= \frac{1}{\sqrt{a^2 - b^2}} \left[1 - \frac{c}{b} \left(a - \sqrt{a^2 - b^2}\right)\right]
 \end{split}
 \end{equation}
We have $c/b = 1/2 I^{+}(z)$ and $1/b c =  1/2 I^{-}(z)$, so averaging \eqn{ld1d_sw_shb_dipmdz} over a physical wavelength yields
 \begin{subequations} \label{eqn:ld1d_sw_shb_dipmdz_avg}
 \begin{align}
\ddz I^{+}(z) &= \frac{\Gnz}{\sqrt{a^2 - b^2}} \left[ 1 - \frac{a - \sqrt{a^2 - b^2}}{2\, I^{+}(z)} \right] I^{+}(z) - \half\, \anz\, I^{+}\z\, , \nd \\
\ddz I^{-}(z) &= -\frac{\Gnz}{\sqrt{a^2 - b^2}} \left[ 1 - \frac{a - \sqrt{a^2 - b^2}}{2\, I^{-}(z)} \right] I^{-}(z) + \half\, \anz\, I^{-}\z\, .
 \end{align}
 \end{subequations}
We note that in the limit of very small gain and intensities, when $\anz = 0$ these equations become
% \begin{subequations} %\label{eqn:ld1d_shb_sw_low_g}
%   \begin{align}
%  \ddz I^{+}(z) &\cong \frac{\overline{G}(z)\,  I^{+}(z)}{1 + I^{+}(z) + 2\,  I^{-}(z)}\, , \nd \\
%  \ddz I^{-}(z) &\cong -\frac{\overline{G}(z)\,  I^{-}(z)}{1 + I^{-}(z) + 2\,  I^{+}(z)}\, .
%   \end{align}
% \end{subequations}
\begin{subequations} \label{eqn:ld1d_shb_sw_low_g}
  \begin{align}
    \ddz I^{+}(z) &\cong \Gnz\, I^{+}\z \left\{\reils -  \left[I^{+}\z + 2\, I^{-}\z\right]\right\}\, , \nd \\
    \ddz I^{-}(z) &\cong -\Gnz\, I^{-}\z \left\{\reils - \left[I^{-}\z + 2\, I^{+}\z\right]\right\}\, .     
  \end{align}
\end{subequations}
so that the net effect of the interference is to \emph{increase} the saturation caused by the counterpropagating field by a factor of 2.

Next, we follow Agrawal and Lax by setting $\anz = 0$ and defining $X \equiv I^{+} + I^{-}$ and $Y \equiv I^{+} - I^{-}$; we find
 \begin{subequations}
 \begin{align}
\label{eqn:ld1d_shb_sw_dXdz} \ddz X &= \frac{G_0\, Y}{\sqrt{a^2 - b^2}}\, , \nd \\
\label{eqn:ld1d_shb_sw_dYdz} \ddz Y &= G_0 \left( 1 - \frac{1}{\rho\, \sqrt{a^2 - b^2}} \right) \, .
 \end{align}
 \end{subequations}
Now define $u = \sqrt{a^2 - b^2}$, and note that \eqn{ld1d_shb_sw_adef} gives
 \begin{equation}
u^2 = \left(\rho^{-1} + I^{+} + I^{-}\right)^2 - 4\, I^{+}\, I^{-} = (\rho^{-1} + X)^2 - (X + Y)(X - Y) = \rho^{-2} + 2\, \rho^{-1}\, X + Y^2\, .
 \end{equation}
Therefore,
 \begin{equation}
2 u \frac{d u}{d X} = 2 + 2 Y \frac{d Y}{d X} = 2 u\, ,
 \end{equation}
where we have divided \eqn{ld1d_shb_sw_dYdz} by \eqn{ld1d_shb_sw_dXdz} to obtain $d Y/d X = \left(u - \rho^{-1}\right) / Y$. Since $d u / d X = 1$, we now have
 \begin{equation} \label{eqn:ld1d_sw_shb_const}
\sqrt{a^2 - b^2} = I^{+} + I^{-} + C\, ,
 \end{equation}
where $C$ is a constant of integration. With \eqn{ld1d_sw_shb_const} in hand, \eqn{ld1d_sw_shb_dipmdz_avg} become
 \begin{subequations}
 \begin{align}
\ddz I^{+}(z) &= \frac{\Gnz}{2}\, \frac{2 I^{+}(z) + C - \reils}{I^{+}(z) + I^{-}(z) + C}\, , \nd \\
\ddz I^{-}(z) &= -\frac{\Gnz}{2}\, \frac{2 I^{-}(z) + C - \reils}{I^{+}(z) + I^{-}(z) + C}\, .
 \end{align}
 \end{subequations}
Now we define
 \begin{subequations} \label{eqn:ld1d_sw_shb_irl_def}
 \begin{align}
\label{eqn:ld1d_sw_shb_ir_def} I_R(z) &\equiv I^{+}(z) - \rels\, \varphi^2\, , \nd \\
\label{eqn:ld1d_sw_shb_il_def} I_L(z) &\equiv I^{-}(z) - \rels\, \varphi^2\, ,
 \end{align}
 \end{subequations}
where $\varphi^2 \equiv [1 - \rels\, C]/2\, \rho^2(\Omega)$. Then
 \begin{subequations} \label{eqn:ld1d_sw_shb_dirldz}
 \begin{align}
\label{eqn:ld1d_sw_shb_dirdz} \ddz I_R(z) &= \frac{\Gnz\, I_R(z)}{\reils + I_R(z) + I_L(z)}\, , \nd \\
\label{eqn:ld1d_sw_shb_dildz} \ddz I_L(z) &= -\frac{\Gnz\, I_L(z)}{\reils + I_R(z) + I_L(z)}\, .
 \end{align}
 \end{subequations}
These equations have exactly the same form as those of the Rigrod model, given by \eqn{ld1d_sw_rigrod_dipmdz}, but with different boundary conditions.

First, we note that substituting \eqn{ld1d_sw_shb_irl_def} into \eqn{ld1d_sw_shb_const} and then squaring both sides yields
 \begin{equation} \label{eqn:ls1d_sw_shb_iril}
I_R(z)\, I_L(z) = \varphi^2\, .
 \end{equation}
Then the solution to \eqn{ld1d_sw_shb_dirdz} follows the same approach as that of \eqn{ld1d_sw_rigrod_dipdz}, and can be read from \eqn{ld1d_sw_rigrod_ipz} as
 \begin{equation} \label{eqn:ld1d_sw_shb_irz}
\reils\, \ln \frac{I_R(z)}{I_R(0)} + \left[I_R(z) - I_R(0)\right] \left[ 1 + \frac{\varphi^2}{I_R(0)\, I_R(z)} \right] = \int_{0}^{z} d z^\prime G_0(z^\prime)\, .
 \end{equation}
We define $I_1 \equiv I_R(0)$ and $I_2 \equiv I_R(1/2)$, and apply the boundary conditions given by \eqn{ld1d_sw_shb_bcs} to \eqn{ld1d_sw_shb_irl_def}. We find
\begin{align} \label{eqn:ld1d_sw_shb_i12_bcs}
  I_1 + \rho\, \varphi^2 &= \varphi^2\, (\rho + 1/I_1)\, R_1\, , \nd \\
  I_2 + \rho\, \varphi^2 &= \frac{\varphi^2\, (\rho + 1/I_2)}{R_2}\, ,
\end{align}
or
 \begin{subequations} \label{eqn:ld1d_sw_shb_i12}
 \begin{align}
\label{eqn:ld1d_shb_sw_i1} I_1 &= \half \left[ \sqrt{4\, R_1\, \varphi^2 + \left(1 - R_1\right)^2 \rho^2\, \varphi^4} - \left(1 - R_1\right) \rho\, \varphi^2 \right]\, , \nd \\
\label{eqn:ld1d_shb_sw_i2} I_2 &= \frac{1}{2\, R_2} \left[ \sqrt{4\, R_2\, \varphi^2 + \left(1 - R_2\right)^2 \rho^2\, \varphi^4} + \left(1 - R_2\right) \rho\, \varphi^2 \right]\, .
 \end{align}
 \end{subequations}
By defining $\Gnb \equiv 2 \int_{0}^{1/2} d z^\prime \Gn(z^\prime)$ and substituting these expressions into \eqn{ld1d_sw_shb_irz} at $z = 1/2$,
 \begin{equation} \label{eqn:ld1d_sw_shb_ir21}
\reils\, \ln \frac{I_2}{I_1} + \left(I_2 - I_1\right) \left( 1 + \frac{\varphi^2}{I_1\, I_2} \right) = \frac{\Gnb}{2}\, ,
 \end{equation}
we can find the value of $\varphi$ corresponding to particular choices of $R_1$, $R_2$, and $\Gn$, and then the values of $I^{+}(z)$ and $I^{-}(z)$ everywhere in the laser resonator.
% \Fig{standing_wave_intensity_1d} plots the counterpropagating intensities found using \eqn{ld1d_sw_shb_ir21} and \eqn{ld1d_sw_shb_irz}, and then computing $I^{+}(z)$ and $I^{-}(z)$ for a symmetric laser cavity with large and constant gain per unit length in the amplifier. The resulting total intracavity intensity computed using the complete one-dimensional model is shown in \fig{standing_wave_intensity_sum_1d}; as in the case of \fig{swl_1d_isum_loss0}, we note that $I^{+}(z) + I^{-}(z)$ does not depend strongly on position.
%
%  \begin{figure}
%   \centering
%   \begin{subfigure}[b]{0.8\textwidth}
%    \centering
%    \includegraphics[width=5.0in]{figures/standing_wave_intensity_1d}
%    \caption{Counterpropagating intensities}
%    \label{fig:standing_wave_intensity_1d}
%   \end{subfigure}
%   \par\vspace{0.25in}
%   \begin{subfigure}[b]{0.8\textwidth}
%    \centering
%    \includegraphics[width=5.0in]{figures/standing_wave_intensity_sum_1d}
%    \caption{Total intracavity intensity}
%    \label{fig:standing_wave_intensity_sum_1d}
%   \end{subfigure}
%   \caption{\label{fig:standing_wave_intensity_1d_all} (a) Counterpropagating intensities found using \eqn{ld1d_sw_shb_ir21} and \eqn{ld1d_sw_shb_irz}, and then computing $I^{+}(z)$ and $I^{-}(z)$ for a symmetric laser cavity with large and constant gain per unit length in the amplifier. (b) The total intracavity intensity computed using the complete one-dimensional model; note that $I^{+}(z) + I^{-}(z)$ does not depend strongly on position.}
%  \end{figure}
If we follow the same approach to output coupling as in \sct{laser_statics_shb_1d_rigrod}, leading to \eqn{ld1d_sw_rigrod_ipm_out} in Rigrod's model, we find
% \begin{subequations}
%   \begin{align}
%  I^{+}_\text{out} &= T_2\, I^{+}(1/2) = \left(1 - R_2 - A_2\right) \left[I_2 + \rels\, \varphi^2\right]\, , \nd \\
%  I^{-}_\text{out} &= T_1\, I^{-}(0) = \left(1 - R_1 - A_1\right) \varphi^2 \left[ \rels + \frac{1}{I_1} \right]\, .
%   \end{align}
%  \end{subequations}
 \begin{subequations} \label{eqn:ld1d_sw_shb_ipm_out}
  \begin{align}
    \label{eqn:ld1d_sw_shb_ip_out} I^{+}_\text{out} &= T_2\, I^{+}(1/2) = \left(1 - R_2 - A_2\right) \left[I_2 + \rels\, \varphi^2\right]\, , \nd \\
    \label{eqn:ld1d_sw_shb_im_out} I^{-}_\text{out} &= T_1\, I^{-}(0) = \frac{1 - R_1 - A_1}{R_1} \left[ I_1 + \rels\, \varphi^2 \right]\, .
  \end{align}
\end{subequations}
 We compare the output intensity of a laser resonator with $R_1 = R$ and $R_2 = 1$ calculated using \eqn{ld1d_sw_rigrod_ip_out} and \eqn{ld1d_sw_shb_im_out} in \fig{standing_wave_comparison_1d}. It is not surprising that Rigrod's model substantially overestimates the output power of the laser by ignoring the effects of interference in the amplifier.
\begin{figure}
  \centering
  \includegraphics[width=5.0in]{figures/standing_wave_comparison_1d}
  \caption{\label{fig:standing_wave_comparison_1d} Comparison of the output intensity of a laser resonator calculated using \eqn{ld1d_sw_rigrod_im_out} and \eqn{ld1d_sw_shb_im_out}. Here $R_1 = R$, $R_2 = 1$, $A_1 = 0.01$, $\anb = 0$, and $\Omega = 0$.}
\end{figure}

When interference effects in standing-wave lasers cannot be ignored, to what extent can we continue to rely on the simple approximate model that we constructed in \sct{laser_statics_1d_approx}? Let's focus on \eqn{ls1d_i_out_approx}, and seek a modification of the saturation factor $\kappa$ that will capture the general behavior of a single-mode laser as the gain and losses vary significantly. We begin by defining $\kappa$ in terms of the intracavity intensities incident on the output coupler as
\begin{equation} \label{eqn:ld1d_sw_shb_kappa_def}
  \kappa \equiv 2\, \frac{I_\text{rr}}{I_\text{al}} = 2\, \frac{\sqrt{R_1}\, \varphi_0}{I_1 + \rho\, \varphi^2}\, ,
\end{equation}
% \begin{equation}
%  \kappa \equiv 2\, \frac{I_\text{rr}}{I_\text{al}} \equiv 2 + \Delta \kappa\, ,
% \end{equation}
where the subscripts $\text{rr} \equiv $~``Rigrod'' and $\text{al} \equiv $~``Agrawal and Lax,'' $\varphi_0$ is given by \eqn{ld1d_sw_rigrod_phi_def}, and we have used \eqn{ld1d_sw_rigrod_bcs}, \eqn{ld1d_sw_shb_irl_def}, and \eqn{ld1d_sw_shb_i12_bcs}.
%  \begin{equation} \label{eqn:ld1d_sw_shb_dkappa_def}
% \Delta \kappa \equiv 2 \left( \frac{I_\text{rr}}{I_\text{al}} - 1 \right) = 2 \left[ \frac{\varphi_0}{\sqrt{R_2}\, (I_2 + \rho\, \varphi^2)} - 1 \right]\, ,
%  \end{equation}
Let's find an approximate expression for $\varphi$ by replacing $1 - R_1$ and $1 - R_2$ (explicitly) with $\delta_1$ and $\delta_2$ in \eqn{ld1d_sw_shb_i12}, substitute these expressions into \eqn{ld1d_sw_shb_ir21}, and then expand the result to second order in $\delta_1$ and $\delta_2$ to obtain
%  \begin{equation*}
% \left[ \left(1 + \frac{1}{r_1^2}\right) \frac{\delta_1}{2} + \left(1 + \frac{1}{r_2^2}\right) \frac{\delta_2}{2} \right] v^2 + \left( \frac{1}{r_1} + \frac{1}{r_2} - r_1 - r_2 + \frac{\delta_1}{2 r_1} + \frac{\delta_2}{2 r_2} \right) v = \frac{\Gn}{2} - \ln \frac{1}{r_1\, r_2}\, ,
%  \end{equation*}
\begin{multline*}
  \frac{\rho^2}{4} \left[ \left(\frac{1 - R_1}{\sqrt{R_1}}\right)^3 + \left(\frac{1 - R_2}{\sqrt{R_2}}\right)^3 \right] \varphi^3 + \rho\, \frac{\left(R_1 + R_2\right) (1 - R_1\, R_2)}{R_1\, R_2}\, \varphi^2 \\
  + 3 \left( \frac{1 - R_1}{\sqrt{R_1}} + \frac{1 - R_2}{\sqrt{R_2}} \right) \varphi = \Gnb - \rho^{-1}\, \ln \frac{1}{R_1\, R_2}\, .
\end{multline*}
Suppose that $R_2 = 1$, and consider values of $R_1$ greater than $0.2$ Then the coefficient of $\varphi^3$ is much smaller than those of $\varphi$ and $\varphi^2$, and we can neglect that term. Assuming that this is generally true, after some straightforward algebra, the equation for $\varphi$ becomes
 \begin{equation} \label{eqn:ld1d_sw_shb_phi_qe}
\beta\, \varphi^2 + 3\, \varphi - 2\, \varphi_0 = 0\, ,
 \end{equation}
where
 \begin{equation}
\beta \equiv \rels\, \frac{\left(R_1 + R_2\right) \left(1 + \sqrt{R_1\, R_2}\right)}{\sqrt{R_1\, R_2} \left(\sqrt{R_1} + \sqrt{R_2}\right)}\, .
 \end{equation}
Therefore, we find that $\varphi$ is approximately given by
 \begin{equation} \label{eqn:ld1d_sw_shb_phi_approx}
\varphi \cong \frac{3}{2\, \beta} \left( \sqrt{1 + \frac{8}{9}\, \beta\, \varphi_0} - 1 \right)\, .
 \end{equation}

% with
%  \begin{equation} \label{eqn:ld1d_sw_shb_phi0p_def}
% \varphi_0^\prime \equiv \frac{\varphi_0}{1 + \Omega^2}\, ,
%  \end{equation}
% where $\varphi_0$ is given by \eqn{ld1d_sw_rigrod_phi_def}. In \fig{delta_kappa_1d}, we plot $\Delta \kappa$ as a function of reflectance for a set of four gains and both $\Omega = 0$ and $\Omega = 1$ in the cases where $R_1 = R_2 \equiv R$ and $R_1 = 1$, $R_2 \equiv R$. First, we note that $\kappa \longrightarrow 3$ as the reflectance approaches the threshold value, which is consistent with the low-intracavity-intensity limit described by \eqn{ld1d_shb_sw_low_g}. Second, as the reflectance approaches unity, $\kappa \longrightarrow 2$, which allows us to identify Rigrod's model as the very high-intracavity-intensity limit of Agrawal and Lax's result. Finally, we observe the similarity between the examples with $\Gn = 2, \Omega = 0$ and $\Gn = 4, \Omega = 1$, as well as $\Gn = 4, \Omega = 0$ and $\Gn = 8, \Omega = 1$, which provides us with a hint that $\Delta \kappa$ depends strongly on $\varphi^\prime_0$.

%  \begin{figure}
%   \centering
%   \begin{subfigure}[b]{0.8\textwidth}
%    \centering
%    \includegraphics[width=5.0in]{figures/delta_kappa_1d_sym}
%    \caption{$R_1 = R_2 \equiv R$}
%    \label{fig:delta_kappa_1d_sym}
%   \end{subfigure}
%   \par\vspace{0.25in}
%   \begin{subfigure}[b]{0.8\textwidth}
%    \centering
%    \includegraphics[width=5.0in]{figures/delta_kappa_1d_oc2}
%    \caption{$R_1 = 1$, $R_2 \equiv R$}
%    \label{fig:delta_kappa_1d_oc2}
%   \end{subfigure}
%   \caption{\label{fig:delta_kappa_1d} $\Delta \kappa$ --- defined by \eqn{ld1d_sw_shb_kappa_def} --- as a function of reflectance for a set of four gains and both $\Omega = 0$ and $\Omega = 1$ in the cases where $R_1 = R_2 \equiv R$ and $R_1 = 1$, $R_2 \equiv R$. We note the similarity between the examples with $\Gn = 2, \Omega = 0$ and $\Gn = 4, \Omega = 1$, as well as $\Gn = 4, \Omega = 0$ and $\Gn = 8, \Omega = 1$, provide a hint that $\Delta \kappa$ depends strongly on $\varphi^\prime_0$.}
%  \end{figure}

%  \begin{figure}
%   \centering
%   \begin{subfigure}[b]{0.8\textwidth}
%    \centering
%    \includegraphics[width=5.0in]{figures/omrp_1d_sym}
%    \caption{$R_1 = R_2 \equiv R$}
%    \label{fig:omrp_1d_sym}
%   \end{subfigure}
%   \par\vspace{0.25in}
%   \begin{subfigure}[b]{0.8\textwidth}
%    \centering
%    \includegraphics[width=5.0in]{figures/omrp_1d_oc2}
%    \caption{$R_1 = 1$, $R_2 \equiv R$}
%    \label{fig:omrp_1d_oc2}
%   \end{subfigure}
%   \caption{\label{fig:omrp_1d} $(1 - R)^2\, \varphi / 4 R$ as a function of reflectance for a set of four gains and both $\Omega = 0$ and $\Omega = 1$ in the cases where $R_1 = R_2 \equiv R$ and $R_1 = 1$, $R_2 \equiv R$.}
%  \end{figure}

Given this analytic approximation for $\varphi$, we seek a corresponding expression for $I_1$ that will allow us to estimate $\kappa$ using \eqn{ld1d_sw_shb_kappa_def}. When $\Gnb \lesssim 4$, 
then $(1 - R)^2 \varphi^2 / 4 R_1 \ll 1$, and we can approximate \eqn{ld1d_shb_sw_i1} as
\begin{equation} \label{eqn:ld1d_shb_sw_i1_approx}
  I_1 \cong \sqrt{R_1}\, \varphi - \frac{1 - R_1}{2}\, \rho\, \varphi^2\, .
\end{equation}
When $\Gnb$ is large and $R_1$ is small, we should expect that  our approximation will become less accurate. We note that $\beta  = (1 + R)/\sqrt{R}$ in two cases of practical interest ($R_1 = R$, $R_2 = 1$ and $R_1 = R_2 = R$), and use \eqn{ld1d_sw_shb_phi_qe} to write
\begin{equation}
  \frac{I_1 + \rho\, \varphi^2}{\sqrt{R_1}} \approx \varphi + \frac{\beta}{2}\, \rho\, \varphi^2 = \varphi_0 - \half\, \varphi\, ,
\end{equation}
giving
\begin{equation} \label{eqn:ld1d_sw_shb_kappa}
  \kappa \approx \frac{2\, \varphi_0}{\varphi_0 - \varphi / 2}\, .
\end{equation}
In the limit $\varphi_0 \longrightarrow 0$,
\begin{equation*}
  \varphi \longrightarrow \frac{3}{2}\, \varphi_0\, , \quad \kappa \longrightarrow 3\, ,
\end{equation*}
which is consistent with the low-intracavity-intensity limit described by \eqn{ld1d_shb_sw_low_g}. On the other hand, if $\varphi_0 \longrightarrow \infty$, then
\begin{equation*}
  \varphi \longrightarrow \sqrt{\frac{2\, \varphi_0}{\beta}}\, , \quad \kappa \longrightarrow 2\, ,
\end{equation*}
which allows us to identify Rigrod's model as the very high-intracavity-intensity limit of Agrawal and Lax's result. In \fig{shb_1d_kappa_loss0} we plot the effective saturation parameter defined by \eqn{ld1d_sw_shb_kappa_def} and the approximation given by \eqn{ld1d_sw_shb_kappa} for a range of output coupler reflectances $R_1 = R$ (with $R_2 = 1$) and gains. We see that the approximation is quite accurate for a wide range of parameters, and (as we expected) the error is largest for low reflectances and high gains.

\begin{figure}
  \centering
  \includegraphics[width=5.0in]{figures/shb_1d_kappa_loss0}
  \caption{\label{fig:shb_1d_kappa_loss0} Comparison of the effective saturation parameter defined by \eqn{ld1d_sw_shb_kappa_def} with the approximation given by \eqn{ld1d_sw_shb_kappa}.}
\end{figure}

In \fig{shb_1d_iz_loss0}, we have assumed that $\rho(0) = 1$ and plotted the intracavity and output intensity for a lossless standing-wave laser in one dimension. The intracavity intensity is computed using three methods: the direct numerical integration of \eqn{ld1d_sw_shb_dipmdz_avg} using the Scientific Python routine \href{https://docs.scipy.org/doc/scipy/reference/generated/scipy.integrate.solve\_bvp.html}{\texttt{scipy.integrate.solve\_bvp}}; the numerical solution of \eqn{ld1d_sw_shb_irz} and \eqn{ld1d_sw_shb_i12} using the Scientific Python routine \href{https://docs.scipy.org/doc/scipy/reference/generated/scipy.optimize.brentq.html}{\texttt{scipy.optimize.brentq}}; and the approximate solution provided by \eqn{ld1d_shb_sw_sol_approx} and \eqn{ld1d_sw_shb_kappa}.
The agreement between the two numerical solvers is expected for $\anb = 0$, and the accuracy of the simple approximation is reasonable for high gain and $R_1 \, R_2 = 0.32$. In \fig{shb_1d_isum_loss0}, we compare the sum of the intracavity intensities using direct numerical integration and the approximation. We see that the sum of the intensities is nearly constant even when spatial interference is not negligible, and the approximate model does a good job of matching the average value of the counterpropagating intensities. The accuracy of the approximate model for $I^{-}(0)$ is consistent with the comparison of \eqn{ls1d_i_out_approxx} and \eqn{ld1d_sw_shb_im_out} plotted in \fig{shb_1d_ir_loss0}, which shows a significant improvement over Rigrod's model when predicting output intensity as a function of reflectance.

\begin{figure}
  \centering
  \begin{subfigure}[b]{0.8\textwidth}
      \centering
      \includegraphics[width=5.0in]{figures/shb_1d_iz_loss0}
      \caption{Intracavity intensities}
      \label{fig:shb_1d_iz_loss0}
  \end{subfigure}
  \par\vspace{0.25in}
  \begin{subfigure}[b]{0.8\textwidth}
      \centering
      \includegraphics[width=5.0in]{figures/shb_1d_isum_loss0}
      \caption{Sum and average of the intracavity intensities}
      \label{fig:shb_1d_isum_loss0}
  \end{subfigure}
  \caption{\label{fig:shb_1d_izs_loss0} Intracavity and total intensities for a lossless standing-wave laser in one dimension with $R_1 = 0.4$, $R_2 = 0.8$, $\Omega = 0$, and $\Gnb = 4$. We have assumed that $\Re[\mathcal{L}(0)] = 1$. (a) The intracavity intensity computed using three methods: direct numerical integration of \eqn{ld1d_sw_shb_dipmdz_avg}; numerical solution of \eqn{ld1d_sw_shb_irz}; and the approximate solution provided by \eqn{ld1d_shb_sw_sol_approx} and \eqn{ld1d_sw_shb_kappa}. (b) Comparison of the sum of the intracavity intensities using direct numerical integration and approximation. }
\end{figure}

\begin{figure}
  \centering
  \includegraphics[width=5.0in]{figures/shb_1d_ir_loss0}
  \caption{\label{fig:shb_1d_ir_loss0} Output intensity as a function of $R$ comparing \eqn{ls1d_i_out_approxx} and \eqn{ld1d_sw_shb_im_out}. }
\end{figure}

In the general case of a one-dimensional standing-wave laser with intracavity scattering and absorption, when we include spatial hole-burning the counterpropagating intensities obey the differential equations given by \eqn{ld1d_sw_shb_dipmdz_avg}, subject to the boundary conditions given by \eqn{ld1d_sw_shb_bcs}.  In \fig{shb_1d_izr_loss1}, our approach is identical to that of \fig{shb_1d_iz_loss0} and \fig{shb_1d_ir_loss0}. We have plotted the intracavity intensities in \fig{shb_1d_iz_loss1} for the same laser as in \fig{shb_1d_iz_loss0}, but now with $\anb \ne 0$. We see that the numerical solution of \eqn{ld1d_sw_shb_irz} is no longer useful, and that using \eqn{ld1d_sw_shb_kappa} our approximate model does a remarkable job reproducing the values of the counterpropagating intensities computed everywhere in the cavity by direct integration of \eqn{ld1d_sw_shb_dipmdz_avg}. In \fig{shb_1d_ir_loss1}, we set $R_2 = 1$, $R_1 \equiv R$, and $A_1 \equiv A \ne 0$, and compare the result of direct integration and \eqn{ls1d_i_out_approxx}. Even at relatively high gains, our simple model very closely matches the exact numerical result. In \fig{shb_1d_opt_r}, we compare the optimum output coupler reflectance as a function of $\Gnb$ computed using \eqn{la1d_r_opt} and a numerical solution of \eqn{ld1d_sw_shb_dipmdz_avg}. Although the approximate model slightly underestimates the value of $R_\mathrm{opt}$ at high gains, it reliably predicts the output intensity computed using \eqn{la1d_i_opt} shown in \fig{shb_1d_opt_i}.

\begin{figure}
  \centering
  \begin{subfigure}[b]{0.8\textwidth}
      \centering
      \includegraphics[width=5.0in]{figures/shb_1d_iz_loss1}
      \caption{Intracavity intensity}
      \label{fig:shb_1d_iz_loss1}
  \end{subfigure}
  \par\vspace{0.25in}
  \begin{subfigure}[b]{0.8\textwidth}
      \centering
      \includegraphics[width=5.0in]{figures/shb_1d_ir_loss1}
      \caption{Output intensity}
      \label{fig:shb_1d_ir_loss1}
  \end{subfigure}
  \caption{\label{fig:shb_1d_izr_loss1} Intracavity and output intensity for a standing-wave laser in one dimension with both background and mirror absorption loss. (a) The intracavity intensity is computed using three methods: direct numerical integration of \eqn{ld1d_sw_shb_dipmdz_avg}; numerical solution of \eqn{ld1d_sw_shb_irz}; and the approximate solution provided by \eqn{ld1d_shb_sw_sol_approx} and \eqn{ld1d_sw_shb_kappa}. (b) Output intensity as a function of $R$ comparing the result of direct numerical integration and \eqn{ls1d_i_out_approxx}. Here $R_2 = 1$, $R_1 \equiv R$, and $A_1 \equiv A$.}
\end{figure}
  
\begin{figure}
  \centering
  \begin{subfigure}[b]{0.8\textwidth}
      \centering
      \includegraphics[width=5.0in]{figures/shb_1d_opt_r}
      \caption{Optimum output coupler reflectance}
      \label{fig:shb_1d_opt_r}
  \end{subfigure}
  \par\vspace{0.25in}
  \begin{subfigure}[b]{0.8\textwidth}
      \centering
      \includegraphics[width=5.0in]{figures/shb_1d_opt_i}
      \caption{Optimum output intensity}
      \label{fig:shb_1d_opt_i}
  \end{subfigure}
  \caption{\label{fig:shb_1d_opt} Optimum output coupler reflectance and intensity for a standing-wave laser in one dimension with both background and mirror absorption loss, limited by spatial hole-burning. (a) The optimum reflectance as a function of $\Gnb$ computed using both direct numerical optimization of \eqn{ld1d_sw_shb_dipmdz_avg} and \eqn{la1d_r_opt}. (b) The corresponding output intensities, with the approximation using \eqn{la1d_i_opt}. }
\end{figure}

Again, let's suppose that the amplifier in the laser resonator is spatially nonuniform with a gain given by \eqn{ld1d_sw_nonuniform_gnz}. As in the case where we neglected interference effects, we replace $|u^\pm_0\z|^2$ with $|u^{\pm \prime}_0(z)|^2$ defined by \eqn{sml_1d_u_swl_nu}. The result is shown in \fig{shb_1d_iz_trap} for $z_1 = 0.125$ and $z_2 = 0.375$, and we rely on $\kappa$ computed with \eqn{ld1d_sw_shb_kappa}. The numerical solution of \eqn{ld1d_sw_shb_irz} fails to predict the correct intensities everywhere, but our approximate model using \eqn{sml_1d_u_swl_nu} and \eqn{ld1d_shb_sw_sol_approx} does a very reasonable job of matching the result of a direct numerical integration of \eqn{ld1d_sw_shb_dipmdz_avg}.

\begin{figure}
  \centering
  \includegraphics[width=5.0in]{figures/shb_1d_iz_trap}
  \caption{\label{fig:shb_1d_iz_trap} Intracavity intensity as a function of $z$ for a standing-wave laser with a constant gain region that extends from $z = 0.125$ to $z = 0.375$. The numerical solution of \eqn{ld1d_sw_shb_irz} predicts incorrect intensities everywhere. On the other hand, an approximate model using \eqn{sml_1d_u_swl_nu} and \eqn{ld1d_sw_shb_kappa} does a credible job of matching the result of a direct numerical integration of \eqn{ld1d_sw_shb_dipmdz_avg}. }
\end{figure}

% Next, we seek approximate versions of \eqn{ld1d_sw_shb_i12} that will help us find an analytic approximation for $\varphi$ that is reasonably accurate and allows us to update our simple model to include spatial interference in the amplifier. As a numerical experiment, we compare the terms like $(1 - R)^2 \varphi^2$ to $4\, R\, \varphi$ in \fig{omrp_1d}. We see that --- except for very high gains in the case of asymmetric output coupling --- it is safe to neglect the terms that are quadratic in $\varphi$, giving
% \begin{subequations}
%   \begin{align}
%   I_1 &\cong \sqrt{R_1\, \varphi} - \half \left(1 - R_1\right) \varphi \, , \nd \\
%  I_2 &\cong \sqrt{\frac{\varphi}{R_2}} + \frac{1 - R_2}{2\, R_2}\, \varphi \, .
%   \end{align}
%   \end{subequations}
% \begin{subequations} \label{eqn:ld1d_sw_shb_i12_approx}
%   \begin{align}
%     \label{eqn:ld1d_shb_sw_i1_approx} I_1 &\cong \sqrt{R_1}\, \varphi - \frac{1 - R_1}{2}\, \rho\, \varphi^2 = \sqrt{R_1}\, \varphi \left( 1 - \frac{1 - R_1}{2 \sqrt{R_1}}\, \rho\, \varphi \right) \, , \nd \\
%     \label{eqn:ld1d_shb_sw_i2_approx} I_2 &\cong \frac{\varphi}{\sqrt{R_2}} + \frac{1 - R_2}{2\, R_2}\, \rho\, \varphi^2 = \frac{\varphi}{\sqrt{R_2}} \left( 1 + \frac{1 - R_2}{2 \sqrt{R_2}}\, \rho\, \varphi \right)\, .
%   \end{align}
% \end{subequations}
% We replace $1 - R_1$ and $1 - R_2$ (explicitly) with $\delta_1$ and $\delta_2$,
% \begin{align*}
%I_1 &= \half \left[ \sqrt{4\, v^2\, r_1^2 + v^4 \delta_1^2} - v^2 \delta_1 \right]\, , \nd \\
%I_2 &= \frac{1}{2\, r_2^2} \left[ \sqrt{4\, v^2\, r_2^2 + v^4 \delta_2^2} + v^2 \delta_2 \right]\, ,
% \end{align*}
% substitute these expressions into \eqn{ld1d_sw_shb_ir21}, expand the result to second order in $\delta_1$ and $\delta_2$, and then neglect the term proportional to $\varphi^3$ to obtain
%  \begin{equation*}
% \left[ \left(1 + \frac{1}{r_1^2}\right) \frac{\delta_1}{2} + \left(1 + \frac{1}{r_2^2}\right) \frac{\delta_2}{2} \right] v^2 + \left( \frac{1}{r_1} + \frac{1}{r_2} - r_1 - r_2 + \frac{\delta_1}{2 r_1} + \frac{\delta_2}{2 r_2} \right) v = \frac{\Gn}{2} - \ln \frac{1}{r_1\, r_2}\, ,
%  \end{equation*}
% \begin{multline*}
%   \rho \left[ \left(1 + \frac{1}{R_1}\right) \delta_1 + \left(1 + \frac{1}{R_2}\right) \delta_2\right] \varphi^2 \\
%   + \left[ \frac{2 \left(1 - R_1\right) + \delta_1}{\sqrt{R_1}} + \frac{2 \left(1 - R_2\right) + \delta_2}{\sqrt{R_2}} \right] \varphi = \Gnb - \rho^{-1}\, \ln \frac{1}{R_1\, R_2}\, .
% \end{multline*}
%  After some straightforward algebra, this equation becomes
%  \begin{equation}
% \beta\, \varphi^2 + 3\, \varphi - 2\, \varphi_0 = 0\, ,
%  \end{equation}
% where
%  \begin{equation}
% \beta \equiv \rels\, \frac{\left(R_1 + R_2\right) \left(1 + \sqrt{R_1\, R_2}\right)}{\sqrt{R_1\, R_2} \left(\sqrt{R_1} + \sqrt{R_2}\right)}\, .
%  \end{equation}
% In \fig{phi_phi0_1d}, we plot $\varphi$ as a function of $\varphi_0^\prime$ for a set of four gains and both $\Omega = 0$ and $\Omega = 1$ in the cases where $R_1 = R_2 \equiv R$ and $R_1 = 1$, $R_2 \equiv R$.  Even in the high-gain cases, it is clear that $\varphi$ depends functionally \emph{only} on $\varphi^\prime_0$. We compare these curves to the approximation given by \eqn{ld1d_sw_shb_phi_approx} with the additional simplification $\beta \longrightarrow 2$. It is surprising that the approximate formula for $\varphi$ works so well even in cases where the gain is large and the output coupler reflectance differs significantly from 1.

%  \begin{figure}
%   \centering
%   \begin{subfigure}[b]{0.8\textwidth}
%    \centering
%    \includegraphics[width=5.0in]{figures/phi_phi0_1d_sym}
%    \caption{$R_1 = R_2 \equiv R$}
%    \label{fig:phi_phi0_1d_sym}
%   \end{subfigure}
%   \par\vspace{0.25in}
%   \begin{subfigure}[b]{0.8\textwidth}
%    \centering
%    \includegraphics[width=5.0in]{figures/phi_phi0_1d_oc2}
%    \caption{$R_1 = 1$, $R_2 \equiv R$}
%    \label{fig:phi_phi0_1d_oc2}
%   \end{subfigure}
%   \caption{\label{fig:phi_phi0_1d} $\varphi$ as a function of $\varphi_0^\prime$ for a set of four gains and both $\Omega = 0$ and $\Omega = 1$ in the cases where $R_1 = R_2 \equiv R$ and $R_1 = 1$, $R_2 \equiv R$. We compare these curves to the approximation given by \eqn{ld1d_sw_shb_phi_approx} with the substitution $\beta \longrightarrow 2$. }
%  \end{figure}

% From \eqn{ld1d_sw_shb_phi_approx}, we can rewrite $\varphi$ in terms of both $\varphi_0^\prime$ and $\sqrt{\varphi}$ as
%  \begin{equation} \label{eqn:ld1d_sw_shb_rphi2}
% \varphi = \frac{2}{\beta} \left( \varphi_0^\prime - \frac{3}{2}\, \sqrt{\varphi} \right) .
%  \end{equation}
%  \begin{equation}
% \sqrt{R_2}\, \left(I_2 + \varphi\right) = \sqrt{\varphi} + \frac{\beta}{2}\, \varphi = \varphi_0^\prime - \frac{1}{2}\, \sqrt{\varphi}\, ,
%  \end{equation}
% and find from \eqn{ld1d_sw_shb_dkappa_def} that
%  \begin{equation} \label{eqn:ld1d_sw_shb_dkappa}
% \Delta \kappa\left(\varphi_0^\prime\right) = \frac{\sqrt{\varphi\left(\varphi_0^\prime\right)}}{\varphi_0^\prime - \sqrt{\varphi\left(\varphi_0^\prime\right)}/2}\, .
%  \end{equation}
% In \fig{standing_wave_approx_1d}, we plot the right-hand output intensity of a standing-wave laser as a function of reflectance in the cases where $R_1 = R_2 \equiv R$ and $R_1 = 1$, $R_2 \equiv R$. We have included the predictions of both the Agrawal \& Lax and Rigrod models, and our simple model given by \eqn{ls1d_i_out_approx} with $\kappa \equiv 2 + \Delta \kappa$, where $\Delta \kappa$ has been computed using \eqn{ld1d_sw_shb_dkappa}. Although our approximate expression for $\Delta \kappa$ lacks the elegance of a simple factor of 2, it is nevertheless remarkable that a standing-wave laser's output intensity can be understood in a straightforward fashion by considering only the degree of intracavity gain saturation.

%  \begin{figure}
%   \centering
%   \begin{subfigure}[b]{0.8\textwidth}
%    \centering
%    \includegraphics[width=5.0in]{figures/standing_wave_approx_1d_sym}
%    \caption{$R_1 = R_2 \equiv R$}
%    \label{fig:standing_wave_approx_sym}
%   \end{subfigure}
%   \par\vspace{0.25in}
%   \begin{subfigure}[b]{0.8\textwidth}
%    \centering
%    \includegraphics[width=5.0in]{figures/standing_wave_approx_1d_oc2}
%    \caption{$R_1 = 1$, $R_2 \equiv R$}
%    \label{fig:standing_wave_approx_1d_oc2}
%   \end{subfigure}
%   \caption{\label{fig:standing_wave_approx_1d} The right-hand output intensity of a standing-wave laser as a function of reflectance in the cases where $R_1 = R_2 \equiv R$ and $R_1 = 1$, $R_2 \equiv R$. We have included the predictions of both the Agrawal \& Lax and Rigrod models, and our simple model given by \eqn{ls1d_i_out_approx} with $\kappa \equiv 2 + \Delta \kappa$, where $\Delta \kappa$ has been computed using \eqn{ld1d_sw_shb_dkappa}. }
%  \end{figure}

%%%%%%%%%%%%%%%%%%%%%%%%%%%%%%%%%%%%%%%%%%%%%%%%%%%%%%%%%%%%%%%%%%%%%%%%%%%%%%
%
% Section file included in chapter file using \input{}
%
% Assumes that LaTeX2e macros and packages defined in rgb_laser_physics.sty
%   are available
%
%%%%%%%%%%%%%%%%%%%%%%%%%%%%%%%%%%%%%%%%%%%%%%%%%%%%%%%%%%%%%%%%%%%%%%%%%%%%%%

\section{One-Dimensional Single Mode Semiconductor Laser Models\label{sct:laser_statics_1d_scl}}

\subsection{Linewidth Enhancement Factor\label{sct:laser_statics_1d_lef}}

In the previous sections of this chapter, we've been careful to allow arbitrary lineshape functions in our single-mode one-dimensional continuous-wave models, but our examples and plots have assumed a symmetric Lorentzian lineshape function $\Lo$. We can extend these results to the asymmetric lineshapes typical of broad classes of semiconductor lasers by relying on the evolution equation of the macroscopic polarization $\widetilde{F}\zt$ and the corresponding lineshape function $\Lao$ given by \eqn{cw_sml_ftz_scaled} and \eqn{laser_statics_1d_lef_lineshape}, respectively.  In general, this function has the real and imaginary parts
\begin{subequations}\label{eqn:laser_statics_1d_lef_ril}
  \begin{align}
    \label{eqn:laser_statics_1d_lef_rel} \begin{split} \rho(\Omega) = \Re\left[\Lao\right] &= \frac{1 + \alpha^2 + 2\, \alpha\, \Omega}{1 + (\Omega + \alpha)^2} \\
    &= 1 - \frac{\Omega^2}{1 + (\Omega + \alpha)^2}\, , \nd \end{split} \\
    \label{eqn:laser_statics_1d_lef_iml} \begin{split} \mu(\Omega) = \Im\left[\Lao\right] &= -\frac{\alpha \left(1 + \alpha^2\right) + \left(\alpha^2 - 1\right) \Omega}{1 + (\Omega + \alpha)^2} \\
    &= -\alpha + \frac{\left( 1 + \alpha^2 + \alpha\, \Omega \right) \Omega}{1 + (\Omega + \alpha)^2}\, . \end{split}
  \end{align}
\end{subequations}

The real part of $\Lao$ is plotted as a function of $\Omega$ for two different ranges of $\alpha$ in \fig{scl_lmc_real}. When $\alpha \lesssim 1$, we observe a dramatic asymmetry for modest detunings, but as $\alpha$ increases the predominant effect becomes a substantial broadening of the lineshape by a factor of $\sqrt{1 + \alpha^2}$. In \fig{scl_lmc_imag} we plot the imaginary part of $\Lao$ (offset vertically by $\alpha$) as a function of $\Omega$ for the same two ranges of $\alpha$.  Note that at $\Omega = 0$, the slope of $\Im{\Lao}$ is unity.

\begin{figure}
  \centering
  \begin{subfigure}[b]{0.8\textwidth}
    \centering
    \includegraphics[width=5.0in]{figures/scl_lmc_real_small_alpha}
    \caption{Real part of the lineshape function for small $\alpha$}
    \label{fig:scl_lmc_real_small_alpha}
  \end{subfigure}
  \par\vspace{0.25in}
  \begin{subfigure}[b]{0.8\textwidth}
    \centering
    \includegraphics[width=5.0in]{figures/scl_lmc_real_large_alpha}
    \caption{Real part of the lineshape function for large $\alpha$}
    \label{fig:scl_lmc_real_large_alpha}
  \end{subfigure}
  \caption{\label{fig:scl_lmc_real} Real part of the lineshape function given by \eqn{laser_statics_1d_lef_lineshape} for $\tau_\perp/\tau_\mathrm{o} = 0.0024$ and $\tau_p/\tau_\mathrm{o} = 1.0$. When $\alpha \lesssim 1$, the lineshape asymmetry is dramatic, but as $\alpha$ increases $\Re[\mathcal{L}(\Omega, \alpha)]$ is predominantly broadened by a factor of $\sqrt{1 + \alpha^2}$.}
\end{figure}

\begin{figure}
  \centering
  \begin{subfigure}[b]{0.8\textwidth}
   \centering
   \includegraphics[width=5.0in]{figures/scl_lmc_imag_small_alpha}
   \caption{Imaginary part of the lineshape function for small $\alpha$}
   \label{fig:scl_lmc_imag_small_alpha}
  \end{subfigure}
  \par\vspace{0.25in}
  \begin{subfigure}[b]{0.8\textwidth}
   \centering
   \includegraphics[width=5.0in]{figures/scl_lmc_imag_large_alpha}
   \caption{Imaginary part of the lineshape function for large $\alpha$}
   \label{fig:scl_lmc_imag_large_alpha}
  \end{subfigure}
  \caption{\label{fig:scl_lmc_imag} Imaginary part of the lineshape function given by \eqn{laser_statics_1d_lef_lineshape}. Note that at $\Omega = 0$, the slope of $\Im{\Lao}$ is unity.}
 \end{figure}

Consistent with our definition of the macroscopic polarization $\widetilde{F}\zt$, we define the \emph{effective linewidth enhancement factor} for a particular detuning $\Omega$ as
\begin{equation} \label{eqn:scl_alpha_eff}
  \begin{split}
    \alpha_\mathrm{eff}(\alpha, \Omega) &\equiv -\frac{\Im\left[\Lao\right]}{\Re\left[\Lao\right]} \\
    &= \alpha - \Omega + \frac{2\, \alpha\, \Omega^2}{1 + \alpha^2 + 2\, \alpha\, \Omega}\, .
  \end{split}
\end{equation}
In \fig{scl_alpha_eff}, we show that as $\alpha$ increases, the gain dispersion shifts downward linearly with $\alpha$, broadens, and maintains a slope of approximately unity near $\Omega = 0$. The corresponding effective linewidth enhancement factor --- and therefore the laser linewidth, proportional to $1 + \alpha^{\prime\, 2}(\Omega, \alpha)$ --- decreases relative to $|\alpha|$ as the laser is detuned to the blue.

 \begin{figure}
  \centering
  \includegraphics[width=5.0in]{figures/scl_alpha_eff}
  \caption{\label{fig:scl_alpha_eff} Plot of the effective linewidth enhancement factor given by \eqn{scl_alpha_eff} as a function of $\Omega$ for the case where $\tau_p / \tau_\mathrm{o} = 1$ and $\tau_\perp / \tau_\mathrm{o} = 2.5 \times 10^{-3}$. The corresponding linewidth --- proportional to $1 + \alpha^2_\mathrm{eff}(\alpha, \Omega)$ --- increases with $\alpha$, but decreases as the laser is detuned to the blue.}
\end{figure}

\subsection{Frequency Pulling and Dispersion\label{sct:laser_statics_1d_frq}}

% Using the quasi-normal modal expansion defined by \eqn{mml_e_field_1d_t}, in~\cite{ref:beausoleil2020flm} we show that the evolution of the field coefficient $E_q(t)$ when driven by a corresponding (properly normalized) macroscopic polarization $F_q(t)$ is described by
%   \begin{equation} \label{eqn:mml_edot}
%   \dot{E}_q(t) = \frac{1}{1 + \delta \tau_q\wn} \left\{ \left[-\frac{1}{2\, \tau_p} + i \left[\delta \omega_q + \delta D_q\wn\right]\right] E_q(t) + F_q(t) \right\}\, ,
%   \end{equation}
%  where $\tau_p$ is the photon lifetime of the bare cavity expressed in units of the group round-trip time $\tau_\mathrm{o}$ as given by \eqn{laser_resonator_1d_tau_p_def}, the small shift $\delta \omega_q$ is discussed in \sct{qdcl_1d_frq_pull}, and the dispersion factors $\delta \tau_q\wn$ and $\delta D_q\wn$ are defined in \sct{qdcl_1d_frq_disp}.
 
%  The term $\delta \omega_q$ in \eqn{mml_edot} is a small frequency shift that we can choose to compensate for gain-dependent phase shifts caused by $F_q(t)$, enabling more rapid numerical convergence. When $\alpha = 0$, we select $\delta \omega_q$ to offset an effect known as ``frequency pulling.'' Using \eqn{mml_1d_delta_w_q_def}, we define
%   \begin{equation}
%  \epsilon \equiv \frac{\tau_\perp}{2\, \tau_p}\, ,
%   \end{equation}
%  and then
%   \begin{subequations}
%   \begin{align}
%  \label{eqn:mml_1d_freq_shift} \delta \omega_q &= -\frac{\Omega_q}{2\, \tau_p} = -\frac{\epsilon}{1 + \epsilon}\, 2 q \pi\, , \\
%  \label{eqn:mml_1d_freq_mode} \Delta \omega_q &= \frac{2 q \pi}{1 + \epsilon}\, , \nd \\
%  \label{eqn:mml_1d_omega_q_def} \Omega_q &\equiv \Delta \omega_q\, \tau_\perp\, .
%   \end{align}
%   \end{subequations}
%  Here $\tau_\perp$ is the dipole relaxation time of the quantum dot laser transition discussed in \sct{qdcl_1d_leq}. If the frequency of the bare cavity mode $q$, given by $2 q \pi$, does not coincide with the quantum dot resonance frequency, the total frequency of that mode will be pulled away from $2 q \pi$ toward the resonance of the medium.
 
%  However, when $\alpha \ne 0$, we must return to the analysis in~\cite{ref:beausoleil2020flm} and write $\delta \omega_q$ in terms of the real and imaginary parts of the macroscopic polarization, which gives
%  where we have used \eqn{qdcl_1d_lef_lineshape} to define
%   \begin{equation} \label{eqn:qdcl_1d_lmc_qa_def}
%  \mathcal{L}_q(\alpha) \equiv \frac{\left(1 - i\, \alpha\right)^2}{1 - i (\Omega_q + \alpha)}\, .
%   \end{equation}
%  Note that this result and \eqn{qdcl_1d_lef_def} implies that frequency pulling and linewidth enhancement are closely related, since
%   \begin{equation} \label{eqn:qdcl_1d_dwq_exact}
%  \delta \omega_q \approx \frac{\alpha^\prime(\Omega_q, \alpha)}{2\, \tau_p} = \frac{1}{2\, \tau_p}\left[\alpha - \frac{\left(1 + \alpha^2\right) \Omega_q}{1 + \alpha^2 + 2\, \alpha\, \Omega_q}\right]\, .
%   \end{equation}
Using \eqn{la1d_dw0_def} to determine the presumably small phase shift arising from frequency pulling, we find
\begin{equation}
  \delta \omega_0 = -\frac{1}{2\, \tau_p}\, \frac{\Im\left[\Lao\right]}{\Re\left[\Lao\right]}\, .
\end{equation}
Given that $\Omega = \Omega_0 + \delta \omega_0\, \tau_\perp$, we can solve this equation accurately to second order in $\tau_\perp / \tau_p$ and obtain
\begin{equation} \label{eqn:laser_statics_1d_dw0_approx}
  \begin{split}
    \delta \omega_0 &\approx \frac{1}{2\, \tau_p + \tau_\perp}\, \frac{\alpha \left(1 + \alpha^2\right) + \left(\alpha^2 - 1\right) \Omega_0}{1 + \alpha^2 + 2\, \alpha\, \Omega_0} \\
    &= \frac{1}{2\, \tau_p + \tau_\perp} \left[ \alpha - \Omega_0 + \frac{2\, \alpha\, \Omega_0^2}{1 + \alpha^2 + 2\, \alpha\, \Omega_0} \right]\, .
  \end{split}
\end{equation}
The Lorentzian contribution to the frequency shift --- $-\Omega_0 / 2\, \tau_p$ --- is now offset by a large linear shift proportional to $\alpha$. A plot of \eqn{laser_statics_1d_dw0_approx} --- relative to the total linear contribution $(\alpha - \Omega_0)/(2\, \tau_p + \tau_\perp)$ --- is shown in \fig{scl_freq_pull_nlo}.
 
  %  \begin{figure}
  %  \centering
  %  \includegraphics[width=5.0in]{figures/scl_freq_pull}
  %  \caption{\label{fig:scl_freq_pull} Plot of $\delta \omega_0$ --- given by \eqn{laser_statics_1d_dw0_approx} --- relative to its linear contribution $(\alpha - \Omega_0)/2\, \tau_p$ as a function of $\Omega_0$.}
  % \end{figure}
 
%  The effects of \emph{frequency dispersion} are included in \eqn{mml_edot} through the terms $\delta D_q\wn$ and $\delta \tau_q\wn$, which are defined as
%   \begin{subequations} \label{eqn:mml_1d_delta_dt_q_def}
%   \begin{align}
%  \label{eqn:mml_1d_delta_d_q_def} \delta D_q\wn &\equiv \sum_{m = 2}^\infty \frac{(2 q \pi)^m}{m!}\, D_m\wn\, , \nd \\
%  \label{eqn:mml_1d_delta_tau_q_def} \delta \tau_q\wn &\equiv \sum_{m = 2}^\infty \frac{(2 q \pi)^{m - 1}}{(m - 1)!}\, D_m\wn\, .
%   \end{align}
%   \end{subequations}
%  The dispersion constants $D_m\wn$ are given by
%   \begin{equation} \label{eqn:mml_1d_disp_coeff}
%  D_m\wn = \frac{L}{\tau_\mathrm{o}^m} \frac{d^m}{d \omega_0^m} \Re\left[\beta\wn\right]\, ,
%   \end{equation}
%  where $\beta\wn$ is the propagation eigenvalue of the transverse spatial mode of the electric field~\cite{ref:beausoleil2020flm}. Recall that $z$ is expressed in units of $L$ (the round-trip physical length of the laser resonator). Similarly, $t$ has units of $\tau_\mathrm{o}$ (the group round-trip propagation time), and $\omega_0$ has units of $\tau_\mathrm{o}^{-1}$. In this context, we have chosen the form given by \eqn{mml_1d_disp_coeff} because the value of $d^m \Re[\beta\wn] / d \omega_0^m$ is usually given in published tables in units of second$^m$/meter. For example, suppose that we have a material with a \emph{group velocity dispersion} $d^2 \Re[\beta\wn] / d \omega_0^2 = 1000$~fs$^2$/mm in a laser cavity with $L = 4$~mm and $\tau_\mathrm{o} = 60$~ps. Then $D_2\wn \cong 10^{-6}$, and is dimensionless. Therefore, we see two main effects of dispersion. First, there is an additional frequency shift for each mode that increases (in magnitude) nonlinearly with mode number $q$. Second, the group round-trip time is slightly different for each mode, changing with $q$ by a factor of $1 + \delta \tau_q\wn$. Although it is not obvious from the form of \eqn{mml_edot}, as the magnitudes of the dispersion coefficients increase, the primary effect will be to change the phase of the nonlinear coupling driving the evolution of each mode.
 
We're going to treat the \emph{nonlinear} contribution to \eqn{laser_statics_1d_dw0_approx} a little differently. Nonlinear frequency shifts correspond to dispersion, so we're going to replace $\Omega_0$ in this expression with a general frequency $\omega\, \tau_\perp$, expand the result in a series in $\omega$, and then take the inverse Fourier transform. We obtain
\begin{equation}
  \frac{1}{2\, \tau_p + \tau_\perp}\, \frac{2\, \alpha\, \Omega_0^2}{1 + \alpha^2 + 2\, \alpha\, \Omega_0} \longrightarrow \sum_{m = 2}^\infty \frac{A_m}{m!} \left(i\, \ppt\right)^m\, ,
\end{equation}
% As is clear from \fig{freq_pull}, it is worth noting that the linewidth enhancement effect on frequency pulling is equivalent to that of dispersion. \Eqn{qdcl_1d_dwq_approx} can be written as the series
%   \begin{equation}
%  \delta \omega_q = \frac{\alpha - \tau_\perp (2 q \pi)}{2\, \tau_p} + \sum_{m = 2}^\infty \frac{(2 q \pi)^m}{m!}\, A_m
%   \end{equation}
 where
  \begin{equation} \label{eqn:laser_statics_1d_am_def}
 A_m \equiv \frac{m!}{2\, \tau_p + \tau_\perp}\, \left(\frac{2\, \alpha}{1 + \alpha^2}\right)^{m - 1} (-\tau_\perp)^m\, .
  \end{equation}
A plot of the ``effective dispersion'' $A_m$ as a function of $\alpha$ for the case where $\tau_p / \tau_\mathrm{o} = 1$ and $\tau_\perp / \tau_\mathrm{o} = 2.5 \times 10^{-3}$ is shown in \fig{scl_freq_pull_dm}. Recall from \sct{laser_amp_1d_pdes} that to convert $A_m$ to an equivalent dispersion in units of fs$^m$/mm, we multiply by a factor of $\tau_0^m / L$. For a laser cavity with $L = 4$~mm and $\tau_0 = 65$~ps, we obtain (e.g.) $A_2 \approx 4000$~fs$^2$/mm. The extremum of $A_m$ occurs at $\alpha = \pm 1$, where $2\, |\alpha| / \left(1 + \alpha^2\right) = 1$. The full-width at half-maximum of $A_m$ is $2 \sqrt{2^{2/(m - 1)} - 1}$, which approaches $2 \sqrt{2 \ln(2)/(m - 1)}$ as $m$ becomes large.

\begin{figure}
  \centering
  \begin{subfigure}[b]{0.8\textwidth}
   \centering
   \includegraphics[width=5.0in]{figures/scl_freq_pull_nlo}
   \caption{Nonlinear frequency pulling/shift}
   \label{fig:scl_freq_pull_nlo}
  \end{subfigure}
  \par\vspace{0.25in}
  \begin{subfigure}[b]{0.8\textwidth}
   \centering
   \includegraphics[width=5.0in]{figures/scl_freq_pull_dm}
   \caption{Effective dispersion of nonlinear frequency pulling/shift}
   \label{fig:scl_freq_pull_dm}
  \end{subfigure}
  \caption{\label{fig:scl_freq_pull} Nonlinear frequency pulling/shift for the case where $\tau_p / \tau_\mathrm{o} = 1$ and $\tau_\perp / \tau_\mathrm{o} = 2.5 \times 10^{-3}$. (a) Plot of $\delta \omega_0$ --- given by \eqn{laser_statics_1d_dw0_approx} --- relative to its linear contribution $(\alpha - \Omega_0)/2\, \tau_p$ as a function of $\Omega_0$. (b) Plot of the corresponding effective dispersion coefficients given by \eqn{laser_statics_1d_am_def} as a function of $\alpha$.}
  % To convert $A_2$ to an equivalent dispersion in units of fs$^2$/mm for these parameters, multiply it by a factor of $10^9$. The extremum of $A_m$ occurs at $\alpha = \pm 1$, where $2\, \alpha / (1 + \alpha^2) = 1$.}
\end{figure}

% \begin{figure}
%   \centering
%   \includegraphics[width=5.0in]{figures/scl_freq_pull_dm}
%   \caption{\label{fig:scl_1d_d2} Plot of $A_m$ given by \eqn{laser_statics_1d_am_def} as a function of $\alpha$ for the case where $\tau_p / \tau_\mathrm{o} = 1$ and $\tau_\perp / \tau_\mathrm{o} = 2.5 \times 10^{-3}$. To convert $A_2$ to an equivalent dispersion in units of fs$^2$/mm for these parameters, multiply it by a factor of $10^9$. The extremum of $A_m$ occurs at $\alpha = \pm 1$, where $2\, \alpha / (1 + \alpha^2) = 1$.}
% \end{figure}

% , the second-order ``effective dispersion'' is $A_2 \cong 4 \times 10^{-6}$, or about $4000$~fs$^2$/mm for the cavity discussed above.

Collecting the results we've derived using continuous-wave laser models, we'll apply them to dynamic models by updating \eqn{cw_sml_etz_scaled} to read
\begin{multline} \label{eqn:scl_etz_scaled}
  \ppt E^\pm\zt \pm \ppz E^\pm\zt \\
  = \left[ \frac{i}{2\, \tau_p}\, (\alpha - \Omega_0) + i\, \sum_{l = 2}^\infty \frac{A_l + D_l\wn}{l!} \left(i\, \frac{\partial}{\partial t}\right)^l E^\pm\zt - \half\, \anz \right] E^\pm\zt + F^\pm\zt\, .
\end{multline}
We should remember that in rapidly-varying systems it is likely that $\alpha$ will change with time. Nevertheless, this equation represents a good starting point to understand semiconductor laser systems.

