%%%%%%%%%%%%%%%%%%%%%%%%%%%%%%%%%%%%%%%%%%%%%%%%%%%%%%%%%%%%%%%%%%%%%%%%%%%%%%
%
% Chapter file included in main project file using \input{}
%
% Assumes that LaTeX2e macros and packages defined in rgb_laser_physics.sty
%   are available
%
% $Id:$
%
%%%%%%%%%%%%%%%%%%%%%%%%%%%%%%%%%%%%%%%%%%%%%%%%%%%%%%%%%%%%%%%%%%%%%%%%%%%%%%

 \chapter*{Preface}
This text is adapted from a set of lecture notes hand-written during 1989--1995 for graduate classes taught at the University of Washington. I am updating these notes as I type them into \LaTeX, because they were quite uneven, and didn't incorporate many of the details that I used to build \textit{Melody}, a library of Fortran codes that I once used to model a variety of laser components, devices, and systems. Over the next few years, I'll add as many of those details as I can remember (and cull from old notebooks) to these notes, and I'll try to update the concepts so that there's a higher likelihood that they'll remain relevant in a era that includes extraordinary advances in new areas such as photonic crystals, negative index materials, and plasmonics. My goal is to rebuild \textit{Melody} as an object-oriented library in MATLAB (any day now), enabling applications in fields ranging from microphotonic lasers to gravitational-wave interferometers.
