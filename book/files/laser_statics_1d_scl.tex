%%%%%%%%%%%%%%%%%%%%%%%%%%%%%%%%%%%%%%%%%%%%%%%%%%%%%%%%%%%%%%%%%%%%%%%%%%%%%%
%
% Section file included in chapter file using \input{}
%
% Assumes that LaTeX2e macros and packages defined in rgb_laser_physics.sty
%   are available
%
%%%%%%%%%%%%%%%%%%%%%%%%%%%%%%%%%%%%%%%%%%%%%%%%%%%%%%%%%%%%%%%%%%%%%%%%%%%%%%

\section{One-Dimensional Single Mode Semiconductor Laser Models\label{sct:laser_statics_1d_scl}}

\subsection{Linewidth Enhancement Factor\label{sct:laser_statics_1d_lef}}

In the previous sections of this chapter, we've been careful to allow arbitrary lineshape functions in our single-mode one-dimensional continuous-wave models, but our examples and plots have assumed a symmetric Lorentzian lineshape function $\Lo$. We can extend these results to the asymmetric lineshapes typical of broad classes of semiconductor lasers by relying on the evolution equation of the macroscopic polarization $\widetilde{F}\zt$ and the corresponding lineshape function $\Lao$ given by \eqn{cw_sml_ftz_scaled} and \eqn{laser_statics_1d_lef_lineshape}, respectively.  In general, this function has the real and imaginary parts
\begin{subequations}\label{eqn:laser_statics_1d_lef_ril}
  \begin{align}
    \label{eqn:laser_statics_1d_lef_rel} \begin{split} \rho(\Omega) = \Re\left[\Lao\right] &= \frac{1 + \alpha^2 + 2\, \alpha\, \Omega}{1 + (\Omega + \alpha)^2} \\
    &= 1 - \frac{\Omega^2}{1 + (\Omega + \alpha)^2}\, , \nd \end{split} \\
    \label{eqn:laser_statics_1d_lef_iml} \begin{split} \mu(\Omega) = \Im\left[\Lao\right] &= -\frac{\alpha \left(1 + \alpha^2\right) + \left(\alpha^2 - 1\right) \Omega}{1 + (\Omega + \alpha)^2} \\
    &= -\alpha + \frac{\left( 1 + \alpha^2 + \alpha\, \Omega \right) \Omega}{1 + (\Omega + \alpha)^2}\, . \end{split}
  \end{align}
\end{subequations}

The real part of $\Lao$ is plotted as a function of $\Omega$ for two different ranges of $\alpha$ in \fig{scl_lmc_real}. When $\alpha \lesssim 1$, we observe a dramatic asymmetry for modest detunings, but as $\alpha$ increases the predominant effect becomes a substantial broadening of the lineshape by a factor of $\sqrt{1 + \alpha^2}$. In \fig{scl_lmc_imag} we plot the imaginary part of $\Lao$ (offset vertically by $\alpha$) as a function of $\Omega$ for the same two ranges of $\alpha$.  Note that at $\Omega = 0$, the slope of $\Im{\Lao}$ is unity.

\begin{figure}
  \centering
  \begin{subfigure}[b]{0.8\textwidth}
    \centering
    \includegraphics[width=5.0in]{figures/scl_lmc_real_small_alpha}
    \caption{Real part of the lineshape function for small $\alpha$}
    \label{fig:scl_lmc_real_small_alpha}
  \end{subfigure}
  \par\vspace{0.25in}
  \begin{subfigure}[b]{0.8\textwidth}
    \centering
    \includegraphics[width=5.0in]{figures/scl_lmc_real_large_alpha}
    \caption{Real part of the lineshape function for large $\alpha$}
    \label{fig:scl_lmc_real_large_alpha}
  \end{subfigure}
  \caption{\label{fig:scl_lmc_real} Real part of the lineshape function given by \eqn{laser_statics_1d_lef_lineshape} for $\tau_\perp/\tau_\mathrm{o} = 0.0024$ and $\tau_p/\tau_\mathrm{o} = 1.0$. When $\alpha \lesssim 1$, the lineshape asymmetry is dramatic, but as $\alpha$ increases $\Re[\mathcal{L}(\Omega, \alpha)]$ is predominantly broadened by a factor of $\sqrt{1 + \alpha^2}$.}
\end{figure}

\begin{figure}
  \centering
  \begin{subfigure}[b]{0.8\textwidth}
   \centering
   \includegraphics[width=5.0in]{figures/scl_lmc_imag_small_alpha}
   \caption{Imaginary part of the lineshape function for small $\alpha$}
   \label{fig:scl_lmc_imag_small_alpha}
  \end{subfigure}
  \par\vspace{0.25in}
  \begin{subfigure}[b]{0.8\textwidth}
   \centering
   \includegraphics[width=5.0in]{figures/scl_lmc_imag_large_alpha}
   \caption{Imaginary part of the lineshape function for large $\alpha$}
   \label{fig:scl_lmc_imag_large_alpha}
  \end{subfigure}
  \caption{\label{fig:scl_lmc_imag} Imaginary part of the lineshape function given by \eqn{laser_statics_1d_lef_lineshape}. Note that at $\Omega = 0$, the slope of $\Im{\Lao}$ is unity.}
 \end{figure}

Consistent with our definition of the macroscopic polarization $\widetilde{F}\zt$, we define the \emph{effective linewidth enhancement factor} for a particular detuning $\Omega$ as
\begin{equation} \label{eqn:scl_alpha_eff}
  \begin{split}
    \alpha_\mathrm{eff}(\alpha, \Omega) &\equiv -\frac{\Im\left[\Lao\right]}{\Re\left[\Lao\right]} \\
    &= \alpha - \Omega + \frac{2\, \alpha\, \Omega^2}{1 + \alpha^2 + 2\, \alpha\, \Omega}\, .
  \end{split}
\end{equation}
In \fig{scl_alpha_eff}, we show that as $\alpha$ increases, the gain dispersion shifts downward linearly with $\alpha$, broadens, and maintains a slope of approximately unity near $\Omega = 0$. The corresponding effective linewidth enhancement factor --- and therefore the laser linewidth, proportional to $1 + \alpha^{\prime\, 2}(\Omega, \alpha)$ --- decreases relative to $|\alpha|$ as the laser is detuned to the blue.

 \begin{figure}
  \centering
  \includegraphics[width=5.0in]{figures/scl_alpha_eff}
  \caption{\label{fig:scl_alpha_eff} Plot of the effective linewidth enhancement factor given by \eqn{scl_alpha_eff} as a function of $\Omega$ for the case where $\tau_p / \tau_\mathrm{o} = 1$ and $\tau_\perp / \tau_\mathrm{o} = 2.5 \times 10^{-3}$. The corresponding linewidth --- proportional to $1 + \alpha^2_\mathrm{eff}(\alpha, \Omega)$ --- increases with $\alpha$, but decreases as the laser is detuned to the blue.}
\end{figure}

\subsection{Frequency Pulling and Dispersion\label{sct:laser_statics_1d_frq}}

% Using the quasi-normal modal expansion defined by \eqn{mml_e_field_1d_t}, in~\cite{ref:beausoleil2020flm} we show that the evolution of the field coefficient $E_q(t)$ when driven by a corresponding (properly normalized) macroscopic polarization $F_q(t)$ is described by
%   \begin{equation} \label{eqn:mml_edot}
%   \dot{E}_q(t) = \frac{1}{1 + \delta \tau_q\wn} \left\{ \left[-\frac{1}{2\, \tau_p} + i \left[\delta \omega_q + \delta D_q\wn\right]\right] E_q(t) + F_q(t) \right\}\, ,
%   \end{equation}
%  where $\tau_p$ is the photon lifetime of the bare cavity expressed in units of the group round-trip time $\tau_\mathrm{o}$ as given by \eqn{laser_resonator_1d_tau_p_def}, the small shift $\delta \omega_q$ is discussed in \sct{qdcl_1d_frq_pull}, and the dispersion factors $\delta \tau_q\wn$ and $\delta D_q\wn$ are defined in \sct{qdcl_1d_frq_disp}.
 
%  The term $\delta \omega_q$ in \eqn{mml_edot} is a small frequency shift that we can choose to compensate for gain-dependent phase shifts caused by $F_q(t)$, enabling more rapid numerical convergence. When $\alpha = 0$, we select $\delta \omega_q$ to offset an effect known as ``frequency pulling.'' Using \eqn{mml_1d_delta_w_q_def}, we define
%   \begin{equation}
%  \epsilon \equiv \frac{\tau_\perp}{2\, \tau_p}\, ,
%   \end{equation}
%  and then
%   \begin{subequations}
%   \begin{align}
%  \label{eqn:mml_1d_freq_shift} \delta \omega_q &= -\frac{\Omega_q}{2\, \tau_p} = -\frac{\epsilon}{1 + \epsilon}\, 2 q \pi\, , \\
%  \label{eqn:mml_1d_freq_mode} \Delta \omega_q &= \frac{2 q \pi}{1 + \epsilon}\, , \nd \\
%  \label{eqn:mml_1d_omega_q_def} \Omega_q &\equiv \Delta \omega_q\, \tau_\perp\, .
%   \end{align}
%   \end{subequations}
%  Here $\tau_\perp$ is the dipole relaxation time of the quantum dot laser transition discussed in \sct{qdcl_1d_leq}. If the frequency of the bare cavity mode $q$, given by $2 q \pi$, does not coincide with the quantum dot resonance frequency, the total frequency of that mode will be pulled away from $2 q \pi$ toward the resonance of the medium.
 
%  However, when $\alpha \ne 0$, we must return to the analysis in~\cite{ref:beausoleil2020flm} and write $\delta \omega_q$ in terms of the real and imaginary parts of the macroscopic polarization, which gives
%  where we have used \eqn{qdcl_1d_lef_lineshape} to define
%   \begin{equation} \label{eqn:qdcl_1d_lmc_qa_def}
%  \mathcal{L}_q(\alpha) \equiv \frac{\left(1 - i\, \alpha\right)^2}{1 - i (\Omega_q + \alpha)}\, .
%   \end{equation}
%  Note that this result and \eqn{qdcl_1d_lef_def} implies that frequency pulling and linewidth enhancement are closely related, since
%   \begin{equation} \label{eqn:qdcl_1d_dwq_exact}
%  \delta \omega_q \approx \frac{\alpha^\prime(\Omega_q, \alpha)}{2\, \tau_p} = \frac{1}{2\, \tau_p}\left[\alpha - \frac{\left(1 + \alpha^2\right) \Omega_q}{1 + \alpha^2 + 2\, \alpha\, \Omega_q}\right]\, .
%   \end{equation}
Using \eqn{la1d_dw0_def} to determine the presumably small phase shift arising from frequency pulling, we find
\begin{equation}
  \delta \omega_0 = -\frac{1}{2\, \tau_p}\, \frac{\Im\left[\Lao\right]}{\Re\left[\Lao\right]}\, .
\end{equation}
Given that $\Omega = \Omega_0 + \delta \omega_0\, \tau_\perp$, we can solve this equation accurately to second order in $\tau_\perp / \tau_p$ and obtain
\begin{equation} \label{eqn:laser_statics_1d_dw0_approx}
  \begin{split}
    \delta \omega_0 &\approx \frac{1}{2\, \tau_p + \tau_\perp}\, \frac{\alpha \left(1 + \alpha^2\right) + \left(\alpha^2 - 1\right) \Omega_0}{1 + \alpha^2 + 2\, \alpha\, \Omega_0} \\
    &= \frac{1}{2\, \tau_p + \tau_\perp} \left[ \alpha - \Omega_0 + \frac{2\, \alpha\, \Omega_0^2}{1 + \alpha^2 + 2\, \alpha\, \Omega_0} \right]\, .
  \end{split}
\end{equation}
The Lorentzian contribution to the frequency shift --- $-\Omega_0 / 2\, \tau_p$ --- is now offset by a large linear shift proportional to $\alpha$. A plot of \eqn{laser_statics_1d_dw0_approx} --- relative to the total linear contribution $(\alpha - \Omega_0)/(2\, \tau_p + \tau_\perp)$ --- is shown in \fig{scl_freq_pull_nlo}.
 
  %  \begin{figure}
  %  \centering
  %  \includegraphics[width=5.0in]{figures/scl_freq_pull}
  %  \caption{\label{fig:scl_freq_pull} Plot of $\delta \omega_0$ --- given by \eqn{laser_statics_1d_dw0_approx} --- relative to its linear contribution $(\alpha - \Omega_0)/2\, \tau_p$ as a function of $\Omega_0$.}
  % \end{figure}
 
%  The effects of \emph{frequency dispersion} are included in \eqn{mml_edot} through the terms $\delta D_q\wn$ and $\delta \tau_q\wn$, which are defined as
%   \begin{subequations} \label{eqn:mml_1d_delta_dt_q_def}
%   \begin{align}
%  \label{eqn:mml_1d_delta_d_q_def} \delta D_q\wn &\equiv \sum_{m = 2}^\infty \frac{(2 q \pi)^m}{m!}\, D_m\wn\, , \nd \\
%  \label{eqn:mml_1d_delta_tau_q_def} \delta \tau_q\wn &\equiv \sum_{m = 2}^\infty \frac{(2 q \pi)^{m - 1}}{(m - 1)!}\, D_m\wn\, .
%   \end{align}
%   \end{subequations}
%  The dispersion constants $D_m\wn$ are given by
%   \begin{equation} \label{eqn:mml_1d_disp_coeff}
%  D_m\wn = \frac{L}{\tau_\mathrm{o}^m} \frac{d^m}{d \omega_0^m} \Re\left[\beta\wn\right]\, ,
%   \end{equation}
%  where $\beta\wn$ is the propagation eigenvalue of the transverse spatial mode of the electric field~\cite{ref:beausoleil2020flm}. Recall that $z$ is expressed in units of $L$ (the round-trip physical length of the laser resonator). Similarly, $t$ has units of $\tau_\mathrm{o}$ (the group round-trip propagation time), and $\omega_0$ has units of $\tau_\mathrm{o}^{-1}$. In this context, we have chosen the form given by \eqn{mml_1d_disp_coeff} because the value of $d^m \Re[\beta\wn] / d \omega_0^m$ is usually given in published tables in units of second$^m$/meter. For example, suppose that we have a material with a \emph{group velocity dispersion} $d^2 \Re[\beta\wn] / d \omega_0^2 = 1000$~fs$^2$/mm in a laser cavity with $L = 4$~mm and $\tau_\mathrm{o} = 60$~ps. Then $D_2\wn \cong 10^{-6}$, and is dimensionless. Therefore, we see two main effects of dispersion. First, there is an additional frequency shift for each mode that increases (in magnitude) nonlinearly with mode number $q$. Second, the group round-trip time is slightly different for each mode, changing with $q$ by a factor of $1 + \delta \tau_q\wn$. Although it is not obvious from the form of \eqn{mml_edot}, as the magnitudes of the dispersion coefficients increase, the primary effect will be to change the phase of the nonlinear coupling driving the evolution of each mode.
 
We're going to treat the \emph{nonlinear} contribution to \eqn{laser_statics_1d_dw0_approx} a little differently. Nonlinear frequency shifts correspond to dispersion, so we're going to replace $\Omega_0$ in this expression with a general frequency $\omega\, \tau_\perp$, expand the result in a series in $\omega$, and then take the inverse Fourier transform. We obtain
\begin{equation}
  \frac{1}{2\, \tau_p + \tau_\perp}\, \frac{2\, \alpha\, \Omega_0^2}{1 + \alpha^2 + 2\, \alpha\, \Omega_0} \longrightarrow \sum_{m = 2}^\infty \frac{A_m}{m!} \left(i\, \ppt\right)^m\, ,
\end{equation}
% As is clear from \fig{freq_pull}, it is worth noting that the linewidth enhancement effect on frequency pulling is equivalent to that of dispersion. \Eqn{qdcl_1d_dwq_approx} can be written as the series
%   \begin{equation}
%  \delta \omega_q = \frac{\alpha - \tau_\perp (2 q \pi)}{2\, \tau_p} + \sum_{m = 2}^\infty \frac{(2 q \pi)^m}{m!}\, A_m
%   \end{equation}
 where
  \begin{equation} \label{eqn:laser_statics_1d_am_def}
 A_m \equiv \frac{m!}{2\, \tau_p + \tau_\perp}\, \left(\frac{2\, \alpha}{1 + \alpha^2}\right)^{m - 1} (-\tau_\perp)^m\, .
  \end{equation}
A plot of the ``effective dispersion'' $A_m$ as a function of $\alpha$ for the case where $\tau_p / \tau_\mathrm{o} = 1$ and $\tau_\perp / \tau_\mathrm{o} = 2.5 \times 10^{-3}$ is shown in \fig{scl_freq_pull_dm}. Recall from \sct{laser_amp_1d_pdes} that to convert $A_m$ to an equivalent dispersion in units of fs$^m$/mm, we multiply by a factor of $\tau_0^m / L$. For a laser cavity with $L = 4$~mm and $\tau_0 = 65$~ps, we obtain (e.g.) $A_2 \approx 4000$~fs$^2$/mm. The extremum of $A_m$ occurs at $\alpha = \pm 1$, where $2\, |\alpha| / \left(1 + \alpha^2\right) = 1$. The full-width at half-maximum of $A_m$ is $2 \sqrt{2^{2/(m - 1)} - 1}$, which approaches $2 \sqrt{2 \ln(2)/(m - 1)}$ as $m$ becomes large.

\begin{figure}
  \centering
  \begin{subfigure}[b]{0.8\textwidth}
   \centering
   \includegraphics[width=5.0in]{figures/scl_freq_pull_nlo}
   \caption{Nonlinear frequency pulling/shift}
   \label{fig:scl_freq_pull_nlo}
  \end{subfigure}
  \par\vspace{0.25in}
  \begin{subfigure}[b]{0.8\textwidth}
   \centering
   \includegraphics[width=5.0in]{figures/scl_freq_pull_dm}
   \caption{Effective dispersion of nonlinear frequency pulling/shift}
   \label{fig:scl_freq_pull_dm}
  \end{subfigure}
  \caption{\label{fig:scl_freq_pull} Nonlinear frequency pulling/shift for the case where $\tau_p / \tau_\mathrm{o} = 1$ and $\tau_\perp / \tau_\mathrm{o} = 2.5 \times 10^{-3}$. (a) Plot of $\delta \omega_0$ --- given by \eqn{laser_statics_1d_dw0_approx} --- relative to its linear contribution $(\alpha - \Omega_0)/2\, \tau_p$ as a function of $\Omega_0$. (b) Plot of the corresponding effective dispersion coefficients given by \eqn{laser_statics_1d_am_def} as a function of $\alpha$.}
  % To convert $A_2$ to an equivalent dispersion in units of fs$^2$/mm for these parameters, multiply it by a factor of $10^9$. The extremum of $A_m$ occurs at $\alpha = \pm 1$, where $2\, \alpha / (1 + \alpha^2) = 1$.}
\end{figure}

% \begin{figure}
%   \centering
%   \includegraphics[width=5.0in]{figures/scl_freq_pull_dm}
%   \caption{\label{fig:scl_1d_d2} Plot of $A_m$ given by \eqn{laser_statics_1d_am_def} as a function of $\alpha$ for the case where $\tau_p / \tau_\mathrm{o} = 1$ and $\tau_\perp / \tau_\mathrm{o} = 2.5 \times 10^{-3}$. To convert $A_2$ to an equivalent dispersion in units of fs$^2$/mm for these parameters, multiply it by a factor of $10^9$. The extremum of $A_m$ occurs at $\alpha = \pm 1$, where $2\, \alpha / (1 + \alpha^2) = 1$.}
% \end{figure}

% , the second-order ``effective dispersion'' is $A_2 \cong 4 \times 10^{-6}$, or about $4000$~fs$^2$/mm for the cavity discussed above.

Collecting the results we've derived using continuous-wave laser models, we'll apply them to dynamic models by updating \eqn{cw_sml_etz_scaled} to read
\begin{multline} \label{eqn:scl_etz_scaled}
  \ppt E^\pm\zt \pm \ppz E^\pm\zt \\
  = \left[ \frac{i}{2\, \tau_p}\, (\alpha - \Omega_0) + i\, \sum_{l = 2}^\infty \frac{A_l + D_l\wn}{l!} \left(i\, \frac{\partial}{\partial t}\right)^l E^\pm\zt - \half\, \anz \right] E^\pm\zt + F^\pm\zt\, .
\end{multline}
We should remember that in rapidly-varying systems it is likely that $\alpha$ will change with time. Nevertheless, this equation represents a good starting point to understand semiconductor laser systems.
