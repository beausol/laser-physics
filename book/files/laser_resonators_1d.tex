%%%%%%%%%%%%%%%%%%%%%%%%%%%%%%%%%%%%%%%%%%%%%%%%%%%%%%%%%%%%%%%%%%%%%%%%%%%%%%
%
% Chapter file included in main project file using \input{}
%
% Assumes that LaTeX2e macros and packages defined in rgb_laser_physics.sty
%   are available
%
% $Id$
%
%%%%%%%%%%%%%%%%%%%%%%%%%%%%%%%%%%%%%%%%%%%%%%%%%%%%%%%%%%%%%%%%%%%%%%%%%%%%%%

 \chapter{Laser Beams and Resonators: One-Dimensional Models\label{chp:laser_resonators_1d}}

The approximate description of a laser electromagnetic field as a transverse plane-polarized wave with the complex amplitude function given by \eqn{e_field_1d_w} allows us to build simple dynamical models of resonant cavities that are commonly used to build laser oscillators.  In this chapter, we will ignore the transverse dimensions of mirrors and other optical components, and rely on the \emph{scattering matrix}\index{Scattering matrix} \cite{ref:haus1984wfo,ref:siegman1986l} to build models of common ``two-port'' linear optical systems\index{Two-port optical component}.

 \red{Expand on this paragraph to include the discussions of time/frequency properties and quasi-normal modes.}

   %%%%%%%%%%%%%%%%%%%%%%%%%%%%%%%%%%%%%%%%%%%%%%%%%%%%%%%%%%%%%%%%%%%%%%%%%%%%%%
%
% Section file included in chapter file using \input{}
%
% Assumes that LaTeX2e macros and packages defined in rgb_laser_physics.sty
%   are available
%
% $Id$
%
%%%%%%%%%%%%%%%%%%%%%%%%%%%%%%%%%%%%%%%%%%%%%%%%%%%%%%%%%%%%%%%%%%%%%%%%%%%%%%

 \section{Scattering Matrices in the Frequency Domain\label{sct:laser_resonators_1d_smat}}

 \subsection{Dielectric Regions and Waveguides\label{sct:wg_smat}}

 \begin{figure}
  \centering
  \includegraphics[width=5.0in]{figures/waveguide_smat}
  \caption{\label{fig:waveguide_smat}Schematic diagram of a linear one-dimensional dielectric propagation region with a constant relative permittivity and permeability. The effective refractive index $n\wn$, the group refractive index $n^\prime\wn$, and the linear absorption coefficient $\alpha\wn$ are defined by \eqn{idm_ref_index_def}, \eqn{idm_grp_ref_index_def}, and \eqn{idm_abs_coeff_def_rw}, respectively, at the carrier frequency $\omega_0$. In the absence of a macroscopic polarization, we use $\mathbf{F}$ to represent an incident electric field.}
 \end{figure}

Let's apply the wave equation in the frequency domain given by the Fourier Transform of \eqn{cw_sml_etz_scaled} to the linear dielectric region (e.g., free space or a waveguide such as an optical fiber) shown in \fig{waveguide_smat} and recapitulate our treatment of the experiment shown in \fig{beers_law_schematic}. For a forward-propagating field (i.e., traveling in the $+\hatb{z}$ direction), in one dimension the rapidly-varying complex envelope function has the form of \eqn{e_field_1d_w}, or in the frequency domain
 \begin{equation} \label{eqn:e_field_1d_pz}
 \mathbf{E}^{+}\rw \equiv \hatb{\epsilon}\, e^{+i\, k_0\wn z}\, E^{+}\zw ,
 \end{equation}
where, from \eqn{idm_k_def}, $k_0\wn = \Re\left[\beta\wn\right] = (\omega_0/c) n\wn$. The corresponding wave equation for the complex scalar envelope function that varies slowly in both space and time is therefore given by\footnote{We recall from \sct{laser_amp_1d_pdes} that the coordinate $z$ has been scaled by a physical length $L$ chosen for convenience, and therefore $\alpha\wn$ and $k_0\wn$ have been scaled by $L^{-1}$. Similarly, the time $t$ has been scaled by the corresponding group propagation time $\tau_g \equiv n^\prime\wn L/c$, and all frequencies by $\tau_g^{-1}$.}
 \begin{equation} \label{eqn:wave_eqn_1d_w_lin_pz}
\ppz E^{+}\zw -i\, \omega\, E^{+}\zw + \half\, \alpha\wn\, E^{+}\zw = 0\, .
 \end{equation}
As in \sct{em_wave_prop_idm}, the effective refractive index $n\wn$, the group refractive index $n^\prime\wn$, and the linear absorption coefficient $\alpha\wn$ are defined by \eqn{idm_ref_index_def}, \eqn{idm_grp_ref_index_def}, and \eqn{idm_abs_coeff_def_rw}, respectively, at the carrier frequency $\omega_0$. In the absence of a macroscopic polarization arising from an internal gain region, in this chapter we use $\mathbf{F}$ to represent an incident electric field. For the wave traveling from a \emph{reference plane} at $z = z_1$ to a reference plane at $z = z_2$, in the case where we can ignore backscattering, we find the solution
 \begin{equation} \label{eqn:forward_prop_w}
E^{+}(z_2, \omega) = \exp\left[ i \omega \left(z_2 - z_1\right) - \half \alpha\wn \left(z_2 - z_1\right) \right] E^{+}(z_1, \omega) .
 \end{equation}
Therefore, the one-dimensional complex vector envelope function (that varies rapidly in space but slowly in time) given by \eqn{e_field_1d_w} in the frequency domain becomes
 \begin{equation} \label{eqn:e_prop_1d_p}
\mathbf{E}_2(\omega) = \frac{G(z_2, \omega)}{G(z_1, \omega)}\, \mathbf{F}_1(\omega) ,
 \end{equation}
where
 \begin{equation} \label{eqn:prop_gzw_def}
G(z, \omega) \equiv \exp\left[ i\, k_0\wn z + i \omega z - \half \alpha\wn z \right] ,
 \end{equation}
and, as in \fig{waveguide_smat}, we have defined the field amplitudes at the reference planes as $\mathbf{E}^{+}(z_1, \omega) \equiv \mathbf{F}_1(\omega)$ and $\mathbf{E}^{+}(z_2, \omega) \equiv \mathbf{E}_2(\omega)$.

For a backward-propagating field traveling in the $-\hatb{z}$ direction, in one dimension the complex vector envelope function has the form
 \begin{equation} \label{eqn:e_field_1d_mz}
 \mathbf{E}^{-}\rt \equiv \hatb{\epsilon}\, e^{-i\, k_0\wn z}\, E^{-}\zt ,
 \end{equation}
which results in a scalar frequency-domain wave equation similar to that of \eqn{wave_eqn_1d_w_lin_pz}, with a negated spatial derivative:
 \begin{equation} \label{eqn:wave_eqn_1d_w_lin_mz}
-\ppz E^{-}\zw -i\, \omega\, E^{-}\zw + \half\, \alpha\wn\, E^{-}\zw = 0 .
 \end{equation}
Integrating this linear differential equation from $z_2$ to $z_1$, we find
 \begin{equation} \label{eqn:e_prop_1d_m}
\mathbf{E}_1(\omega) = \frac{G(z_2, \omega)}{G(z_1, \omega)}\, \mathbf{F}_2(\omega) ,
 \end{equation}
where, as above, we have defined the field amplitudes at the reference planes as $\mathbf{E}^{-}(z_1, \omega) \equiv \mathbf{E}_1(\omega)$ and $\mathbf{E}^{-}(z_2, \omega) \equiv \mathbf{F}_2(\omega)$. Therefore, in matrix form, we have found that\index{Scattering matrix!General two-port component}
 \begin{equation} \label{eqn:two_port_smat_formal}
 \begin{bmatrix} \mathbf{E}_1(\omega) \\ \mathbf{E}_2(\omega) \end{bmatrix} =
 \begin{bmatrix} M_{11}(\omega) & M_{12}(\omega)  \\ M_{21}(\omega) & M_{22}(\omega) \end{bmatrix}\,
 \begin{bmatrix} \mathbf{F}_1(\omega) \\ \mathbf{F}_2(\omega) \end{bmatrix} \equiv \mathbf{M}(\omega) \begin{bmatrix} \mathbf{F}_1(\omega) \\ \mathbf{F}_2(\omega) \end{bmatrix}
 \end{equation}
where, in the case of the \emph{two-port} region\index{Two-port component!waveguide} shown in \fig{waveguide_smat}, the \emph{scattering-matrix}\index{Scattering matrix!waveguide} $\mathbf{M}(\omega)$ is given by
 \begin{equation} \label{eqn:waveguide_smat}
\mathbf{M}(\omega) = \begin{bmatrix} 0 & G(1, \omega)  \\ G(1, \omega) & 0 \end{bmatrix} \qquad \text{(Waveguide)},
 \end{equation}
for the case $z_1 = 0$ and $z_2 = 1$\footnote{As we shall see in \sct{mir_smat}, there are subtleties in the definition of $\mathbf{M}(\omega)$ if $\eta(z_1) \neq \eta(z_2)$.}.

 \subsection{Mirrors and General Two-Port Components\label{sct:mir_smat}}

 \begin{figure}
  \centering
  \includegraphics[width=5.0in]{figures/mirror_schematic}
  \caption{\label{fig:mirror_schematic}Schematic diagram of a laser mirror with high-reflection and anti-reflection dielectric coatings on opposite faces.}
 \end{figure}

 \begin{figure}
  \centering
  \begin{subfigure}[t]{0.5\textwidth}
   \centering
   \includegraphics[width=3.25in]{figures/mirror_smat}
   \caption{Normal-incidence mirror}
   \label{fig:mirror_smat}
  \end{subfigure}
  \qquad
  \begin{subfigure}[t]{0.5\textwidth}
   \centering
   \includegraphics[width=3.25in]{figures/bs_smat}
   \caption{Off-axis mirror (beamsplitter)}
   \label{fig:bs_smat}
  \end{subfigure}
  \caption{\label{fig:mirror_bs_smat} Common choices of reference planes for (a) simple two-port optical components (like laser mirrors) and (b) four-port optical components (like a tilted laser mirror or beamsplitter).}
 \end{figure}

In modern macroscopic lasers, the mirrors which form the laser resonator resemble the schematic shown in \fig{mirror_schematic}. A mirror typically consists of a transparent cylindrical substrate (such as pure fused silica) with a high-reflectance (HR) or partially-reflecting multilayer dielectric coating applied to the inner face of the cylinder; in the partially-reflecting case, transmission through the mirror substrate is usually necessary, so an antireflection (AR) coating will be applied to the outer surface as well. When the mirror is designed, Maxwell's equations are solved for electromagnetic fields propagating from air through the coating to the substrate (and/or vice versa) to optimize the reflectance for a particular application at a target carrier frequency $\omega_0$. However, as we try to build accurate models of laser oscillators, we would like to ``abstract'' the optical characteristics of a mirror in such a way that our predictions do not depend significantly on the physical details of either the coatings or the substrate. Following the example of the one-dimensional waveguide treated in \sct{wg_smat}, we'll attempt to find a general relationship between the input fields shown in \fig{mirror_smat},
 \begin{subequations}  \label{eqn:two_port_inputs}
 \begin{align}
 \label{eqn:two_port_f1} \mathbf{F}_1(t) &\equiv \hatb{\epsilon}\, \exp\left[+i\, k_1\wn z_1\right] E^{+}(z_1, t) , \nd \\
 \label{eqn:two_port_f2} \mathbf{F}_2(t) &\equiv \hatb{\epsilon}\, \exp\left[-i\, k_2\wn z_2\right] E^{-}(z_2, t) ,
 \end{align}
 \end{subequations}
and the corresponding output fields
 \begin{subequations}  \label{eqn:two_port_outputs}
 \begin{align}
 \label{eqn:two_port_e1} \mathbf{E}_1(t) &\equiv \hatb{\epsilon}\, \exp\left[-i\, k_1\wn z_1\right] E^{-}(z_1, t) , \nd \\
 \label{eqn:two_port_e2} \mathbf{E}_2(t) &\equiv \hatb{\epsilon}\, \exp\left[+i\, k_2\wn z_2\right] E^{+}(z_2, t) ,
 \end{align}
 \end{subequations}
where we have defined $k_j\wn \equiv (\omega_0/c) n_j\wn$ for $j \in \{1, 2\}$. Our goal will be to find a relationship between the input and output fields that has a \emph{form} that is independent of the angles of incidence (and polarizations) of the input fields, but we will note that in practice non-normal incidence may convert our simple mirror into the beamsplitter shown in \fig{bs_smat}.

Suppose that we have isolated a volume $\mathcal{V}$---bounded by $z_1 \le z \le z_2$ in our one-dimensional example---with a surface $\mathcal{S}$ that contains no free charges or currents, and that in principle we can solve Maxwell's equations everywhere in $\mathcal{V}$ for some external source configuration (e.g., one or more lasers). Referring to the general two-port component in \fig{two_port_fwd_smat}, we assume that the space in the region $z < z_1$ is filled with a material that has spatially uniform relative permittivity and permeability $\varepsilon_1\wn$ and $\mu_1\wn$ at the carrier frequency $\omega_0$, and that the corresponding constants have the values $\varepsilon_2\wn$ and $\mu_2\wn$ for $z > z_2$. Then the Poynting vectors evaluated just outside $\mathcal{V}$ at each of the reference planes $z_1$ and $z_2$ of $\mathcal{S}$ have the values
 \begin{subequations}  \label{eqn:poynting_vec_recip_1d}
 \begin{align}
 \label{eqn:poynting_vec_z1_1d} \mathbf{S}(z_1, t) &= \hatb{z}\, \left(\left|\mathbf{a}_1(t)\right|^2 - \left|\mathbf{b}_1(t)\right|^2\right) , \nd \\
 \label{eqn:poynting_vec_z2_1d} \mathbf{S}(z_2, t) &= \hatb{z}\, \left(\left|\mathbf{a}_2(t)\right|^2 - \left|\mathbf{b}_2(t)\right|^2\right) ,
 \end{align}
 \end{subequations}
where
 \begin{subequations}  \label{eqn:e_field_recip_1d}
 \begin{align}
 \label{eqn:a_field_z1_1d} \mathbf{a}_1(t) &= \left[ \frac{\varepsilon_0 c}{2 \eta_1\wn}\right]^\half\, \mathbf{F}_1(t) , \\
 \label{eqn:b_field_z1_1d} \mathbf{b}_1(t) &= \left[ \frac{\varepsilon_0 c}{2 \eta_1\wn}\right]^\half\, \mathbf{E}_1(t) , \\
 \label{eqn:a_field_z2_1d} \mathbf{a}_2(t) &= \left[ \frac{\varepsilon_0 c}{2 \eta_2\wn}\right]^\half\, \mathbf{F}_2(t) , \\
 \label{eqn:b_field_z2_1d} \mathbf{b}_2(t) &= \left[ \frac{\varepsilon_0 c}{2 \eta_2\wn}\right]^\half\, \mathbf{E}_2(t) ,
 \end{align}
 \end{subequations}
and $\eta_j\wn \equiv \Re\left[\mu_j\wn\right]/n_j\wn$. \red{\eqn{poynting_vec_recip_1d} may not be obvious to the student, and I probably need to discuss the appropriate spatial averaging concepts in \sct{em_wave_eqn_idm} right after \eqn{poynting_vector_idm_1d}.}

 \begin{figure}
  \centering
  \begin{subfigure}[t]{0.45\textwidth}
   \centering
   \includegraphics[height=3.25in]{figures/two_port_fwd_smat}
   \caption{Schematic of a general two-port element}
   \label{fig:two_port_fwd_smat}
  \end{subfigure}
  \qquad
  \begin{subfigure}[t]{0.45\textwidth}
   \centering
   \includegraphics[height=3.25in]{figures/two_port_rev_smat}
   \caption{Time-reversed excitation of a two-port element}
   \label{fig:two_port_rev_smat}
  \end{subfigure}
  \caption{\label{fig:two_port_smat} A schematic of a general two-port linear optical element under both forward and time-reversed excitations in the frequency domain.}
 \end{figure}

Let's now conduct a \emph{gedanken} (``thought'') experiment where we first configure sources to generate input fields with complex vector envelope functions $\mathbf{F}_1(t)$ and $\mathbf{F}_2(t)$, and then we measure the resulting output fields $\mathbf{E}_1(t)$ and $\mathbf{E}_2(t)$.  In other words, from \eqn{poynting_vec_recip_1d} and \eqn{e_field_recip_1d}, at time $t$ the total energy flux (intensity) incident on $\mathcal{S}$ is $|\mathbf{a}_1(t)|^2 + |\mathbf{a}_2(t)|^2$, and the total intensity emerging from $\mathcal{S}$ is $|\mathbf{b}_1(t)|^2 + |\mathbf{b}_2(t)|^2$. Since the two-port optical system under investigation may incorporate resonant elements that cause time delays, these intensities are not necessarily equal at any given time. However, if all fields have amplitudes which approach 0 as $t \longrightarrow \pm \infty$, and $\Im[\epsilon\wn \mu\wn] = 0$ everywhere in $\mathcal{V}$, then we can express conservation of \emph{energy fluence}\index{Energy fluence} (i.e., energy per unit area) as
 \begin{equation} \label{eqn:two_port_fluence_conserv_t}
 \int_{-\infty}^{\infty} d t\, \left[|\mathbf{a}_1(t)|^2 + |\mathbf{a}_2(t)|^2 - |\mathbf{b}_1(t)|^2 - |\mathbf{b}_2(t)|^2\right] = 0 .
 \end{equation}
We can use the Fourier Power Theorem given by \eqn{fourier_power_thm} to write this equation in the frequency domain as
 \begin{equation} \label{eqn:two_port_fluence_conserv_w}
 \int_{-\infty}^{\infty} \frac{d \omega}{2 \pi}\, \left[|\mathbf{a}_1(\omega)|^2 + |\mathbf{a}_2(\omega)|^2 - |\mathbf{b}_1(\omega)|^2 - |\mathbf{b}_2(\omega)|^2\right] = 0 ,
 \end{equation}
where $\mathbf{a}_1(\omega)$, $\mathbf{a}_2(\omega)$, $\mathbf{b}_1(\omega)$, and $\mathbf{b}_2(\omega)$ are the Fourier transforms of their counterparts in \eqn{e_field_recip_1d}. But the key difference between the time and frequency domains is that we expect the input and output fields to be coupled by simple constitutive relationships similar to \eqn{drw_def}, or
 \begin{equation} \label{eqn:two_port_fwd_smat}
 \mathbf{b}(\omega) = \mathbf{M}(\omega)\, \mathbf{a}(\omega) ,
 \end{equation}
where we have defined the (concatenated) vectors
 \begin{equation} \label{eqn:ab_def_w}
\mathbf{a}(\omega) \equiv \begin{bmatrix} \mathbf{a}_1(\omega) \\ \mathbf{a}_2(\omega) \end{bmatrix}  \quad \nd \quad \mathbf{b}(\omega) \equiv \begin{bmatrix} \mathbf{b}_1(\omega) \\ \mathbf{b}_2(\omega) \end{bmatrix} ,
 \end{equation}
and $\mathbf{M}(\omega)$ is the square \emph{scattering} matrix\index{Scattering matrix!General two-port component}
 \begin{equation} \label{eqn:two_port_smat}
\mathbf{M}(\omega) \equiv \begin{bmatrix} M_{11}(\omega) & M_{12}(\omega)  \\ M_{21}(\omega) & M_{22}(\omega) \end{bmatrix} .
 \end{equation}
Therefore, in the frequency domain, the \emph{integrand} of \eqn{two_port_fluence_conserv_w} should be zero at all values of $\omega$, and we must have
 \begin{equation} \label{eqn:two_port_intensity_conserv_w}
\mathbf{a}^\dagger(\omega) \mathbf{a}(\omega) = \mathbf{b}^\dagger(\omega) \mathbf{b}(\omega) ,
 \end{equation}
where $\mathbf{a}^\dagger(\omega) \equiv [\mathbf{a}_1^\dagger(\omega)~~\mathbf{a}_2^\dagger(\omega)]$ and $\mathbf{b}^\dagger(\omega) \equiv [\mathbf{b}_1^\dagger(\omega)~~\mathbf{b}_2^\dagger(\omega)]$ are the conjugate transposes of $\mathbf{a}(\omega)$ and $\mathbf{b}(\omega)$, respectively\footnote{In the case where the elements of $\mathbf{a}(\omega)$ are scalars, $\mathbf{a}^\dagger(\omega) \equiv [a_1^\ast(\omega)~~a_2^\ast(\omega)]$.}. \Eqn{two_port_intensity_conserv_w} places a significant constraint on the possible values that the elements of $\mathbf{M}(\omega)$ can have, since
 \begin{equation}
 \begin{split}
 \mathbf{b}^\dagger(\omega) \mathbf{b}(\omega) - \mathbf{a}^\dagger(\omega) \mathbf{a}(\omega) &= \left[ \mathbf{a}^\dagger(\omega)\, \mathbf{M}^\dagger(\omega) \right] \left[ \mathbf{M}(\omega)\, \mathbf{a}(\omega) \right] - \mathbf{a}^\dagger(\omega) \mathbf{a}(\omega) \\
 &= \mathbf{a}^\dagger(\omega) \left[ \mathbf{M}^\dagger(\omega)\, \mathbf{M}(\omega) - \mathbf{1}\right] \mathbf{a}(\omega),
 \end{split}
 \end{equation}
or
 \begin{equation} \label{eqn:two_port_smat_cons}
\mathbf{M}^\dagger(\omega) = \mathbf{M}^{-1}(\omega) .
 \end{equation}
In other words, the scattering matrix $\mathbf{M}(\omega)$ is a unitary matrix, with the properties $|\det[\mathbf{M}(\omega)]| = 1$, and its row (or column) vectors (of length $n$) form an orthonormal set in $\mathbb{C}^n$. In the two-port case, this means that
 \begin{subequations} \label{eqn:two_port_unitary_constraint_w}
 \begin{gather}
 \label{eqn:two_port_intensity_constraint_w} \left|M_{11}(\omega)\right|^2 + \left|M_{21}(\omega)\right|^2 = \left|M_{12}(\omega)\right|^2 + \left|M_{22}(\omega)\right|^2 = 1, \nd \\
 \label{eqn:two_port_phase_constraint_w} \quad M_{11}^\ast(\omega) M_{12}(\omega) + M_{21}^\ast(\omega) M_{22}(\omega) = M_{11}^\ast(\omega) M_{21}(\omega) + M_{12}^\ast(\omega) M_{22}(\omega) = 0 .
 \end{gather}
 \end{subequations}

There are additional constraints placed on $\mathbf{M}(\omega)$ by the time-reversal symmetry of Maxwell's equations discussed in \sct{time_reversal_idm}. Suppose that we have solved \eqn{macro_maxwell_c} for the complex field amplitude functions $[\Erw, \Hrw]$ in a volume $\mathcal{V}$ for the case where the bounding surface $\mathcal{S}$ contains no free charges or currents, as well as no nonlinear polarization $\mathbf{P}\rw$. \Eqn{e_t_rw} and \eqn{h_t_rw} show that, in the frequency domain, the solutions of the time-reversed macroscopic Maxwell equations are $[\mathbf{E}^\ast\rw, -\mathbf{H}^\ast\rw]$ given appropriately time-reversed boundary conditions \emph{and} that $\Im[\varepsilon\rw] = \Im[\mu\rw] = 0$. In this case, \eqn{e_t_rt_1d} and the corresponding analog of \eqn{h_field_1d} show that the solutions to the time-reversed equations will propagate in the reverse direction, and from \eqn{poynting_vector_idm_tr} the Poynting vector will have reversed direction as well. Therefore, as shown in \fig{two_port_rev_smat}, we can make the replacements $\mathbf{a}(\omega) \longrightarrow \mathbf{b}^\ast(\omega)$ and $\mathbf{b}(\omega) \longrightarrow \mathbf{a}^\ast(\omega)$ in \eqn{two_port_fwd_smat} and it will remain valid. Therefore
 \begin{equation} \label{eqn:two_port_rev_smat}
\mathbf{a}^\ast(\omega) = \mathbf{M}(\omega)\, \mathbf{b}^\ast(\omega) ,
 \end{equation}
or, multiplying both sides of this equation by $\mathbf{M}^{-1}(\omega) = \mathbf{M}^\dagger(\omega)$ and then taking the complex conjugate,
 \begin{equation}
\mathbf{b}(\omega) = \mathbf{M}^T(\omega)\, \mathbf{a}(\omega) ,
 \end{equation}
where $\mathbf{M}^T(\omega)$ is the transpose of $\mathbf{M}(\omega)$. If we compare \eqn{two_port_smat_symm} with \eqn{two_port_fwd_smat}, we see immediately that we must have
 \begin{equation} \label{eqn:two_port_smat_symm}
\mathbf{M}^T(\omega) = \mathbf{M}(\omega) ,
 \end{equation}
so that the scattering matrix is symmetric.\footnote{The principle of fluence conservation and the time-reversal symmetry of Maxwell's equations imply \emph{reciprocity}~\cite{ref:haus1984wfo}, a relationship between solutions of Maxwell's equations obtained on $\mathcal{S}$ for two different arrangements of sources external to $\mathcal{S}$.} In the case of our two-port scattering matrix, together with \eqn{two_port_intensity_constraint_w} this condition requires that
 \begin{equation} \label{eqn:two_port_symmetry_constraint_w}
M_{12}(\omega) = M_{21}(\omega), \quad \nd \quad \left|M_{11}(\omega)\right| = \left|M_{22}(\omega)\right| .
 \end{equation}

We are now equipped to find the scattering matrix of a laser mirror by applying \eqn{two_port_smat_cons} and \eqn{two_port_smat_symm} to \eqn{two_port_smat}. Let us specify that the mirror has an intensity reflectance $R$ and transmittance $T$ at frequencies near the carrier frequency $\omega_0$ that (in the lossless mirror case) satisfy $R + T = 1$. We choose our reference planes $z_1$ and $z_2$ so that $M_{11} = M_{22} = \sqrt{R}$, and we set $M_{12} = M_{21} = e^{i \varphi} \sqrt{T}$ for some real number $\varphi$. Comparing $\mathbf{M}^{-1}$ and $\mathbf{M}^\dagger$, we find that
 \begin{equation*}
R - e^{i 2 \varphi} T = 1 \quad \nd \quad e^{-i \varphi} = -e^{i \varphi} ,
 \end{equation*}
which requires $\varphi = (q + \half)\pi$, where $q \in \mathbb{Z}$. We choose $q = 0$, and find
 \begin{equation} \label{eqn:mirror_smat}
\mathbf{M} = \begin{bmatrix} \sqrt{R} & i \sqrt{T}  \\ i \sqrt{T} & \sqrt{R} \end{bmatrix} \qquad \text{(Mirror)}
 \end{equation}
at the carrier frequency $\omega_0$. Note that it is straightforward to update $\mathbf{M}$ for the two-port laser mirror when the locations of the reference planes are changed. For example, suppose that we choose the new reference planes $z'_1 = z_1 + \Delta z_1$ and $z'_2 = z_2 + \Delta z_2$. Then, from \eqn{e_field_recip_1d}, with \eqn{two_port_inputs} and \eqn{two_port_outputs}, we find that the input and output fields at the new input planes are
 \begin{equation}
\mathbf{a}'(\omega) = \begin{bmatrix} e^{+i k_1 \Delta z_1} & 0 \\ 0 & e^{-i k_2 \Delta z_2} \end{bmatrix} \mathbf{a}(\omega) \quad \nd \quad \mathbf{b}'(\omega) = \begin{bmatrix} e^{-i k_1 \Delta z_1} & 0 \\ 0 & e^{+i k_2 \Delta z_2} \end{bmatrix}  \mathbf{b}(\omega)
 \end{equation}
and that the new scattering matrix is given by
 \begin{equation} \label{eqn:mirror_smat_new}
 \begin{split}
\mathbf{M}' &= \begin{bmatrix} e^{-i k_1 \Delta z_1} & 0 \\ 0 & e^{+i k_2 \Delta z_2} \end{bmatrix} \mathbf{M} \begin{bmatrix} e^{-i k_1 \Delta z_1} & 0 \\ 0 & e^{+i k_2 \Delta z_2} \end{bmatrix} \\ &= \begin{bmatrix} e^{-i 2 k_1 \Delta z_1} \sqrt{R} & e^{-i (k_1 \Delta z_1 - k_2 \Delta z_2)} i \sqrt{T} \\ e^{-i (k_1 \Delta z_1 - k_2 \Delta z_2)} i \sqrt{T} & e^{i 2 k_2 \Delta z_2} \sqrt{R} \end{bmatrix} .
 \end{split}
 \end{equation}
For example, if we choose $k_1 \Delta z_1 = k_2 \Delta z_2 = \pi/2$, we will change the sign of the $\sqrt{R}$ elements in \eqn{mirror_smat}.

Although we developed the fluence conservation and symmetry principles given by \eqn{two_port_smat_cons} and \eqn{two_port_smat_symm} for two-port components, they are generally applicable to lossless, time-reversible systems with any number of ports. For example, the four-port beamsplitter shown in \fig{bs_smat} has the scattering matrix
 \begin{equation}
\begin{bmatrix} \mathbf{b}_1(\omega) \\ \mathbf{b}_2(\omega) \\ \mathbf{b}_3(\omega) \\ \mathbf{b}_4(\omega) \end{bmatrix} = \begin{bmatrix} 0 & M_{12} & M_{13} & 0 \\ M_{21} & 0 & 0 & M_{24} \\ M_{31} & 0 & 0 & M_{34} \\ 0 & M_{42} & M_{43} & 0 \end{bmatrix} \begin{bmatrix} \mathbf{a}_1(\omega) \\ \mathbf{a}_2(\omega) \\ \mathbf{a}_3(\omega) \\ \mathbf{a}_4(\omega) \end{bmatrix} ,
 \end{equation}
where we have used the schematic in the figure to determine \emph{a priori} which matrix elements are zero. First, we apply \eqn{two_port_smat_symm} to symmetrize the matrix, and then we apply \eqn{two_port_smat_cons} to find
 \begin{equation*}
M_{12} = -M^\ast_{34}, \quad M_{13} = M^\ast_{24}, \quad \nd \quad \left|M_{12}\right|^2 + \left|M_{13}\right|^2 = 1 ,
 \end{equation*}
where we have chosen the locations of the four reference planes shown in \fig{bs_smat} so that $M_{13} M_{24} - M_{12} M_{34} = 1$, giving $\Det[M] = (M_{13} M_{24} - M_{12} M_{34})^2 = 1$. Following our derivation of the scattering matrix for a laser mirror, we choose $M_{13} = M_{24} = \sqrt{R}$ and $M_{12} = M_{34} = i \sqrt{T}$ (consistent with the lossless condition $R + T = 1$), and obtain
 \begin{equation} \label{eqn:bs_smat}
\mathbf{M} = \begin{bmatrix} 0 & i \sqrt{T} & \sqrt{R} & 0 \\ i \sqrt{T} & 0 & 0 & \sqrt{R} \\ \sqrt{R} & 0 & 0 & i \sqrt{T} \\ 0 & \sqrt{R} & i \sqrt{T} & 0 \end{bmatrix} \qquad \text{(Beamsplitter)}
 \end{equation}
at the carrier frequency $\omega_0$.

Finally, we note that no real laser mirror satisfies the lossless requirement that $R + T = 1$. The dielectric coatings applied to the substrate---as well as the substrate itself---generally will absorb a fraction $A$ of incident light and convert it to heat, as well as scatter a fraction $S$ of that light (usually incoherently) into spatial modes that we aren't studying. In practice, then, $R + T + A + S = 1$, and therefore $R + T < 1$. However, when we model systems incorporating mirrors and dielectric propagation regions such as waveguides, we can build scattering matrices as we have above for arbitrary values of (for example) $R$ and $T$, and then check that these matrices are unitary and symmetric when we assume that all media are lossless.

 \subsection{Resonant Cavities\label{sct:resonator_1d_smat}}

Consider the schematic representation of the scattering matrices of the resonant cavities shown in \fig{resonator_1d_smat}. We have constructed each resonator using two partially-reflecting mirrors separated by one or more dielectric propagation regions characterized by a uniform relative permeability and a longitudinally-varying relative permittivity. We assume for convenience (consistent with typical macroscopic laser cavities) that $\eta\wn = 1$ at the external reference plane of each mirror, allowing us to work with electric field amplitudes rather than the scaled amplitudes given by \eqn{e_field_recip_1d}, and that the electromagnetic impedance has the same value $\eta\wn \equiv \eta$ at the internal reference plane of each mirror\footnote{\red{These assumptions can be relaxed, but special care must be taken to ensure that---consistent with Maxwell's equations---the transversely-polarized electric field amplitude is preserved across dielectric interfaces, and that intracavity gain dipoles interact with the local laser electric field.}}. We will use the scattering matrix of the mirror given by \eqn{mirror_smat} and rely on the discussion of the solutions of the dielectric propagation equation provided in \sct{wg_smat} to build a two-port scattering matrix applicable to three different types of one-dimensional resonator:

 \begin{figure}
  \centering
  \begin{subfigure}[b]{0.8\textwidth}
   \centering
   \includegraphics[width=4.5in]{figures/resonator_1d_sw_smat}
   \caption{Standing-wave resonator}
   \label{fig:resonator_1d_sw_smat}
  \end{subfigure}
  \par\vspace{0.25in}
  \begin{subfigure}[b]{0.8\textwidth}
   \centering
   \includegraphics[width=4.5in]{figures/resonator_1d_ring_smat}
   \caption{Ring resonator}
   \label{fig:resonator_1d_ring_smat}
  \end{subfigure}
  \par\vspace{0.25in}
  \begin{subfigure}[b]{0.8\textwidth}
   \centering
   \includegraphics[width=4.5in]{figures/resonator_1d_microring_smat}
   \caption{Microring resonator}
   \label{fig:resonator_1d_microring_smat}
  \end{subfigure}
  \caption{\label{fig:resonator_1d_smat}Schematic diagram of the mirror reflection and transmission coefficients used to determine the intracavity enhancement and output fields of common one-dimensional resonators.}
 \end{figure}

 \begin{description}
 \item[Standing-Wave Resonator.] A special case of the Fabry-Perot Interferometer, the standing-wave resonator shown schematically in \fig{resonator_1d_sw_smat} is a two-port one-dimensional optical component built using two laser mirrors and a dielectric propagation region. In this chapter, we adopt the convention that the \emph{round-trip} physical length of the propagation region enclosed within the mirrors is $L$, rather than the single-pass length, for consistency with both ring resonators and paraxial models of three-dimensional intracavity transverse modes. Toward this end, we calculate a single complex field envelope function $\mathbf{E}(z, \omega)$ as a function of position in the cavity, with $0 < z < 1/2$ as the coordinate for propagation from the reference plane at $\mathcal{M}_1$ (just \emph{after} reflection) to the reference plane at $\mathcal{M}_2$ (just \emph{before} reflection), and $1/2 < z < 1$ as the coordinate for propagation from the reference plane at $\mathcal{M}_2$ (just \emph{after} reflection) to the reference plane at $\mathcal{M}_1$ (just \emph{before} reflection). We note that this approach is completely consistent with the formulation of wave propagation through the dielectric region given in \sct{wg_smat}, and in particular with \eqn{e_prop_1d_p} and \eqn{e_prop_1d_m}. We will link $\mathbf{E}(0, \omega)$ to $\mathbf{E}(1, \omega)$ using the scattering matrix for $\mathcal{M}_1$ given by \eqn{mirror_smat}. We will find that the scattering matrix for the standing-wave resonator is symmetric, and (in the lossless case) unitary.
 \item[Ring Resonator.] We will treat the one-dimensional ring resonator shown schematically in \fig{resonator_1d_ring_smat} as a two-port optical component, even though (strictly speaking) it is a four-port element with reference planes similar to those of the beamsplitter shown in \fig{bs_smat}. Usually, the ring geometry is chosen for laser applications with the intention of operating it unidirectionally by incorporating nonreciprocal elements such as Faraday rotation devices with polarizers. As an interferometer, this configuration is generally excited in only one propagation direction by choosing noncoincident input and output planes at the two mirrors, and we ignore backscattering by mirrors and other intracavity surfaces. In that case, the parameters and reference planes of this system can be treated just as in the case of the standing-wave resonator, but there is no guarantee that the $2 \times 2$ scattering matrix will in fact be symmetric (whereas a properly constructed $4 \times 4$ scattering matrix for this configuration would be explicitly symmetric.) Although we do not have to place mirror $\mathcal{M}_2$ at $z = 1/2$ (and this certainly is not always the case in practice), we follow this approach here for pedagogical convenience.
 \item[Microring Resonator.] For completeness, our discussion will necessarily apply to the one-dimensional microring resonator shown schematically in \fig{resonator_1d_microring_smat} as a two-port optical component, consisting of a ring waveguide adjacent to two bus waveguides. As in the case of the ring resonator, we reduce the $4 \times 4$ scattering matrix to a $2 \times 2$ matrix by judicious choice of excitation direction. The perimeter of the ring---measured at the center of the curved waveguide---is $L$. As implied by the insets of \fig{resonator_1d_microring_smat}, the broadband power reflectances $R_j$ and transmittances $T_j$ at ports 1 and 2 are determined by the dimensions of the waveguides and the separation between them. We assume that the relative (effective) permittivity is uniform in the ring and the adjacent waveguides; thus, the scattering matrix is symmetric, and (again, in the lossless case) unitary.
 \end{description}

Let us begin by finding the propagating intracavity field amplitude at the reference planes in the ring resonator shown in \fig{resonator_1d_ring_smat}; our result will apply to the other two cases in \fig{resonator_1d_smat} by construction. Using \eqn{mirror_smat}, the boundary condition satisfied by $\mathbf{E}(0, \omega)$ and $\mathbf{E}(1, \omega)$ is
 \begin{equation} \label{eqn:resonator_1d_w_bc}
\mathbf{E}(0, \omega) = \sqrt{R_1}\, \mathbf{E}(1, \omega) + i \sqrt{T_1}\, \sqrt{\eta}\, \mathbf{F}_1(\omega) ,
 \end{equation}
with a similar condition for the field at $z = 1/2$. Propagating $\mathbf{E}(0, \omega)$ to $z = 1/2$ (just prior to reflection at $\mathcal{M}_2$) and then $\mathbf{E}(1/2, \omega)$ to $z = 1$ yields
 \begin{subequations} \label{eqn:fpi_icf_eqn}
 \begin{align}
 \mathbf{E}(1/2, \omega) &= G(1/2, \omega) \mathbf{E}(0, \omega) , \nd \\
 \mathbf{E}(1, \omega) &= \frac{G(1, \omega)}{G(1/2, \omega)} \left[ \sqrt{R_2}\, \mathbf{E}(1/2, \omega) + i \sqrt{T_2} \, F_2(\omega) \right],
 \end{align}
 \end{subequations}
where $G(1/2, \omega)$ and $G(1, \omega)$ are found using \eqn{prop_gzw_def}\footnote{Since --- by assumption --- $\eta(z) \equiv \eta$ at each internal mirror reference plane, we can use \eqn{prop_gzw_def} without modification.}. These coupled linear equations are readily solved to obtain
 \begin{equation} \label{eqn:resonator_1d_enhancement_soln}
\begin{bmatrix} \mathbf{E}(1, \omega) \\ \mathbf{E}(1/2, \omega) \end{bmatrix} =
\sqrt{\eta}\, \mathbf{H}(\omega) \begin{bmatrix} \mathbf{F}_1(\omega) \\ \mathbf{F}_2(\omega) \end{bmatrix}
 \end{equation}
where the \emph{enhancement matrix}\index{Enhancement matrix!One-dimensional resonator} for a one-dimensional resonant cavity is given by
 \begin{equation} \label{eqn:resonator_1d_enhancement_matrix}
\mathbf{H}(\omega) = \frac{1}{1 - \sqrt{R_1 R_2}\, G(1, \omega)} \begin{bmatrix} i \sqrt{T_1 R_2}\, G(1, \omega) & i \sqrt{T_2}\, \frac{G(1, \omega)}{G(1/2, \omega)}  \\ i \sqrt{T_1}\, G(1/2, \omega) & i \sqrt{R_1 T_2}\, G(1, \omega) \end{bmatrix}
 \end{equation}
Applying \eqn{mirror_smat} to the exterior reference planes of mirrors $\mathcal{M}_1$ and $\mathcal{M}_2$ allows us to calculate the output fields $\mathbf{E}_1$ and $\mathbf{E}_2$ as
 \begin{equation}
\begin{bmatrix} \mathbf{E}_1(\omega) \\ \mathbf{E}_2(\omega) \end{bmatrix} = \begin{bmatrix} \sqrt{R_1} & 0 \\ 0 & \sqrt{R_2} \end{bmatrix} \begin{bmatrix} \mathbf{F}_1(\omega) \\ \mathbf{F}_2(\omega) \end{bmatrix} + \begin{bmatrix} i \sqrt{T_1} & 0 \\ 0 & i \sqrt{T_2} \end{bmatrix}\, \frac{1}{\sqrt{\eta}}\, \begin{bmatrix} \mathbf{E}(1, \omega) \\ \mathbf{E}(1/2, \omega) \end{bmatrix} ,
 \end{equation}
or, after substitution of \eqn{resonator_1d_enhancement_soln} and straightforward algebra,
 \begin{equation} \label{eqn:resonator_1d_w_soln}
 \begin{bmatrix} \mathbf{E}_1(\omega) \\ \mathbf{E}_2(\omega) \end{bmatrix} =
\mathbf{M}(\omega) \begin{bmatrix} \mathbf{F}_1(\omega) \\ \mathbf{F}_2(\omega) \end{bmatrix} ,
 \end{equation}
where the two-port scattering matrix for a one-dimensional resonant cavity is\index{Scattering matrix!Resonant cavity (1D)}
 \begin{equation} \label{eqn:resonator_1d_smat_two_port}
 \begin{split}
 \mathbf{M}(\omega) &= \begin{bmatrix} \sqrt{R_1} & 0 \\ 0 & \sqrt{R_2} \end{bmatrix} + \begin{bmatrix} i \sqrt{T_1} & 0 \\ 0 & i \sqrt{T_2} \end{bmatrix} \mathbf{H}(\omega) \\
 &= \frac{1}{1 - \sqrt{R_1 R_2}\, G(1, \omega)}
\begin{bmatrix} \sqrt{R_1} - \sqrt{R_2} \left(R_1 + T_1\right) G(1, \omega) & -\sqrt{T_1 T_2}\, \frac{G(1, \omega)}{G(1/2, \omega)} \\ -\sqrt{T_1 T_2}\, G(1/2, \omega) & \sqrt{R_2} - \sqrt{R_1} \left(R_2 + T_2\right) G(1, \omega) \end{bmatrix} .
 \end{split}
 \end{equation}
In the special case where $\varepsilon(1 - z, \omega_0) = \varepsilon(z, \omega_0)$, we have $G(1, \omega)/G(1/2, \omega) = G(1/2, \omega) = \sqrt{G(1, \omega)}$, and the scattering matrix becomes manifestly symmetric. If we also assume that the mirrors are lossless, so that $R_1 + T_1 = R_2 + T_2 = 1$, then we have
 \begin{equation} \label{eqn:resonator_1d_smat}
 \mathbf{M}(\omega) = \frac{1}{1 - \sqrt{R_1 R_2}\, G(1, \omega)}
\begin{bmatrix} \sqrt{R_1} - \sqrt{R_2}\, G(1, \omega) & -\sqrt{T_1 T_2\, G(1, \omega)} \\ -\sqrt{T_1 T_2\, G(1, \omega)} & \sqrt{R_2} - \sqrt{R_1}\, G(1, \omega) \end{bmatrix} .
 \end{equation}
The determinant of this matrix is $-[G(1, \omega) - \sqrt{R_1 R_2}]/[1 - \sqrt{R_1 R_2}\, G(1, \omega))]$, which has an absolute value of unity if $\alpha(z) = 0$ and therefore $|G(1, \omega)| = 1$. In this case, it is also straightforward to show that $\mathbf{M}^\dagger(\omega) \mathbf{M}(\omega) = \mathbf{1}$, demonstrating that $\mathbf{M}(\omega)$ is unitary.

   %%%%%%%%%%%%%%%%%%%%%%%%%%%%%%%%%%%%%%%%%%%%%%%%%%%%%%%%%%%%%%%%%%%%%%%%%%%%%%
%
% Section file included in chapter file using \input{}
%
% Assumes that LaTeX2e macros and packages defined in rgb_laser_physics.sty
%   are available
%
% $Id$
%
%%%%%%%%%%%%%%%%%%%%%%%%%%%%%%%%%%%%%%%%%%%%%%%%%%%%%%%%%%%%%%%%%%%%%%%%%%%%%%

 \section{Properties of Resonant Cavities in the Time and Frequency Domains\label{sct:laser_resonators_1d_freq}}

 \subsection{Input-Output Relations for Resonant Cavities in the Time Domain\label{sct:laser_resonators_1d_time}}

If we define the functions
 \begin{subequations} \label{eqn:fpi_fghw_def}
 \begin{align}
 \label{eqn:fpi_fw_def} f(\omega) &= \frac{1}{1 - \Gamma\, e^{i \omega}}\, , \\
 \label{eqn:fpi_gw_def} g(\omega) &= \frac{\sqrt{\Gamma}\, e^{i \omega/2}}{1 - \Gamma\, e^{i \omega}}\, , \nd \\
 \label{eqn:fpi_hw_def} h(\omega) &= \frac{\Gamma\, e^{i \omega}}{1 - \Gamma\, e^{i \omega}}\, ,
 \end{align}
 \end{subequations}
where
 \begin{equation} \label{eqn:fpi_gamma_def}
\Gamma \equiv \sqrt{R_1 R_2} \exp\left[ -\half \alpha\wn + i \omega_0 \tau \right]\, ,
 \end{equation}
then after straightforward algebra we can rewrite \eqn{resonator_1d_smat} as
 \begin{equation} \label{eqn:resonator_1d_smat_fgh}
 \mathbf{M}(\omega) =
\begin{bmatrix} \sqrt{R_1} f(\omega) - \frac{1}{\sqrt{R_1}} h(\omega) & -\sqrt{ \frac{T_1 T_2}{\sqrt{R_1 R_2}} }\, g(\omega) \\ -\sqrt{ \frac{T_1 T_2}{\sqrt{R_1 R_2}} }\, g(\omega) & \sqrt{R_2} f(\omega) - \frac{1}{\sqrt{R_2}} h(\omega) \end{bmatrix} .
 \end{equation}
In \eqn{fpi_gamma_def}, we have defined the phase round-trip time
 \begin{equation} \label{eqn:tau_phase_def}
\tau \equiv \frac{n\wn L}{c}\, ,
 \end{equation}
which (like $\omega_0^{-1}$) has been scaled by the group round-trip time
 \begin{equation}\label{eqn:tau_group_def}
\tau_g \equiv \frac{n^\prime\wn L}{c}\, .
 \end{equation}

Now we would like to substitute \eqn{resonator_1d_smat_fgh} into \eqn{resonator_1d_w_soln}, and then apply an inverse Fourier transform to find $\mathbf{E}_1(t)$ and $\mathbf{E}_2(t)$. Since $|\Gamma e^{i \omega}| < 1$, we can expand $f(\omega)$ as
 \begin{equation} \label{eqn:f_expansion}
f(\omega) = \sum_{n = 0}^\infty \left( \Gamma e^{i \omega} \right)^n ,
 \end{equation}
giving the inverse transform of $f(\omega)\, \mathbf{F}(\omega)$ as
 \begin{subequations} \label{eqn:fhg_transform_pairs}
 \begin{equation}
 \begin{split} \label{eqn:f_transform_pair}
\int_{-\infty}^{+\infty} \frac{d \omega}{2 \pi}\, \emwt f(\omega)\, \mathbf{F}(\omega) &= \int_{-\infty}^{+\infty} \frac{d \omega}{2 \pi}\, \emwt \sum_{n = 0}^\infty \Gamma^n e^{i n \omega}\, \mathbf{F}(\omega) \\
     &= \sum_{n = 0}^\infty \Gamma^n \int_{-\infty}^{+\infty} \frac{d \omega}{2 \pi}\, e^{-i \omega (t - n)}\, \mathbf{F}(\omega) \\
     &= \sum_{n = 0}^\infty \Gamma^n\, \mathbf{F}\left[t - n\right] ,
 \end{split}
 \end{equation}
where we have applied the Fourier Shift Theorem\index{Fourier Shift Theorem} given by \eqn{fourier_shift_thm}. Following the same approach for both $g(\omega)$ and $h(\omega)$ yields
 \begin{align}
\int_{-\infty}^{+\infty} \frac{d \omega}{2 \pi}\, \emwt g(\omega)\, \mathbf{F}(\omega) &= \sum_{n = 0}^\infty \Gamma^{\left(n + \half\right)}\, \mathbf{F}\left[t - \left(n + \half\right)\right] , \nd \\
\int_{-\infty}^{+\infty} \frac{d \omega}{2 \pi}\, \emwt h(\omega)\, \mathbf{F}(\omega) &= \sum_{n = 0}^\infty \Gamma^{\left(n + 1\right)}\, \mathbf{F}\left[t - \left(n + 1\right)\right] .
 \end{align}
 \end{subequations}
These transform pairs allow us to determine the output fields in the time domain given specified input fields. For example, suppose that $\mathbf{F}_2(t) = 0$; then together \eqn{resonator_1d_w_soln}, \eqn{resonator_1d_smat_fgh}, and \eqn{fhg_transform_pairs} predict that
 \begin{align}
\mathbf{E}_1(t) &= \sqrt{R_1}\, \mathbf{F}_1(t) - \frac{T_1}{\sqrt{R_1}} \sum_{n = 1}^\infty \Gamma^n\, \mathbf{F}_1\left[t - n\right] , \nd \\
\mathbf{E}_2(t) &= -\sqrt{ \frac{T_1 T_2}{\sqrt{R_1 R_2}} } \sum_{n =0}^\infty \Gamma^{\left(n + \frac{1}{2}\right)}\, \mathbf{F}_1\left[t - \left(n + \frac{1}{2}\right)\right] .
 \end{align}
Both of these input-output relations have a straightforward physical interpretation. $\mathbf{E}_1(t)$ is a coherent superposition of a prompt reflection from mirror $\mathcal{M}_1$ and fields that have experienced an integer number of past resonator round-trips. $\mathbf{E}_2(t)$ is a coherent superposition of past round-trips followed by a final, single propagation step from $\mathcal{M}_1$ to $\mathcal{M}_2$.

\red{Show the $\Theta(t)$ and gaussian pulse examples here.}

 \subsection{Input-Output Relations for Resonant Cavities in the Frequency Domain\label{sct:laser_resonators_1d_freq_io}}

 \subsubsection{Partial-Fraction Expansion of the Scattering Matrix\label{sct:resonator_1d_smat_pfe}}

Let's allow $\omega$ to take on complex values, and seek partial-fraction expansions of $f(\omega)$, $g(\omega)$, and $h(\omega)$ that will guide us to their approximations as finite sums of simple analytic functions. For example, since $1/f(\omega)$ is an analytic function, we can apply the Mittag-Leffler partial-fraction expansion~\cite{ref:stone2009mfp}
 \begin{equation} \label{eqn:mittag_leffler}
f(\omega) = f(0) + \sum_q r_q \left(\frac{1}{\omega - \nu_q} + \frac{1}{\nu_q}\right) ,
 \end{equation}
where $\nu_q$ is a pole of $f(\omega)$ with residue $r_q$. In the case of \eqn{fpi_fw_def}, we have
 \begin{subequations} \label{eqn:f_poles}
 \begin{align}
\nu_q &= 2 q \pi + i \ln \Gamma , \nd \label{eqn:f_pole_q} \\
r_q &= i , \label{eqn:f_residue_q}
 \end{align}
 \end{subequations}
so that
 \begin{subequations} \label{eqn:fgh_ml}
 \begin{equation} \label{eqn:f_ml}
 \begin{split}
f(\omega) &= \left[\frac{1}{1 - \Gamma} + \sum_{q = -\infty}^{+\infty}  \frac{i}{\nu_q}\right] + \sum_{q = -\infty}^{+\infty}  \frac{i}{\omega - \nu_q} \\
     &= \frac{1}{2} + \sum_{q = -\infty}^{+\infty}  \frac{i}{\omega - \nu_q} .
 \end{split}
 \end{equation}
The final \rhs of this representation of $f(\omega)$ is a superposition of individual \emph{Lorentzian}\index{Lorentzian} resonance modes that are each centered at the frequency $\omega = \Re(\nu_q) = 2 q \pi$ if $\Im(\Gamma) = 0$. Following the same approach for \eqn{fpi_gw_def} and \eqn{fpi_hw_def}, we find
 \begin{align}
g(\omega) &= \sum_{q = -\infty}^{+\infty}  \frac{(-1)^q\, i}{\omega - \nu_q} , \nd  \label{eqn:g_ml} \\
h(\omega) &= -\frac{1}{2} + \sum_{q = -\infty}^{+\infty}  \frac{i}{\omega - \nu_q} .  \label{eqn:h_ml}
 \end{align}
 \end{subequations}

Suppose that we have chosen the carrier frequency $\omega_0$ to be aligned with mode $q = 0$ of the unloaded cavity, such that $\exp(\pm i\, \omega_0 \tau) = 1$. The shape of the magnitude-squared of an individual Lorentzian mode, given by
 \begin{equation} \label{eqn:f_lorentz_shape}
\left| \frac{i}{\omega - \nu_0} \right|^2 = \frac{1}{\omega^2 + \left(\ln|\Gamma|\right)^2}\, ,
 \end{equation}
is characterized by the full-width at half-maximum (FWHM)\index{Full-width at half-maximum (FWHM)}, or the difference between the positive and negative frequencies that reduce \eqn{f_lorentz_shape} to $1/2 \left(\ln|\Gamma|\right)^2$. We find
 \begin{equation} \label{eqn:f_fwhm}
\Delta \omega_\text{FWHM} \equiv \frac{1}{\tau_p} = \ln\left( \frac{1}{|\Gamma|^2} \right)\, ,
 \end{equation}
where $\tau_p$ is the \emph{photon lifetime}\index{Photon lifetime} (scaled by the group round-trip time $\tau_g$), or the time duration of light storage in the resonator. A common parameter used to represent this storage time is the cavity ``quality factor'' $Q$, defined by
 \begin{equation} \label{eqn:f_q_def}
Q \equiv \frac{\omega_0}{\Delta \omega_\text{FWHM}} = \frac{\omega_0}{\ln(1/|\Gamma|^2)} = \omega_0\, \tau_p\, .
 \end{equation}
Therefore, we can express the complex resonance frequency of pole $q$ as
 \begin{equation} \label{eqn:f_pole_q_redef}
\nu_q = 2 q \pi - \frac{i}{2\, \tau_p} = 2 q \pi - i\, \frac{\omega_0}{2\, Q}\, .
 \end{equation}

In practice, we will approximate \eqn{fgh_ml} by a finite sum of frequency modes over some range of values of $q$ centered on the carrier frequency (which corresponds to $q = 0$ in our formulation of the wave equation). Truncating the infinite series introduces an error that would ordinarily require many more modes than are needed for a particular representation of a problem of interest, but it is straightforward to include a simple correction term. For example, by calculating the difference between \eqn{f_ml} and the corresponding truncated series to first order in $\omega$, we obtain
\begin{equation} \label{eqn:f_lorentzian_approx}
    f(\omega) \approx \half - \frac{\ln \Gamma + i \omega}{2 \pi^2}\, \psi^{(1)}(q_\mathrm{max} + 1) + \sum_{q = -q_\mathrm{max}}^{+q_\mathrm{max}}  \frac{i}{(\omega - 2 q \pi) - i \ln \Gamma}\, ,    
\end{equation}
where
\begin{equation*}
    \psi^{(1)}(n + 1) \approx \frac{2 n + 1}{2 (n + 0.125) (n + 1)}
\end{equation*}
is the first-order polygamma function. In \fig{f_lorentzian_approx} we have plotted \eqn{f_ml} and \eqn{f_lorentzian_approx} for the case $\Gamma = 0.5$ and $q_\mathrm{max} = 2$. Even for this moderate value of $|\Gamma|$ (i.e., far from unity), \eqn{f_lorentzian_approx} is a reasonably good approximation of $f(\omega)$ near $\omega = \Re(\nu_q)$ for $|q| \le q_\mathrm{max}$.
%  \begin{equation}
% f(\omega) \approx \frac{1}{2} - i\, \frac{\omega}{(2 \pi)^2}\, \frac{2 q_\mathrm{max} + 1}{q_\mathrm{max} (q_\mathrm{max} + 1)} + \sum_{q = -q_\mathrm{max}}^{+q_\mathrm{max}}  \frac{i}{(\omega - 2 q \pi) - i \ln \Gamma} ,
%  \end{equation}

% for the case $\Gamma = 0.5$ and $q_\mathrm{max} = 2$. Note that we have added an imaginary correction term --- linear in $\omega$ --- to compensate for sums with small $q_\mathrm{max}$ because $\Im[f(\omega)]$ is odd in $\omega - 2 q \pi$. 
 \begin{figure}
  \centering
  \begin{subfigure}[b]{0.8\textwidth}
   \centering
   \includegraphics[width=5.5in]{figures/real_fpml}
   \caption{ {$\Re[f(\omega)]$} }
   \label{fig:real_fpml}
  \end{subfigure}
  \par\vspace{0.25in}
  \begin{subfigure}[b]{0.8\textwidth}
   \centering
   \includegraphics[width=5.5in]{figures/imag_fpml}
   \caption{ {$\Im[f(\omega)]$} }
   \label{fig:imag_fpml}
  \end{subfigure}
  \caption{\label{fig:f_lorentzian_approx} A comparison of the value of the resonant function given by \eqn{f_ml} and the Lorentzian approximation defined by \eqn{f_lorentzian_approx} for the case where $\Gamma = 0.5$ and $q_\mathrm{max} = 2$. }
 \end{figure}

 \subsubsection{Finesse and Other Properties of Resonant Cavities\label{sct:resonator_1d_finesse}}

Let's explore the optical properties of the simple one-dimensional resonators shown in \fig{resonator_1d_smat} in the case where $F_2(\omega) = 0$. Relying on \eqn{prop_gzw_def}, \eqn{resonator_1d_w_bc}, \eqn{resonator_1d_w_soln}, and \eqn{resonator_1d_smat_two_port}, we find
 \begin{equation} \label{eqn:forward_prop_w_0}
E(0, \omega) = \frac{1}{1 - \Gamma\, e^{i \omega}} \left[ i\, \sqrt{\eta T_1}\, F(\omega) \right]\, ,
 \end{equation}
where $\eta \equiv \eta(0) = \eta(1)$ is the dimensionless electromagnetic impedance coefficient at the internal reference plane of $\mathcal{M}_1$, and from \eqn{fpi_gamma_def} we have $\Gamma^2 = R_1 R_2 e^{-\alpha\wn}$ when $\omega_0$ is chosen such that $\exp(i \omega_0 \tau) = 1$. The denominator
 \begin{equation}
\mathcal{D}(\omega) \equiv 1 - \Gamma\, e^{i \omega}
 \end{equation}
is so common in discussions of optical resonators that we should invest some time to understand its properties in more detail. We begin by defining $\phi(\omega) \equiv \omega/2$, and writing
 \begin{equation}
 \begin{split}
\frac{1}{\mathcal{D}(\omega)} &= \frac{1 - \Gamma\, e^{-i 2 \phi(\omega)}}{\left(1 - \Gamma\, e^{+i 2 \phi(\omega)}\right)\left(1 - \Gamma\, e^{-i 2 \phi(\omega)}\right)} \\
&\equiv \frac{e^{i \left[ \theta(\omega) - \phi(\omega) \right]}}{|\mathcal{D}(\omega)|}\, ,
 \end{split}
 \end{equation}
where $e^{i \theta(\omega)} \equiv [ e^{+i \phi(\omega)} - \Gamma\, e^{-i \phi(\omega)} ] / |\mathcal{D}(\omega)|$. We note that
 \begin{equation}
 \begin{split}
|\mathcal{D}(\omega)|^2 &= 1 + \Gamma^2 - 2\, \Gamma\, \cos [ 2 \phi(\omega) ] \\
&= \left(1 - \Gamma\right)^2 + 4\, \Gamma\, \sin^2 \phi(\omega) \\
&\equiv \left(1 - \Gamma\right)^2 \left[ 1 + \widetilde{F}\, \sin^2 \phi(\omega) \right]\, ,
 \end{split}
 \end{equation}
 where
 \begin{equation}\label{eqn:coefficient_of_finesse}
\widetilde{F} \equiv \frac{4\, \Gamma}{(1 - \Gamma)^2}
 \end{equation}
is known as the coefficient of finesse. Therefore
 \begin{equation} \label{eqn:fpi_denom_abs}
|\mathcal{D}(\omega)| = \left(1 - \Gamma\right)\, \sqrt{ 1 + \widetilde{F}\, \sin^2 \phi(\omega) }\, , \nd
 \end{equation}
 \begin{equation}
e^{i \theta(\omega)} = \frac{(1 - \Gamma)\, \cos \phi(\omega) + i\, (1 + \Gamma)\, \sin \phi(\omega)}{\left(1 - \Gamma\right)\, \sqrt{ 1 + \widetilde{F}\, \sin^2 \phi(\omega) }}\, .
 \end{equation}
Solving \eqn{coefficient_of_finesse} for $\Gamma$ yields
 \begin{equation}
\Gamma = 1 + \frac{2}{\widetilde{F}} - \frac{2}{\widetilde{F}}\, \sqrt{1 + \widetilde{F}}\, ,
 \end{equation}
giving $(1 + \Gamma)/(1 - \Gamma) = \sqrt{1 + \widetilde{F}}$ and
 \begin{equation}\label{eqn:resonator_1d_fp_enhance}
E(0, \omega) = \frac{i\, \sqrt{T_1}}{1 - \Gamma}\, \frac{e^{i [\theta(\omega) - \phi(\omega)]}}{\sqrt{1 + \widetilde{F} \sin^2 \phi(\omega)}} \, \sqrt{\eta}\, F_1(\omega)\, ,
 \end{equation}
where now
 \begin{equation} \label{eqn:sin_cos}
 \cos \theta(\omega) \equiv \frac{\cos \phi(\omega)}{\sqrt{1 + \widetilde{F} \sin^2 \phi(\omega)}}\, , \qquad \nd \qquad \sin \theta(\omega) \equiv \frac{\sqrt{1 + \widetilde{F}}\, \sin \phi(\omega)}{\sqrt{1 + \widetilde{F} \sin^2 \phi(\omega)}}\, .
 \end{equation}

We can use \eqn{resonator_1d_w_soln} and the same ideas that led us to \eqn{resonator_1d_fp_enhance} to solve for the \emph{reflected} field $E_1(\omega)$ and the \emph{transmitted} field $E_2(\omega)$. We find
 \begin{subequations} \label{eqn:resonator_1d_fp_out}
 \begin{align}
 \label{eqn:resonator_1d_fp_refl}
E_1(\omega) &= \frac{1}{\sqrt{R_1} (1 - \Gamma)}\, \frac{R_1 e^{-i \phi(\omega)} - (R_1 + T_1)\, \Gamma\, e^{i \phi(\omega)} }{\sqrt{1 + \widetilde{F} \sin^2 \phi(\omega)}}\, e^{i \theta(\omega)}\, F_1(\omega)\, , \nd \\
 \label{eqn:resonator_1d_fp_trans}
E_2(\omega) &= -\sqrt{\frac{T_1 T_2 \Gamma}{\sqrt{R_1 R_2} (1 - \Gamma)^2}}\, \frac{e^{i \theta(\omega)}}{\sqrt{1 + \widetilde{F} \sin^2 \phi(\omega)}}\, F_1(\omega)\, .
 \end{align}
 \end{subequations}
In the case where $R_1 + T_1 = R_2 + T_2 = 1$ and $\alpha\wn = 0$, it is straightforward to show that power is conserved for all $\omega$, since $|E_1(\omega)|^2 + |E_2(\omega)|^2 = |F_1(\omega)|^2$.

Whenever the frequency $\omega = \omega_q = 2 q \pi$ for some integer $q$, then $\phi(\omega) = q \pi$, and the magnitudes of both the reflected and transmitted fields reach their maximum values with $\theta(\omega_q) = \phi(\omega_q)$. Restoring the scaling by $\tau_g^{-1}$, these adjacent maxima are separated by the \emph{free spectral range}
 \begin{equation}\label{eqn:delta_w_fsr_def}
\Delta \omega_\mathrm{FSR} = \frac{2 \pi}{\tau_g}\, .
 \end{equation}
The power reflected and transmitted by the cavity is reduced by a factor of two relative to these maxima whenever $\widetilde{F} \sin^2 \phi(\omega) = 1$. The corresponding \emph{full width at half-maximum} (FWHM) is therefore
 \begin{equation}\label{eqn:delta_w_fwhm_def}
\Delta \omega_\mathrm{FWHM} = \frac{4}{\tau_g}\, \sin^{-1} \frac{1}{\sqrt{\widetilde{F}}} \equiv \frac{\Delta \omega_\mathrm{FSR}}{\mathcal{F}}\, ,
 \end{equation}
where
 \begin{equation} \label{eqn:finesse_def}
\mathcal{F} \equiv \frac{\pi}{2\, \sin^{-1}\left(1/\sqrt{\widetilde{F}}\right)} \approx \frac{\pi}{2}\, \sqrt{\widetilde{F}}\, ,
 \end{equation}
is known as the \emph{finesse} of the resonator, with the last approximation valid if $\widetilde{F} \gg 1$. At first glance, \eqn{delta_w_fwhm_def} does not appear to be completely consistent with that of \eqn{f_fwhm}. The difference arises because \eqn{resonator_1d_fp_enhance} includes the factor of $1/2$ in \eqn{f_ml}, so the way that we have defined ``half-maximum'' is not the same in the two cases. In the limit $\Gamma \longrightarrow 1$, both methods yield $\Delta \omega_\mathrm{FWHM} \approx 2 (1 - \Gamma) / \tau_g$. The relationship between $Q$, defined by \eqn{f_q_def}, and the finesse is
 \begin{equation} \label{eqn:q_finesse}
Q = \frac{L}{\lambda_g}\, \mathcal{F}\, ,
 \end{equation}
where $\lambda_g \equiv \lambda_0 / n^\prime\wn$ is the ``group wavelength'' of an input field with vacuum wavelength $\lambda_0$. In macroscopic resonators ($L \sim 1$~m), the ratio $L/\lambda_g$ is so large that $Q$ isn't a particularly useful metric of cavity quality, but $Q$ is routinely chosen to describe the optical quality of microscale and nanoscale resonators.

 \subsubsection{Resonant Cavities as Linear Filters\label{sct:resonator_1d_filter}}

Comparing \eqn{f_expansion} and \eqn{f_ml}, note that we have demonstrated the identity
 \begin{equation} \label{eqn:f_identity_w}
 f(\omega) = \sum_{n = 0}^\infty \left( \Gamma e^{i \omega} \right)^n = \frac{1}{2} + \sum_{q = -\infty}^{+\infty}  \frac{i}{\omega - \nu_q} ,
 \end{equation}
subject to our usual disclaimers about mathematical rigor. Therefore, taking the Fourier transform of all terms in this expression, we find
 \begin{equation} \label{eqn:f_identity_t}
 f(t) = \sum_{n = 0}^\infty \Gamma^n \delta(t - n) = \frac{1}{2} \delta(t) + \Theta(t) \sum_{q = -\infty}^{+\infty}  e^{-i \nu_q t} ,
 \end{equation}
where we have used the Fourier Shift Theorem given by \eqn{fourier_shift_thm} and the form of the Dirac delta function shown in \eqn{dirac_delta_1d_ft} adapted for the time domain. The sum over $n$ is completely consistent with \eqn{f_transform_pair} in the context of the Fourier Convolution Theorem, \eqn{fourier_conv_thm}. The sum over $q$ is easily calculated using contour integration, or by the Mathematica command shown in \lst{laser_resonators_1d_mma_ift_complex_lorentzian}. Both series are consistent with the causality requirements~\cite{ref:stone2009mfp} discussed in \sct{math_prelim_fourier_conv_thm}.

 \lstinputlisting[language=Mathematica,caption={Mathematica Command for Inverse Fourier Transform of a Complex Lorentzian},label=lst:laser_resonators_1d_mma_ift_complex_lorentzian]{files/laser_resonators_1d_mma_ift_complex_lorentzian.txt}

\red{Note that the frequency dependence of $R_1$ and $R_2$ can be captured by letting $\Gamma \longrightarrow \Gamma_q \equiv \Gamma(\omega_0 + 2 q \pi/\tau_g)$. Also, this subsubsection really needs to include a reason to work in the frequency domain at all, given that the time domain is so simple; this is a good place to use the power theorem to estimate the transmitted and reflected power.}

   %%%%%%%%%%%%%%%%%%%%%%%%%%%%%%%%%%%%%%%%%%%%%%%%%%%%%%%%%%%%%%%%%%%%%%%%%%%%%%
%
% Section file included in chapter file using \input{}
%
% Assumes that LaTeX2e macros and packages defined in rgb_laser_physics.sty
%   are available
%
% $Id: $
%
%%%%%%%%%%%%%%%%%%%%%%%%%%%%%%%%%%%%%%%%%%%%%%%%%%%%%%%%%%%%%%%%%%%%%%%%%%%%%%

 \section{Quasi-Normal Modes\label{sct:laser_resonators_1d_qnm}}

In \eqn{bmc_e_def}, we have defined the real vector electric field that is rapidly varying in both space and time as
 \begin{equation}
\bmc{E}\rt = \frac{\emwnt}{2}\, \Ert + \cc ,
 \end{equation}
where $\Ert$ is a complex vector electric field amplitude that varies rapidly in space but slowly in time. In turn, following \eqn{e_field_1d_pz}, we express this vector amplitude for the case of forward and backward propagating waves in one dimension in terms of the complex scalar amplitudes $E^+\zt$ and $E^-\zt$ of fields that varies slowly in both space and time as
 \begin{subequations}
 \begin{align}
 \mathbf{E}^{+}\rt &\equiv \hatb{\epsilon}\, e^{+i\, k_0\wn\, z}\, E^{+}\zt , \nd \\
 \mathbf{E}^{-}\rt &\equiv \hatb{\epsilon}\, e^{-i\, k_0\wn\, z}\, E^{-}\zt
 \end{align}
 \end{subequations}
where $\hatb{\epsilon}$ is a complex polarization vector and $k_0\wn = \Re\left[\beta\wn\right] = (\omega_0/c) n\wn$ is the effective propagation vector magnitude. These slowly-varying envelope functions solve the appropriate homogeneous wave equation, given by \eqn{wave_eqn_1d_w_lin_pz} and \eqn{wave_eqn_1d_w_lin_mz} in the frequency domain. In the following discussion, our goal will be to use the solutions to these equations --- given in \sct{wg_smat} --- to find the frequencies of the eigenmodes of \eqn{wave_eqn_1d_w_lin_pz} and \eqn{wave_eqn_1d_w_lin_mz} that satisfy the internal boundary conditions appropriate for a laser resonator.
%\eqn{wave_eqn_1d} in the time domain as
% \begin{subequations} \label{eqn:wave_eqn_1d_fb}
% \begin{align}
% \label{eqn:wave_eqn_1d_forward} \ppt E^+\zt + \frac{c}{n^\prime\wn}\, \ppz E^+\zt + \frac{c}{2\, n^\prime\wn}\, \alpha\wn\, E^+\zt
%&= i \frac{\omega_0}{2\, \varepsilon_0}\, \frac{\eta\wn}{n^\prime\wn}\, P^+\zt , \nd \\
% \label{eqn:wave_eqn_1d_backward} \ppt E^-\zt - \frac{c}{n^\prime\wn}\, \ppz E^-\zt + \frac{c}{2\, n^\prime\wn}\, \alpha\wn\, E^-\zt
%&= i \frac{\omega_0}{2\, \varepsilon_0}\, \frac{\eta\wn}{n^\prime\wn}\, P^-\zt .
% \end{align}
% \end{subequations}
%In the following discussion, our goal will be to use the solutions to the corresponding homogenous wave equation in the frequency domain --- given in \sct{wg_smat} --- to find the frequencies of the eigenmodes of the \lhs of \eqn{wave_eqn_1d_forward} and \eqn{wave_eqn_1d_backward} that satisfy the internal boundary conditions appropriate for a laser resonator.

 \red{This section needs a better introduction than this. The concept of ``modes of the universe'' and the history of approaches to this problem should be described, so that the utility of the approach here is highlighted. It would probably be a good idea to move \eqn{forward_prop_w_z} here.}

 \subsection{Eigenfunctions of a 1D Unidirectional Ring Resonator\label{sct:laser_resonators_1d_url}}
Consider a laser cavity selected from the set of one-dimensional resonators shown in \fig{resonator_1d_smat} with $R_2 = 1$ and $\mathbf{F}_2(t) = 0$. We will define the reflectivity of the output coupling mirror $\mathcal{M}_1$ as $R \equiv R_1$.
%, and we will incorporate any scattering and absorption losses in the coatings of the other mirrors using the approach followed in developing \eqn{alpha_general}.
In the absence of the injected field $\mathbf{F}_1(t)$, the boundary condition satisfied by the forward-propagating intracavity field amplitude $\mathbf{E}^+(z, \omega)$ is given by \eqn{resonator_1d_w_bc} as
 \begin{equation} \label{eqn:laser_resonator_1d_w_bc_pz}
\mathbf{E}^+(0, \omega) = \sqrt{R}\, \mathbf{E}^+(1, \omega) ,
 \end{equation}
As described in \sct{wg_smat}, the solution to \eqn{wave_eqn_1d_w_lin_pz} in the frequency domain is given by \eqn{forward_prop_w}, which when substituted into \eqn{e_field_1d_pz} yields
 \begin{equation} \label{eqn:laser_resonator_1d_w_forward}
\mathbf{E}^+(1, \omega) = G(1, \omega)\, \mathbf{E}^+(0, \omega) ,
 \end{equation}
as given by \eqn{e_prop_1d_p} for the case where $z_1 = 0$ and $z_2 = 1$. If we substitute \eqn{laser_resonator_1d_w_forward} into the \rhs of \eqn{laser_resonator_1d_w_bc_pz}, we obtain the constraint
 \begin{equation} \label{eqn:laser_resonator_1d_eigen_eqn}
\sqrt{R}\, G(1, \omega) = \sqrt{R}\, \exp\left[ i\, \omega_0\, \tau/\tau_g + i\, \omega - \half \alpha\wn \right] = \Gamma\, e^{i \omega} = 1 ,
 \end{equation}
where $\Gamma$ is given by \eqn{fpi_gamma_def} with $R_1 \equiv R$ and $R_2 = 1$. We note that the contribution of $\omega_0\, \tau = k_0\wn$ in $\Gamma$ is in fact the result of the rapidly-varying factor $e^{+i\, k_0\wn\, z}$ accumulating phase over one full cavity round-trip, and should be neglected in our construction of slowly-varying spatial eigenmodes of the cavity. We can do this by modifying the boundary condition given by \eqn{laser_resonator_1d_w_bc_pz} to include a factor of $e^{i \omega_0 \tau}$, or by choosing $\omega_0$ such that $e^{i \omega_0 \tau} = 1$.

By comparing this result with \eqn{fpi_fw_def} and \eqn{mittag_leffler}, we see immediately that the eigenfrequencies $\omega \equiv \nu_q$ which satisfy \eqn{laser_resonator_1d_eigen_eqn} are those of the poles of \eqn{fpi_fghw_def}, with the rapidly-varing phase removed. The corresponding complex propagation constant is given by \eqn{f_pole_q} as
 \begin{equation} \label{eqn:laser_resonator_1d_kappa_def}
\frac{\nu_q}{c} \equiv \frac{2 q \pi + i \ln \left|\Gamma\right|}{L_g} ,
 \end{equation}
where the round-trip group cavity length is defined in terms of \eqn{tau_group_def} as
 \begin{equation} \label{eqn:laser_resonator_1d_L_group_def}
L_g \equiv c \tau_g = n^\prime\wn L .
 \end{equation}
The eigenfunctions $u_q\z$ we seek satisfy \eqn{wave_eqn_1d_w_lin_pz} with $\omega = \nu_q$ (and $e^{i \omega_0 \tau} = 1$), or
 \begin{equation} \label{eqn:laser_resonator_1d_u_hlde}
\ddz u_q\z -i\, \nu_q\, u_q\z + \half\, \alpha\wn\, u_q\z = 0 ,
 \end{equation}
which has the solution
 \begin{equation} \label{eqn:laser_resonator_1d_u_unnorm}
u_q\z = \mathcal{C} \exp\left[ \left( i 2 q \pi + \ln\frac{1}{\sqrt{R}} \right) z \right] ,
 \end{equation}
where $\mathcal{C}$ is a normalization factor. We note that $u_q(0) = \mathcal{C}$, and $u_q(1) = \mathcal{C}/\sqrt{R}$, so an expansion of $E^+\zt$ as a series of these eigenfunctions satisfies the boundary condition given by \eqn{laser_resonator_1d_w_bc_pz} for the complex vector envelope function.

In fact, we would like to follow the usual eigenfunction expansion approach for a self-adjoint differential operator, where we might expect that $\int_0^{1} dz\, u_q\z\, u^\ast_{p}\z  = |\mathcal{C}|^2 \delta_{q p}$. However, when $u_q\z$ is given by \eqn{laser_resonator_1d_u_unnorm}, we find
 \begin{equation}
 \begin{split}
\int_0^1 dz\, u_q\z\, u_{p}^\ast\z &= |\mathcal{C}|^2 \int_0^1 dz\, e^{\left[i 2 \left(q - p\right) \pi + \ln(1/R)\right] z} \\
&= |\mathcal{C}|^2\, \frac{e^{i 2 \left(q - p\right) \pi + \ln(1/R)} - 1}{i 2 \left(q - p\right) \pi + \ln(1/R)} ,
 \end{split}
 \end{equation}
giving
 \begin{equation}
\int_0^1 dz\, u_q\z\, u_{p}^\ast\z = \Delta_{q - p}(R) ,
 \end{equation}
where
\begin{equation} \label{eqn:laser_resonator_1d_Delta_qR}
\Delta_q(R) \equiv \frac{1}{1 + i 2 q \pi / \ln\left( 1/R \right)} \approx
  \begin{cases}
    1 & \textrm{if } q = 0 \\
    \left[\ln(1/R) / (2 q \pi)\right]^2 - i \left[\ln(1/R) / (2 q \pi)\right] & \textrm{if } q \ne 0
  \end{cases}
 \end{equation}
and we have chosen the normalization constant
 \begin{equation}\label{eqn:laser_resonator_1d_u_norm_url}
 \mathcal{C} \equiv \sqrt{\frac{R \ln(1/R)}{1 - R}} .
 \end{equation}
We are forced to conclude that the forward-propagating eigenmodes are not strictly self-orthogonal. As shown in \fig{laser_resonator_1d_Delta_qR}, $\Delta_q(R)$ can be reasonably approximated as a Kronecker delta function only as $R \rightarrow 1$. This nonorthogonality of the longitudinal eigenmodes has observable physical consequences, such as excess spontaneous emission into each mode of a laser~\cite{ref:hamel1990oef}.

 \begin{figure}
  \centering
  \includegraphics[width=4.5in]{figures/Delta_qR}
  \caption{\label{fig:laser_resonator_1d_Delta_qR} Plot of the real and imaginary parts of $\Delta_q(R)$, defined by \eqn{laser_resonator_1d_Delta_qR}, as a function of the output coupler reflectivity $R$.}
 \end{figure}

 \begin{figure}
  \centering
  \begin{subfigure}[b]{0.8\textwidth}
   \centering
   \includegraphics[width=5.00in]{figures/resonator_1d_forward}
   \caption{Forward propagation}
   \label{fig:resonator_1d_forward}
  \end{subfigure}
  \par\vspace{0.25in}
  \begin{subfigure}[b]{0.8\textwidth}
   \centering
   \includegraphics[width=5.00in]{figures/resonator_1d_backward}
   \caption{Backward propagation}
   \label{fig:resonator_1d_backward}
  \end{subfigure}
  \caption{\label{fig:resonator_1d_prop} Diagrams of ``unfolded'' cavities depicting round-trip propagation in the (a) forward and (b) backward directions. Although each of these pictures resembles a standing-wave cavity, they in fact represent any of the resonator configurations shown in \fig{resonator_1d_smat}.}
 \end{figure}

The difficulty arises because the boundary condition at the output coupler turns the laser resonator into an ``open system:'' although the differential operator governing our wave equation is self-adjoint, the boundary condition given by \eqn{laser_resonator_1d_w_bc_pz} breaks this symmetry. We can see this immediately for the simple case where the interior of the cavity is vacuum, so that $n^\prime = 1$ and $\alpha = 0$. If we formally integrate the \lhs of \eqn{wave_eqn_1d_w_lin_pz}, then we obtain
 \begin{equation}
E^+\zw = -i \frac{\omega}{c} \int_{0}^{1} d z^\prime\, K\left(z, z^\prime\right)\, E^+\left(z^\prime, \omega\right) ,
 \end{equation}
where $K\left(z, z^\prime\right)$ is the Green's function (or \emph{propagator}) that satisfies both the differential equation
 \begin{equation}
\ppz K\left(z, z^\prime\right) = -\delta\left(z - z^\prime\right)
 \end{equation}
and the boundary condition $K\left(0, z^\prime\right) = \sqrt{R}\, K\left(1, z^\prime\right)$. By inspection, we construct this propagator as
 \begin{equation}
K\left(z, z^\prime\right) = -\frac{\Theta\left(z - z^\prime\right) + \sqrt{R}\, \Theta\left(z^\prime - z\right)}{1 - \sqrt{R}} ,
 \end{equation}
where $\Theta\z$ is the Heaviside Theta function. We note immediately that $K\left(z, z^\prime\right)$ is \emph{not} Hermitian: since $\Theta\z$ is not even in its argument, $K\left(z, z^\prime\right) \neq K\left(z^\prime, z\right)$. This property is ubiquitous in laser physics; we shall see in \chp{laser_resonators_3d} that the transverse eigenmodes of open-sided laser resonators are not self-orthogonal.

 \subsection{Biorthogonality and Completeness of the Eigenfunctions}

 \red{This subsection needs a proper introduction.}
 \begin{equation} \label{eqn:laser_resonator_1d_w_bc_mz}
\mathbf{E}^-(1, \omega) = \sqrt{R}\, \mathbf{E}^-(0, \omega) ,
 \end{equation}

Therefore, the slowly-varying backward-propagating eigenfunctions we seek satisfy \eqn{wave_eqn_1d_w_lin_mz} with $\omega = \nu_q$, or
 \begin{equation} \label{eqn:laser_resonator_1d_v_hlde}
\ddz v_q\z + i\, \nu_q\, v_q\z - \half\, \alpha\wn\, v_q\z = 0 ,
 \end{equation}
which has the solution
 \begin{equation} \label{eqn:laser_resonator_1d_v_unnorm}
v_q\z \equiv \mathcal{C}^\prime \exp\left[ -\left( i 2 q \pi + \ln\frac{1}{\sqrt{R}} \right) z \right] .
 \end{equation}
In this case, we find that $v_q(0) = \mathcal{C}^\prime$, and $v_q(1) = \mathcal{C}^\prime \sqrt{R}$, so these eigenfunctions satisfy the same boundary condition given by \eqn{laser_resonator_1d_w_bc_mz} as the backward-propagating complex vector envelope function.

%\red{so an expansion of $E^-\zt$ as a series of these eigenfunctions satisfies the boundary condition given by \eqn{laser_resonator_1d_w_bc_mz} for the backward-propagating complex vector envelope function.}

\red{Let us now try to find an integral relationship that allows us to \dots} If we multiply \eqn{laser_resonator_1d_u_hlde} by $v_{p}\z$ and \eqn{laser_resonator_1d_v_hlde} (with $q \rightarrow p$) by $u_q\z$ and then add the results, we obtain
 \begin{equation}
\ddz \left[u_q\z v_{p}\z\right] - i\, 2 \pi \left(q - p\right) u_q\z v_{p}\z = 0 .
 \end{equation}
Integrating both sides of this equation from $z = 0$ to $z = 1$ yields
 \begin{equation}
2 \pi \left(q - p\right) \int_{0}^{1} dz\, u_q\z\, v_{p}\z = -i\left[ u_q(1)\, v_{p}(1) - u_q(0)\, v_{p}(0) \right] = 0 ,
 \end{equation}
since $u_q(1) v_{p}(1) = u_q(0) v_{p}(0) = \mathcal{C} \mathcal{C}^\prime$. Therefore, if $q \neq p$, the integral on the \lhs of this equation must vanish. We can verify this explicitly using the functional forms of $u_q\z$ and $v_q\z$ given by \eqn{laser_resonator_1d_u_unnorm} and \eqn{laser_resonator_1d_v_unnorm}, respectively; we find that the result of the integral is given by $\mathcal{C} \mathcal{C}^\prime \delta_{q p}$. If we choose the scaling constant $\mathcal{C}^\prime = 1/\mathcal{C}$, where $\mathcal{C}$ is given by \eqn{laser_resonator_1d_u_norm_url}, and define
 \begin{subequations} \label{eqn:laser_resonator_1d_uv}
 \begin{align}
 u_q\z &\equiv \mathcal{C}\, e^{+\left[ i 2 q \pi + \ln(1/\sqrt{R}) \right] z} , \nd \label{eqn:laser_resonator_1d_u} \\
 v_q\z &\equiv \frac{1}{\mathcal{C}}\, e^{-\left[ i 2 q \pi + \ln(1/\sqrt{R}) \right] z} , \label{eqn:laser_resonator_1d_v}
 \end{align}
 \end{subequations}
then $u_q\z$ and $v_q\z$ jointly satisfy the weighted \emph{biorthogonality relation}\index{Biorthogonality}
 \begin{equation} \label{eqn:laser_resonator_1d_uv_biortho}
\int_{0}^{1} dz\, u_q\z\, v_{p}\z = \delta_{q p} .
 \end{equation}

Before we can develop a self-consistent approach to expanding the intracavity slowly-varying scalar envelope function $E^+\zt$, we need to find a relation that establishes the completeness of the eigenfunctions defined by \eqn{laser_resonator_1d_uv}. Consider the sum
 \begin{equation} \label{eqn:laser_resonator_1d_uv_precomp}
\sum_{q = -\infty}^\infty u_q\z\, v_q\zp = e^{- \ln\left(\sqrt{R}\right)\, (z - z^\prime)}\, \sum_{q = -\infty}^\infty e^{i 2 q \pi (z - z^\prime)} .
 \end{equation}
Relying on the definition of the Dirac comb~\cite{ref:stone2009mfp},
 \begin{equation} \label{eqn:dirac_comb}
\sum_{q = -\infty}^\infty e^{i 2 q \pi (z - z^\prime)} = \sum_{m = -\infty}^\infty \delta\left(z - z^\prime + m\right) ,
 \end{equation}
we find
 \begin{equation} \label{eqn:laser_resonator_1d_uv_precomp2}
\sum_{q = -\infty}^\infty u_q\z\, v_q\zp = \sum_{m = -\infty}^\infty R^{m/2} \delta\left(z - z^\prime + m\right) .
 \end{equation}
If we confine $\abs{z - z^\prime} < 1$, then terms on the \rhs of this expression with $m \neq 0$ will not contribute, and we have the \emph{completeness relation}\index{Completeness}
 \begin{equation} \label{eqn:laser_resonator_1d_uv_complete}
\sum_{q = -\infty}^\infty u_q\z\, v_q\zp = \delta\left(z - z^\prime\right) ,
 \end{equation}
valid for $0 < z < 1$ (i.e., everywhere in the cavity \emph{except} at the boundary).

We now have the machinery at our disposal to find a valid expansion of the forward-propagating field $E^+\zt$ as a series of time-dependent coefficients multiplied by the functions $u_q\z$. Dropping the subscript, we have
 \begin{equation}% \label{eqn:f_ml}
 \begin{split}
E\zt &= \int_{0}^{1} d z^\prime\, \delta\left(z - z^\prime\right) E(z^\prime, t) \\
     &= \sum_{q = -\infty}^\infty u_q\z \int_{0}^{1} d z^\prime\, v_q\zp  E(z^\prime, t) ,
 \end{split}
 \end{equation}
or
 \begin{equation}
E\zt = \sum_{q = -\infty}^\infty u_q\z\, E_q(t) ,
 \end{equation}
where
 \begin{equation}
E_q(t) \equiv \int_{0}^{1} d z\, v_q\z  E\zt .
 \end{equation}
We make two adjustments to these expressions before choosing the final form of our slowly-varying eigenfunction expansions. The argument $2 q \pi \equiv \Delta k_q$ in the exponent of $u_q\z$ represents a small change in the propagation constant $k_0\wn$. There must be a corresponding small shift in the carrier frequency $\omega_0$, since
 \begin{equation}
 \begin{split}
k_0\left(\omega_0 + \delta \omega\right) &= \frac{\left(\omega_0 + \delta \omega\right) n\left(\omega_0 + \delta \omega\right)}{c} \\
&\cong k_0\wn + \frac{\delta \omega}{c}\, \left[ n\wn + \omega_0 \frac{d n\wn}{d \omega_0} \right] \\
&= k_0\wn + \frac{n^\prime\wn}{c}\, \delta \omega \\
&\equiv k_0\wn + \delta k .
 \end{split}
 \end{equation}
If $\delta k = \Delta k_q$, then we must have
 \begin{equation}\label{eqn:delta_w_q_def}
\delta \omega = \Delta \omega_q \equiv \frac{c}{n^\prime\wn}\, \Delta k_q = \frac{2 q \pi}{\tau_g} = q \Delta \omega_\text{FSR} ,
 \end{equation}
where $\Delta \omega_\text{FSR}$ is defined by \eqn{delta_w_fsr_def}.
Therefore, we can define the effective rapidly-varying propagation vector and frequency of each eigenmode as
 \begin{subequations}
 \begin{align}
 \label{eqn:k_q_def} k_q &\equiv k_0\wn + 2 q \pi , \nd \\
 \label{eqn:w_q_def} \omega_q &\equiv \omega_0 + 2 q \pi ,
 \end{align}
 \end{subequations}
For convenience, then, we make this frequency shift explicit, and write our expansions as
 \begin{equation} \label{eqn:laser_resonator_1d_ezt_expansion}
E\zt = \sum_{q = -\infty}^\infty u_q\z\, e^{-i \Delta \omega_q t}\, E_q(t) ,
 \end{equation}
where
 \begin{equation} \label{eqn:laser_resonator_1d_eq_def}
E_q(t) \equiv e^{+i \Delta \omega_q t} \int_{0}^{1} d z\, v_q\z  E\zt .
 \end{equation}
The analogs of these two equations in the frequency domain are straightforward.

The output electric field $E_1(t) = \sqrt{1 - R}\, E(1, t)$ takes on a particularly elegant form given \eqn{laser_resonator_1d_u} and \eqn{laser_resonator_1d_ezt_expansion}. Since $u_q(1) = \mathcal{C}/\sqrt{R}$, using \eqn{laser_resonator_1d_u_norm_url} we obtain
 \begin{equation}\label{eqn:laser_resonator_1d_url_out}
E_1(t) = \sqrt{\ln\left(\frac{1}{R}\right)}\, \sum_{q = -\infty}^\infty e^{-i \Delta \omega_q t}\, E_q(t) .
 \end{equation}

%Corresponding to the spatially slowly-varying eigenfunctions given by \eqn{laser_resonator_1d_uv}, we have the rapidly-varying basis functions
% \begin{subequations} \label{eqn:laser_resonator_1d_rv}
% \begin{align}
%\tilde{u}_q\z &\equiv \mathcal{C}\, e^{+\left[ i k_q + \ln\left(1/\sqrt{R}\right)\right] z} , \nd \label{eqn:laser_resonator_1d_u_rv} \\
%\tilde{v}_q\z &\equiv \frac{1}{\mathcal{C}}\, e^{-\left[ i k_q + \ln\left(1/\sqrt{R}\right)\right] z} , \label{eqn:laser_resonator_1d_v_rv}
% \end{align}
% \end{subequations}
%where $k_q$ is given by \eqn{k_q_def}. By inspection, these spatially rapidly-varying eigenmodes satisfy the biorthogonality relation given by \eqn{laser_resonator_1d_uv_biortho} and the completeness relation given by \eqn{laser_resonator_1d_uv_complete}. Therefore, we can define the spatially rapidly-varying eigenfunction expansion
% \begin{equation} \label{eqn:laser_resonator_1d_ezt_expansion_rv}
%\tilde{E}\zt = \sum_{q = -\infty}^\infty \tilde{u}_q\z\, e^{-i \Delta \omega_q t}\, E_q(t) ,
% \end{equation}
%where
% \begin{equation} \label{eqn:laser_resonator_1d_eq_def_rv}
%E_q(t) \equiv e^{+i \Delta \omega_q t} \int_{0}^{1} d z\, \tilde{v}_q\z  \tilde{E}\zt .
% \end{equation}
%We note that both $\tilde{E}\zt$ and $E_q(t)$ continue to vary slowly in time.

We close this section with the subtle but important observation that the functions $v_{q}\z$ are \emph{not} normalized. Rather, they are scaled to allow us to project field amplitudes such as $E^+\zt$ and $P^+\zt$ onto the space of functions $u_q\z$ using \eqn{laser_resonator_1d_eq_def}. Although $v_{q}\z$ has the functional form of backward-propagating fields, a bidirectional ring laser would need two properly normalized eigenfunction basis sets, $u^{+}_{q}\z$ and $u^{-}_{q}\z$, with the corresponding biorthogonal projectors, $v^{+}_{q}\z$ and $v^{-}_{q}\z$, to expand the counterpropagating fields within the resonator. We see this explicitly in the next section for the case of a standing-wave resonator.

 \subsection{Eigenfunctions of a 1D Standing-Wave Resonator\label{sct:laser_resonators_1d_swl}}
Let us now construct a similar set of eigenmodes for the standing-wave laser resonator shown in \fig{resonator_1d_sw_smat}, where both mirror reflection coefficients differ from unity. When we were analyzing the scattering matrices of resonant cavities in \sct{resonator_1d_smat}, it was convenient to ``fold'' the coordinate system so that $0 < z < 1/2$ represented the coordinate for propagation from the reference plane at $\mathcal{M}_1$ (just \emph{after} reflection) to the reference plane at $\mathcal{M}_2$ (just \emph{before} reflection), and $1/2 < z < 1$ represented the coordinate for propagation from the reference plane at $\mathcal{M}_2$ (just \emph{after} reflection) to the reference plane at $\mathcal{M}_1$ (just \emph{before} reflection). But this convenience isn't as helpful when we begin to construct physical models of lasers, where the total electric field at each point within the laser amplifier drives the evolution of the local population inversion.

In the absence of the injected fields $\mathbf{F}_1(t)$ and $\mathbf{F}_1(t)$, the boundary conditions satisfied by the intracavity field amplitudes $\mathbf{E}^\pm(z, \omega)$ are given by \eqn{resonator_1d_w_bc} as
 \begin{subequations} \label{eqn:laser_resonator_1d_w_bc_sw}
 \begin{align}
\mathbf{E}^+(0, \omega) &= -\sqrt{R_1}\, \mathbf{E}^-(0, \omega) , \nd \\
\mathbf{E}^-(1/2, \omega) &= -\sqrt{R_2}\, \mathbf{E}^+(1/2, \omega) ,
 \end{align}
 \end{subequations}
where we have repositioned the reference planes using \eqn{mirror_smat_new} to be consistent with the sign convention chosen in many of the discussions of standing-wave cavities in the literature. Then, in this case, the most convenient choice is to represent the rapidly-varying eigenmodes as a ``vector'' with components representing the solution of the full wave equations on $0 < z < 1/2$ subject to the boundary conditions. If we write~\cite{ref:hamel1989nle}
%where $\Gamma$ is given by \eqn{fpi_gamma_def} with $R_1 \equiv R$ and $R_2 = 1$, and $\tau$ and $\tau_g$ are defined by \eqn{tau_def}.
 \begin{equation} \label{eqn:laser_resonator_1d_u_sw_vec}
\mathbf{u}_q\z = \begin{bmatrix} u^{+}_q\z \\ u^{-}_q\z \end{bmatrix} ,
 \end{equation}
then, following the analysis of \sct{laser_resonators_1d_url}, we find
 \begin{subequations} \label{eqn:laser_resonator_1d_u_sw}
 \begin{align}
u^{+}_q\z &=\mathcal{C} e^{+\left[ i 2 q \pi + \ln\left(1/\sqrt{R_1 R_2}\right) \right] z} , \nd \\
 \begin{split}
u^{-}_q\z &=-\mathcal{C} \sqrt{R_2}\, e^{+\left[ i 2 q \pi + \ln\left(1/\sqrt{R_1 R_2}\right) \right] (1 - z)}  \\
&=-\frac{\mathcal{C}}{\sqrt{R_1}} e^{-\left[ i 2 q \pi + \ln\left(1/\sqrt{R_1 R_2}\right) \right] z} .
 \end{split}
 \end{align}
 \end{subequations}
It is clear that both of the boundary conditions --- at $z = 0$ and $z = 1/2$ --- are satisfied when we assume that $e^{i \omega_0 \tau} = e^{i k_0\wn} = 1$.  The normalization integral
 \begin{equation}
 \begin{split}
\int_{0}^{1/2} dz\, \mathbf{u}_{q}\z \dotp \mathbf{u}_q^\ast\z &= |\mathcal{C}|^2 \left[ \int_{0}^{1/2} dz\, e^{z \ln\left( 1/ R_1 R_2 \right)} + \frac{1}{R_1} \int_{0}^{1/2} dz\, e^{-z \ln\left( 1/ R_1 R_2 \right)} \right] \\
&= |\mathcal{C}|^2\, \frac{\left( \sqrt{R_1} + \sqrt{R_2} \right) \left( 1 - \sqrt{R_1 R_2} \right)}{R_1 \sqrt{R_2} \ln\left( 1/ R_1 R_2 \right)}
 \end{split}
 \end{equation}
is unity if we choose the normalization constant
 \begin{equation}\label{eqn:laser_resonator_1d_u_norm_swl}
\mathcal{C} = \left[ \frac{2\, R_1 \sqrt{R_2}\, \ln\left( 1/\sqrt{R_1 R_2} \right)}{\left( \sqrt{R_1} + \sqrt{R_2} \right) \left( 1 - \sqrt{R_1 R_2} \right)} \right]^{1/2} .
 \end{equation}
Then the nonorthogonality integral is given by
\begin{equation}
  \begin{split}
    \int_{0}^{1/2} dz\, \mathbf{u}_{p}\z \dotp \mathbf{u}_q^\ast\z &= \mathcal{C}^2 \left[ \int_{0}^{1/2} dz\, e^{[i\, 2\, (p - q)\, \pi + \ln\left( 1/ R_1 R_2 \right)]\, z} + \frac{1}{R_1} \int_{0}^{1/2} dz\, e^{-[i\, 2\, (p - q)\, \pi + \ln\left( 1/ R_1 R_2 \right)]\, z} \right] \\
    &= \Delta^\prime_{p - q}(R_1\, R_2)\, ,
  \end{split}
\end{equation}
where
\begin{equation}  \label{eqn:laser_resonator_1d_Deltap_qR}
\Delta^\prime_q(R_1\, R_2) \equiv 
  \begin{cases}
    \Delta_q(R_1\, R_2) & \textrm{if } q \textrm{ is even} \\
    \frac{\left(1 + \sqrt{R_1\, R_2}\right) \left(\sqrt{R_2} - \sqrt{R_1}\right)}{\left(\sqrt{R_1} + \sqrt{R_2}\right) \left(1 - \sqrt{R_1\, R_2}\right)}\, \Delta_q(R_1\, R_2) & \textrm{if } q \textrm{ is odd}
  \end{cases}
\end{equation}
and $\Delta_q(R)$ is given by \eqn{laser_resonator_1d_Delta_qR}. Once again, we see that the eigenmodes are not strictly orthogonal unless both $R_1$ and $R_2$ approach unity. Note that when $R_1 = R_2$, all the odd-$q$ modes vanish, and when $R_2 = 1$, $\Delta^\prime_q(R_1\, R_2) = \Delta_q(R_1\, R_2)$ for any value of $q$.

We define the backward-propagating projection eigenmodes in vector form as
 \begin{equation} \label{eqn:laser_resonator_1d_v_sw_vec}
\mathbf{v}_q\z = \begin{bmatrix} v^{+}_q\z \\ v^{-}_q\z \end{bmatrix} ,
 \end{equation}
and simply reverse the directions of the standing-wave functions given by \eqn{laser_resonator_1d_u_sw} to obtain
 \begin{subequations} \label{eqn:laser_resonator_1d_v_sw}
 \begin{align}
v^{+}_q\z &=\mathcal{C}^\prime e^{-\left[ i 2 q \pi + \ln\left(1/\sqrt{R_1 R_2}\right) \right] z} , \nd \\
 \begin{split}
v^{-}_q\z &=-\frac{\mathcal{C}^\prime}{\sqrt{R_2}}\, e^{-\left[ i 2 q \pi + \ln\left(1/\sqrt{R_1 R_2}\right) \right] (1 - z)}  \\
&=-\mathcal{C}^\prime \sqrt{R_1} e^{+\left[ i 2 q \pi + \ln\left(1/\sqrt{R_1 R_2}\right) \right] z} .
 \end{split}
 \end{align}
 \end{subequations}
It is easy to verify that the two sets of eigenmodes are biorthogonal, since
 \begin{equation} \label{eqn:laser_resonator_1d_uv_biortho_sw}
\int_{0}^{1/2} dz\, \mathbf{u}_q\z \dotp \mathbf{v}_{p}\z = \delta_{q p} ,
 \end{equation}
where again $\mathcal{C}^\prime = 1/\mathcal{C}$.

At first glance, it isn't entirely obvious how to craft a completeness relation that would allow us to expand a spatially rapidly-varying, single-transverse-mode field in a series over the eigenfunctions $\mathbf{u}_q\z$. As a guide, we note that the sum $\sum_{q} \mathbf{u}_q\z \mathbf{v}^{T}_q\z$ contains diagonal terms proportional to $\sum_q e^{\pm i 2 q \pi (z - z^\prime)}$ that will simplify to $\delta(z - z^\prime)$ using \eqn{dirac_comb}, and off-diagonal terms proportional to $\sum_q e^{\pm i 2 q \pi (z + z^\prime)}$ that will become $\sum_m \delta(z + z^\prime + m)$. These latter arguments are never zero for any $m$ over the interval $0 < z < 1/2$, so we have
 \begin{equation} \label{eqn:laser_resonator_1d_sw_complete}
\sum_{q = -\infty}^\infty \mathbf{u}_q\z \mathbf{v}^{T}_q\zp = \begin{bmatrix} \delta\left(z - z^\prime\right) & 0  \\ 0 & \delta\left(z - z^\prime\right) \end{bmatrix} ,
 \end{equation}
Therefore, if we write the coefficients of the total spatially rapidly-varying electric field $\widetilde{E}\zt = E^+\zt e^{+i k_0\wn z} + E^-\zt  e^{-i k_0\wn z}$ in vector form as
 \begin{equation}\label{eqn:laser_resonators_1d_e_sw_def}
\mathbf{E}\zt = \begin{bmatrix} E^{+}\zt \\ E^{-}\zt \end{bmatrix} ,
 \end{equation}
then
 \begin{equation}% \label{eqn:f_ml}
 \begin{split}
\mathbf{E}\zt &= \int_{0}^{1/2} d z^\prime\,  \begin{bmatrix} \delta\left(z - z^\prime\right) & 0  \\ 0 & \delta\left(z - z^\prime\right) \end{bmatrix} \mathbf{E}(z^\prime, t) \\
     &= \sum_{q = -\infty}^\infty \mathbf{u}_q\z \int_{0}^{1/2} d z^\prime\, \mathbf{v}^{T}_q\zp  \mathbf{E}(z^\prime, t) ,
 \end{split}
 \end{equation}
or
 \begin{equation} \label{eqn:laser_resonator_1d_ezt_expansion_sw}
\mathbf{E}\zt = \sum_{q = -\infty}^\infty \mathbf{u}_q\z\, e^{-i \Delta \omega_q t}\, E_q(t) ,
 \end{equation}
where
 \begin{equation} \label{eqn:laser_resonator_1d_eq_def_sw}
E_q(t) \equiv e^{+i \Delta \omega_q t} \int_{0}^{1/2} d z\, \mathbf{v}_q\z \dotp \mathbf{E}\zt .
 \end{equation}

The output fields from each mirror can be calculated given \eqn{laser_resonator_1d_u_sw} and \eqn{laser_resonator_1d_ezt_expansion_sw}:
 \begin{subequations}%\label{}
 \begin{align}%\label{}
 E_1 &= i\, \sqrt{1 - R_1}\, \sum_q u_q^-(0)\, e^{-i \Delta \omega_q t}\, E_q(t) , \nd\\
 E_2 &= i\, \sqrt{1 - R_2}\, \sum_q u_q^+(1/2)\, e^{-i \Delta \omega_q t}\, E_q(t) .
 \end{align}
 \end{subequations}
Since $u_q^-(0) = -\mathcal{C}/\sqrt{R_1}$, and $u_q^+(1/2) = (-1)^q\, C/(R_1 R_2)^{1/4}$, using \eqn{laser_resonator_1d_u_norm_swl} we find
 \begin{subequations} \label{eqn:laser_resonator_1d_swl_out}
 \begin{align}
\label{eqn:laser_resonator_1d_swl_out_1} E_1 &= -i\, \left[ \frac{2 \left(1 - R_1\right) \sqrt{R_2}\, \ln\left( 1/\sqrt{R_1 R_2} \right)}{\left( \sqrt{R_1} + \sqrt{R_2} \right) \left( 1 - \sqrt{R_1 R_2} \right)} \right]^{1/2}\, \sum_q e^{-i \Delta \omega_q t}\, E_q(t) , \nd\\
\label{eqn:laser_resonator_1d_swl_out_2} E_2 &= i\, \left[ \frac{2\, \sqrt{R_1} \left(1 - R_2\right) \ln\left( 1/\sqrt{R_1 R_2} \right)}{\left( \sqrt{R_1} + \sqrt{R_2} \right) \left( 1 - \sqrt{R_1 R_2} \right)} \right]^{1/2}\, \sum_q (-1)^q\, e^{-i \Delta \omega_q t}\, E_q(t) .
 \end{align}
 \end{subequations}
Although in general each of these expressions lacks the elegance of \eqn{laser_resonator_1d_url_out}, they each give exactly the same result in two cases of practical interest:
 \begin{enumerate}
   \item
    In the limit where one of the mirrors has unit reflectivity --- either the substitution $R_1 = R$ and $R_2 = 1$ for $E_1$, or $R_1 = 1$ and $R_2 = R$ for $E_2$, will simplify the magnitude of the corresponding coefficient on the left of the sum to $\sqrt{\ln(1/R)}$.
   \item
    When $R_1 = R_2 \equiv R,$ the magnitude of each coefficient on the left of the corresponding sum becomes $\sqrt{\ln(1/R)}$.
 \end{enumerate}
In fact, it is remarkable that over a wide range of values of $R_1$ and $R_2$, we can make the same approximation without significant loss of accuracy. If we write $R_1 \equiv 1 - \delta_1$ and $R_2 \equiv 1 - \delta_2$, then we find as $\{\delta_1, \delta_2\} \longrightarrow \{0, 0\}$ that
 \begin{equation}
\left[ \frac{2 \sqrt{R_1 R_2}\, \ln\left( 1/\sqrt{R_1 R_2} \right)}{\left( \sqrt{R_1} + \sqrt{R_2} \right) \left( 1 - \sqrt{R_1 R_2} \right)} \right]^{1/2} \approx 1 - \frac{\delta_1^2 - \delta_1 \delta_2 + \delta_2^2}{48}\, ,
 \end{equation}
so that we can almost always approximate the product of this coefficient and the factor $\sqrt{(1 - R_j)/R_j}$ as simply $\sqrt{\ln(1/R_j)}$.


%It is tempting to follow the example of the unidirectional ring laser and construct a set of rapidly-varying eigenfunctions for the standing-wave laser simply by multiplying $u_q^\pm(z)$ by $e^{\pm i k_0\wn z}$. These new functions would trivially satisfy the biorthogonality relation given by \eqn{laser_resonator_1d_uv_biortho_sw} and the completeness relation given by \eqn{laser_resonator_1d_sw_complete}. However, the primary physical effect that we would study using rapidly-varying standing-wave eigenmodes would be the spatial interference of counterpropagating electric fields within the laser amplifier. Therefore, we approximate these effects using the functions
% \begin{subequations} \label{eqn:laser_resonator_1d_uv_sw_rv}
% \begin{align}
%\label{eqn:laser_resonator_1d_u_sw_rv} \tilde{u}_q\z &=\mathcal{C} \left\{ e^{+\left[ i k_q + \ln\left(1/\sqrt{R_1 R_2}\right) \right] z} - \frac{1}{\sqrt{R_1}}\, e^{-\left[ i k_q + \ln\left(1/\sqrt{R_1 R_2}\right) \right] z} \right\}, \nd \\
%\label{eqn:laser_resonator_1d_v_sw_rv} \tilde{v}_q\z &=\mathcal{C}^\prime \left\{ e^{-\left[ i k_q + \ln\left(1/\sqrt{R_1 R_2}\right) \right] z} - \sqrt{R_1}\, e^{+\left[ i k_q + \ln\left(1/\sqrt{R_1 R_2}\right) \right] z} \right\} .
% \end{align}
% \end{subequations}
%We recall from \eqn{k_q_def} that $k_q = k_0\wn + 2 q \pi$, and that $k_0\wn = 2 \pi n\wn/\lambda_0$. In a standing-wave laser, each resonance corresponds to an integer number $q_0$ of half-physical-wavelengths stored within half the vacuum round-trip length of the laser cavity, or $L/2 = q_0 \lambda_0 / 2 n\wn$. This means that we can write $k_q = 2(q_0 + q) \pi$, and therefore the normalization constant $\mathcal{C}$ defined by \eqn{laser_resonator_1d_u_norm_swl} doesn't change, since
% \begin{equation}
%\int_{0}^{1/2} dz\, \tilde{u}_{q}\z \tilde{u}_q^\ast\z = 1 .
% \end{equation}
%But the biorthogonality relation given by \eqn{laser_resonator_1d_uv_biortho_sw} must be updated to read
% \begin{equation}
%\int_{0}^{1/2} dz\, \tilde{u}_q\z \tilde{v}_{p}\z = \delta_{q p} - \frac{\left( \sqrt{R_1} + \sqrt{R_2} \right) \left( 1 - \sqrt{R_1 R_2} \right)}{2 \sqrt{R_1 R_2} \ln\left( 1/ R_1 R_2 \right)}\, \Delta_{2 q_0}\left( R_1 R_2 \right)\, ,
% \end{equation}
%where we have assumed that $2 q_0 \gg q + p$, and $\Delta_q(R)$ is defined by \eqn{laser_resonator_1d_Delta_qR}. As indicated by \fig{laser_resonator_1d_Delta_qR}, the second term on the \rhs of this equation is negligible whenever $4 q_0 \pi/\ln(1/ R_1 R_2) \gg 1$, so the rapidly-varying modes given by \eqn{laser_resonator_1d_uv_sw_rv} are approximately biorthogonal. However, the completeness relation for the standing-wave eigenmodes requires a nontrivial update. The arguments presented for the neglect of the rapidly-varying terms in \eqn{laser_resonator_1d_sw_complete} remain valid, but because of the structure of \eqn{laser_resonator_1d_u_sw_rv} we must take the trace of \eqn{laser_resonator_1d_sw_complete} to obtain
% \begin{equation} \label{eqn:laser_resonator_1d_sw_rv_complete}
%\sum_{q = -\infty}^\infty \tilde{u}_q\z\, \tilde{v}_q\zp = 2\, \delta\left(z - z^\prime\right)\, .
% \end{equation}

In \sct{laser_dynamics_1d_mml_mll}, we will describe the intracavity spatial dependence of the total field of longitudinal mode $q$ as
 \begin{equation} \label{eqn:laser_resonator_1d_u_sw_rv}
 \begin{split}
 \widetilde{u}_q\z &= u_q^+\z\, e^{+i k_0 z} + u_q^-\z\, e^{-i k_0 z} \\
 &=\mathcal{C} \left\{ e^{+\left[ i k_q + \ln\left(1/\sqrt{R_1 R_2}\right) \right] z} - \frac{1}{\sqrt{R_1}}\, e^{-\left[ i k_q + \ln\left(1/\sqrt{R_1 R_2}\right) \right] z} \right\}\, ,
 \end{split}
 \end{equation}
where $k_q$ is given by \eqn{k_q_def}. In the case where $R_1 = 1$ and $R_2 \equiv R$, \eqn{laser_resonator_1d_u_sw_rv} can be approximated quite accurately by assuming that $\ln(1/\sqrt{R}) \ll 1$, and expanding $\exp[\ln(1/\sqrt{R}) z]$ linearly in $z$ to obtain
 \begin{equation}\label{eqn:laser_resonator_1d_u_sw_rv_approx}
\widetilde{u}_q\z \approx i 2 \mathcal{C} \left[ \sin\left(k_q z\right) - i \ln\left( \frac{1}{\sqrt{R}} \right)\, z\, \cos\left(k_q z\right) \right] .
 \end{equation}
In \fig{field_approx}, we have plotted \eqn{laser_resonator_1d_u_sw_rv} with \eqn{laser_resonator_1d_u_sw_rv_approx} for $k_q = 10\, \pi$, $R_1 = 1$, and $R_2 \equiv R = 0.3$. We have scaled all functions by a factor of $i\, 2\, \mathcal{C}$ to allow a direct comparison with the simple function $\sin(k_q z)$ that is often chosen in the literature. The relative accuracy of \eqn{laser_resonator_1d_u_sw_rv_approx} is surprising for a reflectivity this low, but it is clear that the simple approximation fails badly near the output coupler.

 \begin{figure}
  \centering
  \begin{subfigure}[b]{0.8\textwidth}
   \centering
   \includegraphics[width=5.0in]{figures/field_approx_re}
   \caption{Real part of $\widetilde{u}_q(z)/(i\, 2\, \mathcal{C})$}
   \label{fig:field_approx_re}
  \end{subfigure}
  \par\vspace{0.25in}
  \begin{subfigure}[b]{0.8\textwidth}
   \centering
   \includegraphics[width=5.0in]{figures/field_approx_im}
   \caption{Imaginary part of $\widetilde{u}_q(z)/(i\, 2\, \mathcal{C})$}
   \label{fig:field_approx_im}
  \end{subfigure}
  \caption{Plot of \eqn{laser_resonator_1d_u_sw_rv} with \eqn{laser_resonator_1d_u_sw_rv_approx} for $k_q = 10\, \pi$, $R_1 = 1$, and $R_2 \equiv R = 0.3$. We have scaled all functions by a factor of $i\, 2\, \mathcal{C}$ to allow a direct comparison with the simple function $\sin(k_q z)$ that is often chosen in the literature. The relative accuracy of \eqn{laser_resonator_1d_u_sw_rv_approx} is surprising for a reflectivity this low, but it is clear that the simple approximation fails badly near the output coupler.\label{fig:field_approx}}
 \end{figure}

%As in the case of the unidirectional ring laser, we can also construct a set of rapidly-varying eigenfunctions for the standing-wave laser. Defining the eigenstate vector
% \begin{equation} \label{eqn:laser_resonator_1d_u_sw_vec_rv}
%\mathbf{\tilde{u}}_q\z = \begin{bmatrix} \tilde{u}^{+}_q\z \\ \tilde{u}^{-}_q\z \end{bmatrix}
% \end{equation}
%and then following the analysis of \sct{laser_resonators_1d_url}, we find
% \begin{subequations} \label{eqn:laser_resonator_1d_u_sw_rv}
% \begin{align}
%\tilde{u}^{+}_q\z &=\mathcal{C} e^{+\left[ i k_q + \ln\left(1/\sqrt{R_1 R_2}\right) \right] z} , \nd \\
% \begin{split}
%\tilde{u}^{-}_q\z &=-\mathcal{C} \sqrt{R_2}\, e^{+\left[ i k_q + \ln\left(1/\sqrt{R_1 R_2}\right) \right] \left(L - z\right)/L}  \\
%&=-\frac{\mathcal{C}}{\sqrt{R_1}} e^{-\left[ i k_q + \ln\left(1/\sqrt{R_1 R_2}\right) \right] z} .
% \end{split}
% \end{align}
% \end{subequations}
%As in the traveling-wave case of the unidirectional ring laser, these spatially rapidly-varying eigenmodes satisfy the .

 \section{Temporal Coupled-Mode Theory\label{sct:laser_resonators_1d_tcm}}
As an example of the utility of the quasi-normal mode methods discussed above, we expand the intracavity fields of the resonators shown in \fig{resonator_1d_smat} for the scalar case where $R_2 = 1$ and, therefore, $F_2(\omega) = 0$.

 \subsection{One-Dimensional Unidirectional Ring Resonator}
We begin by using \eqn{fourier_freq} to calculate the Fourier Transform of the expansion coefficient in \eqn{laser_resonator_1d_eq_def} as
 \begin{equation} \label{eqn:eqw_url}
E_q(\omega) = \int_{-\infty}^{+\infty} d t\, \epwt\, E_q(t) = \int_0^1 d z\, v_q\z\, E(z, \omega + 2 q \pi)\, ,
 \end{equation}
where we have reapplied the Fourier Shift Theorem given by \eqn{fourier_shift_thm} in the frequency domain. From \eqn{forward_prop_w},
 \begin{equation} \label{eqn:forward_prop_w_z}
E(z, \omega) = \exp\left[ i\, \omega\, z - \half \alpha\wn\, z \right] E(0, \omega)\, ,
 \end{equation}
where $E(0, \omega)$ is given by \eqn{forward_prop_w_0} for the ring resonator under consideration here. Substituting \eqn{forward_prop_w_z} into \eqn{eqw_url}, we find
 \begin{equation} \label{eqn:forward_prop_e_q_w}
 \begin{split}
E_q(\omega) &= \frac{i \sqrt{\eta T}}{\mathcal{C}} \frac{F(\omega + 2 q \pi)}{1 - \Gamma\, e^{i \omega}} \int_{0}^{1} d z\, e^{(i \omega + \ln \Gamma) z} \\
&= \frac{i \sqrt{\eta T}}{\mathcal{C}} \frac{F(\omega + 2 q \pi)}{1 - \Gamma\, e^{i \omega}}\, \frac{e^{i \omega + \ln \Gamma} - 1}{i \omega + \ln \Gamma} \\
&= -\frac{i \sqrt{\eta T}}{\mathcal{C}} \frac{F(\omega + 2 q \pi)}{i \omega + \ln \Gamma}\, .
 \end{split}
 \end{equation}

We can follow either of two approaches to determine $E_q(t)$ for general $F(\omega)$. Choosing the method of integration, we can take the Fourier transform of \eqn{forward_prop_e_q_w} and apply the shift theorem to obtain
 \begin{equation} \label{eqn:eqt_int}
E_q(t) = \frac{\sqrt{\eta T}}{\mathcal{C}} e^{i 2 q \pi t} \int_{-\infty}^{+\infty} \frac{d \omega}{2 \pi}\, \emwt \frac{F(\omega)}{\nu_q - \omega}\, ,
 \end{equation}
where $\nu_q$ is defined by \eqn{f_pole_q}. Alternatively, we can multiply both sides of \eqn{forward_prop_e_q_w} by $i \omega + \ln \Gamma$, take the Fourier transform, and then apply the Fourier derivative theorem to find the ordinary differential equation
 \begin{equation} \label{eqn:forward_prop e_q_t}
\dot{E}_q(t) + \frac{1}{2 \tau_p}\, E_q(t) = \frac{i \sqrt{\eta T}}{\mathcal{C}}\, e^{i 2 q \pi t} F(t)\, ,
 \end{equation}
where we have used \eqn{f_fwhm} to write $\ln(1/\Gamma) = 1/2 \tau_p$. The damping term on the \lhs of this equation has two contributions:
 \begin{equation}
\frac{1}{2 \tau_p} = \half\, \alpha\wn + \half\, \ln \left( \frac{1}{1 - T} \right)\, ,
 \end{equation}
First, $\alpha\wn/2$ describes the rate (scaled by $\tau_g^{-1}$) at which the intracavity field in \eqn{forward_prop e_q_t} decays due to internal dissipative losses. Second, in the limit $T \ll 1$, the term $T/2$ represents the rate at which cavity fields ``leak'' out of the resonator. The relationship between this leakage rate --- proportional to $T$ --- and the coupling coefficient for the field incident on the external reference plane of the mirror $\mathcal{M}_1$ --- proportional to $\sqrt{T}$ --- is consistent with the formulations of temporal coupled-mode theory developed in \cite{ref:haus1984wfo} and \cite{ref:fan2003tcm}.

Consider the simple case where $F(\omega)$ describes a single-mode input field, given by
 \begin{equation}
F(\omega) = \frac{2 \pi}{\sqrt{\eta}} \, \delta(\omega - \Delta \omega)\, .
 \end{equation}
Direct application of \eqn{eqt_int} gives
 \begin{equation} \label{eqn:eqt_inj}
E_q(t) = \frac{\sqrt{T}}{\mathcal{C} (\nu_q - \Delta \omega)}\, e^{i (2 q \pi - \Delta \omega) t}\, .
 \end{equation}
Therefore, substitution of this expression into \eqn{laser_resonator_1d_eq_def} yields the time-domain expansion
 \begin{equation} \label{eqn:eqt_exp}
E\zt = \sqrt{T}\, e^{-i \Delta \omega t} \sum_q \frac{e^{\left[ i 2 q \pi + \ln(1/\sqrt{R}) \right] z}}{\nu_q - \Delta \omega}\, ,
 \end{equation}
while the Fourier transform of \eqn{forward_prop_w_z} given the same single-mode input field yields
 \begin{equation} \label{eqn:eqt_exact}
E\zt = i\, \sqrt{T}\, e^{-i \Delta \omega t} \frac{e^{\left[ i \Delta \omega - \alpha\wn/2\right] z}}{1 - \Gamma\, e^{i \Delta \omega}}\, .
 \end{equation}
In \fig{qnm_inj_jq}, we show a plot of \eqn{eqt_inj} for 11 modes at $t = 0$ with $R = 0.5$, $\alpha\wn = 0.1$, and $\Delta \omega = 0.5$. Using the same parameters, \fig{qnm_inj_ez} shows plots of \eqn{eqt_exp} and \eqn{eqt_exact} at $t = 0$, where we have truncated the sum in \eqn{eqt_exp} at three different values of $\pm q_\mathrm{max}$. Note that the eigenmode expansion for $\Re[E(z, 0)]$ is quite accurate even for $q_\mathrm{max} = 5$, but larger sets of modes improve the estimates of $\Im[E(z, 0)]$, particularly near the mirrors.
 \begin{figure}
  \centering
  \begin{subfigure}[b]{0.8\textwidth}
   \centering
   \includegraphics[width=5.0in]{figures/qnm_inj_jq_abs}
   \caption{Absolute value of $E_q(0)$}
   \label{fig:qnm_inj_jq_abs}
  \end{subfigure}
  \par\vspace{0.25in}
  \begin{subfigure}[b]{0.8\textwidth}
   \centering
   \includegraphics[width=5.0in]{figures/qnm_inj_jq_ang}
   \caption{Phase angle of $E_q(0)$}
   \label{fig:qnm_inj_jq_ang}
  \end{subfigure}
  \caption{Plot of \eqn{eqt_inj} for 11 modes at $t = 0$ with $R = 0.5$, $\alpha\wn = 0.1$, and $\Delta \omega = 0.5$. We have defined $E_q(t) \equiv |E_q(t)|\, e^{i \phi_q(t)}$, and graphed $|E_q(0)|$ and $\phi_q(0)$ separately.\label{fig:qnm_inj_jq}}
 \end{figure}

 \begin{figure}
  \centering
  \begin{subfigure}[b]{0.8\textwidth}
   \centering
   \includegraphics[width=5.0in]{figures/qnm_inj_ez_re}
   \caption{Real part of $E(z, 0)$}
   \label{fig:qnm_inj_ez_re}
  \end{subfigure}
  \par\vspace{0.25in}
  \begin{subfigure}[b]{0.8\textwidth}
   \centering
   \includegraphics[width=5.0in]{figures/qnm_inj_ez_im}
   \caption{Imaginary part of $E(z, 0)$}
   \label{fig:qnm_inj_ez_im}
  \end{subfigure}
  \caption{Plots of \eqn{eqt_exp} and \eqn{eqt_exact} at $t = 0$ using the same values of $R$, $\alpha\wn$, and $\Delta \omega$ as in \fig{qnm_inj_jq}. We have truncated the sum in \eqn{eqt_exp} at three different values of $\pm q_\mathrm{max}$. Note that the eigenmode expansion for $\Re[E(z, 0)]$ is quite accurate even for $q_\mathrm{max} = 5$, but larger sets of modes improve the estimates of $\Im[E(z, 0)]$, particularly near the mirrors. \label{fig:qnm_inj_ez}}
 \end{figure}

% \red{\begin{itemize}
%   \item Show that the convergence of \eqn{laser_resonator_1d_ezt_expansion} --- using \eqn{forward_prop e_q_w} --- is quite good everywhere \emph{except} at the boundaries. Although it looks like a straightforward example of Gibbs' phenomenon, it is actually a result of setting $z = 0$ or $z = L$ in \eqn{laser_resonator_1d_ezt_expansion}, computing the value of $u_q\z$ for some values of $q$, and then performing the sum. In the spirit of \eqn{laser_resonator_1d_uv_complete}, introduce the idea of a regularization scheme where the value of the series is computed at a value of $z$ that is \emph{almost} at the reference plane.
% \end{itemize}}
 \subsection{One-Dimensional Standing-Wave Resonator}

For a standing-wave resonator, we compute the expansion coefficient in \eqn{laser_resonator_1d_eq_def_sw} in the frequency domain as
 \begin{equation} \label{eqn:eqw_swl_def}
E_q(\omega) = \int_{-\infty}^{+\infty} d t\, \epwt\, E_q(t) = \int_0^{1/2} d z\, \mathbf{v}_q\z \dotp \mathbf{E}(z, \omega + 2 q \pi)\, .
 \end{equation}
Both \eqn{forward_prop_w_z} and \eqn{forward_prop_w_0} remain valid for $E^{+}\zw$ in the standing-wave case where $0 < z < 1/2$, provided that we define $\Gamma^2 = R_1 R_2\, e^{-\alpha\wn}$. Rewriting \eqn{resonator_1d_w_bc} to be consistent with the sign convention chosen in \eqn{laser_resonator_1d_w_bc_sw}, we calculate the field amplitude just inside the reference plane of mirror $\mathcal{M}_1$ as
 \begin{equation}
E^{-}(0, \omega) = -\frac{1}{\sqrt{R_1}}\, \left[ E^{+}(0, \omega) - i\, \sqrt{\eta T_1}\, F(\omega) \right] = -i\, \sqrt{\frac{\eta T_1}{R_1}}\, \frac{\Gamma\, e^{i \omega}}{1 - \Gamma\, e^{i \omega}}\, F(\omega)\, ,
 \end{equation}
and then write $E^{-}\zw$ as
 \begin{equation}
E^{-}\zw = \exp\left[ -i\, \omega\, z + \half \alpha\wn\, z \right] E^{-}(0, \omega)\, .
 \end{equation}

Substituting this expression and \eqn{forward_prop_w_z} into \eqn{eqw_swl_def} yields
 \begin{equation} \label{eqn:eqw_swl}
 \begin{split}
E_q(\omega) &= \frac{i}{\mathcal{C}}\, \frac{\sqrt{\eta T_1}\, F(\omega + 2 q \pi)}{1 - \Gamma\, e^{i \omega}}\, \int_0^{1/2} d z\, \left[ e^{(i \omega + \ln \Gamma) z} + \Gamma e^{i \omega} e^{-(i \omega + \ln \Gamma) z} \right] \\
&= \frac{i}{\mathcal{C}}\, \frac{\sqrt{\eta T_1}\, F(\omega + 2 q \pi)}{\left(1 - \Gamma\, e^{i \omega}\right)\left(i \omega + \ln \Gamma\right)}\, \left\{ \left[ e^{(i \omega + \ln \Gamma)/2} - 1 \right] - \Gamma e^{i \omega} \left[ e^{-(i \omega + \ln \Gamma)/2} - 1 \right] \right\} \\
&= -\frac{i \sqrt{\eta T_1}}{\mathcal{C}} \frac{F(\omega + 2 q \pi)}{i \omega + \ln \Gamma}\, ,
 \end{split}
 \end{equation}
which is identical to \eqn{forward_prop_e_q_w} with $T \longrightarrow T_1$ and a properly generalized definition of $\Gamma$.


   %\input{files/laser_resonators_1d_sml}
