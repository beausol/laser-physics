%%%%%%%%%%%%%%%%%%%%%%%%%%%%%%%%%%%%%%%%%%%%%%%%%%%%%%%%%%%%%%%%%%%%%%%%%%%%%%
%
% Subsection file included in section file using \input{}
%
% Assumes that LaTeX2e macros and packages defined in rgb_laser_physics.sty
%   are available
%
%%%%%%%%%%%%%%%%%%%%%%%%%%%%%%%%%%%%%%%%%%%%%%%%%%%%%%%%%%%%%%%%%%%%%%%%%%%%%%
 \section{Injection-Seeded Gain-Switched Lasers\label{sct:laser_dynamics_1d_mml_qsl}}

We consider the ideal four-level laser dynamical equations developed in \sct{laser_amp_1d_pdes}, and we assume that $\gamma_\perp \longrightarrow \infty$, so that $\Omega = 0$. The formal integral of \eqn{cw_sml_ftz_scaled} becomes
 \begin{equation} \label{eqn:qsl_ftzt_formal}
\widetilde{F}\zt = \frac{\gamma_\perp}{2}\, e^{-\gamma_\perp t} \int_{-\infty}^{t} d t^\prime\, e^{\gamma_\perp t^\prime}\, \widetilde{G}\left(z, t^\prime\right) \widetilde{E}\left(z, t^\prime\right) \, .
 \end{equation}
We now apply the strong REA in the same limit, and assume that $|\partial \widetilde{E}\zt / \partial t| \ll \gamma_\perp |\widetilde{E}\zt|$, and $|\partial \widetilde{G}\zt / \partial t| \ll \gamma_\perp |\widetilde{G}\zt|$. In this case, both $\widetilde{E}\left(z, t^\prime\right)$ and $\widetilde{G}\left(z, t^\prime\right)$ can be moved outside of the integral, yielding
 \begin{equation} \label{eqn:qsl_ftzt_rea}
\widetilde{F}\zt = \frac{1}{2}\, \widetilde{G}\zt\, \widetilde{E}\zt \, .
 \end{equation}
Substituting this result into \eqn{cw_sml_gtz_scaled} gives
 \begin{equation} \label{eqn:qsl_dgdt_rea}
\ppt \widetilde{G}\zt = \frac{1}{\tau_\parallel} \left[ \overline{G}\zt - \widetilde{G}\zt - \widetilde{G}\zt \left| \widetilde{E}\zt\right|^2 \right]\, ,
 \end{equation}

Let's allow the intracavity field to be supplemented by a quantity $\widetilde{J}\zt$ arising from a very weak input $F_1(t)$ injected through the output coupler mirror $\mathcal{M}_1$, as shown in \fig{resonator_1d_smat}. This additional field will contribute to the total macroscopic polarization, so that
 \begin{equation} \label{eqn:qsl_ftzt_inj}
\widetilde{F}\zt = \frac{1}{2}\, \widetilde{G}\zt\, \left[\widetilde{E}\zt + \widetilde{J}\zt\right]\, .
 \end{equation}
In \sct{laser_resonators_1d_tcm}, we learned how to expand $\widetilde{J}\zt$ as a series of quasi-normal spatial modes in both the unidirectional ring and standing-wave resonator cases. Because the injected field is so weak, we do not need to include it in the saturation term in \eqn{qsl_dgdt_rea}.

 \subsection{Unidirectional Ring Lasers\label{sct:laser_dynamics_1d_mml_qsl_url}}
As discussed in the introduction to \chp{laser_dynamics_1d_mml}, in the case of the URL the rapidly-varying spatial function $\exp(+i k_0 z)$ is common to both $\widetilde{E}\zt$ and $\widetilde{F}\zt$, and can therefore be ignored in \eqn{qsl_ftzt_rea} and \eqn{qsl_dgdt_rea}. Therefore, we find $F_q(t)$ in \eqn{mml_edot_temp} by substituting \eqn{mml_e_field_1d_t} --- and the corresponding expression for $J\zt$ --- into \eqn{qsl_ftzt_inj}, and then the result into \eqn{mml_1d_fq_def_url}. We obtain
 \begin{equation} \label{eqn:qsl_url_fqt}
F_q(t) = \half\, \sum_p e^{i 2 ( q - p ) \pi t}\, G_{q - p}(t) \left[ E_p(t) + J_p(t) \right]\, ,
 \end{equation}
where
 \begin{equation} \label{eqn:qsl_gqp_def}
G_{q - p}(t) \equiv \int_0^1 d z\, v_q\z\, u_p\z\, G\zt = \int_0^1 d z\, e^{-i 2 (q - p) \pi z}\, G\zt\, ,
 \end{equation}
and $J_p(t)$ is given by \eqn{eqt_inj}. \Eqn{qsl_url_fqt} and the slowly-spatially-varying partial differential equation
 \begin{equation} \label{eqn:qsl_dgdt_url}
\ppt G\zt = \frac{1}{\tau_\parallel} \left[ \overline{G}\zt - G\zt - G\zt \left| E\zt\right|^2 \right]\,
 \end{equation}
are the only tools we'll need to solve numerically a wide variety of gain-switched URL problems.

 \subsection{Standing-Wave Lasers\label{sct:laser_dynamics_1d_mml_qsl_swl}}
The calculation of the macroscopic polarization for a multimode standing-wave laser requires that we pay attention to the interference between the counterpropagating fields. We'll follow a strategy similar to that of the continuous-wave case described in \sct{laser_statics_1d_shb}. We begin with an explicit expression for the spatially rapidly-varying polarization of \eqn{qsl_ftzt_inj}, written as
 \begin{multline} \label{eqn:qsl_1d_fzt_swl}
F^+\zt\, e^{+i k_0 z} + F^-\zt\, e^{-i k_0 z} = \\ \half\, \widetilde{G}\zt \left\{\left[E^+\zt + J^+\zt\right] e^{+i k_0 z} + \left[E^-\zt + J^-\zt\right] e^{-i k_0 z}\right\}\, .
 \end{multline}
The envelope functions $F^\pm\zt$, $E^\pm\zt$, and $J^\pm\zt$ are spatially slowly varying, but we will need to average $\widetilde{G}\zt$ over a physical wavelength. Following the procedure outlined in \eqn{ld1d_sw_shb_pzp_full}, we find
 \begin{subequations} \label{eqn:qsl_1d_fpmzt_swl}
 \begin{align}
F^+\zt &= \half\, \mathcal{G}^{[0]}\zt \left[E^+\zt + J^+\zt\right] + \half\, \mathcal{G}^{[-2]}\zt \left[E^-\zt + J^-\zt\right]\, , \nd \\
F^-\zt &= \half\, \mathcal{G}^{[+2]}\zt \left[E^+\zt + J^+\zt\right] + \half\, \mathcal{G}^{[0]}\zt \left[E^-\zt + J^-\zt\right]\, ,
 \end{align}
 \end{subequations}
where
 \begin{equation} \label{eqn:qsl_1d_gnzt}
\mathcal{G}^{[n]}\zt \equiv \frac{k_0}{2 \pi} \int_{z - \pi/k_0}^{z + \pi/k_0} d z^\prime\, e^{+i n k_0 z^\prime}\, \widetilde{G}(z^\prime, t)\, .
 \end{equation}
Substituting \eqn{qsl_1d_fpmzt_swl} into \eqn{mml_1d_fq_def_swl} yields
 \begin{multline}
F_q(t) = \half \sum_p e^{i 2 (q - p) \pi t} \left[E_p(t) + J_p(t)\right]
\int_0^{1/2} d z\, \left[ \mathbf{v}_q\z \dotp \mathbf{u}_p\z\, \mathcal{G}^{[0]}\zt \right. \\
\left. + v_q^+\z\, u_p^-\z\, \mathcal{G}^{[-2]}\zt + v_q^-\z\, u_p^+\z\, \mathcal{G}^{[+2]}\zt \right]\, .
 \end{multline}

We'll need to construct partial differential equations for $\mathcal{G}^{[0]}\zt$ and $\mathcal{G}^{[\pm 2]}\zt$. We expand $|\widetilde{E}\zt|^2$ as
 \begin{equation}
\left|\widetilde{E}\zt\right|^2 = \left|E^{+}\zt\right|^2 +  \left|E^{-}\zt\right|^2 + E^{+}\zt\, E^{- \ast}\zt\, e^{+i 2 k_0 z} + E^{-}\zt\, E^{+ \ast}\zt\, e^{-i 2 k_0 z}\, ,
 \end{equation}
substitute this expression into \eqn{qsl_dgdt_rea}, and then apply the average specified by \eqn{qsl_1d_gnzt} to obtain
 \begin{equation} \label{eqn:qsl_dgndt_swl}
 \begin{split}
\ppt \mathcal{G}^{[n]}\zt &= \frac{1}{\tau_\parallel} \left\{ \delta_{n, 0}\, \overline{G}\zt - \mathcal{G}^{[n]}\zt - \mathcal{G}^{[n]}\zt \left[ \left|E^{+}\zt\right|^2 +  \left|E^{-}\zt\right|^2 \right] \right. \\
&\qquad \left. -~\mathcal{G}^{[n + 2]}\zt\, E^{+}\zt\, E^{- \ast}\zt - \mathcal{G}^{[n - 2]}\zt\, E^{-}\zt\, E^{+ \ast}\zt \right\}\, .
 \end{split}
 \end{equation}
If the pump has a short duration compared to $\tau_\parallel$, then rapid temporal oscillations in $E^{\pm}\zt\, E^{\mp \ast}\zt$ will diminish the contributions of higher-order spatial averages and allow us to neglect $\mathcal{G}^{[\pm 4]}\zt$. 